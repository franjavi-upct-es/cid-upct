\documentclass[12pt, a4paper]{book}
\title{\textbf{LOS ELEMENTALES}}
\date{}
\author{Francisco Javier Mercader Martínez}
\renewcommand{\chaptername}{Capítulo}
\setlength{\parskip}{0.3cm}
\setlength{\parindent}{12pt}

\begin{document}
\maketitle
\chapter{}
Antes de todo, el mundo era distinto a cómo lo podíamos imaginar. 

Al principio, sólo había oscuridad.

Pero de esa oscuridad llegó la luz, poderosa, brillante y envolvente. Seguido de ella, llegó el mundo. 

Fruto de una creación salvaje, descontrolada, capaz de encoger hasta el corazón más valiente. 

Primero fue la tierra, surgida de las violentas erupciones volcánicas que provocaban que todo se agitase, el humo y el calor del magma  envolvieron el mundo como una fuerza de la naturaleza que no encontraba rival alguno, hasta ser detenida por su enemigo natural, el agua y el frío. Debido al gran desequilibrio que en el aquel momento regía las lluvias, como un ser viviente, calmaron el estruendo incesante que asolaba al planeta y lo cubrieron todo. 

Pero al igual que cualquier otra fuerza de la naturaleza, su presencia se descontroló, amenazando al mismo equilibrio que había desafiado su antecesor. Entonces, el mundo se empezó a llenar de aquel color celeste que perduraría a través de los siglos y los milenios. De aquello surgieron los océanos que actuarían como grandes fronteras naturales que dejaron al mundo casi dividido en cinco grandes continentes, cada uno de ellos separados por largas distancias de aguas turbulentas.

Pero aún quedaba algo más por llegar: \emph{la vida...}

Para encontrar el equilibrio, el mundo se fragmentó y de esa división llegaron \textit{los primeros}. Eran seres majestuosos, sin forma, y cada uno representaba una fuerza primordial del mundo primigenio: el fuego, el agua, el aire, la tierra y la naturaleza.

Su nacimiento y su razón de existir era desconocida para ellos, pero, de un modo extraño, todos compartían el mismo objetivo: poblar el mundo. Entonces comenzaron su labor.

Dieron forma a las montañas, a los bosques, a los ríos... y en aquel momento se sintieron satisfechos. Miles de años pasaron mientras ellos veían como su obra se iba desarrollando. Veían cómo la vida se iba desarrollando, cómo los peces ocupaban los mares y ríos, como las aves alcanzaban los cielos, y cómo las bestias se alzaban sobre la tierra. Pero aún con aquellos logros, su satisfacción iba en decadencia. Todos sentían que aún faltaba algo, una cualidad de la que carecían aquellas criaturas movidas nada más que por sus instintos, simples y mundanos.

Los cinco se reunieron, movidos por un sentimiento ardiente de que algo aún faltaba por hacer, algo que marcaría un punto de inflexión en el mundo, y crearon a la humanidad...

Les dotaron de grandes cualidades que los diferenciaban del resto, les dieron inteligencia, les enseñaron el lenguaje y algo más, les dotaron de poder.

De esa forma nacieron los cinco grandes linajes, los herederos de aquel poder ancestral tan antiguo como el mundo, para que una parte de los creadores originales estuviera siempre en compañía de ellos para que sus ideales perduraran hasta el fin de los días.

Pero como en toda historia y como en la vida misma, surgieron los problemas.

Todo llegó de forma inesperada, cuando una gran brecha de colores negro y violeta partió el espacio y el tiempo, como una puerta que dio la bienvenida a unos seres que cambiarían el curso de la historia de forma irreparable. No era posible describirlos con palabras puesto que eran criaturas sin forma, hechas de energía cuya presencia ensombrecía hasta la más brillante luz del día.

Los señores de la creación estaban confundidos, pues había algo en aquellos extraños visitantes que les cautivaba y atemorizaba al mismo tiempo. Lo más curioso era una sensación que se incrustó dentro de sus mentes, como un leve pinchazo que no les dejaba pensar con claridad, sentían como si se mirasen en un espejo... 

Aquel característico parecido nubló el juicio de los creadores y no les permitió ver más allá. La primera chispa de curiosidad se encendió en sus corazones, una sensación que jamás habían experimentado y la cual no eran capaces de describir, era como si se sintieran atraidos hacia aquellos extranjeros. 

Finalmente estaban todos frente a frente, pero permanecieron quietos como estatuas sin que ninguno tratara de pronunciar una sola palabra.

Y fue entonces, al extenderles la mano a sus visitantes que descubrieron la verdad.

Al momento de hacer contacto, pudieron verlo todo: sus nombres, su poder y lo más importante, su origen. En él vieron un mundo devastado y sumido en el caos, pero de los visitantes no emanaba ningún sentimiento de pena, sino de júbilo.

Inmediatamente los soltaron y conocieron sus verdaderas intenciones, no estabán aquí por casualidad ya que su objetivo era apoderarse de este nuevo mundo para corromperlo y oscurecerlo a placer.

Todos retrocedieron en aquel preciso momento pero hubo uno que no retrocedió, uno que no sintió miedo, el fuego.


\end{document}
