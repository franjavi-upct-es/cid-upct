\documentclass{article}
\usepackage{fullpage}
\usepackage[utf8]{inputenc}
\usepackage{pict2e}
\usepackage{amsmath}
\usepackage{enumitem}
\usepackage{eurosym}
\usepackage{mathtools}
\usepackage{amssymb, amsfonts, latexsym, cancel}
\setlength{\parskip}{0.3cm}
\usepackage{graphicx}
\usepackage{fontenc}
\usepackage{slashbox}
\usepackage{setspace}
\usepackage{gensymb}
\usepackage{accents}
\usepackage{adjustbox}
\setstretch{1.35}
\renewcommand*{\arraystretch}{1.5}
\usepackage{bold-extra}
\usepackage{subcaption}
\usepackage{tcolorbox}
\usepackage{xcolor, colortbl}
\usepackage{wrapfig}
\usepackage{empheq}
\usepackage{array}
\usepackage{parskip}
\usepackage{arydshln}
\graphicspath{ {images/} }
\renewcommand*\contentsname{\color{black}Índice} 
\usepackage{array, multirow, multicol}
\definecolor{lightblue}{HTML}{007AFF}
\usepackage{color}
\usepackage{etoolbox}
\usepackage{listings}
\usepackage{mdframed}
\setlength{\parindent}{0pt}
\usepackage{underscore}
\usepackage{hyperref}
\usepackage{tikz}
\usepackage{tikz-cd}
\usetikzlibrary{shapes, positioning, patterns}
\usepackage{tikz-qtree}
\usepackage{biblatex}
\usepackage{pdfpages}
\usepackage{pgfplots}
\usepackage{pgfkeys}
\addbibresource{biblatex-examples.bib}
\usepackage[a4paper, left=1cm, right=1cm, top=1cm, bottom=1.5cm]{geometry}
\usepackage{titlesec}
\usepackage{titletoc}
\usepackage{tikz-3dplot}
\usepackage{kbordermatrix}
\usetikzlibrary{decorations.pathreplacing}
\newcommand{\Ej}{\textcolor{lightblue}{\underline{Ejemplo}}}
\setlength{\fboxrule}{1.5pt}

\newcommand{\bboxed}[1]{\fcolorbox{lightblue}{lightblue!10}{$#1$}}
\newcommand{\rboxed}[1]{\fcolorbox{red}{red!10}{$#1$}}

\DeclareMathOperator{\N}{\mathbb{N}}
\DeclareMathOperator{\Z}{\mathbb{Z}}
\DeclareMathOperator{\R}{\mathbb{R}}
\DeclareMathOperator{\Q}{\mathbb{Q}}
\DeclareMathOperator{\K}{\mathbb{K}}
\DeclareMathOperator{\im}{\imath}
\DeclareMathOperator{\jm}{\jmath}
\DeclareMathOperator{\col}{\mathrm{Col}}
\DeclareMathOperator{\fil}{\mathrm{Fil}}
\DeclareMathOperator{\rg}{\mathrm{rg}}
\DeclareMathOperator{\nuc}{\mathrm{nuc}}
\DeclareMathOperator{\dimf}{\mathrm{dimFil}}
\DeclareMathOperator{\dimc}{\mathrm{dimCol}}
\DeclareMathOperator{\dimn}{\mathrm{dimnuc}}
\DeclareMathOperator{\dimr}{\mathrm{dimrg}}
\DeclareMathOperator{\dom}{\mathrm{Dom}}
\DeclareMathOperator{\infi}{\int_{-\infty}^{+\infty}}
\newcommand{\dint}[2]{\int_{#1}^{#2}}

\newcommand{\bu}[1]{\textcolor{lightblue}{\underline{#1}}}
\newcommand{\lb}[1]{\textcolor{lightblue}{#1}}
\newcommand{\db}[1]{\textcolor{blue}{#1}}
\newcommand{\rc}[1]{\textcolor{red}{#1}}

\renewcommand{\CancelColor}{\color{lightblue}}
\newcommand{\code}[1]{\texttt{\textbf{#1}}}

\usepackage{textgreek}

\newcommand{\dx}{\:\mathrm{d}x}
\newcommand{\dt}{\:\mathrm{d}t}
\newcommand{\dy}{\:\mathrm{d}y}
\newcommand{\dz}{\:\mathrm{d}z}
\newcommand{\dth}{\:\mathrm{d}\theta}
\newcommand{\dr}{\:\mathrm{d}\rho}
\newcommand{\du}{\:\mathrm{d}u}
\newcommand{\dv}{\:\mathrm{d}v}
\newcommand{\tozero}[1]{\cancelto{0}{#1}}
\newcommand{\lbb}[2]{\textcolor{lightblue}{\underbracket[1pt]{\textcolor{black}{#1}}_{#2}}}
\newcommand{\dbb}[2]{\textcolor{blue}{\underbracket[1pt]{\textcolor{black}{#1}}_{#2}}}
\newcommand{\rub}[2]{\textcolor{red}{\underbracket[1pt]{\textcolor{black}{#1}}_{#2}}}

\titleformat{\section}{\normalfont\LARGE\bfseries}{\thesection.}{10 pt}{}

\title{\textbf{\huge Práctica 4 de Señales y Sistemas}\\ Señales y sistemas discretos en el dominio de la frecuencia}
\author{Francisco Javier Mercader Martínez\\ Rubén Gil Martínez}
\date{}

\usepackage{matlab-prettifier}

\lstset{
language=matlab,
basicstyle=\ttfamily\small,
keywordstyle=\color{blue},
commentstyle=\color{green!70!black},
stringstyle=\color{red},
showstringspaces=false,
breaklines=true,
frame=single,
backgroundcolor=\color{lightgray!10},
captionpos=b,
tabsize=2,
inputencoding=utf8,
literate={á}{{\'a}}1 {é}{{\'e}}1 {í}{{\'i}}1 {ó}{{\'o}}1 {ú}{{\'u}}1{ñ}{{\~n}}1
}

\everymath{\displaystyle}

\begin{document}
\maketitle

\section{Transformada de Fourier de secuencias}
\subsection*{Cuestiones}
Realice un script en Matlab para representar una secuencia \code{x} y su transformada de Fourier \code{X}. Las secuencias están muestreadas a una frecuencia \code{fs}. Para ello siga los siguientes pasos:
\begin{itemize}
\item Inicialice \code{ft = inline('fftshift(fftn(x))');}
\item Las secuencias de audio y su frecuencia de muestreo son accesibles desde Matlab- Para ello tiene que cargar los ficheros indicados más adelante mediante la instrucción \code{load}.
\item Realice la representación gráfica de la secuencia empleando el eje de tiempo discreto. Etiquete adecuadamente cada uno de los ejes indicando claramente su contenido y en caso de que sea necesario las unidades.
\item Empleando las instrucciones \code{subplot} represente la transformada de Fourier de la señal en dominio continuo y la transformada de Fourier de la secuencia. De nuevo incluya el etiquetado de los ejes.
\item Reproduzca empleando la instrucción \code{sound} a la frecuencia de muestreo indicada la secuencia de audio.
\end{itemize}
Se muestran las secuencias a continuación que se han de estudiar empleando el código programado. Se estudiará la representación en tiempo y frecuencia. De manera cualitativa establezca relaciones entre las frecuencias que escucha y las que se pueden observar en la transformada de Fourier representada:
\begin{itemize}
\item \textbf{Secuencia 'splat' muestreada a 8 KHz:} 
\begin{lstlisting}
load splat.mat
\end{lstlisting}
\item \textbf{Secuencia 'laughter' muestreada a 8 KHz:}
\begin{lstlisting}
load laughter.mat
\end{lstlisting}
\item \textbf{Secuencia 'handel' muestreada a 8 KHz:}
\begin{lstlisting}
load handel.mat
\end{lstlisting}
\end{itemize}
\section{Comparativa del coste computacional DFT y FFT}
\subsection*{Cuestiones}
Implemente dos funciones en Matlab:
\begin{itemize}
\item La primera debe emplear la implementación directa con Matlab y su entrada será la secuencia a transformar. La función proporcionará la transformada. Llame a la función \code{mdft.m}
\item La segunda debe implementar la versión eficiente basada en el algoritmo de diezmado en tiempo. El nombre de la función será \code{ffdt.m}
\end{itemize}
Redacte un script en Matlab que obtenga una secuencia real $x$ de longitud $N$ con muestras aleatorias empleando el comando \code{rand}.

Empleando los comandos \code{tic} y \code{toc} establezca el tiempo empleado en calcular su DFT utilizando la función de Matlab \code{mdft.m} y mediante la versión eficaz del mismo denominada \code{ffdt.m}

Empleando lo siguientes valores para $N=2^{10},2^{11},2^{12},2^{13},2^{14},2^{15},2^{16}$ y $2^{17}$ represente en una misma gráfica los tiempos de cálculo empleados por las dos implementaciones de la DFT respecto al valor de $N$. Etiquete correctamente ambos ejes e incluya la correspondiente leyenda para diferenciar ambas curvas.
\section{Comparativa de convolución lineal vs convolución circular}
\subsection*{Cuestiones}
Implemente el código necesario para generar y representar la convolución lineal de:
\begin{itemize}
\item Una secuencia de números aleatorios de 64 muestras definida en $n=0..63$.
\item Un filtro FIR casual que realice el promedio deslizante de 8 muestras.
\end{itemize}
Utilice la función \code{convcirc(x1,x2,N)} para realizar la convolución circular de $N$ puntos de secuencia y la respuesta al impulso del filtro. Considere el mínimo $N$ posible y represente gráficamente el resultado, etiquetando correctamente el eje temporal, $n=0..N-1$.

Defina la convolución circular de la siguiente manera
\begin{lstlisting}
convcirc = inline('real(ifft(fft(x1, N) .* fft(x2, N), N))', 'x1', 'x2', 'N');
\end{lstlisting}
\begin{itemize}
\item Especifique si proporcionan el mismo resultado la convolución lineal y la convolución circular. Indique las muestras que difieren.
\item Incremente el tamaño $N$ de la convolución circular. Indique el valor de $N$ para el que se obtiene un mismo resultado para ambas convoluciones.
\end{itemize}
\end{document}