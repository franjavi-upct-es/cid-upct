\documentclass[12pt]{article}
\usepackage{fullpage}
\usepackage[utf8]{inputenc}
\usepackage{pict2e}
\usepackage{amsmath}
\usepackage{enumitem}
\usepackage{eurosym}
\usepackage{mathtools}
\usepackage{amssymb, amsfonts, latexsym, cancel}
\setlength{\parskip}{0.3cm}
\usepackage{graphicx}
\usepackage{fontenc}
\usepackage{slashbox}
\usepackage{setspace}
\usepackage{gensymb}
\usepackage{accents}
\usepackage{adjustbox}
\setstretch{1.35}
\usepackage{bold-extra}
\usepackage{subcaption}
\usepackage{tcolorbox}
\usepackage{xcolor, colortbl}
\usepackage{wrapfig}
\usepackage{empheq}
\usepackage{array}
\usepackage{parskip}
\usepackage{arydshln}
\graphicspath{ {images/} }
\renewcommand*\contentsname{\color{black}Índice} 
\usepackage{array, multirow, multicol}
\definecolor{lightblue}{HTML}{007AFF}
\usepackage{color}
\usepackage{etoolbox}
\usepackage{listings}
\usepackage{mdframed}
\setlength{\parindent}{0pt}
\usepackage{underscore}
\usepackage{hyperref}
\usepackage{tikz}
\usepackage{tikz-cd}
\usetikzlibrary{shapes, positioning, patterns}
\usepackage{tikz-qtree}
\usepackage{biblatex}
\usepackage{pdfpages}
\usepackage{pgfplots}
\usepackage{pgfkeys}
\addbibresource{biblatex-examples.bib}
\usepackage[a4paper, left=1cm, right=1cm, top=1cm, bottom=1.5cm]{geometry}
\usepackage{titlesec}
\usepackage{titletoc}
\usepackage{tikz-3dplot}
\usepackage{kbordermatrix}
\usetikzlibrary{decorations.pathreplacing}
\newcommand{\Ej}{\textcolor{lightblue}{\underline{Ejemplo}}}
\setlength{\fboxrule}{1.5pt}

\newcommand{\bboxed}[1]{\fcolorbox{lightblue}{lightblue!10}{$#1$}}
\newcommand{\rboxed}[1]{\fcolorbox{red}{red!10}{$#1$}}

\DeclareMathOperator{\N}{\mathbb{N}}
\DeclareMathOperator{\Z}{\mathbb{Z}}
\DeclareMathOperator{\R}{\mathbb{R}}
\DeclareMathOperator{\Q}{\mathbb{Q}}
\DeclareMathOperator{\K}{\mathbb{K}}
\DeclareMathOperator{\im}{\imath}
\DeclareMathOperator{\jm}{\jmath}
\DeclareMathOperator{\col}{\mathrm{Col}}
\DeclareMathOperator{\fil}{\mathrm{Fil}}
\DeclareMathOperator{\rg}{\mathrm{rg}}
\DeclareMathOperator{\nuc}{\mathrm{nuc}}
\DeclareMathOperator{\dimf}{\mathrm{dimFil}}
\DeclareMathOperator{\dimc}{\mathrm{dimCol}}
\DeclareMathOperator{\dimn}{\mathrm{dimnuc}}
\DeclareMathOperator{\dimr}{\mathrm{dimrg}}
\DeclareMathOperator{\dom}{\mathrm{Dom}}
\DeclareMathOperator{\infi}{\int_{-\infty}^{+\infty}}
\newcommand{\dint}[2]{\int_{#1}^{#2}}

\newcommand{\bu}[1]{\textcolor{lightblue}{\underline{#1}}}
\newcommand{\lb}[1]{\textcolor{lightblue}{#1}}
\newcommand{\db}[1]{\textcolor{blue}{#1}}
\newcommand{\rc}[1]{\textcolor{red}{#1}}

\renewcommand{\CancelColor}{\color{lightblue}}
\newcommand{\code}[1]{\texttt{\textbf{#1}}}

\newcommand{\dx}{\:\mathrm{d}x}
\newcommand{\dt}{\:\mathrm{d}t}
\newcommand{\dy}{\:\mathrm{d}y}
\newcommand{\dz}{\:\mathrm{d}z}
\newcommand{\dth}{\:\mathrm{d}\theta}
\newcommand{\dr}{\:\mathrm{d}\rho}
\newcommand{\du}{\:\mathrm{d}u}
\newcommand{\dv}{\:\mathrm{d}v}
\newcommand{\tozero}[1]{\cancelto{0}{#1}}
\newcommand{\lbb}[2]{\textcolor{lightblue}{\underbracket[1pt]{\textcolor{black}{#1}}_{#2}}}
\newcommand{\dbb}[2]{\textcolor{blue}{\underbracket[1pt]{\textcolor{black}{#1}}_{#2}}}
\newcommand{\rub}[2]{\textcolor{red}{\underbracket[1pt]{\textcolor{black}{#1}}_{#2}}}

\titleformat{\section}{\normalfont\LARGE\bfseries}{\thesection.}{10 pt}{}

\title{\textbf{\huge Práctica 2 de Señales y Sistemas}\\ Convolución y análisis de sistemas LTI}
\author{Francisco Javier Mercader Martínez\\ Rubén Gil Martínez}

\usepackage{matlab-prettifier}

\lstset{
	language=matlab,
	basicstyle=\ttfamily,
	keywordstyle=\color{blue},
	commentstyle=\color{green!70!black},
	stringstyle=\color{red},
	showstringspaces=false,
	breaklines=true,
	frame=single,
	backgroundcolor=\color{lightgray!10},
	captionpos=b,
	tabsize=2,
	inputencoding=utf8,
	literate={á}{{\'a}}1 {é}{{\'e}}1 {í}{{\'i}}1 {ó}{{\'o}}1 {ú}{{\'u}}1{ñ}{{\~n}}1
}

\begin{document}
	\maketitle
	\newpage
	\section{Convolución de señales discretas}
La convolución de dos señales discretas viene dada por la expresión \[ y[n]=h[n]\cdot x[n]=\sum_{k=-\infty}^{\infty}x[k]h[n-k] \]La convolución de dos señales se puede entender de dos maneras desde el punto de vista analítico:
\begin{enumerate}[label=\arabic*)]
	\item En la primera a cada impulso de la señal de entrada, el sistema responde con la respuesta al impulso ponderada por el valor de la señal en ese momento, así: \[ y[n]=\cdots+x[-1]h[n+1]+x[0]h[n]+x[1]h[n-1]+x[2]h[n-2]+\cdots \]
	\item Para la segunda en cada instante de tiempo discreto $n$ la señal de salida $y[n]$ se calcula asumiendo que el eje de tiempos es $k$, se queda fija la señal de entrada, y se invierte y  se desplaza a $n$ la respuesta al impulso, multiplicándose finalmente ambas señales y sumando todos sus valores. De esta manera podemos obtener el resultado:
	\[ \begin{array}{l}
		\cdots\\
		y[-1]=\sum_{k=-\infty}^{\infty}x[k]h[-1-k]\\
		y[0]=\sum_{k=-\infty}^{\infty}x[k]h[0-k]\\
		y[1]=\sum_{k=-\infty}^{\infty}x[k]h[1-k]\\
		\cdots
	\end{array} \]
\end{enumerate}
\subsection*{Cuestiones}
\begin{itemize}[leftmargin=*]
	\item \textbf{Calcule previamente de manera gráfica y a mano la convolución de las dos señales casuales $\mathbf{x[n]}$ y $\mathbf{h[n]}$} que definiremos en MATLAB de la siguiente manera.
	\begin{itemize}[label=\textbullet]
		\item \code{x = [1 2 -2];}
		\item \code{h = [1 3 0 1 2 1 2];}
	\end{itemize}
	
\begin{lstlisting}
x = [1 2 -2];
h = [1 3 0 1 2 1 2];

y = conv(x, h);
\end{lstlisting}
	\item \textbf{Teniendo en cuneta que la longitud de la secuencia $\mathbf{x[n]}$ es $\mathbf{N}$ y la de $\mathbf{h[n]}$ es $\mathbf{M}$, deduzca una expresión para la longitud de $\mathbf{y[n]}$.}
	
	\begin{lstlisting}
longitud_y = length(x) + length(h) - 1;  % = 9
	\end{lstlisting}
	
	\item Programa dos funciones MATLAB, denominadas \code{conv1} y \code{conv2}, que implementen la convolución de dos señales dsicretas mediante el método 1 y el método 2, explicados anteriormente. Las funciones tendrán el formato \code{y=conv1(x,h)} y \code{y=conv2(x,h)}. Puede inicializar la longitud de la señal de salida deducida en el punto anterior al implementear las funciones. Proporcione el código desarrollado.
	\item Compruebe el correcto funcionamiento de las funciones empleando las señales $x[n]$ y $h[n]$ generadas en MATLAB previamente. Consulta la ayuda (con el comnado \code{help}) de la función \code{conv} de MATLAB, que implementa la convolución discreta de dos secuencias. El resultado con \code{conv}, \code{conv1} y \code{conv2} ha de ser el mismo con cualquier par de señales de entrada.
\end{itemize}
\end{document}