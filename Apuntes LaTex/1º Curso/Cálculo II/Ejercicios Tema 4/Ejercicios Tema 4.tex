\documentclass[12pt]{article}
\usepackage{fullpage}
\usepackage[utf8]{inputenc}
\usepackage{pict2e}
\usepackage{amsmath}
\usepackage{enumitem}
\usepackage{eurosym}
\usepackage{pict2e}
\usepackage{mathtools}
\usepackage{amssymb, amsfonts, latexsym, cancel}
\setlength{\parskip}{0.3cm}
\usepackage{graphicx}
\usepackage{fontenc}
\usepackage{slashbox}
\usepackage{setspace}
\usepackage{gensymb}
\usepackage{accents}
\usepackage{adjustbox}
\setstretch{1.5}
\usepackage{bold-extra}
\usepackage[document]{ragged2e}
\usepackage{subcaption}
\usepackage{tcolorbox}
\usepackage{xcolor, colortbl}
\usepackage{wrapfig}
\usepackage{empheq}
\usepackage{array}
\usepackage{parskip}
\usepackage{arydshln}
\graphicspath{ {images/} }
\renewcommand*\contentsname{\color{black}Índice} 
\usepackage{array, multirow, multicol}
\definecolor{lightblue}{HTML}{007AFF}
\usepackage{color}
\usepackage{etoolbox}
\usepackage{listings}
\usepackage{mdframed}
\setlength{\parindent}{0pt}
\usepackage{underscore}
\usepackage{hyperref}
\usepackage{tikz}
\usepackage{tikz-cd}
\usetikzlibrary{shapes, positioning, patterns}
\usepackage{tikz-qtree}
\usepackage{biblatex}
\usepackage{pdfpages}
\usepackage{pgfplots}
\usepackage{pgfkeys}
\addbibresource{biblatex-examples.bib}
\usepackage[a4paper, left=1.5cm, right=1.5cm, top=1cm,
bottom=1.5cm]{geometry}
\everymath{\displaystyle}
\usetikzlibrary{decorations.pathreplacing}
\usepackage{titlesec}
\usepackage{titletoc}
\usepackage{tikz-3dplot}
\usetikzlibrary{decorations.pathreplacing}
\newcommand{\Ej}{\textcolor{lightblue}{\underline{Ejemplo}}}
\setlength{\fboxrule}{1.5pt}
\renewcommand{\arraystretch}{1.35}
\setlength{\arraycolsep}{0.3cm}

% Configura el formato de las secciones utilizando titlesec
\titleformat{\section}
{\color{red}\normalfont\LARGE\bfseries}
{Tema \thesection:}
{10 pt}
{}

% Ajusta el formato de las entradas de la tabla de contenidos
\addtocontents{toc}{\protect\setcounter{tocdepth}{4}}
\addtocontents{toc}{\color{black}}

\titleformat{\subsection}
{\normalfont\Large\bfseries\color{red}}{\thesubsection)}{1em}{\color{lightblue}}

\titleformat{\subsubsection}
{\normalfont\large\bfseries\color{red}}{\thesubsubsection)}{1em}{\color{lightblue}}

\newcommand{\bboxed}[1]{\fcolorbox{lightblue}{lightblue!10}{$#1$}}

\DeclareMathOperator{\N}{\mathbb{N}}
\DeclareMathOperator{\Z}{\mathbb{Z}}
\DeclareMathOperator{\R}{\mathbb{R}}
\DeclareMathOperator{\Q}{\mathbb{Q}}
\DeclareMathOperator{\K}{\mathbb{K}}
\DeclareMathOperator{\im}{\imath}
\DeclareMathOperator{\jm}{\jmath}
\DeclareMathOperator{\col}{\mathrm{Col}}
\DeclareMathOperator{\fil}{\mathrm{Fil}}
\DeclareMathOperator{\rg}{\mathrm{rg}}
\DeclareMathOperator{\nuc}{\mathrm{nuc}}
\DeclareMathOperator{\dimf}{\mathrm{dimFil}}
\DeclareMathOperator{\dimc}{\mathrm{dimCol}}
\DeclareMathOperator{\dimn}{\mathrm{dimnuc}}
\DeclareMathOperator{\dimr}{\mathrm{dimrg}}

\newcommand{\bu}[1]{\textcolor{lightblue}{\underline{#1}}}
\newcommand{\lb}[1]{\textcolor{lightblue}{#1}}
\newcommand{\db}[1]{\textcolor{blue}{#1}}
\newcommand{\rc}[1]{\textcolor{red}{#1}}
\newcommand{\tr}{^\intercal}

\renewcommand{\CancelColor}{\color{lightblue}}

\newcommand{\dx}{\:\mathrm{d}x}
\newcommand{\dt}{\:\mathrm{d}t}
\newcommand{\dy}{\:\mathrm{d}y}
\newcommand{\dz}{\:\mathrm{d}z}
\newcommand{\dth}{\:\mathrm{d}\theta}
\newcommand{\dr}{\:\mathrm{d}\rho}
\newcommand{\du}{\:\mathrm{d}u}
\newcommand{\dv}{\:\mathrm{d}v}
\newcommand{\tozero}[1]{\cancelto{0}{#1}}
\newcommand{\lbb}[2]{\textcolor{lightblue}{\underbracket[1pt]{\textcolor{black}{#1}}_{#2}}}
\newcommand{\dbb}[2]{\textcolor{blue}{\underbracket[1pt]{\textcolor{black}{#1}}_{#2}}}
\title{Cálculo II\\ Tema 4: Teoría de campos}
\begin{document}
\maketitle
\textbf{\Large Capítulo 1: Integración múltiple}
\begin{enumerate}[label=\color{red}\textbf{\arabic*)}, leftmargin=*]
\item \lb{Calcular para $\Omega=[0,1]\times[0,3]$ las integrales}
\begin{enumerate}[label=\color{red}\textbf{\alph*)}]
\item $\db{\iint_{\Omega}xy\:\mathrm{d}x\:\mathrm{d}y}$

$\int_{0}^{3}\int_{0}^{1}xy\:\mathrm{d}x\:\mathrm{d}y=\int_{0}^{3}\left[ \dfrac{x^{2}}{2}y \right]_{x=0}^{x=1}\:\mathrm{d}y=\int_{0}^{3}\dfrac{1}{2}y\:\mathrm{d}y=\left[ \dfrac{1}{4}y^{2} \right]_{y=0}^{y=3}=\dfrac{9}{4}$  

\item $\db{\iint_{\Omega}xe^{ y }\:\mathrm{d}x\:\mathrm{d}y}$

$\int_{0}^{3}\int_{0}^{1}xe^{ y }\:\mathrm{d}x\:\mathrm{d}y=\int_{0}^{3}e^{ y }\int_{0}^{1}x\:\mathrm{d}x\:\mathrm{d}y=\int_{0}^{3}e^{ y }\left[ \dfrac{x^{2}}{2} \right]_{x=0}^{x=1}\:\mathrm{d}y=\int_{0}^{3}\dfrac{e^{ y }}{2}\:\mathrm{d}y=\left[ \dfrac{e^{ y }}{2} \right]_{y=0}^{y=3}=\dfrac{1}{2}(e^{ 3 }-1)$

\item $\db{\iint_{\Omega}y^{2}\sin x\:\mathrm{d}x\:\mathrm{d}y}$

$\int_{0}^{3}\int_{0}^{1}y^{2}\sin x\:\mathrm{d}x\:\mathrm{d}y=\int_{0}^{3}y^{2}\left[ -\cos x \right]_{x=0}^{x=1}\:\mathrm{d}y=\int_{0}^{3}y^{2}(1-\cos(1))\:\mathrm{d}y=\left[ \dfrac{y^{3}}{3} \right]_{y=0}^{y=3}\cdot(1-\cos(1))=9(1-\cos(1))$ 

\end{enumerate}

\item \lb{Calcular las integrales dobles siguientes en los recintos que se indican}
\begin{enumerate}[label=\color{red}\textbf{\alph*)}]
\item $\db{\iint_{\Omega}y\:\mathrm{d}x\:\mathrm{d}y\text{ en }\Omega=\{ (x,y)\in\mathbb{R}^{2}: x^{2}+y^{2}\leq 1 \}}$

$\Omega$ es el disco de radio 1 centrado en el origen.

En coordenadas polares, las variables $x$ y $y$ se expresan como: $$
\begin{cases}
x=r\cos\theta\\
y=r\sin\theta\\
\end{cases}\longrightarrow x^2+y^2=r^2
$$

El elemento de área diferencial $\dx\dy$ se transforma en: $$r\dr\dth.$$Los límites de integración en coordenadas polares son: $$r\in[0,1],\quad\theta\in[0,2\pi].$$

$\int_{0}^{2\pi}\int_{0}^{1}r^2\sin\theta\dr\dth=\int_{0}^{2\pi}\left[\dfrac{r^3}{3}\right]_{r=0}^{r=1}\sin\theta\dth=\int_0^{2\pi}\dfrac{1}{3}\cdot\sin\theta\dth=\dfrac{1}{3}\cdot[-\cos\theta]_{\theta=0}^{\theta=2\pi}=\dfrac{1}{3}\left(-\cos(2\pi)+\cos(0)\right)=\dfrac{1}{3}(-1+1)=0$

\item $\db{\iint_{\Omega}(3y^{3}+x^{2})\:\mathrm{d}x\:\mathrm{d}y\text{ en }\Omega=\{ (x,y)\in\mathbb{R}^{2}: x^{2}+y^{2}\leq 1 \}.}$

$\Omega$ es el disco de radio 1 centrado en el origen.

En coordenadas polares, las variables $x$ y $y$ se expresan como: $$
\begin{cases}
x=r\cos\theta\\
y=r\sin\theta\\
\end{cases}\longrightarrow x^2+y^2=r^2
$$
El elemento de área diferencial $\dx\dy$ se transforma en: $$r\dr\dth.$$Los límites de integración en coordenadas polares son: $$r\in[0,1],\quad\theta\in[0,2\pi].$$

En estas coordenadas, la función $2y^3+x^2$ se convierte en:
$$2y^3+x^2=3(r\sin\theta)^3+(r\cos\theta)^2.$$

$\begin{aligned}\iint_{\Omega}(2y^3+x^2)\:\mathrm{d}x\:\mathrm{d}y&=\int_{0}^{2\pi}\int_{0}^{1}(3(r\sin\theta)^3+(r\cos\theta)^2)\cdot r\:\mathrm{d}r\:\mathrm{d}\theta=\int_{0}^{2\pi} \int_{0}^{1} 3r^{4}\sin ^{3}\theta+r^{3}\cos ^{2}\theta \:\mathrm{d}r\mathrm{d}\theta\\ &=\int_{0}^{2\pi}\left[ \dfrac{3r^{5}}{5} \right]_{r=0}^{r=1}\cdot \sin ^{3}\theta+\left[ \dfrac{r^{4}}{4} \right]_{r=0}^{r=1}\cdot \cos ^{2}\theta\:\mathrm{d}\theta =\int_{0}^{2\pi}\dfrac{3}{5}\sin ^{3}\theta+\dfrac{1}{4}\cos ^{2}\theta \:\mathrm{d}\theta=\lb{(\ast)}\end{aligned}$

Usamos que $\sin ^{3}\theta=\sin\theta(1-\cos ^{2}\theta)$ y la simetría de $\sin \theta$ en $[0,2\pi]$ implica que: $$
\int_{0}^{2\pi}\sin ^{3}\theta=0 
$$
Usamos la identidad $\cos ^{2}\theta=\dfrac{1+\cos(2\theta)}{2}$. Entonces: $$
\int_{0}^{2\pi} \cos ^{2}\theta \:\mathrm{d}\theta=\int_{0}^{2\pi}\dfrac{1}{2}\:\mathrm{d}\theta+\int_{0}^{2\pi} \dfrac{\cos(2\theta)}{2}\:\mathrm{d}\theta=\left[ \dfrac{1}{2}\theta \right]_{\theta=0}^{\theta=2\pi}+\dfrac{1}{2}\cancelto{0}{\int_{0}^{2\pi}\cos(2\theta)\:\mathrm{d}\theta}=\pi  
$$
$\int_{0}^{2\pi}\cos(2\theta)\:\mathrm{d}\theta=0$ porque $\cos(2\theta)$ es impar en $[0,2\pi]$.

$\lb{(\ast)=}\dfrac{3}{5}\cdot 0+\dfrac{1}{4}\cdot \pi=\dfrac{\pi}{4}$

\item $\db{\iint_{\Omega}\sqrt{ xy }\:\mathrm{d}x\:\mathrm{d}y\text{ en }\Omega=\{ (x,y)\in\mathbb{R}^{2}: x^{2}+y^{2}\leq 1\}.}$

$\Omega$ es el disco de radio 1 centrado en el origen.

La función $\sqrt{ xy }$ depende del producto $xy$. Observamos que:
\begin{itemize}[label=\textbullet]
\item Si $x>0$ y $y>0$, $\sqrt{ xy }>0$.
\item Si $x<0$ o $y<0$, el signo del producto puede cambiar.
\item En particular, en las regiones donde $x>0$, $y<0$ (o viceversa), el producto $xy<0$, y $\sqrt{ xy }$ no está definida para valores negativos.
\end{itemize}
Debido a que $\sqrt{ xy }$ no está definida en $\mathbb{R}^{2}$ cuando $xy<0$, esta integral \textbf{no se puede calcular} sobre $\Omega$ como está formulada, porque incluye regiones donde $xy<0$.

\item $\db{\iint_{\Omega}ye^{ x }\:\mathrm{d}x\:\mathrm{d}y\text{ en }\Omega=\{ (x,y)\in\mathbb{R}^{2}:0\leq y\leq 1,\,0<x\leq y^{2} \}.}$

$\int_{0}^{1}\int_{0}^{y^{2}}ye^{ x }\:\mathrm{d}x\:\mathrm{d}y=\int_{0}^{1}y\cdot\left[ e^{ x } \right]_{x=0}^{x=y^{2}}\:\mathrm{d}y=\int_{0}^{1}y\cdot \left( e^{ y^{2} }-1 \right)\:\mathrm{d}y=\int_{0}^{1}ye^{ y^{2} }-y\:\mathrm{d}y=\int_{0}^{1}ye^{ y }\:\mathrm{d}y-\int_{0}^{1}y\:\mathrm{d}y=\dfrac{1}{2}(e-1)-\dfrac{1}{2}=\dfrac{1}{2}(e-2)$

$\int_{0}^{1}ye^{ y^{2} }\:\mathrm{d}y=\left\{ \begin{array}{l}u=y^{2} \\ \:\mathrm{d}u=2y\:\mathrm{d}y\end{array} \right\}=\int_{0}^{1}e^{ u }\dfrac{\:\mathrm{d}u}{2}=\dfrac{1}{2}\int_{0}^{1}e^{ u }\:\mathrm{d}u=\dfrac{1}{2}\left[ e^{ u } \right]_{u=0}^{u=1}=\dfrac{1}{2}(e-1)$ 

$\int_{0}^{1}y\:\mathrm{d}y=\left[ \dfrac{y^{2}}{2} \right]_{y=0}^{y=1}=\dfrac{1}{2}$

\item $\db{\iint_{\Omega}y+\log x\:\mathrm{d}x\:\mathrm{d}y\text{ en }\Omega=\{ (x,y)\in\mathbb{R}^{2}:0.5\leq x\leq 1,\,x^{2}\leq y\leq x \}.}$

$$\int_{\frac{1}{2}}^{1}\int_{x^{2}}^{x}y+\log x\:\mathrm{d}y\:\mathrm{d}x$$
$\int_{x^{2}}^{x}y+\log x\:\mathrm{d}y=\int_{x^{2}}^{x}y\:\mathrm{d}y+\int_{x^{2}}^{x}\log x\:\mathrm{d}y=\left[ \dfrac{y^{2}}{2} \right]_{y=x^{2}}^{y=x}+\left[ y\log x \right]_{y=x^{2}}^{y=x}=\left( \dfrac{x^{2}}{2}-\dfrac{(x^{2})^{2}}{2} \right)+\log x(x-x^{2})=\dfrac{x^{2}(1-x^{2})}{2}+\log x(x-x^{2})$

$\int_{\frac{1}{2}}^{1}\dfrac{x^{2}(1-x^{2})}{2}+\log x(x-x^{2})\:\mathrm{d}x=\lbb{\int_{\frac{1}{2}}^{1}\dfrac{x^{2}(1-x^{2})}{2}\:\mathrm{d}x}{I_{1}}+\dbb{\int_{\frac{1}{2}}^{1}\log x(x-x^{2})\:\mathrm{d}x}{I_{2}}$
$\lb{I_{1}}=\int_{\frac{1}{2}}^{1}\dfrac{x^{2}(1-x^{2})}{2}\:\mathrm{d}x=\dfrac{1}{2}\int_{\frac{1}{2}}^{1}x^{2}-x^{4}\:\mathrm{d}x=\dfrac{1}{2}\left[ \dfrac{x^{3}}{3}-\dfrac{x^{5}}{5} \right]_{x=0.5}^{x=1}=\dfrac{1}{2}\cdot \left( \dfrac{47}{480} \right)=\dfrac{47}{960}$

$\db{I_{2}}=\int_{\frac{1}{2}}^{1}\log x(x-x^{2})\:\mathrm{d}x=\int_{\frac{1}{2}}^{1}x\log x-x^{2}\log x\:\mathrm{d}x=\lb{(\ast)}=\left( -\dfrac{1}{8}\log \left( \dfrac{1}{2} \right) -\dfrac{3}{16} \right)-\left( -\dfrac{1}{24}\log \left(\dfrac{1}{2}\right) -\dfrac{7}{72}\right)=-\dfrac{1}{12}\log \left( \dfrac{1}{2} \right)-\dfrac{13}{144}$
$$\lb{(\ast)}\begin{aligned}
\int_{\frac{1}{2}}^{1} x\log x\:\mathrm{d}x&=\left\{ \begin{array}{ll}
u=\log x & \mathrm{d}u=\frac{1}{x}\:\mathrm{d}x \\
\mathrm{d}v=x\:\mathrm{d}x & v=\frac{x^{2}}{2}
\end{array} \right\}=\left[ \log x\cdot \dfrac{x^{2}}{2} \right]_{x=0.5}^{x=1}-\int_{\frac{1}{2}}^{1} \dfrac{x^{\cancel{2}}}{2}\cdot \dfrac{1}{\cancel{x}}\:\mathrm{d}x=-\dfrac{1}{8}\log \left( \dfrac{1}{2} \right)-\dfrac{1}{2}\int_{\frac{1}{2}}^{1}x\:\mathrm{d}x\\
&=-\dfrac{1}{8}\log \left( \dfrac{1}{2} \right)-\dfrac{1}{2}\left[ \dfrac{x^{2}}{2} \right]_{x=0.5}^{x=1}=-\dfrac{1}{8}\log \left( \dfrac{1}{2} \right)-\dfrac{3}{16} \\
\int_{\frac{1}{2}}^{1} x^{2}\log x\:\mathrm{d}x&=\left\{ \begin{array}{ll}
u=\log x & \mathrm{d}u=\frac{1}{x}\:\mathrm{d}x \\
\mathrm{d}v=x^{2}\:\mathrm{d}x & v=\frac{x^{3}}{3}
\end{array} \right\}=\left[ \log x\cdot \dfrac{x^{3}}{3} \right]_{x=0.5}^{x=1}-\int_{\frac{1}{2}}^{1} \dfrac{x^{\cancel{3}}}{3}\cdot \dfrac{1}{\cancel{x}}\:\mathrm{d}x=-\dfrac{1}{24}\log \left( \dfrac{1}{2} \right)-\dfrac{1}{3}\int_{\frac{1}{2}}^{1} x^{2}\:\mathrm{d}x\\
&=-\dfrac{1}{24}\log \left( \dfrac{1}{2} \right)-\dfrac{1}{3}\left[ \dfrac{x^{3}}{3} \right]_{x=0.5}^{x=1}=-\dfrac{1}{24}\log \left( \dfrac{1}{2} \right)-\dfrac{7}{72}
\end{aligned}$$

\end{enumerate}

\item \lb{Calcular las integrales dobles siguientes en los recintos que a continuación se dan:}

\begin{enumerate}[label=\color{red}\textbf{\alph*)}]
\item \db{$\iint_{\Omega}(4-y^{2})\:\mathrm{d}x\:\mathrm{d}y$ en el recinto limitado por las ecuaciones $y^{2}=2x$ e $y^{2}=8-2x$.}

Se nos da el recinto limitado por las curvas: \[ y^{2}=2x\quad\text{e}\quad y^{2}=8-2x \]

\includegraphics[width=0.5\textwidth]{"figures/Figure 1"}

Despejamos $x$ en las dos ecuaciones y obtenemos: \[ \begin{array}{l}
y^2=2x\longrightarrow x=\dfrac{y^2}{2}\\
y^2=8-2x\longrightarrow 2x=8-u^2\longrightarrow x=4-\dfrac{y^2}{2}.
\end{array} \]
Por lo tanto, las fronteras de $x$ están dadas por: \[ x_{\mathrm{izq}}(y)=\dfrac{y^{2}}{2},\qquad x_{\mathrm{der}}(y)=4-\dfrac{y^{2}}{2} \]

Para hallar los límites e $y$, igualamos: \[\dfrac{y^2}{2}=4-\dfrac{y^2}{2}\longrightarrow \dfrac{y^2}{2}+\dfrac{y^2}{2}=4\longrightarrow y^2=4\longrightarrow y=\pm2.\]

Por lo tanto, el recinto está comprendido entre $y=-2$ e $y=2$.

Para calcular la integral, dado un valor de $y$ entre $-2$ y $2$, $x$ varía desde $x=\dfrac{y^2}{2}$ hasta $x=4-\dfrac{y^2}{2}$. Por lo tanto: \[\int_{y=-2}^{y=2}\int_{x=\frac{y^2}{2}}^{x=4-\frac{y^2}{2}}(4-y^2)\dx\dy.\]

$\int_{\frac{y^2}{2}}^{4-\frac{y^2}{2}}4-y^2\dx=(4-y^2)\cdot[x]_{x=\frac{y^2}{2}}^{4-\frac{y^2}{2}}=(4-y^2)\cdot\left(4-\dfrac{y^2}{2}-\dfrac{y^2}{2}\right)=(4-y^2)\cdot(4-y^2)=(4-y^2)^2$

Entonces: 

$\int_{-2}^{2}(4-y^2)^2\dy=\int_{-2}^{2}y^4-8y^2+16\dy=\left[\dfrac{y^5}{5}-\dfrac{8y^3}{3}+16y\right]_{y=-2}^{y=2}=\dfrac{512}{15}$

\item \db{$\iint_{\Omega}(x^{4}+y^{2})\:\mathrm{d}x\:\mathrm{d}y$ en el recinto limitado por $y=x^{3}$ e $y=x^{2}$.}

El recinto está limitado por las curvas: 
		\[ y=x ^{3} \quad \text{e}\quad y=x ^{2} .\]  
Primero hallamos, los puntos de intersección: 	
		\[x ^{3}=x ^{2} \longrightarrow x ^{3} -x ^{2} =0 \longrightarrow x ^{2} (1-x)=0 \]
\item \db{$\iint_{\Omega}(x+y)\:\mathrm{d}x\:\mathrm{d}y$ en el recinto limitado por $y=x^{3}$ e $y=x^{4}$ con $-1\leq x\leq 1$.}
\item \db{$\iint_{\Omega}(2xy^{2}-y)\:\mathrm{d}x\:\mathrm{d}y$ en la región limitada por $y=|x|,\,y=-|x|$ y $x \in[-1,1]$.}
\end{enumerate}

\item \lb{Calcular la superficie de las siguientes regiones:}
\begin{enumerate}[label=\color{red}\textbf{\alph*)}]
\item \db{Círculo de radio $R$.}
\item \db{Elipse de semiejes $a,b$.}
\item \db{La región limitada por las ecuaciones $x^{2}=4y$ y $2y-x-4=0$.}
\item \db{La región limitada por las ecuaciones $x+y=5$ y $xy=6$.}
\item \db{La región limitada por las ecuaciones $x=y$ y $x=4y-y^{2}$.}
\end{enumerate}

\item \lb{Calcular el volumen de los siguientes sólidos:}
\begin{enumerate}[label=\color{red}\textbf{\alph*)}]
\item \db{El limitado por $\dfrac{x}{2}+\dfrac{y}{3}+\dfrac{z}{4}=1$ y los planos de coordenadas.}
\item \db{El tronco limitado superiormente por $z=2x+3y$ e inferiormente por el cuadrado $[0,1]\times[0,1]$.}
\item \db{Esfera de radio $R$.}
\item \db{Cono de altura $h$ y radio de la base $R$.}
\item \db{El tronco limitado superiormente por la ecuación $z=2x+1$ e inferiormente por el disco $(x-1)^{2}+y^{2}\leq 1$.}
\end{enumerate}

\item \lb{Calcular cambiando a coordenadas polares:}
\begin{enumerate}[label=\color{red}\textbf{\alph*)}]
\item $\db{\int_{-1}^{1}\int_{0}^{\sqrt{ 1-y^{2} }}\sqrt{ x^{2}+y^{2} }\:\mathrm{d}x\:\mathrm{d}y.}$
\item $\db{\int_{0}^{2}\int_{0}^{\sqrt{ 4-x^{2} }}\sqrt{ x^{2}+y^{2} }\:\mathrm{d}y\:\mathrm{d}x.}$
\item $\db{\int_{\frac{1}{2}}^{1}\int_{0}^{\sqrt{ 1-x^{2} }}(x^{2}+y^{2})^{\frac{3}{2}}\:\mathrm{d}y\:\mathrm{d}x.}$
\item $\db{\int_{0}^{\frac{1}{2}}\int_{0}^{\sqrt{ 1-y^{2} }}xy\sqrt{ x^{2}+y^{2} }\:\mathrm{d}x\:\mathrm{d}y.}$
\end{enumerate}

\item \lb{Calcular para $\Omega=[0,1]\times[0,3]\times[-1,1]$ las integrales}
\begin{enumerate}[label=\color{red}\textbf{\alph*)}]
\item $\db{\iiint_{\Omega}xyz\:\mathrm{d}x\:\mathrm{d}y\:\mathrm{d}z.}$
\item $\db{\iiint_{\Omega}xe^{ y+z }\:\mathrm{d}x\:\mathrm{d}y\:\mathrm{d}z.}$
\item $\db{\iiint_{\Omega} y^{2}z^{3}\sin x\:\mathrm{d}x\:\mathrm{d}y\:\mathrm{d}z.}$
\end{enumerate}

\item \lb{Calcular las integrales que a continuación se piden en los recintos correspondientes:}
\begin{enumerate}[label=\color{red}\textbf{\alph*)}]
\item \db{$\iiint_{\Omega}(y^{3}+z+x)\:\mathrm{d}x\:\mathrm{d}y\:\mathrm{d}z$ en $\Omega=\{ (x,y,z)\in\mathbb{R}^{3}:x^{2}+y^{2}+z^{2}=1 \}$}
\item \db{$\iiint_{\Omega}(y\sin z+x)\:\mathrm{d}x\:\mathrm{d}y\:\mathrm{d}z$ en $\Omega=\{ (x,y,z)\in\mathbb{R}^{3}: y\geq z\geq y^{2},0\leq x,y\leq 1 \}.$}
\item \db{$\iiint_{\Omega}x\:\mathrm{d}x\:\mathrm{d}y\:\mathrm{d}z$ en $\Omega=\{ (x,y,z)\in\mathbb{R}^{3}:1\geq y^{2}+x^{2},0\leq z\leq 1 \}.$}
\item \db{$\iiint_{\Omega}yxz\:\mathrm{d}x\:\mathrm{d}y\:\mathrm{d}z$ en $\Omega=\{ (x,y,z)\in\mathbb{R}^{3}:-5\leq z\leq y^{2}+x,-1\leq x,y\leq 1 \}$.}
\end{enumerate}

\item \lb{Calcular el volumen del sólido limitado superiormente por $z=1$ e inferiormente por $z=\sqrt{ x^{2}+y^{2} }$.}

\item \lb{Calcular el volumen del sólido limitado superiormente por el cilindro parabólico $z=1-y^{2}$, inferiormente por el plano $2x+3y+z+10=0$ y lateralmente por el cilindro circular $x^{2}+y^{2}+x=0$.}

\item \lb{Hallar el volumen del sólido limitado por los paraboloides de ecuaciones $z=2-x^{2}-y^{2}$ y $z=x^{2}+y^{2}$.}

\item \lb{Calcular el volumen del sólido limitado superiormente por la superficie cilíndrica $x^{2}+z=4$, inferiormente por el plano $x+z=2$ y lateralmente por lo planos $y=0$ e $y=3$.}

\item \lb{Haciendo uso de las coordenadas esféricas $x=r\sin \phi \cos\theta,yr\sin \phi \sin\theta$ y $z=r\cos \phi$, calcular:}
\begin{enumerate}[label=\color{red}\textbf{\alph*)}]
\item \db{El volumen de una esfera de radio $R$.}
\item \db{$\iiint_{\Omega}(x^{2}+y^{2}+z^{2})\:\mathrm{d}x\:\mathrm{d}y\:\mathrm{d}z$ en el recinto $\Omega=\{(x,y,z)\in\mathbb{R}^{3}:1\leq x^{2}+y^{2}+z^{2}\leq 2 \}$.}
\item \db{El volumen del recinto del apartado (b).}
\end{enumerate}

\item \lb{Calcular el volumen del cuerpo limitado por las ecuaciones $z=x^{2}+4y^{2}$, el plano $z=0$ y lateralmente por los cilindros $x=y^{2}$ y $x^{2}=y$.}

\item \lb{Calcular $\iint_{\Omega}e^{ \frac{x-y}{x+y} }\:\mathrm{d}x\:\mathrm{d}y$ siendo $\Omega$ el triángulo formado por los ejes de coordenadas y la recta $x+y=1$.}

\item \lb{Calcular el volumen comprendido entre los cilindros $z=x^{2}$ y $z=4-y^{2}$.}

\item \lb{Calcular el volumen del balón de Rugby de ecuaciones $\dfrac{x^{2}}{a^{2}}+\dfrac{y^{2}}{b^{2}}+\dfrac{z^{2}}{c^{2}}=1$.}

\item \lb{Calcular $$
\underset{\Omega}{ \iiint }\dfrac{\:\mathrm{d}x\:\mathrm{d}y\:\mathrm{d}z}{(x^{2}+y^{2}+x^{2})^{\frac{3}{2}}},
$$
donde $\Omega$ es la región limitado por las esferas $x^{2}+y^{2}+z^{2}=a^{2}$ y $x^{2}+y^{2}+z^{2}=b^{2}$, donde $0<b<a$. Indicación: hacer el cambio a coordenadas esféricas.}

\end{enumerate}

\end{document}

