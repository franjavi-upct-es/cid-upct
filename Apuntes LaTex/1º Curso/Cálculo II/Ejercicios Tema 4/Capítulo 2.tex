\textbf{\Large Capítulo 2: Campos escalares y vectoriales}
\begin{enumerate}[label=\color{red}\textbf{\arabic*)}]
  \item \lb{Calcular el gradiente de los siguientes campos escalares:}
    \begin{enumerate}[label=\color{red}\textbf{\alph*)}]
      \item \db{$f(x,y,z)=e^{zyx} $}
      
      \[
          \nabla f=\left( \frac{\partial f}{\partial x} ,\frac{\partial f}{\partial y} ,\frac{\partial f}{\partial z}  \right) =(yze^{xyz},xze^{xyz},xye^{xyz}   )
          \] 
      \item \db{$f(x,y,z)=\dfrac{1}{\sqrt{x^2+y^2+z^2} } $} 
      
      $\begin{array}{l}
        \frac{\partial f}{\partial x} =-\dfrac{1}{2}\dfrac{1}{(x^2+y^2+z^2)^{\frac{3}{2}}} \cdot 2x=-\dfrac{x}{(x^2+y^2+z^2)^{\frac{3}{2} }} \\
        \frac{\partial f}{\partial y} =-\dfrac{1}{2}\dfrac{1}{(x^2+y^2+z^2)^{\frac{3}{2}}} \cdot 2y=-\dfrac{y}{(x^2+y^2+z^2)^{\frac{3}{2} }} \\
        \frac{\partial f}{\partial z} =-\dfrac{1}{2}\dfrac{1}{(x^2+y^2+z^2)^{\frac{3}{2}}} \cdot 2z=-\dfrac{z}{(x^2+y^2+z^2)^{\frac{3}{2} }} 
      \end{array}$

      \[
        \nabla f=\left( \frac{\partial f}{\partial }, \frac{\partial f}{\partial y} ,\frac{\partial f}{\partial z}   \right) =\left(-\dfrac{x}{(x^2+y^2+z^2)^{\frac{3}{2} }},-\dfrac{y}{(x^2+y^2+z^2)^{\frac{3}{2} }},-\dfrac{z}{(x^2+y^2+z^2)^{\frac{3}{2} }}\right) 
          \] 
      \item \db{$f(x,y,z)=\sin(xyz)$} 
        \[
        \nabla f=\left( \frac{\partial f}{\partial x} ,\frac{\partial f}{\partial y} ,\frac{\partial f}{\partial z}  \right) =\left( \cos(xyz)\cdot yz,\cos(xyz)\cdot xz,\cos(xyz)\cdot xy \right) 
        \] 
      \item \db{$\cos(x+y+z)$} 
        \[
        \nabla f=\left( \frac{\partial f}{\partial x} ,\frac{\partial f}{\partial y} ,\frac{\partial f}{\partial z} \right) =(-\sin(x+y+z), -\sin(x+y+z), -\sin(x+y+z))
        \] 
    \end{enumerate}
  \item \lb{Calcular la divergencia y el rotacional de los siguientes campos:}

    Vamos a calcular la \textbf{divergencia} y el \textbf{rotacional}. Las fórmulas son: 
\begin{enumerate}[label=\arabic*)]
  \item \textbf{Divergencia} de $\mathbf{F}=F_x\mathbf{i} +F_y\mathbf{j} +F_z\mathbf{k}:$ 
    \[
    \nabla \cdot \mathbf{F} =\frac{\partial F_x}{\partial x} +\frac{\partial F_y}{\partial y} +\frac{\partial F_z}{\partial z} 
    \] 
  \item \textbf{Rotacional} de $\mathbf{F} =F_x\mathbf{i} +F_y\mathbf{j} +F_z\mathbf{k} $:
    \[
    \nabla \times \mathbf{F} =\begin{vmatrix} 
      \mathbf{i}  & \mathbf{j} & \mathbf{k} \\
      \frac{\partial }{\partial x} & \frac{\partial }{\partial y} & \frac{\partial }{\partial z} \\
      F_x & F_y & F_z
    \end{vmatrix}. 
    \] 
\end{enumerate}
    \begin{enumerate}[label=\color{red}\textbf{\alph*)}]
      \item \db{$\mathbf{F}(x,y,z)=(\sin x)\mathbf{i}+(\cos)\mathrm{j}$.} 
        \begin{enumerate}[label=\arabic*)]
          \item Divergencia:
            \[
              \nabla \cdot \mathbf{F} =\frac{\partial (\sin x)}{\partial x} +\frac{\partial (\cos x)}{\partial y} +\frac{\partial (0)}{\partial z} = \cos x+0+0=\cos x
            \] 
          \item Rotacional:
            \[
            \nabla \times \mathbf{F} = \begin{vmatrix} 
              \mathbf{i}  & \mathbf{j} & \mathbf{k} \\
              \frac{\partial }{\partial x} & \frac{\partial }{\partial y} & \frac{\partial V}{\partial z} \\
              \sin x & \cos x & 0
            \end{vmatrix} =\mathbf{i} (0,0)-\mathbf{j} (0-0)+\mathbf{k} \left( \frac{\partial (\cos x)}{\partial x} -\frac{\partial (\sin )}{\partial y}  \right) =\mathbf{k} (-\sin x-0)=-\sin x\mathbf{k} 
            \] 
        \end{enumerate}
      \item \db{$\mathbf{F}(x,y,z)=x\mathbf{i}-y\mathbf{j}$.} 
        \begin{enumerate}[label=\arabic*)]
          \item Divergencia
            \[
            \nabla \cdot \mathbf{F} =\frac{\partial (x)}{\partial x} +\frac{\partial (-y)}{\partial y} +\frac{\partial 0}{\partial z} =1-1+0=0
            \] 
          \item Rotacional
            \[
            \nabla \times \mathbf{F} =\begin{vmatrix} 
              \mathbf{i} & \mathbf{j}  & \mathbf{k} \\
              \frac{\partial }{\partial x} & \frac{\partial }{\partial y} & \frac{\partial }{\partial z} \\
              x & -y & 0
          \end{vmatrix}=\mathbf{i} (0-0)-\mathbf{j} (0-0)+\mathbf{j} \left( \frac{\partial (-y)}{\partial x} -\frac{\partial (x)}{\partial y}  \right) = 0
            \] 
        \end{enumerate}
      \item \db{$\mathbf{F}(x,y,z)=ax\mathbf{i}+by\mathbf{j}-^3fk$, donde $a,b,c\in \R$.} 
      \begin{enumerate}[label=\arabic*)]
                \item Divergencia
                  \[
                  \nabla \cdot \mathbf{F} =\frac{\partial (ax)}{\partial x} +\frac{\partial (by)}{\partial y} +\frac{\partial (-cz)}{\partial z} =a+b-c
                  \] 
                \item Rotacional
                  \[
                  \nabla \times \mathbf{F} =\begin{vmatrix} 
                    \mathbf{i}  & \mathbf{j} & \mathbf{k} \\
                    \frac{\partial }{\partial x} & \frac{\partial }{\partial y} & \frac{\partial }{\partial z} \\
                    ax & by & -cz
                  \end{vmatrix}=\mathbf{i} (0-0)-\mathbf{j} (0-0)+\mathbf{k} (0-0)=0 
                  \] 
              \end{enumerate}
      \item \db{$\mathbf{F}(x,y,z)=x^2\mathbf{i}+y^2\mathbf{j}-z^2\mathbf{k}$.} 
      \begin{enumerate}[label=\arabic*)]
                \item Divergencia
                  \[
                  \nabla \cdot \mathbf{F} =\frac{\partial (x^2)}{\partial x}+\frac{\partial (y^2)}{\partial y} +\frac{\partial (-z^2)}{\partial z}=2x+2y-2z
                  \] 
                \item Rotacional
                  \[
                  \nabla \times \mathbf{F} =\begin{vmatrix} 
                    \mathbf{i}  & \mathbf{j} &\mathbf{k} \\
                    \frac{\partial }{\partial x} & \frac{\partial }{\partial y} & \frac{\partial }{\partial z} \\
                    x^2 & y^2 & -z^2
                  \end{vmatrix} =\mathbf{i} (0-0)-\mathbf{j} (0-0)+\mathbf{k} (0-0)=0
                  \] 
              \end{enumerate}
      \item \db{$\mathbf{F}(x,y,z)=xy\mathbf{i}+yz\mathbf{j}+xz\mathbf{k}$.} 
        \begin{enumerate}[label=\arabic*)]
          \item Divergencia:
            \[
            \nabla \cdot \mathbf{F} =\frac{\partial (xy)}{\partial x} +\frac{\partial (yz)}{\partial y} +\frac{\partial (xz)}{\partial z} =y+z+x
            \] 
          \item Relacional:
            \[
            \nabla \times \mathbf{F} =\begin{vmatrix} 
              \mathbf{i} & \mathbf{j} & \mathbf{k} \\
              \frac{\partial }{\partial x} & \frac{\partial }{\partial y} & \frac{\partial }{\partial z} \\
              xy & yz & xz
            \end{vmatrix} =\mathbf{i} (z-z)-\mathbf{j} (x-x)+\mathbf{k} (y-y)=0
            \] 
        \end{enumerate}
      \item \db{$\mathbf{F}(x,y,z)=xyz\mathbf{i}+x^2y^2z^2\mathbf{j}+y^2z^3\mathbf{k}$.} 
      \begin{enumerate}[label=\arabic*)]
                \item Divergencia:
                  \[
                  \nabla \cdot F = \frac{\partial (xyz)}{\partial x} +\frac{\partial (x^2y^2z^2)}{\partial y} +\frac{\partial (y^2z^3)}{\partial z} =yz+2x^2yz^2+3y^2z^2
                  \] 
                \item Relacional:
                  \[
                  \begin{aligned}
                  \nabla \times \mathbf{F} =\begin{vmatrix} 
                    \mathbf{i}  & \mathbf{j} & \mathbf{k} \\
                    \frac{\partial }{\partial x} & \frac{\partial }{\partial y}  & \frac{\partial }{\partial z} \\
                    xyz & x^2y^2z^2 & y^2z^3
                  \end{vmatrix} &= \mathbf{i} (2xz^3-2x^2y^2z)-\mathbf{j} (xy-0)+\mathbf{k} (2xy^2z^2-xz)\\
                  &=(2xz^3-2x^2y^2z)\mathbf{i} -xy\mathbf{j} +(2xy^2z^2-xz)\mathbf{k} 
                  \end{aligned}
                  \] 
              \end{enumerate}
    \end{enumerate}
  \item \lb{Sea $f\in C^2(D,\R)$. Se define el \textit{Laplaciano de $f$} como la divergencia del gradiente de $f$, esto es  \[
  \nabla^2f= <\nabla,\nabla f > = \mathrm{div}(\nabla f).
  \] 
Una función $f$ se dice  \textit{armónica} si $\nabla ^2f=0$. Identificar cuáles de las siguientes funciones son armónicas:}
\begin{enumerate}[label=\color{red}\textbf{\alph*)}]
  \item \db{$f(x,y)=e^{z}\cos y$.}

    \[
    \nabla ^2f=\frac{\partial^2 f}{\partial x^2} +\frac{\partial^2 f}{\partial y^2} +\frac{\partial^2 f}{\partial z^2} = 0 + (-e^{z}\cos y )+e^{z}\cos y=0\longrightarrow f(x,y,z)\text{ es armónica} 
    \] 
    
    $\begin{array}{l}
      \frac{\partial f}{\partial x} = 0\longrightarrow \frac{\partial^2 f}{\partial x^2} =0\\
      \frac{\partial f}{\partial y} =e^{z}(-\sin y)\longrightarrow \frac{\partial^2 f}{\partial y^2} =e^{z}(-\cos y)  \\
      \frac{\partial f}{\partial z} =e^{z}\cos y\longrightarrow \frac{\partial^2 f}{\partial z^2} =e^{z} \cos y 
    \end{array}$
  \item \db{$f(x,y,z)=e^{-z}(\cos y-\sin y) $.}
    
    \[
    \nabla ^2f=\frac{\partial^2 f}{\partial x^2} +\frac{\partial^2 f}{\partial y^2} +\frac{\partial^2 f}{\partial z^2} = 0 + e^{-z}(-\cos y+\sin y)+e^{-z}(\cos y -\sin y)=0\longrightarrow f(x,y,z)\text{ es armónica }  
    \] 

    $\begin{array}{l}
      \frac{\partial f}{\partial x} =0\longrightarrow \frac{\partial^2 f}{\partial x^2} =0\\
      \frac{\partial f}{\partial y} =e^{-z}(-\sin y-\cos y)\longrightarrow \frac{\partial^2 f}{\partial y^2} e^{-z}(-\cos y + \sin y)\\
      \frac{\partial f}{\partial z} =e^{-z}(-\cos y + \sin y)\longrightarrow \frac{\partial^2 f}{\partial z^2} =e^{-z}(\cos y -\sin y)  
    \end{array}$
  \item \db{$f(x,y,z)=(x^2+y^2+z^2)^{-\frac{1}{2} }$.} 

    \[
      \begin{aligned}
      \nabla ^2f=\frac{\partial^2 f}{\partial x^2} +\frac{\partial^2 f}{\partial y^2} +\frac{\partial^2 f}{\partial z^2} &=  3x^2(x^2+y^2+z^2)^{-\frac{5}{2} }+3y^2(x^2+y^2+z^2)^{-\frac{5}{2} }+3z^2-3(x^2+y^2+z^2)^{-\frac{3}{2} }\neq 0\\
      & \longrightarrow f(x,y,z) \text{ no es armónica}
      \end{aligned}
    \] 
    $\begin{array}{l}
      \frac{\partial f}{\partial x} =-\dfrac{1}{2}(x^2+y^2+z^2)^{-\frac{3}{2} }\cdot 2x=-x(x^2+y^2+z^2)^{-\frac{3}{2} }\longrightarrow \frac{\partial^2 f}{\partial x^2} = -(x^2+y^2+z^2)^{-\frac{3}{2} }+3x^2(x^2+y^2+z^2)^{-\frac{5}{2} }\\
      \frac{\partial f}{\partial y} =-\dfrac{1}{2}(x^2+y^2+z^2)^{-\frac{3}{2} }\cdot 2y=-y(x^2+y^2+z^2)^{-\frac{3}{2} }\longrightarrow \frac{\partial^2 f}{\partial y^2} = -(x^2+y^2+z^2)^{-\frac{3}{2} }+3y^2(x^2+y^2+z^2)^{-\frac{5}{2} }\\
      \frac{\partial f}{\partial z} =-\dfrac{1}{2}(x^2+y^2+z^2)^{-\frac{3}{2} }\cdot 2z=-z(x^2+y^2+z^2)^{-\frac{3}{2} }\longrightarrow \frac{\partial^2 f}{\partial z^2} = -(x^2+y^2+z^2)^{-\frac{3}{2} }+z^2(x^2+y^2+z^2)^{-\frac{5}{2} }\\
    \end{array}$
\end{enumerate}
\item \lb{Dadas las funciones $\mathbf{F}(x,y,z)=2\mathrm{i}+2x\mathbf{j}+3y\mathbf{k}$ y $\mathbf{G}(x,y,z)=x\mathbf{i}-y\mathbf{j}+z\mathbf{k}$, calcular:}
  \begin{enumerate}[label=\color{red}\textbf{\alph*)}]
    \item \db{$\mathrm{rot}(\mathbf{F}\times \mathbf{G})$} 

      La fórmula del rotaciona del producto vectorial de dos campos $\mathbf{F} $ y $\mathbf{G} $ es: \[
      \mathrm{rot} (\mathbf{F} \times \mathbf{G} )=(\mathbf{G} \cdot \nabla)\mathbf{F} -(\mathbf{F} \cdot \nabla )\mathbf{G} +F(\nabla \cdot \mathbf{G} )-\mathbf{G} (\nabla \cdot \mathbf{F} )
      \] 
      \begin{enumerate}[label=Paso \arabic*:]
        \item Calcular $\nabla \cdot \mathbf{F} $ y $\nabla \cdot \mathbf{G} $ 
          \begin{itemize}[label=\textbullet]
            \item $\mathbf{F} (x,y,z)=2\mathbf{i} +2x\mathbf{j} +3y\mathbf{k} $: \[
            \nabla \cdot \mathbf{F} =\frac{\partial (2)}{\partial x} +\frac{\partial (2x)}{\partial y} +\frac{\partial (3y)}{\partial z} =0
            \] 
          \item $\mathbf{G} (x,y,z)=x\mathbf{i} -y\mathbf{j} +z\mathbf{k} $: \[
          \nabla \cdot \mathbf{G} =\frac{\partial (x)}{\partial x} +\frac{\partial (-y)}{\partial z} +\frac{\partial (z)}{\partial z} =1-1+1=1
          \] 
          \end{itemize}
      \item Calcular $(\mathbf{G} \cdot \nabla )\mathbf{F} $ y $(\mathbf{F} \cdot \nabla )\mathbf{F} $
        \begin{itemize}[label=\textbullet]
          \item Para $(\mathbf{G} \cdot \nabla )\mathbf{F}$: \[
          G\cdot \nabla =x \frac{\partial }{\partial x} -y \frac{\partial }{\partial y} +z \frac{\partial }{\partial z} .
          \] 
          Aplicamos este operacdor a cada componente de $\mathbf{F} $:
          \[
          \begin{array}{c}
            (\mathbf{F} \cdot \nabla)\mathbf{F}_x=x\cdot \frac{\partial (2)}{\partial x} -y\cdot \frac{\partial (2)}{\partial y} +z\cdot \frac{\partial (2)}{\partial z} =0\\
            (\mathbf{G} \cdot \nabla )\mathbf{F} _y=x\cdot \frac{\partial (2x)}{\partial x} -y\cdot \frac{\partial (2x)}{\partial y} +z\cdot \frac{\partial (2x)}{\partial z} =2x\\
            (G\cdot \nabla )\mathbf{F} _z=x\cdot \frac{\partial (3y)}{\partial x} -y\cdot \frac{\partial (3y)}{\partial y} +z\cdot \frac{\partial (3y)}{\partial z} =-3y\\
            (\mathbf{G} \cdot \nabla )\mathbf{F} =O\mathbf{i} +2x\mathbf{j} -3y\mathbf{k} 
          \end{array}
          \] 
        \item Para $(\mathbf{F} \cdot \nabla )\mathbf{G}$:
          \[
          \mathbf{F} \cdot \nabla =2 \frac{\partial }{\partial x} +2x \frac{\partial }{\partial y} +3y \frac{\partial }{\partial z} .
          \]
          Aplicamos este operador a cada componente de $\mathbf{G} $:
          \[
          \begin{array}{c}
            (\mathbf{F} \cdot \nabla )\mathbf{G} _x=2\cdot \frac{\partial (x)}{\partial x} +2x\cdot \frac{\partial (x)}{\partial y} +3y\cdot \frac{\partial (x)}{\partial z} =2\\
            (\mathbf{F} \cdot \nabla )\mathbf{G} _y=2\cdot \frac{\partial (-y)}{\partial x} +2x\cdot \frac{\partial (-y)}{\partial y} +3y\cdot \frac{\partial (-y)}{\partial z} =-2y\\
            (\mathbf{F} \cdot \nabla )\mathbf{G} _z=2\cdot \frac{\partial (z)}{\partial x} +2x\cdot \frac{\partial (z)}{\partial x} +3y\cdot \frac{\partial (z)}{\partial z} =3y\\
            (\mathbf{F} \cdot \nabla )\mathbf{G} =2\mathbf{i} -2x\mathbf{j} +3y\mathbf{k} 
          \end{array}
          \] 
        \end{itemize}
      \item Sustituir en la fórmula \[
      \mathrm{rot} (\mathbf{F} \times \mathbf{G} )=(\mathbf{G} \cdot \nabla )\mathbf{F} -(\mathbf{F} \cdot \nabla)\mathbf{G} +\mathbf{F} (\nabla \cdot G)-\mathbf{G} (\nabla \cdot F)
      \] 
      \begin{enumerate}[label=\arabic*)]
        \item $(\mathbf{G} \cdot \nabla)\mathbf{F} -(F\cdot \nabla )\mathbf{G}=(0\mathbf{i} +2x\mathbf{j}-3y\mathbf{k}  )-(2\mathbf{i} -2x\mathbf{j} +3y\mathbf{k} )=-2\mathbf{i} +4x\mathbf{j} -4y\mathbf{k} $
      \item $\mathbf{F} (\nabla \cdot \mathbf{G} )=1\cdot F=2\mathbf{i} +2x\mathbf{j} +3y\mathbf{k} $
      \item $-\mathbf{G} (\nabla \cdot \mathbf{F} )=-2(x\mathbf{i} -y\mathbf{j} +z\mathbf{k} )=-2\mathbf{i} +2y\mathbf{j}-2z\mathbf{k}$
      \end{enumerate}
      Sumamos todas las contribuciones:
      \[
      \mathrm{rot} (\mathbf{F} \times \mathbf{G} )=(-2+2-2x)\mathbf{i} +(4x+2x+2y)\mathbf{j} +(-yy+3y-2z)\mathbf{k} =(-2x)\mathbf{i} +(6x+2y)\mathbf{j} +(-3y-2z)\mathbf{k} .
      \] 
      \end{enumerate}
    \item \db{$\mathrm{div}(\mathbf{F}\times \mathbf{G})$} 
      \begin{enumerate}[label=Paso \arabic*:]
        \item Calcular $\mathbf{F} \times \mathbf{G} $

          El producto cruzado es: \[
          \begin{aligned}
          \mathbf{F} \times \mathbf{G} =\begin{vmatrix} 
            \mathbf{i}  & \mathbf{j} &\mathbf{k} \\
            2 & 2x & 3y\\
            x & -y & z
          \end{vmatrix}=\mathbf{i} \begin{vmatrix} 
            2x & 3y\\
            -y & z
          \end{vmatrix} -\mathbf{j} \begin{vmatrix} 
            2 & 3y\\
            x & z
          \end{vmatrix}  +\mathbf{k} \begin{vmatrix} 
            2 & 2x\\
            x & -y
          \end{vmatrix} &=\mathbf{i} (2x\cdot z-3y\cdot (-y)-\mathbf{j} (2z-3xy))+\mathbf{k} (-2y-2x^2)\\
          &=(2xz+3y^2)\mathbf{i} -(2z-3xy)\mathbf{j} +(-2y-x^2)\mathbf{k}
          \end{aligned}
          \] 
        \item Calcular $\mathrm{div} (\mathbf{F} \times \mathbf{G} )$

          La divergencia es:
          \[
          \mathrm{div} (\mathbf{F} \times \mathbf{G} )=\frac{\partial }{\partial x} (2xz+3y^2)+\frac{\partial }{\partial y} (-(2z-3xy))+\frac{\partial }{\partial z} (-2y-2x^2)=2z+3x+0=2x+3x
          \] 
      \end{enumerate}
  \end{enumerate}

\item \lb{Sean $f$ y  $g$ dos campos escalares,  $\mathbf{F}$ y $\mathbf{G}$ dos campos vectoriales y,  $\alpha\in \R$. Demostrar las siguientes propiedades:} 
  \begin{enumerate}[label=\color{red}\textbf{\alph*)}]
    \item \db{$\nabla (\alpha f)=\alpha\nabla (f)$.} 

      \[
        \nabla (\alpha f)=\left( \frac{\partial (\alpha f)}{\partial x} ,\frac{\partial (\alpha f)}{\partial y},\frac{\partial (\alpha f)}{\partial z}   \right) =\left(\alpha \frac{\partial f}{\partial x} ,\alpha \frac{\partial f}{\partial y} ,\alpha \frac{\partial f}{\partial z}\right)=\alpha \left( \frac{\partial f}{\partial x} ,\frac{\partial f}{\partial y} ,\frac{\partial f}{\partial z}  \right) =\alpha \nabla (f)
      \] 
    \item \db{$\nabla (f+g)=\nabla (f)+\nabla (g)$} 

      \[
      \nabla (f+g)=\left( \frac{\partial (f+g)}{\partial x},\frac{\partial (f+g)}{\partial y} ,\frac{\partial (f+g)}{\partial z}  \right) =\left( \frac{\partial f}{\partial x} +\frac{\partial g}{\partial x} ,\frac{\partial f}{\partial y} +\frac{\partial g}{\partial y},\frac{\partial f}{\partial z}+\frac{\partial g}{\partial z}   \right) =\left( \frac{\partial f}{\partial x} ,\frac{\partial f}{\partial y} ,\frac{\partial f}{\partial z}  \right) +\left( \frac{\partial g}{\partial x} ,\frac{\partial g}{\partial y} ,\frac{\partial g}{\partial z}  \right)
    \]
    Por lo tanto: \[
    \nabla (f+g) =\nabla (f)+\nabla (g)
    \] 
    \item \db{$\nabla \left( \dfrac{f}{g} \right) =\dfrac{g\nabla (f)-f\nabla (g)}{g^2},\,g\neq 0$.}
    \[
    \nabla \left( \dfrac{f}{g} \right) =\left( \dfrac{\frac{\partial f}{\partial x} g-f \frac{\partial g}{\partial x} }{g^2},\dfrac{\frac{\partial f}{\partial y} g-f \frac{\partial h}{\partial y} }{g^2},\dfrac{\frac{\partial f}{\partial z} g-f \frac{\partial g}{\partial z} }{g^2}   \right) =\dfrac{g\nabla (f)-f\nabla (g)}{g^2} 
    \] 
    \item \db{$\nabla (fg)=f\nabla g+g\nabla (f)$}
      
      \[
      \nabla (fg)=\left( \frac{\partial f}{\partial x} g+f \frac{\partial g}{\partial x} ,\frac{\partial f}{\partial y} g+f \frac{\partial g}{\partial y} ,\frac{\partial f}{\partial z} g+f \frac{\partial g}{\partial z}  \right) =f\nabla (g)+g\nabla (f)
      \] 
    \item \db{$\mathrm{div}(\alpha\mathbf{F})=\alpha\mathrm{div}(\mathbf{F})$.} 

      \[
      \begin{array}{l}
        \alpha \mathbf{F} =\alpha F_x\mathbf{i} +\alpha F_y\mathbf{j} +\alpha F_z\mathbf{k}\\ 
        \mathrm{div} (\alpha \mathbf{F} )=\nabla \cdot (\alpha \mathbf{F} )=\frac{\partial (\alpha F_x)}{\partial x} +\frac{\partial (\alpha F_y)}{\partial y} +\frac{\partial (\alpha F_z)}{\partial z}=\alpha \frac{\partial F_x}{\partial x} + \alpha \frac{\partial F_y}{\partial y} +\alpha \frac{\partial F_z}{\partial z}=\alpha\left( \frac{\partial F_x}{\partial x} +\frac{\partial F_y}{\partial y} +\frac{\partial F_z}{\partial z}  \right) =\alpha \mathrm{div} (\mathbf{F} ) \\
      \end{array}
      \] 
    \item \db{$\mathrm{div}(\mathbf{F}+\mathbf{G})=\mathrm{div}(\mathbf{F})+\mathrm{div}(\mathbf{G})$.} 
      \[
      \begin{array}{l}
        \mathbf{F} +\mathbf{G} =(F_x+G_x)\mathbf{i} +(F_y+G_y)\mathbf{j} +(F_z+G_z)\mathbf{k} \\
        \begin{aligned}
        \mathrm{div} (\mathbf{F} +\mathbf{G} )=\nabla \cdot (\mathbf{F} +\mathbf{G} )&=\frac{\partial (F_x+G_x)}{\partial x} +\frac{\partial (F_y+G_y)}{\partial y} +\frac{\partial (F_z+G_z)}{\partial z} =\frac{\partial F_x}{\partial x}+\frac{\partial G_x}{\partial x} +\frac{\partial F_y}{\partial y} +\frac{\partial G_y}{\partial y} +\frac{\partial F_z}{\partial z} +\frac{\partial G_z}{\partial z}   \\
        &=\left(\frac{\partial F_x}{\partial x}+\frac{\partial F_y}{\partial y}+\frac{\partial F_z}{\partial z} \right)+\left( \frac{\partial G_x}{\partial x}+\frac{\partial G_y}{\partial y} +\frac{\partial G_z}{\partial z}   \right) =\mathrm{div} (\mathbf{F} )+\mathrm{div} (\mathbf{G} )
        \end{aligned}
      \end{array}
      \] 
    \item \db{$\mathrm{rot}(\mathbf{F}+\mathbf{G})=\mathrm{rot}(\mathbf{F})+\mathrm{rot}(\mathbf{G})$.} 

      \[
      \mathrm{rot} (\mathbf{F} +\mathbf{G} )=\nabla \times (\mathbf{F} +\mathbf{G} )=(\nabla\times  \mathbf{F} )+(\nabla \times \mathbf{G} )=\mathrm{rot} (\mathbf{F} )+\mathrm{rot} (\mathbf{G} )
      \] 
    \item \db{$\mathrm{rot}(\alpha\mathbf{F})=\alpha\mathrm{rot}(\mathbf{F})$} 
      \[
      \mathrm{rot} (\alpha \mathbf{F} )=\nabla \times (\alpha\mathbf{F} )=\alpha(\nabla \times \mathbf{F} )=\alpha\mathrm{rot} (\mathbf{F} )
      \] 
    \item \db{$\mathrm{div}(\mathbf{F}\times \mathbf{G})= <\mathrm{rot}(\mathbf{F}), \mathbf{G} > - <\mathbf{F},\mathrm{rot}(G) >$} 

      Sea $\mathbf{F} \times \mathbf{G} =\mathbf{H}=H_x\mathbf{i} +H_y\mathbf{j} +H_z\mathbf{k} $, donde: \[
      \begin{array}{c}
      \mathbf{F} \times \mathbf{G} =\begin{vmatrix} 
        \mathbf{i}  & \mathbf{j}  & \mathbf{k} \\
        F_x & F_y & F_z\\
        G_x & G_y & G_z
      \end{vmatrix}\\
      H_x=F_yG_z-F_zG_y,\quad H_y=F_zG_x-F_xG_z,\quad H_z=F_xG_y-F_yG_x.
      \end{array}
      \] 
      La divergencia de $\mathbf{H}$ es:
      \[
      \mathrm{div} (\mathbf{H})=\frac{\partial H_x}{\partial x} +\frac{\partial H_y}{\partial y} +\frac{\partial H_z}{\partial z} .
      \] 
      Susituimos cada cmponente $H_x,H_y,H_z$ y derivamos:
       \[
      \begin{array}{l}
        \frac{\partial H_x}{\partial x} =\frac{\partial (F_yG_z)}{\partial x} -\frac{\partial (F_zG_y)}{\partial x} =\left( \frac{\partial F_y}{\partial x} G_z+F_y \frac{\partial G_z}{\partial x}  \right) - \left( \frac{\partial F_z}{\partial x} G_y+F_z \frac{\partial G_y}{\partial x}  \right)\\
        \frac{\partial H_y}{\partial y} =\frac{\partial (F_zG_x)}{\partial y} -\frac{\partial (F_xG_z)}{\partial y} =\left( \frac{\partial F_z}{\partial y} G_x+F_z \frac{\partial G_x}{\partial y}  \right) -\left( \frac{\partial F_x}{\partial y} G_z+F_x \frac{\partial G_z}{\partial y}  \right) \\
        \frac{\partial H_z}{\partial z} =\frac{\partial (F_xG_y)}{\partial z} -\frac{\partial (F_yG_x)}{\partial z} =\left( \frac{\partial F_x}{\partial z} G_y+F_x \frac{\partial G_y}{\partial z}  \right) - \left( \frac{\partial F_y}{\partial z} G_x+F_y \frac{\partial G_x}{\partial z}  \right)
      \end{array}
      \] 
      Sumamos todos los términos:
      \[
      \begin{aligned}
      \mathrm{div} (\mathbf{F} \times \mathbf{G} )=&G_z\left( \frac{\partial F_y}{\partial x}-\frac{\partial F_x}{\partial y}   \right) +G_x\left( \frac{\partial F_z}{\partial y} -\frac{\partial F_y}{\partial z}  \right) +G_y\left( \frac{\partial F_x}{\partial z} -\frac{\partial F_z}{\partial x}  \right)\\
      & -F_z\left( \frac{\partial G_y}{\partial x} -\frac{\partial G_x}{\partial y}  \right) -F_x\left( \frac{\partial G_z}{\partial y} -\frac{\partial G_y}{\partial z}  \right) -F_y\left( \frac{\partial G_x}{\partial z} -\frac{\partial G_z}{\partial x}  \right) =\left<\mathrm{rot} (\mathbf{F} ),\mathbf{G}  \right>-\left<\mathbf{F} ,\mathrm{rot} (\mathbf{G} ) \right>
      \end{aligned}
      \] 
    \item \db{$\mathrm{rot}(f\mathbf{F})=f\mathrm{rot}(\mathbf{F})+(\nabla f\times \mathbf{F})$}
      \[
      \mathrm{rot} (f\mathbf{F} )=\nabla \times (f\mathbf{F} )=f(\nabla \times F)+(\nabla f\times \mathbf{F} )=f\mathrm{rot} (\mathbf{F} )+(\nabla f\times \mathbf{F} )
      \] 
    \item \db{$\mathrm{div}(f\mathbf{F})=f\mathrm{div}(\mathbf{F})+<\nabla f,\mathbf{F} >$} 

      \[
      \mathrm{div} (f\mathbf{F} )=\nabla \cdot (f\mathbf{F} )=f(\nabla \cdot \mathbf{F} )+\left<\nabla f, \mathbf{F}  \right> = f\mathrm{div} (\mathbf{F} )+\left< \nabla f,\mathbf{F}  \right>.
      \] 

    \item \db{$\mathrm{div}(f\nabla g)=f\mathrm{div}(\nabla g)+ <\nabla f,\nabla g >$} 
      \[
      \mathrm{div} (f\nabla g)=f(\nabla \cdot \nabla g)+\left<\nabla f,\nabla g \right> =f\nabla ^2g+\left<\nabla f, \nabla g \right>
      \] 
  \end{enumerate}

\item \lb{Determinar si los siguientes campos vectoriales son conservativos y en caso de serlo obtener su función potencial:}
  \begin{enumerate}[label=\color{red}\textbf{\alph*)}]
    \item \db{$\mathbf{F} (x,y,z)=\dfrac{2x}{x^2+y^2+z^2}\mathbf{i} +\dfrac{2y}{x^2+y^2+z^2}\mathbf{j} +\dfrac{2z}{x^2+y^2+z^2}\mathbf{k}$.} 

      El rotacional se define como: \[
      \mathrm{rot} (\mathbf{F} )=\nabla \times \mathbf{F} =\begin{vmatrix} 
        \mathbf{i}  & \mathbf{j}  & \mathbf{k} \\
        \frac{\partial }{\partial x} & \frac{\partial }{\partial y} & \frac{\partial }{\partial z} \\
        F_x & F_y & F_z
      \end{vmatrix}, 
      \] 
      donde: \[
      F_x=\dfrac{2x}{x^2+y^2+z^2},\quad F_y=\dfrac{2y}{x^2+y^2+z^2},\quad F_z=\dfrac{2z}{x^2+y^2+z^2}.   
      \] 
      \begin{enumerate}[label=\arabic*)]
        \item Primera componente en $\mathbf{i} $: \[
        \left( \frac{\partial F_z}{\partial y} -\frac{\partial F_y}{\partial z}  \right)=-\dfrac{4yz}{(x^2+y^2+z^2)^2}-\left( -\dfrac{4yz}{(x^2+y^2+z^2)^2}  \right)  =0
        \] 
        $\begin{array}{l}
          \frac{\partial F_z}{\partial y} =\frac{\partial }{\partial y} \left( \dfrac{2z}{x^2+y^2+z^2} \right) =2z\cdot \frac{\partial }{\partial y} \left( \dfrac{1}{x^2+y^2+z^2} \right) =2z\cdot\left( -\dfrac{2y}{(x^2+y^2+z^2)^2}  \right) = -\dfrac{4yz}{(x^2+y^2+z^2)^2}\\
          \frac{\partial F_y}{\partial z} =\frac{\partial }{\partial z} \left( \dfrac{2y}{x^2+y^2+z^2} \right) =2y\cdot \frac{\partial }{\partial z} \left( \dfrac{1}{x^2+y^2+z^2} \right) =2y\cdot \left( -\dfrac{2z}{(x^2+y^2+z^2)^2}  \right) =-\dfrac{4yz}{(x^2+y^2+z^2)} 
        \end{array}$
      \item Segunda componente en $\mathbf{j} $:
        \[
        \left( \frac{\partial F_x}{\partial z}-\frac{\partial F_z}{\partial x}   \right) =-\dfrac{4xz}{(x^2+y^2+z^2)^2}-\left( -\dfrac{4xz}{(x^2+y^2+z^2)^2}  \right) =0
        \] 
        $\begin{array}{l}
          \frac{\partial F_x}{\partial z} =\frac{\partial }{\partial z} \left( \dfrac{2x}{x^2+y^2+z^2} \right) =2x\cdot \frac{\partial }{\partial z} \left( \dfrac{1}{x^2+y^2+z^2} \right) =2x\cdot \left( -\dfrac{2z}{(x^2+y^2+z^2)^2} \right) =-\dfrac{4xz}{(x^2+y^2+z^2)^2}. \\
          \frac{\partial F_z}{\partial x} =\frac{\partial }{\partial x} \left( \dfrac{2z}{x^2+y^2+z^2} \right) =2z\cdot \frac{\partial }{\partial x} \left( \dfrac{1}{x^2+y^2+z^2} \right) =2z\cdot \left( -\dfrac{2x}{(x^2+y^2+z^2)^2} \right) =-\dfrac{4xz}{(x^2+y^2+z^2)^2}. 
        \end{array}$
      \item Tercera componente en $\mathbf{k} $:
        \[
        \left( \frac{\partial F_y}{\partial x} -\frac{\partial F_x}{\partial y}  \right) =-\dfrac{4xy}{(x^2+y^2+z^2)}-\left( -\dfrac{4xy}{(x^2+y^2+z^2)^2}  \right) =0
        \] 
        $\begin{array}{l}
                  \frac{\partial F_y}{\partial x} =\frac{\partial }{\partial x} \left( \dfrac{2y}{x^2+y^2+z^2} \right) =2y\cdot \frac{\partial }{\partial x} \left( \dfrac{1}{x^2+y^2+z^2} \right) =2y\cdot \left( -\dfrac{2x}{(x^2+y^2+z^2)^2} \right) =-\dfrac{4xy}{(x^2+y^2+z^2)^2}. \\
                  \frac{\partial F_x}{\partial y} =\frac{\partial }{\partial x} \left( \dfrac{2x}{x^2+y^2+z^2} \right) =2x\cdot \frac{\partial }{\partial y} \left( \dfrac{1}{x^2+y^2+z^2} \right) =2x\cdot \left( -\dfrac{2y}{(x^2+y^2+z^2)^2} \right) =-\dfrac{4xy}{(x^2+y^2+z^2)^2}. 
                \end{array}$
      \end{enumerate}

      Como todas las componentes del rotacional son cero, se tiene: \[
      \mathrm{rot} (\mathbf{F} )=0.
      \] 
      Por lo tanto, el campo $\mathbf{F} $ es convervativo. Esto implica que existe una función potencial $f(x,y,z)$ tal que:  \[
      \mathbf{F} =\nabla f.
      \] 
      Esto implica que: \[
      \frac{\partial f}{\partial x} =\dfrac{2x}{x^2+y^2+z^2},\quad \frac{\partial f}{\partial y} =\dfrac{2y}{x^2+y^2+z^2},\quad \frac{\partial f}{\partial z} =\dfrac{2z}{x^2+y^2+z^2} .
      \]
      Observamos que: \[
      \frac{\partial f}{\partial x} =\frac{\partial }{\partial x} (\ln(x^2+y^2+z^2)),\quad \frac{\partial f}{\partial y} =\frac{\partial }{\partial y} (\ln(x^2+y^2+z^2)),\quad \frac{\partial f}{\partial z} =\frac{\partial }{\partial z}(\ln(x^2+y^2+z^2)).
      \] 
Por lo tanto, la función potencial es: \[
f(x,y,z)=\ln(x^2+y^2+z^2)+C,
\] donde $C$ es una constante de integración.
    \item \db{$\mathbf{F} (x,y,z)=\dfrac{1}{x}\mathbf{i} +\dfrac{1}{y}\mathbf{j} +\dfrac{1}{z}\mathbf{k} $} 

      El rotacional se define como: \[
      \mathrm{rot} (\mathbf{F} )=\nabla \times \mathbf{F} =\begin{vmatrix} 
        \mathbf{i}  & \mathbf{j}  & \mathbf{k} \\
        \frac{\partial }{\partial x} & \frac{\partial }{\partial y} & \frac{\partial }{\partial z} \\
        F_x & F_y & F_z
      \end{vmatrix} 
      \] 
      donde:
      \[
      F_x=\dfrac{1}{x},\quad F_y=\dfrac{1}{y}, \quad F_z=\dfrac{1}{z}
      \] 
      \begin{enumerate}[label=\arabic*)]
        \item Primera componente en $\mathbf{i} $:
          \[
          \left( \frac{\partial F_z}{\partial y} -\frac{\partial F_y}{\partial z}  \right) =0-0=0
          \] 
        \item Segunda componente en $\mathbf{j} $:
          \[
          \left( \frac{\partial F_x}{\partial z} -\frac{\partial F_z}{\partial x}  \right) = 0-0=0
          \] 
        \item Tercera componente en $\mathbf{k} $:
          \[
          \left( \frac{\partial F_y}{\partial x} -\frac{\partial F_x}{\partial y}  \right) =0-0=0
          \] 
      \end{enumerate}
      Como todas las componentes son cero, se tiene: \[
      \mathrm{rot} (\mathbf{F} )=0.
      \] 
      Por lo tanto, el campo $\mathbf{F} $ es conservativo. Esto implica que existe una función potencial $f(x,y,z)$ tal que:  \[
      \mathbf{F} =\nabla f.
      \]   
      Esto implica que: \[
      \frac{\partial f}{\partial x} =\dfrac{1}{x},\quad \frac{\partial f}{\partial y} =\dfrac{1}{y},\quad \frac{\partial f}{\partial z} =\dfrac{1}{z}.
      \] 
      Observamos que: \[
      \frac{\partial f}{\partial x} =\frac{\partial }{\partial x} (\ln(x+y+z)),\quad \frac{\partial f}{\partial y} =\frac{\partial }{\partial y} (\ln(x+y+z)), \quad \frac{\partial f}{\partial z} =\frac{\partial }{\partial z} (\ln(x+y+z)).
      \] 
      Por lo tanto, la función potencial es: \[
      f(x,y,z)=\ln(x+y+z)+C,
      \] 
      donde $C$ es una constante de integración.
    \item \db{$\mathbf{F} (x,y,z)=(x+y^2)\mathbf{i} +2yx\mathbf{j} $} 
    
      El rotacional se define como: \[
      \mathrm{rot} (\mathbf{F} )=\nabla \times \mathbf{F} =\begin{vmatrix} 
        \mathbf{i}  & \mathbf{j} & \mathbf{k} \\
        \frac{\partial }{\partial x} & \frac{\partial }{\partial y} & \frac{\partial }{\partial z} \\
        F_x & F_y & F_x
      \end{vmatrix} 
      \] donde: \[      
        F_x=x+y^2, \quad F_y=2yx, \quad F_z=0.
      \] 
      Sustituyendo, tenemos que:
      \[
      \mathrm{rot} (\mathbf{F} )=\nabla \times \mathbf{F} =\begin{vmatrix} 
        \mathbf{i}  & \mathbf{j}  & \mathbf{k} \\
        \frac{\partial }{\partial x} & \frac{\partial }{\partial y} & \frac{\partial }{\partial z} \\
        x+y^2 & 2yx & 0
      \end{vmatrix} = \mathbf{i} (0 - 0) - \mathbf{j}(0-0) +\mathbf{k} (2y-2y)=0
      \] 

      Tenemos que: \[
      \mathrm{rot} (\mathbf{F} )=0.
      \] Por lo tanto, el campo $F$ es conservativo. Esto implica que existe una función potencial  $f(x,y,z)$ tal que:  \[
      \mathbf{F} =\nabla f.
      \] 
Esto implica que: \[
\frac{\partial f}{\partial x} =\frac{\partial }{\partial x} (x+2y), \quad \frac{\partial f}{\partial y} =\frac{\partial }{\partial y} (2xy),\quad \frac{\partial f}{\partial z} =\frac{\partial }{\partial z} (0)=0
\]
Integramos $F_x$ respecto de  $x$:
\[
f(x,y,z)=  \int F_x \dx =\int x+y^2\dx =\dfrac{x^2}{2}+xy^2+C(y,z),
\]
donde $C(y,z)$ es una función que puede depender de  $y$ y  $z$.

Ahora derivamos  $f(x,y,z)$ respecto a  $y$ y comparamos con  $F_y$:
 \[
\frac{\partial f}{\partial y} =\frac{\partial }{\partial y} \left( \dfrac{x^2}{2}+y^2x+C(y,z) \right) =2yx+\frac{\partial C(y,z)}{\partial y} .
\] 
Igualamos con $F_y=2xy$:  \[
2yx+\frac{\partial C(y,z)}{\partial y} =2yx\longrightarrow \frac{\partial C(y,z)}{\partial y} =0\longrightarrow C(y,z)=C(z),
\] 
donde $C(z)$ es una función que depende solo de  $z$.

Para finalizar derivamos  $f(x,y,z)$ respecto de  $z$ y lo comparamos con  $F_z=\frac{\partial f}{\partial z} =0$:
\[
\frac{\partial f}{\partial z} =\frac{\partial }{\partial z} \left( \dfrac{x^2}{2}+y^2x+C(z) \right) =\frac{\partial C(z)}{\partial z} =0\longrightarrow C(z)=\text{constante}.
\] 
Sustituyendo en $f(x,y,z)$:  \[
f(x,y,z)=\dfrac{x^2}{2}+y^2x+\text{constante}=\dfrac{x^2}{2}+y^2x+C,
\] 
donde $C\in \R$ es una constante arbitraria.
    \item \db{$\mathbf{F} (x,y,z)=\dfrac{x}{(x^2+y^2+z^2)^{\frac{3}{2} }}\mathbf{i}+\dfrac{y}{(x^2+y^2+z^2)^{\frac{3}{2} }}\mathbf{j} +\dfrac{z}{(x^2+y^2+z^2)^{\frac{3}{2} }} \mathbf{k} $} 

      El rotacional se define como: \[
      \mathrm{rot} (\mathbf{F} )=\nabla \times \mathbf{F} =\begin{vmatrix} 
        \mathbf{i}  & \mathbf{j}  & \mathbf{k} \\
        \frac{\partial }{\partial x} & \frac{\partial }{\partial y} & \frac{\partial }{\partial z}\\
        F_x & F_y & F_z
      \end{vmatrix} 
      \] 
      donde: \[
      F_x=\dfrac{x}{(x^2+y^2+z^2)^{\frac{3}{2} }},\quad F_y=\dfrac{y}{(x^2+y^2+z^2)^{\frac{3}{2} }},\quad F_z=\dfrac{z}{(x^2+y^2+z^2)^{\frac{3}{2} }}.
      \] 
      \begin{enumerate}[label=\arabic*)]
        \item Primera componente en $\mathbf{i} $:
          \[
          \left( \frac{\partial F_z}{\partial y} -\frac{\partial F_y}{\partial z}  \right) =-\dfrac{3yz}{(x^2+y^2+z^2)^{\frac{5}{2} }}-\left(-\dfrac{3yz}{(x^2+y^2+z^2)^{\frac{5}{2} }}\right)=0
          \] 
          $\begin{array}{l}
            \frac{\partial F_z}{\partial y} =\frac{\partial }{\partial y} \left( \dfrac{z}{(x^2+y^2+z^2)^{\frac{3}{2} }} \right) = z\cdot \frac{\partial }{\partial y} \left( \dfrac{1}{(x^2+y^2+z^2)^{\frac{3}{2} }} \right) =z\cdot \left( -\dfrac{3y}{(x^2+y^2+z^2)^{\frac{5}{2} }} \right)=-\dfrac{3yz}{(x^2+y^2+z^2)^{\frac{5}{2} }}\\
            \frac{\partial F_y}{\partial z} =\frac{\partial }{\partial z} \left( \dfrac{y}{(x^2+y^2+z^2)^{\frac{3}{2} }} \right) = y\cdot \frac{\partial }{\partial z} \left( \dfrac{1}{(x^2+y^2+z^2)^{\frac{3}{2} }} \right) =y\cdot \left( -\dfrac{3z}{(x^2+y^2+z^2)^{\frac{5}{2} }} \right)=-\dfrac{3yz}{(x^2+y^2+z^2)^{\frac{5}{2} }}\\
          \end{array}$
        \item Segunda componente en $\mathbf{j} $:
          \[
          \left( \frac{\partial F_x}{\partial z} -\frac{\partial F_z}{\partial x}  \right) =-\dfrac{3xz}{(x^2+y^2+z^2)^{\frac{5}{2} }}-\left(-\dfrac{3xz}{(x^2+y^2+z^2)^{\frac{5}{2} }}\right)=0
          \] 
          $\begin{array}{l}
            \frac{\partial F_x}{\partial z} =\frac{\partial }{\partial z} \left( \dfrac{x}{(x^2+y^2+z^2)^{\frac{3}{2} }} \right) = x\cdot \frac{\partial }{\partial z} \left( \dfrac{1}{(x^2+y^2+z^2)^{\frac{3}{2} }} \right) =x\cdot \left( -\dfrac{3z}{(x^2+y^2+z^2)^{\frac{5}{2} }} \right)=-\dfrac{3xz}{(x^2+y^2+z^2)^{\frac{5}{2} }}\\
            \frac{\partial F_z}{\partial x} =\frac{\partial }{\partial x} \left( \dfrac{z}{(x^2+y^2+z^2)^{\frac{3}{2} }} \right) = z\cdot \frac{\partial }{\partial x} \left( \dfrac{1}{(x^2+y^2+z^2)^{\frac{3}{2} }} \right) =z\cdot \left( -\dfrac{3x}{(x^2+y^2+z^2)^{\frac{5}{2} }} \right)=-\dfrac{3xz}{(x^2+y^2+z^2)^{\frac{5}{2} }}\\
          \end{array}$
        \item Tercera componente en $\mathbf{k} $:
          \[
          \left( \frac{\partial F_y}{\partial x} -\frac{\partial F_x}{\partial y}  \right) =-\dfrac{3xy}{(x^2+y^2+z^2)^{\frac{5}{2} }}-\left(-\dfrac{3xy}{(x^2+y^2+z^2)^{\frac{5}{2} }}\right)=0
          \] 
          $\begin{array}{l}
            \frac{\partial F_y}{\partial x} =\frac{\partial }{\partial x} \left( \dfrac{y}{(x^2+y^2+z^2)^{\frac{3}{2} }} \right) = y\cdot \frac{\partial }{\partial x} \left( \dfrac{1}{(x^2+y^2+z^2)^{\frac{3}{2} }} \right) =y\cdot \left( -\dfrac{3x}{(x^2+y^2+z^2)^{\frac{5}{2} }} \right)=-\dfrac{3xy}{(x^2+y^2+z^2)^{\frac{5}{2} }}\\
            \frac{\partial F_x}{\partial y} =\frac{\partial }{\partial y} \left( \dfrac{x}{(x^2+y^2+z^2)^{\frac{3}{2} }} \right) = x\cdot \frac{\partial }{\partial y} \left( \dfrac{1}{(x^2+y^2+z^2)^{\frac{3}{2} }} \right) =x\cdot \left( -\dfrac{3y}{(x^2+y^2+z^2)^{\frac{5}{2} }} \right)=-\dfrac{3xy}{(x^2+y^2+z^2)^{\frac{5}{2} }}\\
          \end{array}$
      \end{enumerate}

      Como todas las componentes son cero, se tiene: \[
      \mathrm{rot} (\mathbf{F} )=0
      \] 
      Por lo tanto, el campo $\mathbf{F} $ es conservativo. Esto implica que existe una función potencial $f(x,y,z)$ tal que:  \[
      \mathbf{F} =\nabla f.
      \] 
      Esto implica que: 
      \[
      \frac{\partial f}{\partial x} =\dfrac{x}{(x^2+y^2+z^2)^{\frac{3}{2} }},\quad \frac{\partial f}{\partial y} = \dfrac{y}{(x^2+y^2+z^2)^{\frac{3}{2} }}, \quad \frac{\partial f}{\partial z} =\dfrac{z}{(x^2+y^2+z^2)^{\frac{3}{2} }}.
      \] 

    \item \db{$\mathbf{F} (x,y,z)=(y+z)\mathbf{i} +(x+z)\mathbf{j} +(x+y)\mathbf{k} $} 
    \item \db{$\mathbf{F} (x,y,z)=(4x+2y+2z)\mathbf{i} +(2x+4y+2z)\mathbf{j} +(2x+2y+4z)\mathbf{k} $} 
  \end{enumerate}
\item \lb{Dada $f(x,y)$, calcular el gradiente de  $f$ en coordenadas polares.}

\item \lb{Sea $f:\R^2 \backslash  \{(0,0)\} \to \R$ es un campo escalar de clase $C^2$. Comprobar que \[
\dfrac{1}{x^2+y^2}\left( \frac{\partial^2 f}{\partial x^2} (x,y)+\frac{\partial^2 f}{\partial y^2} (x,y) \right) =4\left( \frac{\partial^2 f}{\partial u^2} (u,v)+\frac{\partial^2 f}{\partial v^2} (u,v) \right) 
\] donde $u=x^2-y^2$ y $v=2xy$. } 

\item \lb{Demostrar que si $f:\R^2\to \R$ es un campo escalar de clase $C^2$ y se verifica la igualdad \[
\frac{\partial^2 f}{\partial x^2} (x,y)+\frac{\partial^2 f}{\partial y^2} (x,y)=0,
\]
entonces también se verifica 
\[
\frac{\partial^2 f}{\partial u^2} (u,v)+\frac{\partial^2 f}{\partial v^2} =0
\]
donde $x=\dfrac{u}{u^2+v^2} $ e $y=\dfrac{v}{u^2+v^2}$.} 
\end{enumerate}
