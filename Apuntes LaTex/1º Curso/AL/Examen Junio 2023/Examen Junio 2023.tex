\documentclass[12pt]{article}
\usepackage{fullpage}
\usepackage[utf8]{inputenc}
\usepackage{pict2e}
\usepackage{amsmath}
\usepackage{enumitem}
\usepackage{eurosym}
\usepackage{pict2e}
\usepackage{mathtools}
\usepackage{amssymb, amsfonts, latexsym, cancel}
\setlength{\parskip}{0.3cm}
\usepackage{graphicx}
\usepackage{fontenc}
\usepackage{slashbox}
\usepackage{setspace}
\usepackage{gensymb}
\usepackage{accents}
\usepackage{adjustbox}
\setstretch{1.5}
\usepackage{bold-extra}
\usepackage[document]{ragged2e}
\usepackage{subcaption}
\usepackage{tcolorbox}
\usepackage{xcolor, colortbl}
\usepackage{wrapfig}
\usepackage{empheq}
\usepackage{array}
\usepackage{parskip}
\usepackage{arydshln}
\graphicspath{ {images/} }
\renewcommand*\contentsname{\color{black}Índice} 
\usepackage{array, multirow, multicol}
\definecolor{lightblue}{HTML}{007AFF}
\usepackage{color}
\usepackage{etoolbox}
\usepackage{listings}
\usepackage{mdframed}
\setlength{\parindent}{0pt}
\usepackage{underscore}
\usepackage{hyperref}
\usepackage{tikz}
\usepackage{tikz-cd}
\usetikzlibrary{shapes, positioning, patterns}
\usepackage{tikz-qtree}
\usepackage{biblatex}
\usepackage{pdfpages}
\usepackage{pgfplots}
\usepackage{pgfkeys}
\addbibresource{biblatex-examples.bib}
\usepackage[a4paper, left=1.5cm, right=1.5cm, top=1cm,
bottom=1.5cm]{geometry}
\everymath{\displaystyle}
\usetikzlibrary{decorations.pathreplacing}
\usepackage{titlesec}
\usepackage{titletoc}
\usepackage{tikz-3dplot}
\usetikzlibrary{decorations.pathreplacing}
\newcommand{\Ej}{\textcolor{lightblue}{\underline{Ejemplo}}}
\setlength{\fboxrule}{1.5pt}
\renewcommand{\arraystretch}{1.35}
\setlength{\arraycolsep}{0.3cm}

% Configura el formato de las secciones utilizando titlesec
\titleformat{\section}
{\color{red}\normalfont\LARGE\bfseries}
{Tema \thesection:}
{10 pt}
{}

% Ajusta el formato de las entradas de la tabla de contenidos
\addtocontents{toc}{\protect\setcounter{tocdepth}{4}}
\addtocontents{toc}{\color{black}}

\titleformat{\subsection}
{\normalfont\Large\bfseries\color{red}}{\thesubsection)}{1em}{\color{lightblue}}

\titleformat{\subsubsection}
{\normalfont\large\bfseries\color{red}}{\thesubsubsection)}{1em}{\color{lightblue}}

\newcommand{\bboxed}[1]{\fcolorbox{lightblue}{lightblue!10}{$#1$}}

\DeclareMathOperator{\N}{\mathbb{N}}
\DeclareMathOperator{\Z}{\mathbb{Z}}
\DeclareMathOperator{\R}{\mathbb{R}}
\DeclareMathOperator{\Q}{\mathbb{Q}}
\DeclareMathOperator{\K}{\mathbb{K}}
\DeclareMathOperator{\im}{\imath}
\DeclareMathOperator{\jm}{\jmath}
\DeclareMathOperator{\col}{\mathrm{Col}}
\DeclareMathOperator{\fil}{\mathrm{Fil}}
\DeclareMathOperator{\rg}{\mathrm{rg}}
\DeclareMathOperator{\nuc}{\mathrm{nuc}}
\DeclareMathOperator{\dimf}{\mathrm{dimFil}}
\DeclareMathOperator{\dimc}{\mathrm{dimCol}}
\DeclareMathOperator{\dimn}{\mathrm{dimnuc}}
\DeclareMathOperator{\dimr}{\mathrm{dimrg}}

\newcommand{\bu}[1]{\textcolor{lightblue}{\underline{#1}}}
\newcommand{\lb}[1]{\textcolor{lightblue}{#1}}
\newcommand{\db}[1]{\textcolor{blue}{#1}}
\newcommand{\rc}[1]{\textcolor{red}{#1}}
\newcommand{\tr}{^\intercal}

\renewcommand{\CancelColor}{\color{lightblue}}

\newcommand{\dx}{\:\mathrm{d}x}
\newcommand{\dt}{\:\mathrm{d}t}
\newcommand{\dy}{\:\mathrm{d}y}
\newcommand{\dz}{\:\mathrm{d}z}
\newcommand{\dth}{\:\mathrm{d}\theta}
\newcommand{\dr}{\:\mathrm{d}\rho}
\newcommand{\du}{\:\mathrm{d}u}
\newcommand{\dv}{\:\mathrm{d}v}
\newcommand{\tozero}[1]{\cancelto{0}{#1}}
\newcommand{\lbb}[2]{\textcolor{lightblue}{\underbracket[1pt]{\textcolor{black}{#1}}_{#2}}}
\newcommand{\dbb}[2]{\textcolor{blue}{\underbracket[1pt]{\textcolor{black}{#1}}_{#2}}}
\title{Álgebra Lineal\\Examen Convocatoria Junio 2023}

\begin{document}
\maketitle
\begin{enumerate}[label=\color{red}\textbf{\arabic*)}]
    \item \lb{Consideremos los número complejos \[
    x_1=-1-j,\quad z_2=\sqrt{2}e^{\frac{\pi}{4} j}  .
    \]Calcula $z_1+z_2,z_1\cdot z_2$ y $\dfrac{z_2}{z_1}$ y expresa el resultado en forma exponencial.}

\item \lb{Dada una matriz cuadrada $D$, se llama de  \textit{similitud producto-escalar} a la matriz $S=DD^\intercal$, con $D^\intercal$ la traspuesta de $D$. Se pide:}
    \begin{enumerate}[label=\color{red}\textbf{\alph*)}]
        \item \db{Demuestra que $S$ simétrica.}
        \item \db{Sea $P$ una matriz ortogonal del mismo tamaño que $D$ y consideremos la matriz $DP$. Denotemos por  $S$ y  $S'$ a las matrices de similitud producto-escalar de  $D$ y  $DP$, respectivamente. Comprueba que  $S=S'$.} 
    \end{enumerate}
\item \lb{Sea $A$ una matriz. Explica con detalle en qué consiste la factorización en valores singulares  $(SVD)$ de  $A$. Por supuesto, se ha de explicar qué son los valores singulares y cómo se calculan las matrices que aparecen en dicha factorización. Pon también un ejemplo de aplicación de la factorización  $SVD$ en Ciencia de datos.}

\item \lb{Consideremos el sistema de ecuaciones \[
\left.\begin{array}{rcc}
        4x-4y & 0 & 8\\
        -4x+5y+z & = & -9\\
        y+2z &=&1
\end{array}  \right\} 
\] } 
\begin{enumerate}[label=\color{red}\textbf{\alph*)}]
    \item \db{Comprueba que la matriz de coeficientes del sistema es simétrica definida positiva.}
    \item \db{Calcula la factorización de Cholesky de la matriz de coeficientes.}
    \item \db{Resuelve el sistema usando la factorización anterior}
\end{enumerate}
\item \lb{Dada la matriz \[
\begin{bmatrix} 
    1 & 1 & 0 & 1\\
    0 & -1 & -1 & 1\\
    1 & 2 & 1 & 0
\end{bmatrix} 
\] calcula una base ortonormal del subespacio $\mathrm{Col}(A)$. Si $B$ es una matriz de tamaño  $4\times m$, ¿qué podemos decir de la dimensión del subespacio $\mathrm{Col}(AB)$?}
\item \lb{Dada la matriz $\begin{bmatrix} 
            a & 0\\b&a 
\end{bmatrix} $ con $b\neq 0$, explica por qué no existe ninguna matriz invertible $P$ tal que la matriz  $P^{-1}AP$ sea una matriz diagonal.} 
\end{enumerate}
\end{document}
