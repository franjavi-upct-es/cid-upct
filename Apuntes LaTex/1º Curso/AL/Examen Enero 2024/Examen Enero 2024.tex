\documentclass[12pt]{article}
\usepackage{fullpage}
\usepackage[utf8]{inputenc}
\usepackage{pict2e}
\usepackage{amsmath}
\usepackage{enumitem}
\usepackage{eurosym}
\usepackage{pict2e}
\usepackage{mathtools}
\usepackage{amssymb, amsfonts, latexsym, cancel}
\setlength{\parskip}{0.3cm}
\usepackage{graphicx}
\usepackage{fontenc}
\usepackage{slashbox}
\usepackage{setspace}
\usepackage{gensymb}
\usepackage{accents}
\usepackage{adjustbox}
\setstretch{1.5}
\usepackage{bold-extra}
\usepackage[document]{ragged2e}
\usepackage{subcaption}
\usepackage{tcolorbox}
\usepackage{xcolor, colortbl}
\usepackage{wrapfig}
\usepackage{empheq}
\usepackage{array}
\usepackage{parskip}
\usepackage{arydshln}
\graphicspath{ {images/} }
\renewcommand*\contentsname{\color{black}Índice} 
\usepackage{array, multirow, multicol}
\definecolor{lightblue}{HTML}{007AFF}
\usepackage{color}
\usepackage{etoolbox}
\usepackage{listings}
\usepackage{mdframed}
\setlength{\parindent}{0pt}
\usepackage{underscore}
\usepackage{hyperref}
\usepackage{tikz}
\usepackage{tikz-cd}
\usetikzlibrary{shapes, positioning, patterns}
\usepackage{tikz-qtree}
\usepackage{biblatex}
\usepackage{pdfpages}
\usepackage{pgfplots}
\usepackage{pgfkeys}
\addbibresource{biblatex-examples.bib}
\usepackage[a4paper, left=1.5cm, right=1.5cm, top=1cm,
bottom=1.5cm]{geometry}
\everymath{\displaystyle}
\usetikzlibrary{decorations.pathreplacing}
\usepackage{titlesec}
\usepackage{titletoc}
\usepackage{tikz-3dplot}
\usetikzlibrary{decorations.pathreplacing}
\newcommand{\Ej}{\textcolor{lightblue}{\underline{Ejemplo}}}
\setlength{\fboxrule}{1.5pt}
\renewcommand{\arraystretch}{1.35}
\setlength{\arraycolsep}{0.3cm}

% Configura el formato de las secciones utilizando titlesec
\titleformat{\section}
{\color{red}\normalfont\LARGE\bfseries}
{Tema \thesection:}
{10 pt}
{}

% Ajusta el formato de las entradas de la tabla de contenidos
\addtocontents{toc}{\protect\setcounter{tocdepth}{4}}
\addtocontents{toc}{\color{black}}

\titleformat{\subsection}
{\normalfont\Large\bfseries\color{red}}{\thesubsection)}{1em}{\color{lightblue}}

\titleformat{\subsubsection}
{\normalfont\large\bfseries\color{red}}{\thesubsubsection)}{1em}{\color{lightblue}}

\newcommand{\bboxed}[1]{\fcolorbox{lightblue}{lightblue!10}{$#1$}}

\DeclareMathOperator{\N}{\mathbb{N}}
\DeclareMathOperator{\Z}{\mathbb{Z}}
\DeclareMathOperator{\R}{\mathbb{R}}
\DeclareMathOperator{\Q}{\mathbb{Q}}
\DeclareMathOperator{\K}{\mathbb{K}}
\DeclareMathOperator{\im}{\imath}
\DeclareMathOperator{\jm}{\jmath}
\DeclareMathOperator{\col}{\mathrm{Col}}
\DeclareMathOperator{\fil}{\mathrm{Fil}}
\DeclareMathOperator{\rg}{\mathrm{rg}}
\DeclareMathOperator{\nuc}{\mathrm{nuc}}
\DeclareMathOperator{\dimf}{\mathrm{dimFil}}
\DeclareMathOperator{\dimc}{\mathrm{dimCol}}
\DeclareMathOperator{\dimn}{\mathrm{dimnuc}}
\DeclareMathOperator{\dimr}{\mathrm{dimrg}}

\newcommand{\bu}[1]{\textcolor{lightblue}{\underline{#1}}}
\newcommand{\lb}[1]{\textcolor{lightblue}{#1}}
\newcommand{\db}[1]{\textcolor{blue}{#1}}
\newcommand{\rc}[1]{\textcolor{red}{#1}}
\newcommand{\tr}{^\intercal}

\renewcommand{\CancelColor}{\color{lightblue}}

\newcommand{\dx}{\:\mathrm{d}x}
\newcommand{\dt}{\:\mathrm{d}t}
\newcommand{\dy}{\:\mathrm{d}y}
\newcommand{\dz}{\:\mathrm{d}z}
\newcommand{\dth}{\:\mathrm{d}\theta}
\newcommand{\dr}{\:\mathrm{d}\rho}
\newcommand{\du}{\:\mathrm{d}u}
\newcommand{\dv}{\:\mathrm{d}v}
\newcommand{\tozero}[1]{\cancelto{0}{#1}}
\newcommand{\lbb}[2]{\textcolor{lightblue}{\underbracket[1pt]{\textcolor{black}{#1}}_{#2}}}
\newcommand{\dbb}[2]{\textcolor{blue}{\underbracket[1pt]{\textcolor{black}{#1}}_{#2}}}
\title{Álgebra Lineal\\Examen Convocatoria Enero 2024}

\begin{document}
\maketitle
\begin{enumerate}[label=\color{red}\textbf{\arabic*)}]
    \item \lb{Sea $A$ una matriz de tamaño  $m\times n$ y cuyo rango es $r$. Se pide:}
        \begin{enumerate}[label=\color{red}\textbf{\alph*)}]
            \item \db{Define los cuatro subespacios fundamentales asociados a $A$.}
            \item \db{¿Qué dimensión tiene cada subespacio fundamental en función de $n,m$ y  $r$? Justifica la respuesta.}
            \item \db{Suponiendo que $n=m$, decide si las siguientes afirmaciones son ciertas o falsas. Para las ciertas ¿es la suma directa? Justifica la respuesta. Para las falsas debes dar un contraejemplo:  \[
            \begin{array}{c}
                \mathrm{Fil}(A)+\mathrm{Nuc}(A)=\R^n\\
                \mathrm{Col}(A)+\mathrm{Nuc}(A)=\R^n
            \end{array}
            \] } 
        \item \db{Explica la relación que existe entre dichos subespacios y las soluciones del sistema lineal $Ax=b$, incluyendo el caso $b=0$.} 
        \end{enumerate}
    \item \lb{Sea $A$ una matriz cuadrada orden  $n$ e invertible. Sean $u,v$ dos vectores no nulos vistos como matrices columna  $n\times 1$. Demuestra que la matriz $B=Auv^\intercal$ es invertible, y que su inversa viene dada por \begin{equation}
    B^{-1}=A^{-1}-\dfrac{A^{-1}uv^\intercal A^{-1}}{1+v^\intercal A^{-1}u}.
\end{equation}
La expresión (1) se denomina fórmula de Sherman-Morrison-Woodbury, y es clave en el éxito de los llamados algoritmos de optimización tipo quasi-Newton, que estudiarás en la asignatura Optimización II.}

\item \lb{Consideremos los números complejos \[
z_1=-1+j,\quad z_2=\sqrt{2}e^{\frac{3\pi}{4}j }  .
\]Calcula $z_1+z_2,z_1\cdot z_2$ y $\dfrac{z_2}{z_1}$ y expresa el resultado en forma exponencial.}
\item \lb{Consideremos la matriz $A$ y el vector  $b$ siguientes:  \[
A=\begin{bmatrix} 
    1 & 0 & 1\\
    -1 & 0 & -1\\
    0 & 1 & 1
\end{bmatrix},\qquad b=\begin{bmatrix} 
0\\1\\0 
\end{bmatrix}  .
\] } 
\begin{enumerate}[label=\color{red}\textbf{\alph*)}]
    \item \db{Demuestra que el sistema $Ax=b$ no tiene solución y calcula la proyección ortogonal de  $b$ sobre el subespacio vectorial  $\mathrm{Col}(A)$.} 
    \item \db{Escribe la proyección que has calculado como $Az$ y explica por qué  $z$ es una solucion aproximada del sistema  $Ax=b$.} 
\end{enumerate}
\item \lb{Sea $f:\R^2\to \R^2$ la aplicación lineal definida por \[
f(x,y)=\dfrac{1}{5}(-3x+4y,4x+3y)
\]y sea $\mathcal{B}=\{u_1,u_2\} $ con $u_1=(1,2)$ y $u_2=(2,-1)$ una base de $\R^2$.}
\begin{enumerate}[label=\color{red}\textbf{\alph*)}]
    \item \db{Halla la matriz de $f$ en la base  $\mathcal{B}$.}
    \item \db{A la vista de la matriz obtenida en el aparato anterior, ¿puedes dar una interpretación geométrica de la aplicación $f$?} 
\end{enumerate}
\item \lb{Consideremos la matriz de datos \[
A=\begin{bmatrix} 
    1 & 1\\ 1 & 0\\0 & 1 
\end{bmatrix} 
\] Se pide:}
\begin{enumerate}[label=\color{red}\textbf{\alph*)}]
    \item \db{Calcula matrices $U,V$ ortogonales y  $\Sigma$ diagonal con los valores singulares de  $A$ en la diagonal tales que  $A=U\Sigma V^\intercal$ (haz cálculos exactos sin tomar aproximaciones decimales).}
    \item \db{Calcula una matriz $A_1$ de rango uno que mejor aproxime a $A$, es decir, que haga mínima la norma  $\|A-A_1\|$.} 
\end{enumerate}
\end{enumerate}
\end{document}
