\documentclass[12pt]{article}
\usepackage{fullpage}
\usepackage[utf8]{inputenc}
\usepackage{pict2e}
\usepackage{amsmath}
\usepackage{enumitem}
\usepackage{eurosym}
\usepackage{pict2e}
\usepackage{mathtools}
\usepackage{amssymb, amsfonts, latexsym, cancel}
\setlength{\parskip}{0.3cm}
\usepackage{graphicx}
\usepackage{fontenc}
\usepackage{slashbox}
\usepackage{setspace}
\usepackage{gensymb}
\usepackage{accents}
\usepackage{adjustbox}
\setstretch{1.5}
\usepackage{bold-extra}
\usepackage[document]{ragged2e}
\usepackage{subcaption}
\usepackage{tcolorbox}
\usepackage{xcolor, colortbl}
\usepackage{wrapfig}
\usepackage{empheq}
\usepackage{array}
\usepackage{parskip}
\usepackage{arydshln}
\graphicspath{ {images/} }
\renewcommand*\contentsname{\color{black}Índice} 
\usepackage{array, multirow, multicol}
\definecolor{lightblue}{HTML}{007AFF}
\usepackage{color}
\usepackage{etoolbox}
\usepackage{listings}
\usepackage{mdframed}
\setlength{\parindent}{0pt}
\usepackage{underscore}
\usepackage{hyperref}
\usepackage{tikz}
\usepackage{tikz-cd}
\usetikzlibrary{shapes, positioning, patterns}
\usepackage{tikz-qtree}
\usepackage{biblatex}
\usepackage{pdfpages}
\usepackage{pgfplots}
\usepackage{pgfkeys}
\addbibresource{biblatex-examples.bib}
\usepackage[a4paper, left=1.5cm, right=1.5cm, top=1cm,
bottom=1.5cm]{geometry}
\everymath{\displaystyle}
\usetikzlibrary{decorations.pathreplacing}
\usepackage{titlesec}
\usepackage{titletoc}
\usepackage{tikz-3dplot}
\usetikzlibrary{decorations.pathreplacing}
\newcommand{\Ej}{\textcolor{lightblue}{\underline{Ejemplo}}}
\setlength{\fboxrule}{1.5pt}
\renewcommand{\arraystretch}{1.35}
\setlength{\arraycolsep}{0.3cm}

% Configura el formato de las secciones utilizando titlesec
\titleformat{\section}
{\color{red}\normalfont\LARGE\bfseries}
{Tema \thesection:}
{10 pt}
{}

% Ajusta el formato de las entradas de la tabla de contenidos
\addtocontents{toc}{\protect\setcounter{tocdepth}{4}}
\addtocontents{toc}{\color{black}}

\titleformat{\subsection}
{\normalfont\Large\bfseries\color{red}}{\thesubsection)}{1em}{\color{lightblue}}

\titleformat{\subsubsection}
{\normalfont\large\bfseries\color{red}}{\thesubsubsection)}{1em}{\color{lightblue}}

\newcommand{\bboxed}[1]{\fcolorbox{lightblue}{lightblue!10}{$#1$}}

\DeclareMathOperator{\N}{\mathbb{N}}
\DeclareMathOperator{\Z}{\mathbb{Z}}
\DeclareMathOperator{\R}{\mathbb{R}}
\DeclareMathOperator{\Q}{\mathbb{Q}}
\DeclareMathOperator{\K}{\mathbb{K}}
\DeclareMathOperator{\im}{\imath}
\DeclareMathOperator{\jm}{\jmath}
\DeclareMathOperator{\col}{\mathrm{Col}}
\DeclareMathOperator{\fil}{\mathrm{Fil}}
\DeclareMathOperator{\rg}{\mathrm{rg}}
\DeclareMathOperator{\nuc}{\mathrm{nuc}}
\DeclareMathOperator{\dimf}{\mathrm{dimFil}}
\DeclareMathOperator{\dimc}{\mathrm{dimCol}}
\DeclareMathOperator{\dimn}{\mathrm{dimnuc}}
\DeclareMathOperator{\dimr}{\mathrm{dimrg}}

\newcommand{\bu}[1]{\textcolor{lightblue}{\underline{#1}}}
\newcommand{\lb}[1]{\textcolor{lightblue}{#1}}
\newcommand{\db}[1]{\textcolor{blue}{#1}}
\newcommand{\rc}[1]{\textcolor{red}{#1}}
\newcommand{\tr}{^\intercal}

\renewcommand{\CancelColor}{\color{lightblue}}

\newcommand{\dx}{\:\mathrm{d}x}
\newcommand{\dt}{\:\mathrm{d}t}
\newcommand{\dy}{\:\mathrm{d}y}
\newcommand{\dz}{\:\mathrm{d}z}
\newcommand{\dth}{\:\mathrm{d}\theta}
\newcommand{\dr}{\:\mathrm{d}\rho}
\newcommand{\du}{\:\mathrm{d}u}
\newcommand{\dv}{\:\mathrm{d}v}
\newcommand{\tozero}[1]{\cancelto{0}{#1}}
\newcommand{\lbb}[2]{\textcolor{lightblue}{\underbracket[1pt]{\textcolor{black}{#1}}_{#2}}}
\newcommand{\dbb}[2]{\textcolor{blue}{\underbracket[1pt]{\textcolor{black}{#1}}_{#2}}}
\title{Álgebra Lineal\\ Ejercicios Tema 3: Sistemas de ecuaciones y determinantes}
\renewcommand{\arraystretch}{1}
\setlength{\arraycolsep}{6pt}

\begin{document}
\maketitle
\begin{enumerate}[label=\color{red}\textbf{\arabic*)}]
    \item \lb{Sea $\{w_1,w_2,w_3\} $ un conjunto independiente de vectores de $\R^3$. Se definen los vectores $v_1=w_1+w_2,v_2=w_1+2w_2+w_3$ y $v_3=w_2+cw_3$. Si $V=[v_1,v_2,v_3]$ y $W=[w_1,w_2,w_3]$, entonces se tiene $V=WC$ con  \[
    C=\begin{bmatrix} 
        1 & 1 & 0\\
        1 & 2 & 1\\
        0 & 1 & c
    \end{bmatrix} 
    \]¿Qué condición debe cumplir $c$ para que los vectores  $v_1,v_2,v_3$ sean linealmente independientes? } 

    Para determinar la condición que debe cumplir $c$ para que los vectores  $v_1,v_2,v_3$ sean linealmente indpendientes, observa que:
\begin{enumerate}[label=\arabic*)]
\item \textbf{Los vectores $\{w_1,w_2,w_3\} $} son linealmente independientes en $\R^3$. Esto implica que la matriz \[
        W=[w_1,w_2,w_3]
    \] es invertible  (tiene determinante distinto de cero).
\item \textbf{Los vectores $\{v_1,v_2,v_3\} $} se pueden expresar como \[
        V=[v_1,v_2,v_3]=WC,
\] donde \[
C=\begin{bmatrix} 
    1 & 1 & 0 \\
    1 & 2 & 1\\
    0 & 1 & c
\end{bmatrix}. 
\] 
\item \textbf{Independencia lineal de $v_1,v_2,v_3$.} Puesto que $W$ es invertible  $v_1,v_2,v_3$ serán linealmente independientes si y solo si la matriz $C$ es invertible; es decir,  si y sólo si $\mathrm{det}(C)\neq 0$.
\item \textbf{Cálculo de $\mathrm{det}(C)$.} Calculamos el determinante de la matriz $C$:  \[
\mathrm{det}(C)=\begin{vmatrix} 
    1 & 1 & 0\\
    1 & 2 & 1\\
    0 & 1 & c
\end{vmatrix}=(2c+0+0)-(0+c+1)=2c-c-1=c-1 
\]  
\item \textbf{Condición de independencia.} Para que $C$ sea invertible, necesitamos  $\mathrm{det}(C)\neq 0$. Dado que \[
\mathrm{det}(C)=c-1,
\] se requiere \[
c-1\neq 0\longrightarrow c\neq 1.
\] 
\end{enumerate}
    \item \lb{Dadas las matrices \[
    A=\begin{bmatrix} 
        1 & 1 & -1\\
        2 & 4 & 2\\
        0 & 1 & 1\\
        3 & 2 & 1
    \end{bmatrix}\qquad B=\begin{bmatrix} 
        2 & 4 & 2\\
        3 & 2 & 1\\
        1 & 1 & -1\\
        0 & 1 & 1
    \end{bmatrix}  
    \]Halla una matriz de permutación $P$ tal que $PA=B$ y escribe  $P$ como producto de matrices de permutación simples.}

    Para encontrar una matriz de permutación $P$ tal que  $PA=B$, procedemos de la siguiente manera: 

     {
\renewcommand{\arraystretch}{1}
\setlength{\arraycolsep}{6pt}
$\begin{aligned}
      \left[ \begin{array}{ccc:cccc}
			1 & 1 & -1 & 1 & 0 & 0 & 0\\
			2 & 4 & 2 & 0 & 1 & 0 & 0\\
			0 & 1 & 1 & 0 & 0 & 1 & 0\\
			3 & 2 & 1 & 0 & 0 & 0 & 1
      \end{array} \right]& \xrightarrow{F_1\leftrightarrow F_2}      \left[ \begin{array}{ccc:cccc}
2 & 4 & 2 & 0 & 1 & 0 & 0\\
1 & 1 & -1 & 1 & 0 & 0 & 0\\
0 & 1 & 1 & 0 & 0 & 1 & 0\\
3 & 2 & 1 & 0 & 0 & 0 & 1
      \end{array} \right]\xrightarrow{F_3\leftrightarrow F_4}\left[ \begin{array}{ccc:cccc}
2 & 4 & 2 & 0 & 1 & 0 & 0\\
1 & 1 & -1 & 1 & 0 & 0 & 0\\
3 & 2 & 1 & 0 & 0 & 0 & 1\\
0 & 1 & 1 & 0 & 0 & 1 & 0
      \end{array} \right]\\
&\xrightarrow{F_2\leftrightarrow F_3}\left[ \begin{array}{ccc:cccc}
2 & 4 & 2 & 0 & 1 & 0 & 0\\
3 & 2 & 1 & 0 & 0 & 0 & 1\\
1 & 1 & -1 & 1 & 0 & 0 & 0 \\
0 & 1 & 1 & 0 & 0 & 1 & 0\\
      \end{array} \right]\Longrightarrow P=\begin{bmatrix}
0 & 1 & 0 & 0\\
0 & 0 & 0 & 1\\
1 & 0 & 0 & 0\\
0 & 0 & 1 & 0
\end{bmatrix}
     \end{aligned}$}
    \item \lb{Se tiene una matriz $A=(a_{ij})$ de tamaño $5\times 5$, donde $a_{ij}$ en la cantidad de mensajes que la persona $i$ manda a la persona $j$. Las filas y columnas siguen el orden: Juan, Ana, Pedro, María, Maite. Halla una matriz de permutación $P$ tal que las columnas y filas de  $PAP^\intercal$ sigan el orden: Ana, María, Juan, Maite, Pedro.}

        Para encontrar la matriz de permutación $P$ tal que las filas y columnas de  $PAP^\intercal$ estén en el orden deseado, seguimos estos paso:
        \begin{enumerate}[label=\arabic*)]
            \item Entender el problema:

                La matriz $A$ tiene filas y columas y columnas ordendas como: \[
                \text{Orden original: }\{\mathrm{Juan},\mathrm{Ana},\mathrm{Pedro}, \text{María},\text{Maite}\} .
                \] 
                Queremos reorganizar filas y columnas para que queden en este orden:
                \begin{center}
                    Nuevo orden: \{Ana, María, Juan, Maite, Pedro\}.
                \end{center}
                La matriz de permutación $P$ realizará este cambio
            \item Definir $P$:

                La matriz de permutación $P$ es una matriz identidad $5\times 5$ con las filas reordenadas de acuerdo con el nuevo orden. Cada fila de $P$ indica la nueva posición de una fila de la matriz identidad:
                \begin{itemize}[label=\textbullet]
                    \item El orden original:
                        \begin{itemize}[label=\textbullet]
                            \item Ana $(i=2)$ pasa a la posición 1.
                            \item María $(i=4)$ pasa a la posición 2.
                            \item Juan $(i=1)$ pasa a la posició 3.
                            \item Maite $(i=5)$ pasa a la posición 4.
                            \item Pedro $(i=3)$ pasa a la posición 5.
                        \end{itemize}
                \end{itemize}
                Entonces, $P$ se construye intercambiando las filas de la identidad en consecuencia: \[
                P=\begin{bmatrix} 
                    0 & 1 & 0 & 0 & 0\\
                    0 & 0 & 0 & 1 & 0\\
                    1 & 0 & 0 & 0 & 0\\
                    0 & 0 & 0 & 0 & 1\\
                    0 & 0 & 1 & 0 & 0
                \end{bmatrix} .
                \] 
            \item Interpretación de $PAP^\intercal$:

                La operación $PAP^\intercal$ realiza dos pasos:
                \begin{enumerate}[label=\arabic*)]
                    \item \textbf{Reorganizar las filas de $A$ según $P$}, es decir, poner las filas de $A$ en el orden especificado.
                    \item  \textbf{Reorganizar las columnas de $A$} (mediante $P^\intercal$) en el mismo orden.
                \end{enumerate}
        \end{enumerate}
    \item \lb{Consideremos la matriz por bloques $\begin{bmatrix} 
                A & B 
    \end{bmatrix} $ con $A$ una matriz de tamaño  $n\times n$ y $B$ una matriz de tamaño  $n\times p$. Supongamos que haciendo operaciones elementales de filas se obtiene la matriz $\begin{bmatrix} 
                I_n & X 
    \end{bmatrix} $. Prueba que $X=A^{-1}B$.}

    Para probar que $X=A^{-1}B$, analizamos el problema paso a paso:
    \begin{enumerate}[label=\arabic*)]
        \item Definición del problema.

            La matriz por bloques es \[
            \begin{bmatrix} 
                A & B 
            \end{bmatrix}, 
            \] donde $A$ es una matriz cuadrada de tamaño  $n\times n$, y $B$ es una matriz de tamaño  $n\times p$.

            Por hipótesis, haciendo operaciones elementales de filas, se transforma en \[
            \begin{bmatrix} 
                I_n & X 
            \end{bmatrix}, 
            \] donde $I_n$ es la matriz identidad de tamaño  $n\times n$ y $X$ es una matriz de tamaño  $n\times p$.

            Queremos demostrar que $X=A^{-1}B$.
        \item Operaciones elementales y equivalencia de filas:

            Hacer operaciones elementales de filas sobre una matriz equivale a multiplicarla a la izquierda por una matriz invertible $P$. Esto significa que existe una matriz  $P$ de tamaño $n\times n$ tal que: \[
            P\begin{bmatrix} 
                A & B 
            \end{bmatrix} =\begin{bmatrix} 
                I_n & X 
            \end{bmatrix} .
            \] 
            Separando los bloques, esta ecuación se escribe como: \[
            P\begin{bmatrix} 
            A & B 
            \end{bmatrix} =\begin{bmatrix} 
            PA & PB 
            \end{bmatrix} .
            \] 
            De la igualdad con $\begin{bmatrix} 
                I_n & X 
            \end{bmatrix} $, concluimos que: \[
            PA=I_n\quad \text{y}\quad PB=X.
            \] 
            De la igualda $PA=I_n$, vemos que  $P$ es la inversa de  $A$:  \[
            P=A^{-1}.
            \] 
            Sustituyendo $P=A^{-1}$ en $PB=X$, obtenemos:  \[
            X=A^{-1}B.
            \] 
    \end{enumerate}
    \item \lb{Halla una relación de dependencia entre los vectores $u_1=(1,0,1,0),\,u_2=(2,1,0,1),\,u_3=(0,2,-1,1)$ y $u_4=(3,-1,2,0)$.}

        Para encontrar una relación de dependencia lineal entre los vectores $u_1,u_2,u_3$ y $u_4$, necesitamos determinar si existen coeficientes $c_1,c_2,c_3,c_4$, no todos cero, tales que: \[
        c_1u_1+c_2u_2+c_3u_3+c_4u_4=0,
        \] o equivalentemente: \[
        c_1(1,0,1,0)+c_2(2,1,0,1)+c_3(0,2-1,1)+c_4(3,-1,2,0)=(0,0,0,0).
        \] 
        Esto genera el sistema de ecuaciones lineales: \[
        \begin{cases}
            c_1+2c_2+3c_4=0\\
            c_2+2c_3-c_4=0\\
            c_1-c_3+2c_4=0\\
            c_2+c_3=0
        \end{cases}
        \] 
        \begin{enumerate}[label=\arabic*)]
            \item Formar la matriz del sistema:

                Escribimos este sistema como una matriz aumentada: 
                \[
                    \begin{bmatrix}
                    1 & 2 & 0 & 3 & 0\\
                    0 & 1 & 2 & -1 & 0\\
                    1 & 0 & -1 & 2 & 0\\
                    0 & 1 & 1 & 0 & 0
                \end{bmatrix}
                \] 
            \item Resolver por eliminación gaussiana:

$\begin{aligned}\begin{bmatrix}
                    1 & 2 & 0 & 3 & 0\\
                    0 & 1 & 2 & -1 & 0\\
                    1 & 0 & -1 & 2 & 0\\
                    0 & 1 & 1 & 0 & 0
                \end{bmatrix}&\xrightarrow{F_3\to F_3-F_1}\begin{bmatrix}
                    1 & 2 & 0 & 3 & 0\\
                    0 & 1 & 2 & -1 & 0\\
                    0 & -2 & -1 & -1 & 0\\
                    0 & 1 & 1 & 0 & 0
                \end{bmatrix}\xrightarrow[F_4\to F_4-F_1]{F_3\to F_3+2F_1}\begin{bmatrix}
                    1 & 2 & 0 & 3 & 0\\
                    0 & 1 & 2 & -1 & 0\\
                    0 & 0 & 3 & -3 & 0\\
                    0 & 0 & -1 & 1 & 0
                \end{bmatrix}\\
&\xrightarrow{F_3\leftrightarrow F_4}\begin{bmatrix}
                    1 & 2 & 0 & 3 & 0\\
                    0 & 1 & 2 & -1 & 0\\
                    0 & 0 & -1 & 1 & 0\\
                    0 & 0 & 3 & -3 & 0\\
                \end{bmatrix}\xrightarrow{F_4\to F_4+3F_3}\begin{bmatrix}
                    1 & 2 & 0 & 3 & 0\\
                    0 & 1 & 2 & -1 & 0\\
                    0 & 0 & -1 & 1 & 0\\
                    0 & 0 & 0 & 0 & 0\\
                \end{bmatrix}\end{aligned}$

Esta matriz reducida resultante muestra que la última fila es cero, indicando una relación de dependencia lineal entre los vectores $u_1,u_2,u_3,u_4$.
\item Relación de dependencia:

    De la matriz reducida, obtenemos las ecuaciones: 
    \begin{itemize}[label=\textbullet]
        \item $c_1+c_4=0\longrightarrow c_1=-c_4$
        \item $c_2+c_4=0\longrightarrow c_2=-c_4$
        \item $c_3-c_4=0\longrightarrow c_3=c_4$
    \end{itemize}
    Al sustituir en la combinación lineal, podemos escribir la relación de dependencia como: \[
    c_1u_1+c_2u_2+c_3u_3+c_4u_4=0\quad \text{con}\quad c_1=c_2=-c_4,c_3=c_4.
    \] 
        \end{enumerate}
    \item \lb{Sean $A$ y  $B$ matrices del mismo tamaño. Sea  $A'$ la matriz que resulta de  $A$ después de intercambiar las columnas  $i,j$ y sea  $B'$ la matriz que resulta de  $B$ después de intercambiar las filas  $i,j$. Escribe las matrices  $A'$ y  $B'$ en términos de  $A,B$ y matrices elementales. ¿Por qué se verifica que $AB=A'B'$?}

        \begin{enumerate}[label=Paso \arabic*:]
            \item Representar $A'$ y  $B'$ en términos de matrices elementales

                Supongamos que  $A$ y  $B$ son matrices de tamaño  $n\times n$. Queremos describir las matrices $A'$ y  $B'$ que resultan al intercambiar columnas y filas, respectivamente.
                 \begin{enumerate}[label=1.\arabic*)]
                    \item Matriz $A'$:

                        Para obtener  $A'$, intercambiamos las columnas  $i$ y $j$ de $A$. Esto se logra multiplicando $A$ por una matriz de permutación  $P_{ij}$  desde la derecha: \[
                        A'=AP_{ij},
                        \] 
                        donde $P_{ij}$ es una matriz identidad $n\times n$ con las columnas $i$ y $j$ intercambiadas.
                    \item Matriz $B'$:

                        Para obtener  $B'$, intercambiamos las filas  $i$ y $j$ de $B$. Esto se logra multiplicando  $B$ por una matriz de permutación  $P_{ij}$  desde la izquierda: \[
                        B'=P_{ij}B.
                        \] 
                \end{enumerate}
            \item Producto $AB$ en términos de  $A'$ y $B'$  

                El producto $AB$ se transforma en $A'B'$. Sustituyendo  $A'=AP_{ij}$ y $B'=P_{ij}B$, tenemos: \[
                A'B'=(AP_{ij})(P_{ij}B).
                \] 
                Por la propiedad asociativa del producto de matrices: \[
                    A'B'=A(P_{ij}P_{ij})B.
                \] 
                La clave es que $P_{ij}$ es una matriz de permutación, y el producto de una matriz de permutación consigo misma es la identidad: \[
                P_{ij}P_{ij}=I.
                \] 
                Por lo tanto: \[
                A'B'=AIB=AB.
                \] 
        \end{enumerate}
    \item \lb{Calcula el rango de la matriz \[
    B=\begin{bmatrix} 
        a & 0 & 0 & b\\
        b & a & 0 & 0\\
        0 & b & a & 0\\
        0 & 0 & b & a
    \end{bmatrix} 
    \]en función de los parámetros $a$ y  $b$.}

    \underline{Posibles casos:}
    \begin{enumerate}[label=\arabic*)]
        \item $a=b=0\longrightarrow B=[0]\longrightarrow \mathrm{rango}(B)=0$ 
        \item $a=0,b\neq 0\longrightarrow B=\begin{bmatrix} 
                0 & 0 & 0 & b\\
                b & 0 & 0 & 0\\
                0 & b & 0 & 0\\
                0 & 0 & b & 0
                \end{bmatrix}\xrightarrow{C_4\to C_1}\underset{\text{forma escalonada}}{\begin{bmatrix} 
                b & 0 & 0 & 0\\
                0 & b & 0 & 0\\
                0 & 0 & b & 0\\
                0 & 0 & 0 & b
        \end{bmatrix}}\longrightarrow \mathrm{rango}(B)=4  $ 
        \item $a\neq 0,b=0\longrightarrow B=\begin{bmatrix} 
                a & 0 & 0 & 0\\
                0 & a & 0 & 0\\
                0 & 0 & a & 0\\
                0 & 0 & 0 & a
        \end{bmatrix}\longrightarrow \mathrm{rango}(B)=4 $
    \end{enumerate}

    \item \lb{Dada la matriz $M=\begin{bmatrix} 
                2 & 0 & 2 & 6\\
                1 & 1 & 0 & 2\\
                3 & 2 & 1 & 7
    \end{bmatrix} $ calcula matrices invertibles $P$ y  $Q$ tales que  \[
    PMQ=\left[ \begin{array}{c|c}
            1_r & 0\\ \hline
            0 & 0
    \end{array} \right] 
    \] con $r$ el rango de $M$.}

$\begin{aligned}
\left[ 
    \begin{array}{cccc|ccc}
        2 & 0 & 2 & 6 & 1 & 0 & 0\\
        1 & 1 & 0 & 2 & 0 & 1 & 0\\
        3 & 2 & 1 & 7 & 0 & 0 & 1
    \end{array}
    \right]& \xrightarrow{F_1\leftrightarrow F_2}\left[ \begin{array}{cccc|ccc}
            1 & 1 & 0 & 2 & 0 & 1 & 0\\
            2 & 0 & 2 & 6 & 1 & 0 & 0\\
            3 & 2 & 1 & 7 & 0 & 0 & 1
    \end{array} \right] \xrightarrow[F_3\to F_3-3F_1]{F_2\to F_2-2F_1} \left[ \begin{array}{cccc|ccc}
            1 & 1 & 0 & 2 & 0 & 1 & 0\\
            0 & -2 & 2 & 2 & 1 & -2 & 0\\
            0 & -1 & 1 & 1 & 0 & -3 & 1\\
    \end{array} \right]\\
           & \xrightarrow{F_2\leftrightarrow F_3}\left[ \begin{array}{cccc|ccc}
            1 & 1 & 0 & 2 & 0 & 1 & 0\\
            0 & -1 & 1 & 1 & 0 & -3 & 1 \\
            0 & -2 & 2 & 2 & 1 & -2 & 0
    \end{array} \right] \xrightarrow{F_3\to F_3-2F_2}\left[ \begin{array}{cccc|ccc}
            1 & 1 & 0 & 2 & 0 & 1 & 0\\
            0 & -1 & 1 & 1 & 0 & -3 & 4\\
            0 & 0 & 0 & 0 & 1 & 4 & -2
    \end{array} \right]\\
           & \xrightarrow{F_2\to -F_2}\left[ \begin{array}{cccc|ccc}
            1 & 1 & 0 & 2 & 0 & 1 & 0\\
            0 & 1 & -1 & -1 & 0 & 3 & 1\\
            0 & 0 & 0 & 0 & 1 & 4 & -2
    \end{array} \right] \longrightarrow P=\begin{bmatrix} 
            0 & 1 & 0\\
            0 & 3 & -1\\
            1 & 4 & -2
    \end{bmatrix} 
\end{aligned}$

$\begin{aligned}\left[\begin{array}{cccc}
    1 & 1 & 0 & 2\\ 
    0 & 1 & -1 & -1\\
    0 & 0 & 0 & 0\\ \hline
    1 & 0 & 0 & 0\\
    0 & 1 & 0 & 0\\
    0 & 0 & 1 & 0\\
    0 & 0 & 0 & 1
\end{array}\right]&\xrightarrow[C_4\to C_4-2C_1]{C_2\to C_2-C_1}\left[ \begin{array}{cccc}
    1 & 0 & 0 & 0\\
    0 & 1 & -1 & -1\\
    0 & 0 & 0 & 0\\ \hline
    1 & -1 & 0 & -2\\
    0 & 1 & 0 & 0\\
    0 & 0 & 1 & 0\\
    0 & 0 & 0 & 1
\end{array} \right] \xrightarrow[C_4\to C_4-C_2]{C_{3}\to C_3-C_2}\left[ \begin{array}{cccc}
    1 & 0 & 0 & 0\\
    0 & 1 & 0 & 0\\
    0 & 0 & 0 & 0\\ \hline
    1 & -1 & 1 & -1\\
    0 & 1 & -1 & -1\\
    0 & 0 & 1 & 0\\
    0 & 0 & 0 & 1
\end{array} \right]\\ 
& \longrightarrow Q=\begin{bmatrix} 
    1 & -1 & 1 & -1\\
    0 & 1 & -1 & -1\\
    0 & 0 & 1 & 0\\
    0 & 0 & 0 & 1
    \end{bmatrix}\end{aligned}$

Finalmente tenemos una matriz $PMQ=\begin{bmatrix} 
    1 & 0 & 0 & 0\\
    0 & 1 & 0 & 0\\
    0 & 0 & 0 & 0
    \end{bmatrix}$ que tiene la forma \[
\left[ \begin{array}{c|c}
        I_n & 0\\ \hline
        0 & 0
\end{array} \right] 
\] donde $I_r$ es la matriz identidad con  $r=\mathrm{rango}(M)=2$.

    \item \lb{Halla la inversa de la matriz $A=\begin{bmatrix} 
                1 & 1 & -3\\
                3 & 4 & -2\\
                -1 & -1 & 2
    \end{bmatrix} $ y expresa $A$ y  $A^{-1}$ como producto de matrices elementales.} 

    Vamos a calcular la inversa de la matriz $A$ usando el método de Gauss-Jordan, que implica reducir $A$ a la matriz identidad $I$ y aplicar las mismas operaciones elementales a $I$. Luego expresaremos $A$ y $A^{-1}$ como producto de matrices elementales.

$\begin{aligned}\left[ \begin{array}{ccc|ccc}
        1 & 1 & -3 & 1 & 0 & 0\\
        3 & 4 & -2 & 0 & 1 & 0\\
        -1 & -1 & 2 & 0 & 0 & 1
\end{array} \right]&\xrightarrow[F_3\to F_3+F_1]{F_2\to F_2-3F_1}\left[ \begin{array}{ccc|ccc}
        1 & 1 & -3 & 1 & 0 & 0\\
        0 & 1 & 7 & -3 & 1 & 0\\
        0 & 0 & -1 & 1 & 0 & 1
\end{array} \right]\xrightarrow[F_3\to F_3+F_2]{F_1\to F_1-F_2} \left[ \begin{array}{ccc|ccc}
    1 & 0 & -10 & 4 & -1 & 0\\
    0 & 1 & 7 & -3 & 1 & 0\\
    0 & 0 & -1 & 1 & 1 & 1
\end{array}\right]\\ 
& \xrightarrow{F_3\to -F_3} \left[ \begin{array}{ccc|ccc}
    1 & 0 & -10 & 4 & -1 & 0\\
    0 & 1 & 7 & -3 & 1 & 0\\
    0 & 0 & 1 & -1 & -1 & -1
\end{array}\right]\xrightarrow[F_2\to F_2-7F_3]{F1\to F1+10F_3}\left[ \begin{array}{ccc|ccc}
    1 & 0 & 0 & -6 & 9 & 10\\
    0 & 1 & 0 & 4 & -6 & -7\\
    0 & 0 & 1 & -1 & -1 & -1
\end{array}\right]\\ 
& \longrightarrow A^{-1}=\begin{bmatrix} 
    -6 & 9 & 10\\
    4 & -6 & -7\\
    -1 & -1 & -1
\end{bmatrix}. \end{aligned}$
\item \lb{Determina el valor del parámetro $a$ para el cual la matriz $A=\begin{bmatrix} 
            1 & 0 & 0 & 0\\
            a & 1 & 0 & 0\\
            a^2 & a & 1 & 0\\
            a^3 & a^2 & a & 1
\end{bmatrix} $ es invertible y calcula su inversa.} 

El determinante de $A$ se puede calcular expandiendo por cofactores, pero notamos que $A$ tiene una estructura triangular. En matrices triangulares, el determinante es igual al producto de los elementos de la diagonal principal.

Por lo tanto: \[
\mathrm{det}(A)=1\cdot 1\cdot 1\cdot 1=1.
\] 
El determinante de $A$ es siempre 1, independientemente del valor de $a$. Esto implica que la matriz  $A$ \textbf{siempre es invertible}, para cualquier valor de $a$.

La inversa de una matriz triangular inferior como $A$ también es triangular inferior, y sus elementos se calculan directamente utilizando las relaciones entre las filas y columnas.

\textbf{Construcción de $A^{-1}$}

La inversa de $A$ tiene la forma:  \[
A^{-1}=\begin{bmatrix} 
    1 & 0 & 0 & 0\\
    -a & 1 & 0 & 0\\
    a^2 & -a & 1 & 0\\
    -a^3 & a^2 & -a & 1
\end{bmatrix} .
\] 
Para verificar que la matriz calculada es la inversa, multiplicamos $A$ y $A^{-1}$, y comprobamos que el resultado es la matriz identidad $I_4$. \[
A\cdot A^{-1}=\begin{bmatrix} 
    1 & 0 & 0 & 0\\
    a & 1 & 0 & 0\\
    a^2 & a & 1 & 0\\
    a^3 & a^2 & a & 1
\end{bmatrix} \cdot \begin{bmatrix} 
    1 & 0 & 0 & 0\\
    -a & 1 & 0 & 0\\
    a^2 & -a & 1 & 0\\
    -a^3 & a^2 & -a & 1
\end{bmatrix}=\begin{bmatrix} 
    1 & 0 & 0 & 0 \\
    0 & 1 & 0 & 0\\
    0 & 0 & 1 & 0\\
    0 & 0 & 0 & 1
\end{bmatrix} . 
\] 

\item \lb{Resuelve el siguiente sistema de ecuaciones lineales \[
\left.\begin{array}{r}
    x+2y-z-2t=5\\
    -2x-4y+2z+4t=-10\\
    y+t=1\\
    x+3y-z-t=6\\
    x-z-4t=3
\end{array}\right\}
\] } 

Para resolver el sistema de ecuaciones, usaremos el método de eliminación gaussiana. Reescribimos el sistema en forma de matriz aumentada: \[
    \begin{aligned}
\left[ \begin{array}{cccc|c}
        1 & 2 & -1 & -2 & 5\\
        -2 & -4 & 2 & 4 & -10\\
        0 & 1 & 0 & 1 & 1 \\
        1 & 3 & -1 & -1 & 6\\
        1 & 0 & -1 & -4 & 3
\end{array} \right] & \xrightarrow[\begin{subarray}{l}
    F_4\to F_4-F_1\\
    F_5\to F_5-F_1
\end{subarray}]{F_2\to F_2+2F_1} \left[\begin{array}{cccc|c} 
1 & 2 & -1 & -2 & 5\\
0 & 0 & 0 & 0 & 0\\
0 & 1 & 0 & 1 & 1\\
0 & 1 & 0 & 1 & 1\\
0 & -2 & 0 & -2 & -2
\end{array}\right] \xrightarrow[F_5\to F_5+2F_3]{F_4\to F_4-F_3}\left[ \begin{array}{cccc|c}
1 & 2 & -1 & -2 & 5\\
0 & 0 & 0 & 0 & 0\\
0 & 1 & 0 & 1 & 1\\
0 & 0 & 0 & 0 & 0\\
0 & 0 & 0 & 0 & 0\\
\end{array} \right] .
    \end{aligned}
\] 
El sistema ahora se simplifica como:
\begin{enumerate}[label=\arabic*)]
    \item $x+2y-z-2t=5$.
    \item $y+t=1$.
\end{enumerate}
De la segunda ecuación: \[
t=1-y.
\] 
Sustituyendo $t=1-y$ en la primera ecuación:  \[
\begin{aligned}
    x+2y-z-2(1-y)&= 5 \\
    x+2y-z-2+2y&= 5 \\
    x+2y-z&=5
\end{aligned}
\] 
Ahora tenemos: \[
\begin{cases}
    x+4y-z=7\\
    t=1-y
\end{cases}
\] 
De las dos ecuaciones restante, el sistema tiene \textbf{dos grados de libertad}. Sea $y=s$ y $z=r$ donde  $s,r \in\R$. Entonces:
\begin{enumerate}[label=\arabic*)]
    \item De $x+2y-z=7$:  \[
    x=7-4s+r.
    \] 
\item De $t=1-y$:  \[
t=1-s.
\] 
El sistema tiene solución paramétrica: \[
x=7-4s+r, \quad y=s, \quad z=r, \quad t=1-s.
\] donde $s,r \in \R$ son parámetros libres.
\end{enumerate}
\item \lb{Discute en el cuerpo de los números reales los siguientes sistemas de ecuaciones en función del parámetro $a$  \[
\left\{ \begin{array}{rcrcrcrcc}
        x & + & ay & & & + & at & = & a\\
        ax & + & y & + & z & + & t & = & a\\
        x & + & y & + & az & + & t & = & 1
\end{array} \right. 
\] } 

Podemos escribir el sistema de ecuaciones lineales en forma matricial como \[
    \underbrace{\begin{bmatrix} 
            1 & a & 0 & a\\
            a & 1 & 1 & 1\\
            1 & 1 & a & 1
    \end{bmatrix} }_M\cdot \underbrace{\begin{bmatrix} 
    x\\
    y\\
    z\\
    t
    \end{bmatrix} }_{\mathbf{v}}=\underbrace{\begin{bmatrix} 
    a\\
    a\\
    1
    \end{bmatrix} }_{\mathbf{b}}.
\] 
Observemos que hay 4 incógnitas pero solo 3 ecuaciones, de modo que, si el sistema es compatible, normalemente esperamos soluciones con (al menos) un grado de libertado. Sin embargo, dependiendo del valor de $a$, podría ocurrir que el sistema:
 \begin{enumerate}[label=\arabic*)]
    \item Sea \textbf{incompatible} (no tenga soluciones).
    \item Sea \textbf{compatible} con una familia infinita de soluciones de dimensión 1.
    \item Sea \textbf{compatible}  pero con una familia de soluciones de mayor dimensión (si se reduce el rango de la matriz de coeficientes).
\end{enumerate}
A continuación discutiremos cada caso.

\begin{enumerate}[label=\arabic*), leftmargin=*]
    \item Reducción y búsqueda de condiciones de compatibilidad

        Partiendo de las ecuaciones:
        \begin{enumerate}[label=\arabic*)]
            \item $x+ay+at=a$.
            \item $ax+y+z+t=a$.
            \item $x+y+az+t=1$.
        \end{enumerate}
        \begin{enumerate}[label=1.\arabic*), leftmargin=*]
            \item De la primera ecuación, \[
            x=a-ay-at.
            \] 
        \item Sustituir e las otras dos
            \begin{itemize}[label=\textbullet]
                \item En la segunda: \[
                ax+y+z+t=a\quad\longrightarrow \quad a(a-ay-at)+y+z+t=a.
                \] 
                Desarrollando, \[
                a^2-a^2y-a^2t+y+z+y=a\longrightarrow z=a-a^2-(1-a^2)(y+t).
                \] 
            \item En la tercera: \[
            x+y+az+t=1\longrightarrow (a-ay-at)+y+az+t=1\longrightarrow a+(1-a)(y-t)+az=1\longrightarrow  az=1-a-(1-a)(y+t),
            \] 
            y si $a\neq 0$, \[
            z=\dfrac{1-a-(1-a)(y+t)}{a}.
            \] 
            \end{itemize}
        \item De las segunda y tercera ecuación obtenemos la condición de compatibilidad: \[
        a-a^2-(1-a^2)(y+t)=\dfrac{1-a-(1-a)(y+t)}{a}\quad \text{asumiendo $a\neq 0$.}
        \] 
        Multiplicando ambos lados por $a$ y reordenando se llega a la siguiente ecuación en $S:= y+t$:  \[
            (a^2-a^3)+(a^3-a)S=(1-a)[1-S]\longrightarrow a^2-a^3+a^3S-aS=1-a-S+aS\longrightarrow\underbrace{(a^2-a^3+a-1)}_{\text{término constante}}+\underbrace{(a^3-2a+1)S}_{\text{término en $S$}} =0.
        \] 
        \begin{itemize}[label=\textbullet]
            \item Se factoriza \[
            a^2-a^3+a-1=(a-1)(1-a^2)=(a-1)(1-a)(1+a).
            \] 
        \item Asimismo, \[
        a^3-2a+1=(a-1)(a^2+a-1).
        \] 
        \end{itemize}
        Por tanto, la ecuación completa se factoriza como \[
            (a-1)\left[ (1-a)(1+a) \right] +S\left[ (a-1)(a^2+a-1) \right] =0,
        \] o \[
        (a-1)\left[ (1-a)(1+a)+S(a^2+a-1) \right] =0.
        \] 
        De aquí surgen \textbf{dos grandes ramas}:
        \begin{enumerate}[label=\arabic*)]
            \item $a-1=0\longrightarrow a=1$.
            \item $(1-a)(1+a)+S(a^2+a-1)=0$.

                Si $a\neq 1$, esto equivale a \[
                1-a^2+S(a^2+a-1)=0\longrightarrow S=\dfrac{a^2-1}{a^2+a-1},
                \] 
                siempre que $a^2+a-1\neq 0$.
        \end{enumerate}
        De modo que, \textbf{salvo casos especiales}, se fija \[
        S=y+t=\dfrac{a^2-1}{a^2+a-1}.
        \]  
        \end{enumerate}
    \item Discusión según el valor de $a$

        A partir de lo anterior, distinguimos:
         \begin{enumerate}[label=2.\arabic*), leftmargin=*]
            \item Caso $a=1$

                Si  $a=1$, las ecuaciones se convierten en:
                 \begin{enumerate}[label=\arabic*.]
                    \item $x+y+t=1$.
                    \item  $x+y+z+t=1$.
                    \item  $x+y+z+t=1$.
                \end{enumerate}
                Obsérvese que las ecuaciones 2 y 3 son \textbf{idénticas}. Por tanto, el sistema realmente queda en: 
                \begin{itemize}[label=\textbullet]
                    \item $x+y+t=1$.
                    \item  $x+y+z+t=1$.
                \end{itemize}
                Restando la primera a la segunda, se ve que $z=0$.

                Entonces la primera ecuación imponer  $x+y+t=1$. Aquí hay 3 incógnitas  $(x,y,t)$ y solo una ecuación, de modo que la dimensión del conjunto de soluciones es  $3-1=2$. En otras palabras,  \textbf{existen infinitas soluciones} (un plano de soluciones en $\R^3$), que además se extiene a la cuarta variable $z=0$ fijo.

                Por tanto,
                 \begin{center}
                     \fbox{Para $a=1$, el sistema es compatible con infinitas soluciones (dimensión 2).}
                \end{center}
            \item Caso $a^2+a-1=0$

                La ecuación $a^2+a-1=0$ tiene dos raíces reales: \[
                a=\dfrac{-1\pm\sqrt{1+4} }{2}=\dfrac{-1\pm\sqrt{5} }{2}=\begin{cases}
                    a_+=\frac{-1+\sqrt{5} }{2} \\
                    a_-=\frac{-1+\sqrt{5} }{2} 
                \end{cases}.
                \] 

                En este caso, la fórmula para $s=\dfrac{a^2-1}{a^2+a-1}$ no es aplicable porque el denominador es cero. Además, si uno examina la factorización final, el término que quedaba era $$(1-a)(1+a)+S(a^2+a-1),$$pero si $a^2+a-1=0$, entonces $(1-a)(1+a)=1-a^2$ se vuelve \[
                1-a^2=1-(1-a)=a\quad(\text{pues }a^2=1-a).
                \] 
                De modo que la condición de cimplatibilidad sería $a+0\cdot S=0$, esto es $a=0$, que  \textbf{no}  coincide con $a_{\pm}$. Por tanto, \textbf{no hay manera de que el sistema sea consistente} en esos valores. 
                \begin{center}
                    \fbox{Para $a=\dfrac{-1\pm\sqrt{5} }{2}$, el sistema es incompatible (no tiene soluciones).}
                \end{center}
            \item Caso restante: $a\neq 1$ y $a^2+a-1\neq 0$

                En este caso, no salta ninguna inconsistencia. Para que el sistema sea compatible basta imponer \[
                y+t=S=\dfrac{a^2-1}{a^2+a-1}.
                \] 
                Es decir, $y+t$ viene  \textbf{fijado} por esa fórmula en función de $a$. Una vez impuesto  $y+t=S$, podemos:
                 \begin{itemize}[label=\textbullet]
                    \item Escoger libremente (por ejemplo) a $y$ como parámetro libre $\lambda$.
                    \item Determinar $t=S-y=S-\lambda$.
                    \item Con ello, de la primera ecuación se halla $x$, y de la segunda/tercera se determina  $z$.
                \end{itemize}
                Como hay 4 incógnitas y finalmente 3 ecuaciones \textbf{independientes}, el espacio de soluciones (cuando es no vacío) tiene dimensión $4-3=1$. Por tanto: 
                 \begin{center}
                     \fbox{Para $a\neq 1$ y $a^2+a-1\neq 0$, existe una familia infinita (1-paramétrica) de soluciones.}
                \end{center}
                Obsérvese que dentro de este caso "restante" también se incluye $a=0$. De hecho, si  $a=0$, la condición  $y+t=1$ (que se verifica fácilmente sustituyendo en las ecuaciones) produce soluciones infinitas de dimensión 1.
        \end{enumerate}
\end{enumerate}
\item \lb{Si $A=[u_1,\dots,u_n]$, expresa el determinante de $B=[u_n u_1\cdots u_{n-1}]$ en función del determinante de $A$.}

    Supongamos que $A=[u_1,u_2,\dots,u_n]$ es una matriz $n\times n$ cuyas columnas son los vectores $u_1,u_2,\dots,u_n$. La matriz \[
        B=[u_n,u_1,u_2,\dots,u_{n-1}]
    \] 
    se obtiene a partir de $A$ al realizar una permutación cíclica de columnas: \[
        (u_1,u_2,\dots,u_{n-1},u_n)\longmapsto (u_n,u_1,u_2,\dots,u_{n-1}).
    \]
    Esta permutación de columnas corresponde al ciclo \[
    1\mapsto 2,\quad 2\mapsto 3,\quad\dots,\quad(n-1)\mapsto n,\quad n\mapsto 1,
    \] 
    que es un ciclo de longitud $n$. Recordemos que el signo de un ciclo de longitud $n,\, \sigma=(-1)^{n-1}$.
    
    Dado que $\mathrm{det}(A)$ cambia por el factor $\sigma$ al permutar sus columnas, obtenemos: \[
    \mathrm{det}(B)=(-1)^{n-1}\mathrm{det}(A).
    \] 
    En conclusión, \[
        \bboxed{\mathrm{det}([u_n,u_1,\dots,u_{n-1}])=(-1)^{n-1}\mathrm{det}([u_1,\dots,u_n]).}
    \] 
\item \lb{Una matriz $A$ que cumple  $A=-A^\intercal$ se llama \textbf{antisimétrica}. Prueba que una matriz antisimétrica e tamaño impar tiene determinante nulo. }

    \textbf{Idea de la demostración}

    La clave está en la relación entre el determinante de una matriz y el determinante de su traspuesta, junto con el hecho de multiplicar una matriz por $-1$.
     \begin{enumerate}[label=\arabic*)]
        \item Sabemos que $\mathrm{det}(A)=\mathrm{det}(A^\intercal)$.
        \item Dado que $A=-A^\intercal$, se tendrá $A^\intercal=-A$.
        \item Por propiedades del determinante, $\mathrm{det}(-A)=(-1)^n\mathrm{det}(A)$.
    \end{enumerate}
    Juntando estas ideas, al ser $n$ impar, concluiremos que $\mathrm{det}(A)=0$.

    \textbf{Demostración detallada}

    \begin{enumerate}[label=\arabic*)]
        \item \textbf{Relación entre $\mathrm{det}(A)$ y $\mathrm{det}(A^\intercal)$.}

            Para cualquier matriz cuadrada $A$ de orden  $n$, se cumple \[
            \mathrm{det}(A)=\mathrm{det}(A^\intercal).
            \] 
            Esto es una propiedad estándar del determinante.
        \item \textbf{Relación entre $A$ y $A^\intercal$ para $A$ antisimétrica.}

            Por hipótesis, $A$ es antisimétrica, es decir, \[
            A=-A^\intercal\longrightarrow A^\intercal=-A.
            \] 
        \item \textbf{Determinante de $-A$.}

            Si multiplicamos una matriz $A$ de orden $n$ por $-1$, el nuevo determinante $\mathrm{det}(-A)$ equivale a \[
            \mathrm{det}(-A)=(-1)^n\mathrm{det}(A).
            \] 
        \item \textbf{Caso $n$ impar.}

            Suponiendo que $n$ es \textit{impar}, se tiene $(-1)^n=-1$. Por tanto,  \[
            \mathrm{det}(A)=(-1)^n\mathrm{det}(A)=-\mathrm{det}(A).
            \]  
            Lo cual implica \[
            \mathrm{det}(A)=-\mathrm{det}(A)\longrightarrow 2\mathrm{det}(A)=0\longrightarrow \mathrm{det}(A)=0.
            \] 
    \end{enumerate}
    En consecuencia, \textbf{toda matriz antisimétrica de orden impar es singular}, es decir, tiene determinante nulo. 
\item \lb{Calcula el determinante \[
\begin{vmatrix} 
    1 & 2 & 3 & \cdots & n\\
    2 & 3 & 4 & \cdots & n+1\\
    \vdots & \vdots & \vdots & & \vdots\\
    n & n+1 & n+2 & \cdots & 2n-1
\end{vmatrix} 
\] }
\begin{enumerate}[label=\arabic*)]
    \item Casos pequeños: $n=1$ y  $n=2$
         \begin{itemize}[label=\textbullet]
            \item Caso $n=1$:
                \[
                \mathrm{det}(M)=1.
                \] 
            \item Caso $n=2$:  \[
            \mathrm{det}(M)=\begin{vmatrix} 
                1 & 2\\
                2 & 3
            \end{vmatrix} =3-4=-1
            \] 
        \end{itemize}
    \item Caso general $n\ge 3$

        Veremos que para $n\ge 3$ el determinante es $0$. Para ello, haremos operaciones elementales sobre las filas que no cambian o solo cambian el determinante por un factor distitnto de cero. En particular, restar una fila de otra \textit{no altera} el valor del determinante.
        \begin{enumerate}[label=\arabic*)]
            \item Restar la primera fila a las demás

                Para $k=2,\dots,n$, hacemos \[
                R_k\longleftarrow R_k-R_1.
                \] 
                \begin{itemize}[label=\textbullet]
                    \item La primera fila $R_1$ se queda igual: \[
                    R_1=(1,2,3,\dots,n).
                    \] 
                \item La fila $R_k$ original era  $(k,k+1,k+2,\dots,k+(n-1))$.

                    Al restarle $R_1$, obtenemos \[
                    R_k-R_1=(k-1,(k+1)-2, (k+2)-3,\dots,(k+n-1)-n).
                    \] 
                    Observando cada componente: \[
                        (k+j-1)-(1+j-1)=(k-1).
                    \] 
                    En efecto, para cada columna $j$, el resultado es  \textbf{constante} e igual a $k-1$. 
                \end{itemize}
                Por tanto, después de esas $n-1$ restas, la matriz queda:  \[
                M'=\begin{vmatrix} 
                    1 & 2 & 3 & \cdots & n\\
                    1 & 1 & 1 & \cdots & 1\\
                    2 & 2 & 2 & \cdots & 2\\
                    \vdots & \vdots & \vdots & & \vdots\\
                    n-1 & n-1 & n-1 & \cdots & n-1
                \end{vmatrix}. 
                \] 
            \item Examinar las filas resultantes

                A partir de la segunda fila, todas están dentro del mismo subespacio de dimensión 1. Esto implica que la matriz $M'$  \textbf{tiene rango menor que $n$} tan pronto como $n\ge 3$. Si el rango es menor que $n$, el determinante debe ser cero.

                Más concretamente, podemos incuso anular una de esas filas con una combinación adecuada. Por ejemplo, si restamos 2 veces la fila 2 a la fila 3, obtenemos una fila completamente nula: \[
                R_3\longleftarrow R_3-2R_2=(2-2\cdot 1,2-2\cdot 1,\dots,2-2\cdot 1)=(0,0,\dots,0).
                \] 
                En cuanto aparece una fila de ceros, el determinante es $0$.
        \end{enumerate}
        Por lo tanto, para $n\ge 3$ el determinante se anula.
\end{enumerate}

\item \lb{
\setlength{\arraycolsep}{1pt}
Considera el sistema de ecuaciones \[
\left.\begin{array}{rl}
2x+y+z&=0\\
4x-6y-2x&=2\\
-2x+15y+7z&=-4
\end{array}\right\}
\]
Observa que podemos eliminar la última ecuación pues es combinación lineal de las dos primeras. Considerando ahora los términos en $z$ como si fuesen términos independientes, observa que el sistema es un sistema de Cramer en  $x,y$.  Resuélvelo con la fórmula de Cramer y después expresa las soluciones en la forma $x_0+u$ ($x_0$ solución parcial y $u$ solución genérica del sistema homogéneo).} 

Se observa que la tercera ecuación es combinación lineal de las dos primeras $(3\cdot (1)-2\cdot (2)=(3))$, por lo que puede eliminarse. Quedamos con el sistema reducido a las dos primeras ecuaciones: \[
\begin{cases}
    2x+y+z=0\\
    4x-4y-2z=2
\end{cases}
\] 
Consideremos los términos en $z$ como si fuesen términos independientes. Esto nos permite escribir el sistema como un sistema de Cramer en  $x$ y $y$, donde los coeficientes de $z$ se tratan como términos constantes.

Podemos expresar el sistema en forma matricial como: \[
\begin{pmatrix} 
    2 & 1\\
    4 & -6
\end{pmatrix} \begin{pmatrix} 
x\\
y
\end{pmatrix} =\begin{pmatrix} 
-z\\
2+2z
\end{pmatrix} .
\] 
El determinante de la matriz de coeficientes es: \[
\nabla =\begin{vmatrix} 
    2 & 1\\
    4 & -6
\end{vmatrix} = -12-4=-16
\] 
Dado que $\nabla \neq 0$, el sistema tiene una única solución en $x$ e $y$ para cualquier valor de $z$.

Usando la fórmula de Cramer:  \[
\begin{array}{c}
    x=\dfrac{\begin{vmatrix} 
            -z & 1\\
            2+2z & -6
    \end{vmatrix} }{-16}=\dfrac{6z-2-2z}{-16}=\dfrac{4z-2}{-16}=\dfrac{1}{8}-\dfrac{z}{4}.\\
    y=\dfrac{\begin{vmatrix} 
            2 & -z\\
            4 & 2+2z
    \end{vmatrix} }{-16}=\dfrac{4+4z+4z}{-16}=\dfrac{4+8z}{-16}=-\dfrac{1}{4}-\dfrac{z}{2}.
\end{array}
\] 
Escribimos las soluciones en la forma $ x_0+u$.
\begin{enumerate}[label=\arabic*)]
    \item Sistema homogéneo:

        Tomemos $z=0$ en el sistema, resolviendo: \[
        \begin{cases}
            2x+y=0\\
           4x-6y=0 
        \end{cases}
        \] 
        De la primera, $y=-2x$. Sustituyendo en la segunda:  \[
        4x-6(-2x)=0\longrightarrow 4x+12x=0\longrightarrow x=0,y=0.
        \] 
        Entonces, la solución homogénea general es: \[
            (x,y,z)=(0,0,t),\quad t\in \R.
        \] 
    \item Solución general:

        Sumamos la solución particular $x_0=\left( \dfrac{1}{8},-\dfrac{1}{4},0 \right) $ y la solución homogénea: \[
            (x,y,z)=\left( \dfrac{1}{8},-\dfrac{1}{4},0 \right) +t(0,0,1),
        \]
        donde $t=z$ es el parámetro libre.
\end{enumerate}
\end{enumerate}
\end{document}
