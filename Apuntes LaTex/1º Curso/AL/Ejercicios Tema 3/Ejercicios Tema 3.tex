\documentclass[12pt]{article}
\usepackage{fullpage}
\usepackage[utf8]{inputenc}
\usepackage{pict2e}
\usepackage{amsmath}
\usepackage{enumitem}
\usepackage{eurosym}
\usepackage{pict2e}
\usepackage{mathtools}
\usepackage{amssymb, amsfonts, latexsym, cancel}
\setlength{\parskip}{0.3cm}
\usepackage{graphicx}
\usepackage{fontenc}
\usepackage{slashbox}
\usepackage{setspace}
\usepackage{gensymb}
\usepackage{accents}
\usepackage{adjustbox}
\setstretch{1.5}
\usepackage{bold-extra}
\usepackage[document]{ragged2e}
\usepackage{subcaption}
\usepackage{tcolorbox}
\usepackage{xcolor, colortbl}
\usepackage{wrapfig}
\usepackage{empheq}
\usepackage{array}
\usepackage{parskip}
\usepackage{arydshln}
\graphicspath{ {images/} }
\renewcommand*\contentsname{\color{black}Índice} 
\usepackage{array, multirow, multicol}
\definecolor{lightblue}{HTML}{007AFF}
\usepackage{color}
\usepackage{etoolbox}
\usepackage{listings}
\usepackage{mdframed}
\setlength{\parindent}{0pt}
\usepackage{underscore}
\usepackage{hyperref}
\usepackage{tikz}
\usepackage{tikz-cd}
\usetikzlibrary{shapes, positioning, patterns}
\usepackage{tikz-qtree}
\usepackage{biblatex}
\usepackage{pdfpages}
\usepackage{pgfplots}
\usepackage{pgfkeys}
\addbibresource{biblatex-examples.bib}
\usepackage[a4paper, left=1.5cm, right=1.5cm, top=1cm,
bottom=1.5cm]{geometry}
\everymath{\displaystyle}
\usetikzlibrary{decorations.pathreplacing}
\usepackage{titlesec}
\usepackage{titletoc}
\usepackage{tikz-3dplot}
\usetikzlibrary{decorations.pathreplacing}
\newcommand{\Ej}{\textcolor{lightblue}{\underline{Ejemplo}}}
\setlength{\fboxrule}{1.5pt}
\renewcommand{\arraystretch}{1.35}
\setlength{\arraycolsep}{0.3cm}

% Configura el formato de las secciones utilizando titlesec
\titleformat{\section}
{\color{red}\normalfont\LARGE\bfseries}
{Tema \thesection:}
{10 pt}
{}

% Ajusta el formato de las entradas de la tabla de contenidos
\addtocontents{toc}{\protect\setcounter{tocdepth}{4}}
\addtocontents{toc}{\color{black}}

\titleformat{\subsection}
{\normalfont\Large\bfseries\color{red}}{\thesubsection)}{1em}{\color{lightblue}}

\titleformat{\subsubsection}
{\normalfont\large\bfseries\color{red}}{\thesubsubsection)}{1em}{\color{lightblue}}

\newcommand{\bboxed}[1]{\fcolorbox{lightblue}{lightblue!10}{$#1$}}

\DeclareMathOperator{\N}{\mathbb{N}}
\DeclareMathOperator{\Z}{\mathbb{Z}}
\DeclareMathOperator{\R}{\mathbb{R}}
\DeclareMathOperator{\Q}{\mathbb{Q}}
\DeclareMathOperator{\K}{\mathbb{K}}
\DeclareMathOperator{\im}{\imath}
\DeclareMathOperator{\jm}{\jmath}
\DeclareMathOperator{\col}{\mathrm{Col}}
\DeclareMathOperator{\fil}{\mathrm{Fil}}
\DeclareMathOperator{\rg}{\mathrm{rg}}
\DeclareMathOperator{\nuc}{\mathrm{nuc}}
\DeclareMathOperator{\dimf}{\mathrm{dimFil}}
\DeclareMathOperator{\dimc}{\mathrm{dimCol}}
\DeclareMathOperator{\dimn}{\mathrm{dimnuc}}
\DeclareMathOperator{\dimr}{\mathrm{dimrg}}

\newcommand{\bu}[1]{\textcolor{lightblue}{\underline{#1}}}
\newcommand{\lb}[1]{\textcolor{lightblue}{#1}}
\newcommand{\db}[1]{\textcolor{blue}{#1}}
\newcommand{\rc}[1]{\textcolor{red}{#1}}
\newcommand{\tr}{^\intercal}

\renewcommand{\CancelColor}{\color{lightblue}}

\newcommand{\dx}{\:\mathrm{d}x}
\newcommand{\dt}{\:\mathrm{d}t}
\newcommand{\dy}{\:\mathrm{d}y}
\newcommand{\dz}{\:\mathrm{d}z}
\newcommand{\dth}{\:\mathrm{d}\theta}
\newcommand{\dr}{\:\mathrm{d}\rho}
\newcommand{\du}{\:\mathrm{d}u}
\newcommand{\dv}{\:\mathrm{d}v}
\newcommand{\tozero}[1]{\cancelto{0}{#1}}
\newcommand{\lbb}[2]{\textcolor{lightblue}{\underbracket[1pt]{\textcolor{black}{#1}}_{#2}}}
\newcommand{\dbb}[2]{\textcolor{blue}{\underbracket[1pt]{\textcolor{black}{#1}}_{#2}}}
\title{Álgebra Lineal\\ Ejercicios Tema 3: Sistemas de ecuaciones y determinantes}
\renewcommand{\arraystretch}{1}
\setlength{\arraycolsep}{6pt}

\begin{document}
\maketitle
\begin{enumerate}[label=\color{red}\textbf{\arabic*)}]
    \item \lb{Sea $\{w_1,w_2,w_3\} $ un conjunto independiente de vectores de $\R^3$. Se definen los vectores $v_1=w_1+w_2,v_2=w_1+2w_2+w_3$ y $v_3=w_2+cw_3$. Si $V=[v_1,v_2,v_3]$ y $W=[w_1,w_2,w_3]$, entonces se tiene $V=WC$ con  \[
    C=\begin{bmatrix} 
        1 & 1 & 0\\
        1 & 2 & 1\\
        0 & 1 & c
    \end{bmatrix} 
    \]¿Qué condición debe cumplir $c$ para que los vectores  $v_1,v_2,v_3$ sean linealmente independientes? } 

    Para determinar la condición que debe cumplir $c$ para que los vectores  $v_1,v_2,v_3$ sean linealmente indpendientes, observa que:
\begin{enumerate}[label=\arabic*)]
\item \textbf{Los vectores $\{w_1,w_2,w_3\} $} son linealmente independientes en $\R^3$. Esto implica que la matriz \[
        W=[w_1,w_2,w_3]
    \] es invertible  (tiene determinante distinto de cero).
\item \textbf{Los vectores $\{v_1,v_2,v_3\} $} se pueden expresar como \[
        V=[v_1,v_2,v_3]=WC,
\] donde \[
C=\begin{bmatrix} 
    1 & 1 & 0 \\
    1 & 2 & 1\\
    0 & 1 & c
\end{bmatrix}. 
\] 
\item \textbf{Independencia lineal de $v_1,v_2,v_3$.} Puesto que $W$ es invertible  $v_1,v_2,v_3$ serán linealmente independientes si y solo si la matriz $C$ es invertible; es decir,  si y sólo si $\mathrm{det}(C)\neq 0$.
\item \textbf{Cálculo de $\mathrm{det}(C)$.} Calculamos el determinante de la matriz $C$:  \[
\mathrm{det}(C)=\begin{vmatrix} 
    1 & 1 & 0\\
    1 & 2 & 1\\
    0 & 1 & c
\end{vmatrix}=(2c+0+0)-(0+c+1)=2c-c-1=c-1 
\]  
\item \textbf{Condición de independencia.} Para que $C$ sea invertible, necesitamos  $\mathrm{det}(C)\neq 0$. Dado que \[
\mathrm{det}(C)=c-1,
\] se requiere \[
c-1\neq 0\longrightarrow c\neq 1.
\] 
\end{enumerate}
    \item \lb{Dadas las matrices \[
    A=\begin{bmatrix} 
        1 & 1 & -1\\
        2 & 4 & 2\\
        0 & 1 & 1\\
        3 & 2 & 1
    \end{bmatrix}\qquad B=\begin{bmatrix} 
        2 & 4 & 2\\
        3 & 2 & 1\\
        1 & 1 & -1\\
        0 & 1 & 1
    \end{bmatrix}  
    \]Halla una matriz de permutación $P$ tal que $PA=B$ y escribe  $P$ como producto de matrices de permutación simples.}

    Para encontrar una matriz de permutación $P$ tal que  $PA=B$, procedemos de la siguiente manera: 

     {
\renewcommand{\arraystretch}{1}
\setlength{\arraycolsep}{6pt}
$\begin{aligned}
      \left[ \begin{array}{ccc:cccc}
			1 & 1 & -1 & 1 & 0 & 0 & 0\\
			2 & 4 & 2 & 0 & 1 & 0 & 0\\
			0 & 1 & 1 & 0 & 0 & 1 & 0\\
			3 & 2 & 1 & 0 & 0 & 0 & 1
      \end{array} \right]& \xrightarrow{F_1\leftrightarrow F_2}      \left[ \begin{array}{ccc:cccc}
2 & 4 & 2 & 0 & 1 & 0 & 0\\
1 & 1 & -1 & 1 & 0 & 0 & 0\\
0 & 1 & 1 & 0 & 0 & 1 & 0\\
3 & 2 & 1 & 0 & 0 & 0 & 1
      \end{array} \right]\xrightarrow{F_3\leftrightarrow F_4}\left[ \begin{array}{ccc:cccc}
2 & 4 & 2 & 0 & 1 & 0 & 0\\
1 & 1 & -1 & 1 & 0 & 0 & 0\\
3 & 2 & 1 & 0 & 0 & 0 & 1\\
0 & 1 & 1 & 0 & 0 & 1 & 0
      \end{array} \right]\\
&\xrightarrow{F_2\leftrightarrow F_3}\left[ \begin{array}{ccc:cccc}
2 & 4 & 2 & 0 & 1 & 0 & 0\\
3 & 2 & 1 & 0 & 0 & 0 & 1\\
1 & 1 & -1 & 1 & 0 & 0 & 0 \\
0 & 1 & 1 & 0 & 0 & 1 & 0\\
      \end{array} \right]\Longrightarrow P=\begin{bmatrix}
0 & 1 & 0 & 0\\
0 & 0 & 0 & 1\\
1 & 0 & 0 & 0\\
0 & 0 & 1 & 0
\end{bmatrix}
     \end{aligned}$}
    \item \lb{Se tiene una matriz $A=(a_{ij})$ de tamaño $5\times 5$, donde $a_{ij}$ en la cantidad de mensajes que la persona $i$ manda a la persona $j$. Las filas y columnas siguen el orden: Juan, Ana, Pedro, María, Maite. Halla una matriz de permutación $P$ tal que las columnas y filas de  $PAP^\intercal$ sigan el orden: Ana, María, Juan, Maite, Pedro.}

        Para encontrar la matriz de permutación $P$ tal que las filas y columnas de  $PAP^\intercal$ estén en el orden deseado, seguimos estos paso:
        \begin{enumerate}[label=\arabic*)]
            \item Entender el problema:

                La matriz $A$ tiene filas y columas y columnas ordendas como: \[
                \text{Orden original: }\{\mathrm{Juan},\mathrm{Ana},\mathrm{Pedro}, \text{María},\text{Maite}\} .
                \] 
                Queremos reorganizar filas y columnas para que queden en este orden:
                \begin{center}
                    Nuevo orden: \{Ana, María, Juan, Maite, Pedro\}.
                \end{center}
                La matriz de permutación $P$ realizará este cambio
            \item Definir $P$:

                La matriz de permutación $P$ es una matriz identidad $5\times 5$ con las filas reordenadas de acuerdo con el nuevo orden. Cada fila de $P$ indica la nueva posición de una fila de la matriz identidad:
                \begin{itemize}[label=\textbullet]
                    \item El orden original:
                        \begin{itemize}[label=\textbullet]
                            \item Ana $(i=2)$ pasa a la posición 1.
                            \item María $(i=4)$ pasa a la posición 2.
                            \item Juan $(i=1)$ pasa a la posició 3.
                            \item Maite $(i=5)$ pasa a la posición 4.
                            \item Pedro $(i=3)$ pasa a la posición 5.
                        \end{itemize}
                \end{itemize}
                Entonces, $P$ se construye intercambiando las filas de la identidad en consecuencia: \[
                P=\begin{bmatrix} 
                    0 & 1 & 0 & 0 & 0\\
                    0 & 0 & 0 & 1 & 0\\
                    1 & 0 & 0 & 0 & 0\\
                    0 & 0 & 0 & 0 & 1\\
                    0 & 0 & 1 & 0 & 0
                \end{bmatrix} .
                \] 
            \item Interpretación de $PAP^\intercal$:

                La operación $PAP^\intercal$ realiza dos pasos:
                \begin{enumerate}[label=\arabic*)]
                    \item \textbf{Reorganizar las filas de $A$ según $P$}, es decir, poner las filas de $A$ en el orden especificado.
                    \item  \textbf{Reorganizar las columnas de $A$} (mediante $P^\intercal$) en el mismo orden.
                \end{enumerate}
        \end{enumerate}
    \item \lb{Consideremos la matriz por bloques $\begin{bmatrix} 
                A & B 
    \end{bmatrix} $ con $A$ una matriz de tamaño  $n\times n$ y $B$ una matriz de tamaño  $n\times p$. Supongamos que haciendo operaciones elementales de filas se obtiene la matriz $\begin{bmatrix} 
                I_n & X 
    \end{bmatrix} $. Prueba que $X=A^{-1}B$.}

    Para probar que $X=A^{-1}B$, analizamos el problema paso a paso:
    \begin{enumerate}[label=\arabic*)]
        \item Definición del problema.

            La matriz por bloques es \[
            \begin{bmatrix} 
                A & B 
            \end{bmatrix}, 
            \] donde $A$ es una matriz cuadrada de tamaño  $n\times n$, y $B$ es una matriz de tamaño  $n\times p$.

            Por hipótesis, haciendo operaciones elementales de filas, se transforma en \[
            \begin{bmatrix} 
                I_n & X 
            \end{bmatrix}, 
            \] donde $I_n$ es la matriz identidad de tamaño  $n\times n$ y $X$ es una matriz de tamaño  $n\times p$.

            Queremos demostrar que $X=A^{-1}B$.
        \item Operaciones elementales y equivalencia de filas:

            Hacer operaciones elementales de filas sobre una matriz equivale a multiplicarla a la izquierda por una matriz invertible $P$. Esto significa que existe una matriz  $P$ de tamaño $n\times n$ tal que: \[
            P\begin{bmatrix} 
                A & B 
            \end{bmatrix} =\begin{bmatrix} 
                I_n & X 
            \end{bmatrix} .
            \] 
            Separando los bloques, esta ecuación se escribe como: \[
            P\begin{bmatrix} 
            A & B 
            \end{bmatrix} =\begin{bmatrix} 
            PA & PB 
            \end{bmatrix} .
            \] 
            De la igualdad con $\begin{bmatrix} 
                I_n & X 
            \end{bmatrix} $, concluimos que: \[
            PA=I_n\quad \text{y}\quad PB=X.
            \] 
            De la igualda $PA=I_n$, vemos que  $P$ es la inversa de  $A$:  \[
            P=A^{-1}.
            \] 
            Sustituyendo $P=A^{-1}$ en $PB=X$, obtenemos:  \[
            X=A^{-1}B.
            \] 
    \end{enumerate}
    \item \lb{Halla una relación de dependencia entre los vectores $u_1=(1,0,1,0),\,u_2=(2,1,0,1),\,u_3=(0,2,-1,1)$ y $u_4=(3,-1,2,0)$.}

        Para encontrar una relación de dependencia lineal entre los vectores $u_1,u_2,u_3$ y $u_4$, necesitamos determinar si existen coeficientes $c_1,c_2,c_3,c_4$, no todos cero, tales que: \[
        c_1u_1+c_2u_2+c_3u_3+c_4u_4=0,
        \] o equivalentemente: \[
        c_1(1,0,1,0)+c_2(2,1,0,1)+c_3(0,2-1,1)+c_4(3,-1,2,0)=(0,0,0,0).
        \] 
        Esto genera el sistema de ecuaciones lineales: \[
        \begin{cases}
            c_1+2c_2+3c_4=0\\
            c_2+2c_3-c_4=0\\
            c_1-c_3+2c_4=0\\
            c_2+c_3=0
        \end{cases}
        \] 
        \begin{enumerate}[label=\arabic*)]
            \item Formar la matriz del sistema:

                Escribimos este sistema como una matriz aumentada: 
                \[
                    \begin{bmatrix}
                    1 & 2 & 0 & 3 & 0\\
                    0 & 1 & 2 & -1 & 0\\
                    1 & 0 & -1 & 2 & 0\\
                    0 & 1 & 1 & 0 & 0
                \end{bmatrix}
                \] 
            \item Resolver por eliminación gaussiana:

$\begin{aligned}\begin{bmatrix}
                    1 & 2 & 0 & 3 & 0\\
                    0 & 1 & 2 & -1 & 0\\
                    1 & 0 & -1 & 2 & 0\\
                    0 & 1 & 1 & 0 & 0
                \end{bmatrix}&\xrightarrow{F_3\to F_3-F_1}\begin{bmatrix}
                    1 & 2 & 0 & 3 & 0\\
                    0 & 1 & 2 & -1 & 0\\
                    0 & -2 & -1 & -1 & 0\\
                    0 & 1 & 1 & 0 & 0
                \end{bmatrix}\xrightarrow[F_4\to F_4-F_1]{F_3\to F_3+2F_1}\begin{bmatrix}
                    1 & 2 & 0 & 3 & 0\\
                    0 & 1 & 2 & -1 & 0\\
                    0 & 0 & 3 & -3 & 0\\
                    0 & 0 & -1 & 1 & 0
                \end{bmatrix}\\
&\xrightarrow{F_3\leftrightarrow F_4}\begin{bmatrix}
                    1 & 2 & 0 & 3 & 0\\
                    0 & 1 & 2 & -1 & 0\\
                    0 & 0 & -1 & 1 & 0\\
                    0 & 0 & 3 & -3 & 0\\
                \end{bmatrix}\xrightarrow{F_4\to F_4+3F_3}\begin{bmatrix}
                    1 & 2 & 0 & 3 & 0\\
                    0 & 1 & 2 & -1 & 0\\
                    0 & 0 & -1 & 1 & 0\\
                    0 & 0 & 0 & 0 & 0\\
                \end{bmatrix}\end{aligned}$

Esta matriz reducida resultante muestra que la última fila es cero, indicando una relación de dependencia lineal entre los vectores $u_1,u_2,u_3,u_4$.
\item Relación de dependencia:

    De la matriz reducida, obtenemos las ecuaciones: 
    \begin{itemize}[label=\textbullet]
        \item $c_1+c_4=0\longrightarrow c_1=-c_4$
        \item $c_2+c_4=0\longrightarrow c_2=-c_4$
        \item $c_3-c_4=0\longrightarrow c_3=c_4$
    \end{itemize}
    Al sustituir en la combinación lineal, podemos escribir la relación de dependencia como: \[
    c_1u_1+c_2u_2+c_3u_3+c_4u_4=0\quad \text{con}\quad c_1=c_2=-c_4,c_3=c_4.
    \] 
        \end{enumerate}
    \item \lb{Sean $A$ y  $B$ matrices del mismo tamaño. Sea  $A'$ la matriz que resulta de  $A$ después de intercambiar las columnas  $i,j$ y sea  $B'$ la matriz que resulta de  $B$ después de intercambiar las filas  $i,j$. Escribe las matrices  $A'$ y  $B'$ en términos de  $A,B$ y matrices elementales. ¿Por qué se verifica que $AB=A'B'$?}

        \begin{enumerate}[label=Paso \arabic*:]
            \item Representar $A'$ y  $B'$ en términos de matrices elementales

                Supongamos que  $A$ y  $B$ son matrices de tamaño  $n\times n$. Queremos describir las matrices $A'$ y  $B'$ que resultan al intercambiar columnas y filas, respectivamente.
                 \begin{enumerate}[label=1.\arabic*)]
                    \item Matriz $A'$:

                        Para obtener  $A'$, intercambiamos las columnas  $i$ y $j$ de $A$. Esto se logra multiplicando $A$ por una matriz de permutación  $P_{ij}$  desde la derecha: \[
                        A'=AP_{ij},
                        \] 
                        donde $P_{ij}$ es una matriz identidad $n\times n$ con las columnas $i$ y $j$ intercambiadas.
                    \item Matriz $B'$:

                        Para obtener  $B'$, intercambiamos las filas  $i$ y $j$ de $B$. Esto se logra multiplicando  $B$ por una matriz de permutación  $P_{ij}$  desde la izquierda: \[
                        B'=P_{ij}B.
                        \] 
                \end{enumerate}
            \item Producto $AB$ en términos de  $A'$ y $B'$  

                El producto $AB$ se transforma en $A'B'$. Sustituyendo  $A'=AP_{ij}$ y $B'=P_{ij}B$, tenemos: \[
                A'B'=(AP_{ij})(P_{ij}B).
                \] 
                Por la propiedad asociativa del producto de matrices: \[
                    A'B'=A(P_{ij}P_{ij})B.
                \] 
                La clave es que $P_{ij}$ es una matriz de permutación, y el producto de una matriz de permutación consigo misma es la identidad: \[
                P_{ij}P_{ij}=I.
                \] 
                Por lo tanto: \[
                A'B'=AIB=AB.
                \] 
        \end{enumerate}
    \item \lb{Calcula el rango de la matriz \[
    B=\begin{bmatrix} 
        a & 0 & 0 & b\\
        b & a & 0 & 0\\
        0 & b & a & 0\\
        0 & 0 & b & a
    \end{bmatrix} 
    \]en función de los parámetros $a$ y  $b$.}
    \begin{enumerate}[label=Paso \arabic*:]
        \item Propiedades básicas

            La matriz $B$ es una matriz $4\times 4$. El rango es actual al número de filas (o columnas) linealmente independientes. Si el determinante de $B$ es no nulo, entonces el rango es 4. Si el determinante es nulo, reduciremos  $B$ para determinar su rango.
        \item Determinante de $B$
\[
\mathrm{det}(B)=a\begin{vmatrix}
a & 0 & 0\\
b & a & 0\\
0 & b & a
\end{vmatrix}-b\begin{vmatrix}
b & a & 0\\
0 & b & a\\
0 & 0 & b
\end{vmatrix}=a^4-3a^2b^2+b^4.
\] 

\item Análisis de $\mathrm{det}(B)$ 
    \begin{enumerate}[label=\arabic*)]
        \item Si $a^4+b^4\neq 0$, entonces $\mathrm{det}(B)\neq 0$, y el rango de $B$ es 4.
        \item Si  $a^4+b^4=0$, entonces $a^4=-b^4$. Esto ocurre si
    \end{enumerate}
    \end{enumerate}

    \item \lb{Dada la matriz $M=\begin{bmatrix} 
                2 & 0 & 2 & 6\\
                1 & 1 & 0 & 2\\
                3 & 2 & 1 & 7
    \end{bmatrix} $ calcula matrices invertibles $P$ y  $Q$ tales que  \[
    PMQ=\left[ \begin{array}{c|c}
            1_r & 0\\ \hline
            0 & 0
    \end{array} \right] 
    \] con $r$ el rango de $M$.}

    \item \lb{Halla la inversa de la matriz $A=\begin{bmatrix} 
                1 & 1 & -3\\
                3 & 4 & -2\\
                -1 & -1 & 2
    \end{bmatrix} $ y expresa $A$ y  $A^{-1}$ como producto de matrices elementales.} 


\item \lb{Determina el valor del parámetro $a$ para el cual la matriz $A=\begin{bmatrix} 
            1 & 0 & 0 & 0\\
            a & 1 & 0 & 0\\
            a^2 & a & 1 & 0\\
            a^3 & a^2 & a & 1
\end{bmatrix} $ es invertible y calcula su inversa} 

\item \lb{Resuelve el siguiente sistema de ecuaciones lineales \[
\left.\begin{array}{r}
    x+2y-z-2t=5\\
    -2x-4y+2z+4t=-10\\
    y+t=1\\
    x+3y-z-t=6\\
    x-z-4t=3
\end{array}\right\}
\] } 

\item \lb{Discute en el cuerpo de los números reales los siguientes sistemas de ecuaciones en función del parámetro $a$  \[
\left\{ \begin{array}{rcrcrcrcc}
        x & + & ay & & & + & at & = & a\\
        ax & + & y & + & z & + & t & = & a\\
        x & + & y & az & + & t & = & 1
\end{array} \right. 
\] } 

\item \lb{Si $A=[u_1,\dots,u_n]$, expresa el determinante de $B=[u_n u_1\cdots u_{n-1}]$ en función del determinante de $A$.}

\item \lb{Una matriz $A$ que cumple  $A=-A^\intercal$ se llama \textbf{antisimétrica}. Prueba que una matriz antisimétrica e tamaño impar tiene determinante nulo. }

\item \lb{Calcula el determinante \[
\begin{vmatrix} 
    1 & 2 & 3 & \cdots & n\\
    2 & 3 & 4 & \cdots & n+1\\
    \vdots & \vdots & \vdots & & \vdots\\
    n & n+1 & n+2 & \cdots & 2n-1
\end{vmatrix} 
\] } 

\item \lb{
\setlength{\arraycolsep}{1pt}
Considera el sistema de ecuaciones \[
\left.\begin{array}{rl}
2x+y+z&=0\\
4x-6y-2x&=2\\
-2x+15y+7z&=-4
\end{array}\right\}
\]
Observa que podemos eliminar la última ecuación pues es combinación lineal de las dos primeras. Considerando ahora los términos en $z$ como si fuesen términos independientes, observa que el sistema es un sistema de Cramer en  $x,y$.  Resuélvelo con la fórmula de Cramer y después expresa las soluciones en la forma $x_0+u$ ($x_0$ solución parcial y $u$ solución genérica del sistema homogéneo).} 
\end{enumerate}
\end{document}
