\section{Recuperación de datos y formatos de serialización}
\subsection{Necesidad de formatos de serialización}
\begin{itemize}
\item Los formatos de \lb{serialización}
\end{itemize}
\subsection{Características a analizar}
\begin{itemize}[leftmargin=*]
	\item \lb{¿Es un protocolo estándar?-} Con ello nos referimos a si está avalado por  cuerpo de estándares o su especificación 
\end{itemize}
\begin{itemize}[label=\color{red}\textbullet, leftmargin=*]
	\item \color{lightblue}Tabla resumen
\end{itemize}
\begin{tabular}{cccccccc}
\hline
 & Estándar & Bin? & Humano? & Ref? & IDL? & Ext? & API? \\
\hline
Apache Avro & Sí & Sí  & No & N/A & Sí (acoplado) & Sí & N/A \\
\hline
CSV & Parcial (RFC4180) & No & Sí & No & No & Parcial  & No \\
\hline
JSON & Sí (RFC7159) & No (BSON) & Sí & Sí (RFC6901) & Parcial  & Sí & No \\
\hline
Thrift & No & Sí & Parcial  & No & Sí (acoplado) &  &  \\
\hline
XML &  &  &  &  &  &  &  \\
\hline
\end{tabular}
\subsubsection{Pandas}
\begin{itemize}
\item La herramienta más conocida y usada de Pandas son los \texttt{DataFrames}.
\item Permite almacenar datos tabulares en dos dimensiones similar a una hoja de cálculo o una base de datos relacional.
\item Las columnas de datos 
\end{itemize}
\begin{lstlisting}[language=Python]
df = pd.DataFrame(rand)
\end{lstlisting}
\subsubsection{XML (eXtensible Markup Language)}
\begin{itemize}
\item Meta-lenguaje de etiquetas derivado de SGML.
\item Motivación:
\begin{itemize}
\item Intercambio de datos en Internet
\end{itemize}
\item Reúne los requisitos de un lenguaje de intercambio de información:
\begin{itemize}
\item Simple: al estar basado en etiquetas y legible
\item Independiente de la plataforma: codificación UNICODE
\item Estándar y amplia difusión: W3C
\item Definición de estructuras complejas: DTD, Schemas
\item Validación y transformación: DTD, XSLT
\item Integración con otros sistemas
\end{itemize}
\item Facilita procesamiento lado cliente.
\item \textbf{No muy utilizado para BigData. Ha perdido tracción, es demasiado complejo finalmente y es muy poco eficiente en cual al porcentaje datos/metadatos}
\end{itemize}
\Ej
\begin{lstlisting}[language=XML, inputencoding=utf8, literate={Ñ}{{\~N}}1]
<?xml version="1.0" encoding="ISO-8859-1" ?>
<DISCO CODIGO="B000067FSG">
	<TITULO>Estrella de Mar</TITULO>
	<ARTISTA>Amaral</ARTISTA>
	<ESTILO>Pop</ESTILO>
	<REFERENCIA>
		<EDITORA>Virgin</EDITORA>
		<AÑO_EDICION>2002<AÑO_EDICION>
	</REFERENCIA>
	<MUSICOS>
		<MUSICO ROL="cantante">Amaral</MUSICO>
		<MUSICO ROL="guitarra">Juan Aguirre</MUSICO>
	</MUSICOS>
</DISCO>
\end{lstlisting}
\begin{itemize}
\item Instrucciones de procesamiento (línea 1)
\item Raíz (línea 2)
\item Etiquetas y atributos
\end{itemize}
\begin{itemize}[label=\color{red}\textbullet, leftmargin=*]
	\item \color{lightblue}DTD
\end{itemize}
\begin{lstlisting}[language=XML, inputencoding=utf8, literate={Ñ}{{\~N}}1]
<!ELEMENT DISCO (TITULO, ARTISTA, ESTILO?, REFERENCIA, MUSICOS)>
<!ATTLIST DISCO CODIGO ID #REQUIERED>
<!ATTLIST DISCO TIPO=(CD | LP | DVD) "CD">
<!ELEMENT TITULO (#PCDATA)>
...
<!ELEMENT REFERENCIA (EDITORA, AÑO_EDICION) >
<!ELEMENT MUSICOS (MUSICO*)>
...
\end{lstlisting}

Describe los documentos XML $\Longrightarrow$ \lb{Validación}
\begin{itemize}[label=\color{red}\textbullet, leftmargin=*]
	\item \color{lightblue}DTD (ii)
\end{itemize}
Documento XML:
\begin{itemize}
\item \textbf{Válido:} sigue la estructura de un DTD
\begin{lstlisting}[language=XML, inputencoding=utf8, literate={Ñ}{{\~N}}1]
<!DOCTYPE web-app
	PUBLIC "-//Sun Microsystems, Inc.//DTD Web Application 2.2//EN"
	"http://java.sun.com/j2ee/dtds/web-app_2_2.dtd">
\end{lstlisting}
\item \textbf{Bien formado:} Sigue las reglas de XML
\end{itemize}
Limitaciones:
\begin{itemize}
\item No es XML
\item Tipado limitado de datos
\item No soporta espacios de nombres
\end{itemize}

\begin{itemize}[label=\color{red}\textbullet, leftmargin=*]
	\item \color{lightblue}API SAX
\end{itemize}

Librería \texttt{\lb{xml.sax}}
\begin{lstlisting}[language=python]
import xml.sax

class XMLHandler(xml.sax.ContentHandler):
    def __init__(self):
        # Inicializamos variables de interés

    # Se llama cuando comienza un nuevo elemento
    def StartElement(self, tag, attributes):
        pass
    # Se llama cuando un elemento acaba
    def endElement(self, tag):
        pass

parser = xml.sax.make_parser()
parser.setFeature(xml.sax.handler.feature_namespaces, 0)
Handler = XMLHandler()
parser.setContentHandler(Handler)
parser.parse('models.xml') # nombre del documento a analizar
\end{lstlisting}
Cómo podríamos procesar con \texttt{\lb{xml.sax}} este documento
\begin{lstlisting}[language=XML]
<collection shelf="New Arrivals">
    <model number="ST001">
        <price>35000</price>
        <qty>12</qty>
        <company>Samsung</company>
    </model>
    <model number="RW345">
        <price>46500</price>
        <qty>14</qty>
        <company>Onida</company>
    </model>
</collection>
\end{lstlisting}
\begin{itemize}[label=\color{red}\textbullet, leftmargin=*]
	\item \color{lightblue}Parser DOM
\end{itemize}
Librería \lb{\texttt{xml.dom}}

\begin{minipage}{0.55\textwidth}
\begin{lstlisting}[language=python]
# Procesar un determinado fichero
file = minidom.parse('model.xml')

# Obtener los elementos con un determinado tag
modelos = fil.getElementosByTagName('modelo')

# Obtener el atributo 'nombre' del segundo modelo
print('modelo #2 atributos:')
print(modelos[1].attributes['nombre'].value)

# El datos de un item específico
print('\nmodelo #2 datos:')
print(modelos[1].firstChild.data)
\end{lstlisting}
\end{minipage}\qquad\begin{minipage}{0.4\textwidth}
\begin{lstlisting}[language=XML]
<data>
    <modelos>
        <modelo name='modelo1'>
            modelo1abc
        </modelo>
        <modelo name="modelo2">
            modelo2abc
        </modelo>
    </modelos>
</data>
\end{lstlisting}
\end{minipage}

\begin{itemize}[label=\color{red}\textbullet, leftmargin=*]
	\item \color{lightblue}Librería \texttt{BeatifulSoup}
\end{itemize}
\begin{minipage}{0.55\textwidth}
\begin{lstlisting}[language=python]
from bs4 import BeatifulSoup

# Leemos el fichero
with open('models.xml', 'r') as f:
    data = f.read()

# Pasamos los datos al parse
bs_data = BeatifulSoup(data, 'xml')

# Buscamos todas las instancias 'unique'
b_unique = bs_data.find_all('unique')
print(b_unique)

# Usamos .find búsquedas más concretas
b_name = bs_data.find('child', {'name':'Acer'})
print(b_name)
\end{lstlisting}
\end{minipage}\qquad\begin{minipage}{0.4\textwidth}
\begin{lstlisting}[language=XML]
<modelo>
    <child name="Acer" qty="12">
        Portátil Acer
    </child>
    <unique>
        Número de modelo
    </unique>
    <child name="Acer" qty="7">
        Exclusive
    </child>
    <unique>
        1.200€
    </unique>
</modelo>
\end{lstlisting}
\end{minipage}
\subsubsection{CSV (Comma-Separated Values)}
\begin{itemize}
\item Formato basado en columnas normalmente separado por coma
\item Sin embargo, el formato admite variaciones:
\begin{itemize}
\item Con o sin cabecera con los nombres de las columnas
\item Separador de columans (comas, tabuladores, etc.)
\item Codificación de caracteres (UTF-8, latin-1, etc.)
\item Escapado de caracteres (por ejemplo, una comilla doble como dos comillas dobles(" ") ó como \textbackslash")
\item Comillas opcionales (sólo si hacen falta) o en todos los campos siempre
\end{itemize}
\end{itemize}
\begin{itemize}[label=\color{red}\textbullet, leftmargin=*]
	\item \color{lightblue}Carga en SQL
\end{itemize}
\begin{lstlisting}[language=SQL]
LOAD DATA [LOW_PRIORITY | CONCURRENT] [LOCAL] INFLINE 'file_name'
    [RELACE | IGNORE]
    INTO TABLE tbl_name
    [PARCTITION (parctition_name, ...)]
    [CHARACTER SET charset_name]
    [{FIELDS | COLUMS}
        [TERMINATE BY 'string']
        [[OPTIONALL] ENCLOSED BY 'char']
        [ESCAPED BY 'char']
    ]
    [LINES
        [STARTING BY 'string']
        [TERMINATE BY 'string']
    ]
    [IGNORE number {LINES | ROWS}]
    [(col_name_or_user_var, ...)]
    [SET col_name = expr, ...]
\end{lstlisting}

\begin{lstlisting}[language=SQL]
LOAD DATA LOCAL INFILE "/tmp/Posts.csv"
    INTO TABLE Posts
    COLUMNAS TERMINATED BY ',' ENCLOSED BY '"' ESCAPED BY '"'
    LINES TERMINATED BY '\r\n'
    IGNORE 1 LINES;
\end{lstlisting}

\begin{itemize}[label=\color{red}\textbullet, leftmargin=*]
	\item \color{lightblue}Programación: Lectura con \texttt{Pandas DataFrame}
\end{itemize}