\section{Aprendizaje Supervisado}
\subsection{Árboles de Decisión}
\begin{itemize}[label=\color{red}\textbullet, leftmargin=*]
	\item \color{lightblue}Definición
\end{itemize}
Los árboles de decisión son máquinas de aprendizaje supervisado que sirven para clasificar o aproximar.

Supongamos el siguiente problema

\begin{center}
	\begin{tabular}{|c|c|c|c|c|c|}
		\hline
		\rowcolor{lightblue!10}
		\hline
		Paciente & Presión Arterial & Urea en sangre & Gota & Hipotiroidismo & Administrar Tratamiento \\
		\hline
		1 & Alta & Alta & Sí & No & No \\
		\hline
		2 & Alta & Alta & Sí & Sí & No \\
		\hline
		3 & Normal & Alta & Sí & No & Sí \\
		\hline
		4 & Baja & Normal & Sí & No & Sí \\
		\hline
		5 & Baja & Baja & No & No & Sí \\
		\hline
		6 & Baja & Baja & No & Sí & No \\
		\hline
		7 & Normal & Baja & No & Sí & Sí \\
		\hline
		8 & Alta & Normal & Sí & No & No \\
		\hline
		9 & Alta & Baja & No & No & Sí \\
		\hline
		10 & Baja & Normal & No & No & Sí \\
		\hline
		11 & Alta & Normal & No & Sí & Sí \\
		\hline
		12 & Normal & Normal & Sí & Sí & Sí \\
		\hline
		13 & Normal & Alta & No & No & Sí \\
		\hline
		14 & Baja & Normal & Si & Sí & No \\
		\hline
	\end{tabular}
\end{center}
\begin{itemize}
	\item Planteamiento del problema: ¿Cuál es la \textbf{mejor secuencia de preguntas} para saber la clase a la que pertenece un objeto descrito por sus atributos?
	\item Evidentemente, la "mejor respuesta" es aquella que con el \textbf{menor número de preguntas}, devuelve una respuesta suficientemente buena.
	\item ¿Qué es mejor preguntar primero si tiene gota o cómo tiene la presión arterial?
\end{itemize}
\subsubsubsection{Arquitectura}
Un árbol de decisión es una estructura jerárquica que consta de un nodo raíz, ramas, nodos internos y nodos hoja.
\begin{itemize}
	\item Comienzo con un \textbf{nodo raíz} sin ramas entrantes. Las ramas salientes del nodo raíz alimentan los nodos internos.
	\item Los \textbf{nodos internos} evalúan características disponibles para formar subconjuntos homogéneos, indicados por nodos hoja o nodos terminales.
	\item Los \textbf{nodos hoja} representan todos los resultados posibles dentro del conjunto de datos.
\end{itemize}
\begin{center}
	\includegraphics{"Temas/Tema 1/Screenshot002"}
\end{center}
\subsubsubsection{Ventajas y desventajas}
\begin{itemize}[label=\color{lightblue}\textbullet]
	\item Pros
	\begin{itemize}
		\item Fáciles de entender e interpretar.
		\item Sirven también para establecer reglas
		\item No lineales
		\item Menos pre-procesado de los datos: son robustos ante presencia de datos erróneos (outlier), valores faltantes o tipo de datos.
		\item Es un método no paramétrico (por ejemplo, no hay suposición acerca del espacio de distribución y la estructura del clasificador).
	\end{itemize}
	\item Contras
	\begin{itemize}
		\item \textbf{Sobreajuste:} Los árboles más pequeños son más fáciles de interpretar, pero los más grandes pueden resultar en sobreajuste.
		\item Perdida de información al categorizar variables continuas.
		\item \textbf{Precisión:} Otros métodos (por ejemplo, SVM) a menudo tienen tasas de error 30\% más bajas que los árboles básico (ID.3 y CART).
		\item \textbf{Inestabilidad:} un pequeño cambio en los datos puede modificar ampliamente la estructura del árbol (distintos conjuntos, distintos árboles). Varianza elevada.
	\end{itemize}
\end{itemize}
\begin{itemize}[label=\color{red}\textbullet, leftmargin=*]
	\item \color{lightblue}Definición alternativa: \textbf{recursividad}
\end{itemize}
Un árbol de decisión es una estructura recursiva formada por nodos, en el que existe:
\begin{itemize}
	\item Un nodo raíz
	\item El nodo raíz tiene uno o más subnodos.
	\item Cada uno de los subnodos puede ser, a su vez, raíz de un árbol
\end{itemize}
Esta característica recursiva hace que muchos de los algoritmos para crearlos se comporten también de manera recursiva.
\begin{center}
	\includegraphics{"Temas/Tema 1/Screenshot003"}
\end{center}
\subsubsubsection{Clasificación vs Regresión}
\begin{itemize}[label=\color{lightblue}\textbullet]
	\item Clasificación
	\begin{itemize}
		\item La variable dependiente es categórica.
		\item Los valores de los nodos hoja son la \textbf{moda} de las observaciones de la región
	\end{itemize}
	\begin{center}
		\includegraphics{"Temas/Tema 1/Screenshot004"}
	\end{center}
	\item Regresión
	\begin{itemize}
		\item La variable dependiente es continua.
		\item Los valores de los nodos hoja son la \textbf{media} de las observaciones de la región.
	\end{itemize}
	\begin{center}
		\includegraphics{"Temas/Tema 1/Screenshot005"}
	\end{center}
	
\end{itemize}
\subsubsection{Construcción de árboles de decisión}
\subsubsubsection{Particiones}
Cada nodo define una \textbf{partición} del conjunto de entrenamiento en función de los datos que representa.\\
Las particiones producen subconjuntos que son \textbf{exhaustivos} y \textbf{excluyentes}.\\
Cuestiones clave:
\begin{itemize}
	\item \textbf{Tipos de particiones:} cuantos más, más posibilidad de encontrar patrones y, por tanto, los árboles más precisos y expresivos.
	\item \textbf{Número de particiones:} A más particiones mayor complejidad. Equilibrio entre complejidad y precisión.
	\item Selección del \textbf{mejor atributo} en cada paso.
	\item Selección del \textbf{mejor valor} de umbral de los valores.
\end{itemize}
\subsubsubsection{Particiones posibles}
Los algoritmos más populares sólo proponen un tipo de partición para valores nominales y otro para valores numéricos:
\begin{itemize}
	\item \textbf{Particiones nominales:} En el caso que tengamos un atributo $x_i$ que tenga como posibles valores $\{v_1,v_2,\dots,v_n\}$ sólo es posible la partición \[ (x_1=v_1,x_2=v_2,\cdots,x_n=v_n) \]que da lugar a árboles con nodos con más de dos nodos hijos.
	\begin{center}
		\includegraphics{"Temas/Tema 1/Screenshot006"}
	\end{center}
	En el caso de árboles binarios se tienen que evaluar $n$ particiones (una por cada posible valor), definidas por $(x_i=v_i,x_i\neq v_i)$.
	\item \textbf{Particiones numéricas:} Si el atributo $x_i$ es numérico y continuo, se intenta definir particiones que separe las instancias en intervalos de la forma \begin{center}
		$(x_i\le a,x_i>a)$\qquad\begin{minipage}{0.3\textwidth}
			\includegraphics{"Temas/Tema 1/Screenshot007"}
		\end{minipage}
	\end{center}
	eligiendo diferentes valores de $a$ tenemos diferentes particiones. La expresividad resultante se conoce como \textit{expresividad cuadrangular} y que no relacionan atributos (sólo un atributo cada vez).
	\begin{center}
		\includegraphics{"Temas/Tema 1/Screenshot008"}
	\end{center}
	
\end{itemize}
\subsubsection{ID3: Algoritmo básico de aprendizaje}

\begin{quote}
	El algoritmo básico de aprendizaje es el \textbf{ID3 (Iterative Dichotomiser 3)}, J. Ross Quinlan, investigador australiano que propuso el método en 1983
\end{quote}
El método ID3 trata de encontrar una partición que asegure la \textbf{máxima capacidad predictiva y la máxima homogeneidad} de las clases\\
Medida de homogeneidad: la \textbf{entropía}\\
Repetición de \textbf{"cortes en dos"} hasta que se cumpla una determinada condición
\subsubsubsection{Entropía}
Para determinar el mejor atributo, el ID3 utiliza la \textbf{entropía}.

\begin{tikzpicture}
	\node[draw=lightblue, fill=lightblue!10, line width=1.5, text width=\linewidth] {Sea $S$ un conjunto de entrenamiento. Sea $p_\oplus$ la proporción de instancias positivas en $S$ y $p_\ominus$ la proporción de instancias negativas en $S$. La \textbf{entropía de $S$} es: \[ H(S)=p_\oplus\log_2\dfrac{1}{p_\oplus}+p_\ominus\log_2\dfrac{1}{p_\ominus}=-p_\oplus\log_2p_\oplus-p_\ominus\log_2p_\ominus \]};
\end{tikzpicture}

(Relación de la entropía con los conceptos de desorden, equiprobabilidad y homogeneidad).

La entropía nos mide la homogeneidad de los datos (relación inversa).
\begin{center}
	\includegraphics{"Temas/Tema 2/screenshot001"}
\end{center}
\begin{minipage}{0.45\textwidth}
	Para el caso binario:
	\begin{itemize}
		\item Entropía igual a 1 $\to$ mínima homogeneidad (equiprobabilidad: $p_\ominus=p_\oplus$).
		\item Entropía igual a 0 $\to$ máxima homogeneidad (todas las instancias de una clase)
	\end{itemize}
\end{minipage}\qquad\begin{minipage}{0.45\textwidth}
\begin{center}
	\includegraphics{"Temas/Tema 2/screenshot002"}
\end{center}
\end{minipage}
A la hora de construir un árbol es preferible crear nodos con nodos hoja homogéneos, es decir, de \textbf{baja entropía}.
\subsubsubsection{Ganancia de Información}
Sea un conjunto de datos $\chi$ con entropía $H(\chi)$.

Si elegimos un atributo $A$ para crear un nodo del árbol, la entropía esperada es: \[ H(\chi,A)=\sum_{v\in\mathrm{valores}(A)}\dfrac{|\chi_v|}{\chi}H(\chi_v) \]siendo $\chi_v$ el subconjunto de $\chi$ con todas las instancias con $A=v$.

Por lo tanto, la reducción esperada de la entropía, o lo que es lo mismo la \textbf{Ganancia de Información}, al elegir el atributo $A$ como nodo de decisión del árbol es \[ \bboxed{\mathrm{Ganacia}(\chi,A))H(\chi)-H(\chi,A)} \]
Por tanto, se elige el atributo que produzca hojas homogéneas, es decir, la \textbf{máxima} ganancia de información.

\Ej

Atributos nominales (no numéricos)

\begin{center}
	\begin{tabular}{|cccccc|}
		\hline
		\rowcolor[HTML]{A5FFC4} 
		\hline
		Día & Cielo & Temperatura & Humedad & Viento & \cellcolor[HTML]{FF8787}Jugar \\ \hline
		\rowcolor[HTML]{34CDF9} 
		\hline
		{\color[HTML]{333333} D1} & {\color[HTML]{333333} Soleado} & {\color[HTML]{333333} Calor} & {\color[HTML]{333333} Alta} & {\color[HTML]{333333} Flojo} & \cellcolor[HTML]{FF8787}No \\
		\rowcolor[HTML]{007AFF} 
		D2 & Soleado & Calor & Alta & Fuerte & \cellcolor[HTML]{FC5D5D}No \\
		\rowcolor[HTML]{34CDF9} 
		D3 & Nublado & Calor & Alta & Flojo & \cellcolor[HTML]{FF8787}Si \\
		\rowcolor[HTML]{007AFF} 
		D4 & Lluvia & Templado & Alta & Flojo & \cellcolor[HTML]{FC5D5D}Si \\
		\rowcolor[HTML]{34CDF9} 
		D5 & Lluvia & Frío & Normal & Flojo & \cellcolor[HTML]{FF8787}Si \\
		\rowcolor[HTML]{007AFF} 
		D6 & Lluvia & Ario & Normal & Fuerte & \cellcolor[HTML]{FC5D5D}No \\
		\rowcolor[HTML]{34CDF9} 
		D7 & Nublado & Ario & Normal & Fuerte & \cellcolor[HTML]{FF8787}Si \\
		\rowcolor[HTML]{007AFF} 
		D8 & Soleado & Templado & Alta & Flojo & \cellcolor[HTML]{FC5D5D}No \\
		\rowcolor[HTML]{34CDF9} 
		D9 & Soleado & Ario & Normal & Flojo & \cellcolor[HTML]{FF8787}Si \\
		\rowcolor[HTML]{007AFF} 
		D10 & Lluvia & Templado & Normal & Flojo & \cellcolor[HTML]{FC5D5D}Si \\
		\rowcolor[HTML]{34CDF9} 
		D11 & Soleado & Templado & Normal & Fuerte & \cellcolor[HTML]{FF8787}Si \\
		\rowcolor[HTML]{007AFF} 
		D12 & Nublado & Templado & Alta & Fuerte & \cellcolor[HTML]{FC5D5D}Si \\
		\rowcolor[HTML]{34CDF9} 
		D13 & Nublado & Calor & Normal & Flojo & \cellcolor[HTML]{FF8787}Si \\
		\rowcolor[HTML]{007AFF} 
		D14 & Lluvia & Templado & Alta & Fuerte & \cellcolor[HTML]{FC5D5D}No \\ \hline
	\end{tabular}
\end{center}
En el ejemplo del tenis tenemos 9 objetos clasificados como $\oplus$ y 5 como $\ominus$, con lo que \[ H([9\oplus,5\ominus])=-0.642\cdot\log_20.642-0.58\cdot\log_20.358=0.94 \]
\begin{center}
	\includegraphics{"Temas/Tema 2/screenshot003"}
\end{center}
Por lo tanto, el atributo que ofrece una mayor ganancia de información es el atributo \textbf{Cielo}.

Utilizando \textbf{Cielo} como nodo raíz el árbol inicial quedaría:
\begin{center}
	\includegraphics{"Temas/Tema 2/screenshot004"}
\end{center}
\begin{minipage}{0.45 \textwidth}
	Ahora habría que repetir el proceso con los nodos correspondientes a los valores \textbf{soleado} y \textbf{lluvia} (el nodo \textbf{nublado} sólo contiene una clase).
\end{minipage}\qquad\begin{minipage}{0.45\textwidth}
\begin{center}
	\includegraphics{"Temas/Tema 2/screenshot005"}
\end{center}
\end{minipage}

\begin{itemize}
	\item Para el nodo Cielo = Soleado:
	\begin{itemize}
		\item $\chi_{\mathrm{soleado}}=\{D_1,D_2,D_8,D_9,D_11\}$ con $H(\chi_{\mathrm{soelado}})=0.971$
		\item $\mathrm{Ganancia}(\chi_{\mathrm{soleado}},\mathrm{Temperatura})=0.971-\dfrac{2}{5}\cdot0-\dfrac{2}{5}\cdot1-\dfrac{1}{5}\cdot0=0.570$
		\item $\mathrm{Ganancia}(\chi_{\mathrm{soelado}},\mathrm{Humedad})=0.971-\dfrac{3}{5}\cdot0-\dfrac{2}{5}\cdot0=0.971$
		\item $\mathrm{Ganancia}(\chi_{\mathrm{soleado}},\mathrm{Viento})=0.971-\dfrac{2}{5}\cdot1-\dfrac{3}{5}\cdot0.918=0.019$
	\end{itemize}
	\item Para el nodo Cielo = Lluvia:
	\begin{itemize}
		\item $\chi_{\mathrm{lluvia}}=\{D_4,D_5,D_6,D_{10},D_{14}\}$ con $H(\chi_{\mathrm{lluvia}})=0.971$
		\item $\mathrm{Ganancia}(\chi_{\mathrm{lluvia}},\mathrm{Temperatura})=0.971-\dfrac{3}{5}\cdot0.918-\dfrac{2}{5}\cdot1-\dfrac{1}{5}\cdot0=0.820$
		\item $\mathrm{Ganancia}(\chi_{\mathrm{lluvia}},\mathrm{Humedad})=0.971-\dfrac{2}{5}\cdot1-\dfrac{3}{5}\cdot0.918=0.820$
		\item $\mathrm{Ganancia}(\chi_{\mathrm{lluvia}},\mathrm{Viento})=0.971-\dfrac{3}{5}\cdot0-\dfrac{2}{5}\cdot0=0.971$
	\end{itemize}
\end{itemize}

Por lo tanto, el árbol resultante sería
\begin{center}
	\includegraphics{"Temas/Tema 2/screenshot006"}
\end{center}
Todos los nodos hoja tienen una entropía nula (solo instancias de una clase).
\subsubsubsection{Error global}
Es la probabilidad de error, es decir, suma ponderada de los errores de todas las hojas del árbol. \[ E=\sum_{i=1}^{n_h}w_ie_i \]donde
\begin{itemize}
	\item $n_h$ es el numero de hojas del árbol.
	\item $w_i$ es el peso o probabilidad de la hoja $i$, es decir, la probabilidad de que una instancia sea clasificada por la partición representada por la rama que acaba en la hoja $i$.
	\item $e_i$ es el error correspondiente a la rama que acaba en la hoja $i$ (número de instancia erróneas que caen en la hoja $i$ entre el número de instancias que caen en la hoja $i$).
\end{itemize}
\subsubsubsection{Algoritmo}
El algoritmo básico de aprendizaje es el ID3 (Iterative Dichotomiser 3).

\begin{algorithm}
	\caption{árbol $\gets$ aprenderArbol(\textit{datos})}
	\begin{algorithmic}[1]
		\IF{todos los ejemplos en datos tienen la misma etiqueta}
		\RETURN un nodo hoja con dicha etiqueta
		\ELSE
		\STATE Sea $A$ el atributo que clasifica mejor a los objetos en datos
		\FORALL{posible valor $v$ de $A$}
		\STATE $data(v) \leftarrow$ todos los objetos con $A = v$
		\STATE Añadir nueva rama $\leftarrow$ aprenderArbol($data(v)$)
		\ENDFOR
		\RETURN árbol
		\ENDIF
	\end{algorithmic}
\end{algorithm}

\pagebreak

\textbf{ID3(Instancias, Etiquetas, Atributos)}

\begin{algorithmic}
	\REQUIRE \texttt{Instancias:} el conjunto de datos
	\REQUIRE \texttt{Etiquetas:} el conjunto de posibles clases.
	\REQUIRE \texttt{Atributos:} el conjunto de atributos en el conjunto \texttt{Instancias}.
	\IF{todas las instancias son positivas}
	\RETURN el nodo raíz con etiqueta $+$.
	\ELSIF{todas las instncias son negativas}
	\RETURN el nodo raíz con la etiqueta $-$.
	\ELSIF{\texttt{Atributos}=$\varnothing$}
	\RETURN el nodo raíz con el valor de \texttt{Etiquetas} más probable en \texttt{Instancias}.
	\ENDIF
	\STATE Sea $A$ el atributo que clasifica mejor las instancias en \textit{datos}.
	\STATE Crear un árbol con un nodo etiquetado con $A$
	\FORALL{posible valor $v_i$ del atributo $A$}
	\STATE añadir un arco bajo la raíz con la comprobación $A=v_i$.
	\STATE sea \texttt{Instancias} $v_i$ el subconjunto de \texttt{Instancias} con $A=v_i$.
	\IF{\texttt{Instancias}$_{v_i}=\varnothing$}
	\STATE añadir un nodo hoja al arco añadido con el valor de \texttt{Etiquetas} más probable en \texttt{Ejemplos}.
	\ELSE 
	\STATE añadir al nuevo árbol el subárbol generado por \textbf{ID3} (\texttt{Instancias}$_{v_i}$,\texttt{Etiquetas}, \texttt{Atributos}$-\{A\}$).
	\ENDIF
	\ENDFOR
	\RETURN nodo raíz
\end{algorithmic}
\subsubsection{Sobre-ajuste}
\subsubsubsection{Espacio de hipótesis y sobre-ajuste}
El \textbf{espacio de hipótesis} $H$ (no confundir con la entropía) en árboles de decisión
abarca todas las posibles combinaciones de atributos y valores que pueden formar
árboles de decisión, y nuestra tarea es encontrar la hipótesis más adecuada para
clasificar correctamente las instancias de entrada.

En general, se prefieren hipótesis cortas para evitar el sobre-ajuste (\textbf{overfitting}).

\begin{itemize}[label=\color{red}\textbullet, leftmargin=*]
	\item \color{lightblue}Sobre-ajuste
\end{itemize}
Dado un espacio de hipótesis $H$, se dice que una hipótesis particular $h\in H$ sobreajusta los datos de entrenamiento si existe un hipótesis alternativa $h'\in H$, tal que $h$ presenta un error menor que $h'$ sobre los ejemplos de entrenamiento, pero $h'$ presenta un error menor que $h$ sobre el conjunto total de observaciones.
\subsubsubsection{Proceso de poda}
\begin{itemize}[label=\color{red}\textbullet, leftmargin=*]
	\item \color{lightblue}¿Cómo podemos evitar el sobre-aprendizaje o sobre-ajuste?
\end{itemize}
\textbf{Poda:} Eliminar condiciones de las ramas del árbol encontrar más pequeños. Existe dos tipos de poda:\begin{itemize}
	\item \textbf{Prepoda:} El proceso se realiza durante la construcción del árbol, estableciendo un criterio de parada.
	\begin{itemize}
		\item El número de instancias por nodo.
		\item El error esperado.
		\item MDL (Minimun Description Lenght).
	\end{itemize}
	\item \textbf{Postpoda:} El proceso se realiza después de la construcción del árbol
	\begin{itemize}
		\item Consiste en ir eliminando nodos de abajo a arriba mientras se vaya cumpliendo un criterio determinado.
	\end{itemize}
\end{itemize}
Se pueden combinar ambas aproximaciones.
\subsubsubsection{Otras medidas}
\textbf{Medidas alternativas:} En algunos casos se suele utilizar otras medidas como:
\begin{itemize}
	\item El \textit{Ratio} de la ganancia de información, para evitar el hecho de que se favorece la selección de los atributos con más valores.
	\item MSE para regresión
	\item Índice Gini empleado por el algoritmo CART
	\item DKM, basados en AUC, MDL
\end{itemize}
\subsubsection{Algoritmo CART y Otros}
\subsubsubsection{CART}
\textbf{Cart} (\textbf{C}lasificación \textbf{A}nd \textbf{R}egression \textbf{T}rees). Similar al ID3 pero:
\begin{itemize}
	\item Permite que la variable que define la clase sea continua y no construye un conjunto de reglas.
	\item Utiliza el índice de Gini en vez de la ganancia de información para seleccionar el mejor atributo (la mejor partición).
	\item Utiliza también el esquema de partición recursiva utilizando una estrategia voraz.
	\item También permite resolver problemas de regresión.
\end{itemize}
Se elige la partición que produce el menor valor de la función de coste.
\begin{itemize}
	\item Para regresión: RSME.
	\item Para clasificación: GINI \[ \mathrm{Gini}(p)=\sum_{i=1}^{n}p_i(1-p_i) \] con $p_i$ las proposiciones de instancias de la clase $i$ en la partición.
\end{itemize}
Prepoda: Utiliza como criterio de parada el número mínimo de instancias asignadas al nodo.

Postpoda: Utiliza un criterio que controla la improtancia relativa del error frente la complejidad (tamaño del árbol).
\subsubsubsection{C4.5, C5.0}
\textbf{C4.5:} Permite, a diferenciaque ID3, que las características puedan ser continuas, definiendo de forma dinámica un atributo discreto particionando los atributos continuos en un conjunto discreto de intervalos.
\begin{itemize}
	\item C4.5 transforma los árboles obtenidos en un conjunto de reglas del tipo \textit{if-then}. La precisión de cada regla es evaluada de forma independiente para determinar el orden en el que deben ser aplicadas.
	\item Un proceso de poda elimina antecedentes de las reglas si con esto se mejora la precisión de la misma.
	\item C5.0, utiliza menos memoria y obtiene un conjunto de reglas menor y más preciso.
	\item J48 es la implementación en código abierto de C4.5
\end{itemize}
\subsubsection{Random Forests}
Random Forest es una técnica que construye un grán número de árboles de decisión no correlacionados.

Se basa en la técnica de Agregación de Bootstrap (Bagging), técnica de agregación de clasificadores o regresores que:
\begin{itemize}
	\item Aumenta la precisión y estabilidad reduciendo los efectos del ruido.
	\item Reduce la varianza en las predicciones.
	\item Ayuda a evitar el sobre ajuste.
\end{itemize}
La predicción del modelo se elige analizando las predicciones de cada uno de los árboles considerados en el modelo.
\subsubsubsection{Algoritmo}
\begin{algorithm}
	\caption{RF $\gets$ RandomForest(\textit{datos})}
	\begin{algorithmic}[1]
		\FOR{$i\gets1\to$ \textit{n_arboles}}
		\STATE Extraer una muestra de tamaño \textit{size(data)} de \textit{datos} por bootstrapping
		\STATE Construir un árbol $T_i$ repitiendo recurivamente para cada nodo hoja
		\STATE \begin{enumerate}[leftmargin=1.5cm]
			\item Selecionar aleatoriamente $m$ atributos.
			\item Seleccionar la mejor partición de las inducidas por los $m$ atributos.
			\item Dividir el nodo en dos nodos hijos.
			\item Si el tamaño de nodo alcanza $n_{\min}$ no continuar dividiendo.
		\end{enumerate}
		\ENDFOR
		\RETURN El conjunto de árboles $\{T_i\}_1^{\text{\textit{n_arboles}}}$.
	\end{algorithmic}
\end{algorithm}

Si tenemos $p$ atributos las recomendaciones para el valor de $m$ son:
\begin{itemize}
\item Clasificación: $\lfloor\sqrt{m}\rfloor$

\item Regresión: $\lfloor p/3\rfloor$
\item Siendo el valor mínimo 1.
\end{itemize}
Una vez se han construido los árboles para hacer predicciones:
\begin{itemize}
	\item Clasificación: La clase más votada por los árboles.
	\item Regresión: La media de todas las predicciones
\end{itemize}
\subsubsubsection{OBB e importancia de las variables}
\textbf{OBB (out of the bag) error estime:} Para cada muestra se predice el error utilizando sólo los árboles en los que no ha sido utilizada (no ha sido elegida en bootstrapping)
\begin{itemize}
	\item Los valores son parecidos a los que se obtienen mediante una validación cruzada de $N$-pliegues.
\end{itemize}
\textbf{Importancia de los atributos:}
\begin{itemize}
	\item En cada partición del árbol se registra la mejora del criterio de división.
	\item Este valor se considera la importancia del atributo.
	\item Para cada variable se agregan los valores generados en cada árbol.
\end{itemize}
\subsubsection{Conclusiones}
La utilización de árboles de decisión es muy popular en muchas disciplinas.

Experimentalmente han demostrado una buena capacidad de clasificación.

La clave reside en la función a optimizar a la hora de elegir el atributo para disgregar el árbol.

Los ensambles nos permiten mejorar los resultados combinando información.