\begin{center}
    \textbf{\large Hoja de ejercicios Tema 4: Contrastes de hipótesis} 
\end{center}
\begin{enumerate}[label=\color{red}\textbf{\arabic*)}]
    \item \lb{Sea $X\sim \mathcal{N}(\mu,\sigma^2)$ y se considera una m.a.s. de tamaño $n$ de $X$, donde  $\sigma^2$ es conocida.}
        \begin{enumerate}[label=\color{red}\textbf{\alph*)}]
            \item \db{Consideremos las hipótesis $H_0:\mu=\mu_0$ y $H_1:\mu>\mu_0$. Demostrar que el test con región de rechazo $S_1=\{\mathbf{x}\in \mathbf{X}|(\overline{x}-\mu_0) / (\sigma /\sqrt{n} )>z_{1-\alpha}\} $ es de extensión $\alpha$ para las hipótesis anteriores.} 
            \item \db{Consideremos las hipótesis $H_0:\mu\le \mu_0$ y $H_1:\mu>\mu_0$. Demostrar que el test con región de rechazo $S_1=\{\mathbf{x}\in \mathbf{X}|(\overline{x}-\mu_0) / (\sigma /\sqrt{n} )>z_{1-\alpha}\} $ es de extensión $\alpha$ para las hipótesis anteriores.} 
        \end{enumerate}
\newpage
    \item \lb{Sean $X\sim \mathcal{N}(\mu_X,\sigma_X^2)$ e $Y\sim \mathcal{N}(\mu_Y,\sigma_Y^2)$ independientes. Se considera una m.a.s. de tamaño $n_X$ de  $X$ y una m.a.s. de tamaño  $n_Y$ de  $Y$.}
        \begin{enumerate}[label=\color{red}\textbf{\alph*)}]
            \item \db{Suponiendo $\sigma_X^2$ y $\sigma_Y^2$ conocidas, obtener un procedimiento para contrastar $H_0:\mu_X=\mu_Y$ frente a $H_1:\mu_X\neq \mu_Y$.}
            \item \db{Suponiendo $\sigma_X^2$ y $\sigma_Y^2$ desconocidas pero iguales, obtener un procedimiento para contrastar $H_0:\mu_X=\mu_Y$ frente a $H_1:\mu_X\neq \mu_Y$.} 
        \end{enumerate}
        \newpage
    \item \lb{Consideremos una m.a.s. $X_1,\dots,X_n$ de una población que tiene una densidad $f(x;\theta)=e^{-(x-\theta)}\chi_{(\theta,+\infty)} $. Obtener el test RVG para $H_0:\theta\le \theta_0$ frente a $H_1:\theta>\theta_0$.}
        \newpage
    \item \lb{Sean $X\sim \mathcal{N}(\mu,\sigma^2)$ y se considera una m.a.s. de tamaño $n$ de $X$. Consideramos las hipótesis  $H_0:\mu=\mu_0$ frente a $H_1:\mu>\mu_0$. Demostrar que el test que rechaza $H_0$ si $\dfrac{\overline{X}-\mu_0}{S /\sqrt{n} }>t_{n-1,1-\alpha}$ es de extensión $\alpha$.} 
        \newpage
    \item \lb{A continuación indicamos para distintas situaciones las hipótesis sobre la media poblacional de una variable normal para $\sigma$ conocida que se plantearon, así como el valor del estadístico de prueba  $z_0=\dfrac{\overline{X}-\mu_0}{\sigma /\sqrt{n} }$, que se obtuvo. Determinar los p-valores correspondientes e indicar cuál es la decisión acerca de $H_0$.}
        \begin{enumerate}[label=\color{red}\textbf{\alph*)}]
            \item \db{$H_0:\mu=120;H_1:\mu\neq 120$, estadístico de prueba: $z_0=2.32$.} 
            \item \db{$H_0:\mu=120;H_1:\mu\neq 120$, estadístico de prueba: $z_0=-1.88$.} 
            \item \db{$H_0:\mu=1000;H_1:\mu>1000$, estadístico de prueba: $z_0=1.6$.} 
            \item \db{$H_0:\mu=1000;H_1:\mu< 1000$, estadístico de prueba: $z_0=-1.24$.} 
        \end{enumerate}
        \newpage
    \item \lb{En un laboratorio, se estudia la velocidad de combustión de dos tipos de combustibles sólidos A y B. Se pretende determinar qué combustibles presentan una velocidad de combustión en promedio mayor de 50 cm/s. Los resultados obtenidos fueron:}
        \begin{center}
            \color{lightblue}
            \begin{tabular}{llll}
                \hline
                Combustible & Tamaño muestral & Media muestral & Varianza muestral\\ \hline
                A & 25 & 51.3 & 3.9\\
                B & 30 & 51.5 & 3.6\\ \hline
            \end{tabular}
        \end{center}
        \lb{Suponiendo que la distribución de las variables es normal, plantear un test para contrastar si la varianza en la población de la velocidad de combustión para A es superior a la de B.}
        \newpage
    \item \lb{Sea $X$ una población con función de densidad gamma:  \[
    f(x;\theta)=\theta^2 x\exp(-\theta x),
    \] con $x>0,\theta>0$. Se considera una muestra de tamaño  $n$ de $X$.}
    \begin{enumerate}[label=\color{red}\textbf{\alph*)}]
        \item \db{Estudiar el modelo anterior tiene cociente de verosimilitudes monótono en algún estadístico.}
        \item \db{En caso afirmativo, obtener el test uniforme de máxima potencia y extensión $\alpha$ para el contraste $H_0:\theta=\theta_0$ frente a $H_1:\theta>\theta_0$.}
        \item \db{Si $n=20,\alpha=0.05,\theta_0=1$ y $\overline{X}=1.2$, ¿qué criterio de decisión de tomaría? Obtener el p-valor para dicha muestra.}
        \item \db{Obtener el test de razón de verosimilitudes generalizado y extensión $\alpha$, para el contraste $H_0:\theta=2$ frente a la hipótesis $H_1:\theta\neq 2$.} 
    \end{enumerate}
\end{enumerate}
