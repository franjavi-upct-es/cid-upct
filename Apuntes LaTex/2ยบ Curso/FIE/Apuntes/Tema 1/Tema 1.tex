\section{Muestreo y distribuciones muestrales}
\subsection{Introducción}
\begin{tcolorbox}[colback=blue!5!white, colframe=blue!75!black, title=El contexto]
\begin{itemize}[label=\textbullet]
    \item Tenemos una pregunta acerca de un fenómenos aleatorio.
    \item Formulamos un modelo para la varaible de interés $X$.
    \item Traducimos la pregunta de interés en términos de uno o varios parámetros del modelo.
    \item Repetimos el experimento varias veces, apuntamos los valores de  $X$.
    \item ¿Cómo usar estos valores para extraer información sobre el parámetro?
\end{itemize}
\end{tcolorbox}

\subsection{Ejemplos}

\begin{tcolorbox}[colback=blue!5!white, colframe=blue!75!black, title=¿Está la moneda trucada?]
\begin{itemize}[label=\textbullet]
    \item Experimento: tirar la modena. $X=$ resultado obtenido.

         $P(X=+)=p,P(X=c)=1-p$  \[
         \text{¿} p = \dfrac{1}{2}?
         \] 
\end{itemize}
\end{tcolorbox}

\begin{tcolorbox}[colback=blue!5!white, colframe=blue!75!black, title=Sondeo sobre intención de participación en unas elecciones]
\begin{itemize}[label=\textbullet]
    \item Queremos estima la tasa de participación antes de unas elecciones generales.
    \item Formulamos un modelo:
        \begin{itemize}[label=\textrightarrow]
            \item Experimento: "escoger una persona al azar en el censo".
            \item $X$: participación, variable dicotómica ("Sí" o "No").  $p=P(X=\text{Si})$.
        \end{itemize}
    \item ¿Cuánto vale $p$?
    \item Censo: aproximadamente 37 000 000. Escogemos aproximadamente 3000 personas.
\end{itemize}
\end{tcolorbox}

\begin{tcolorbox}[colback=blue!5!white, colframe=blue!75!black, title=Determinación de la concentración de un producto]
\begin{itemize}[label=\textbullet]
    \item Quiero determinar la concentración de un producto.
    \item Formulo el modelo:
        \begin{itemize}[label=\textrightarrow]
            \item Experimento: "llevar a cabo una medición".
            \item $X$: "valor proporcionado por el aparato".
            \item  $X\sim \mathcal{N}(\mu,\sigma^2)$.
        \end{itemize}
    \item ¿Qué vale $\mu$?
\end{itemize}
\end{tcolorbox}
\subsection{Surge una pregunta}
En todas estas situaciones donde nos basamos en la repetición de un experimento simple\dots
\begin{itemize}[label=\textbullet]
    \item ¿Cómo sabemos que nuestra estimación es fiable?
    \item ¿Qué confianza tenemos al extrapolar los resultados de una muestra de 3000 personas a una población de 37 millones de personas?
\end{itemize}
\subsection{Esbozo de respuesta: tasa de participación}
\begin{tcolorbox}[colback=blue!5!white, colframe=blue!75!black, title=Para convenceros, un experimento de simulación]
\begin{itemize}[label=\textbullet]
    \item Voy a simular el proceso de extracción de una muestra de 3000 personas en una población de 37 millones de personas.
    \item Construyo a mi antojo los distintos componentes:
        \begin{itemize}[label=\textrightarrow]
            \item \textbf{La población:} defino en mi ordenador un conjunto de 37 000 000 de ceros y unos. ($\Leftrightarrow$ el censo electoral)
                \begin{itemize}[label=\textbullet]
                    \item "1" $\Leftrightarrow$ "la persona piensa ir a votar".
                    \item "0" $\Leftrightarrow$ "la persona \textbf{no} peinsa ir a votar"
                \end{itemize}
            \item \textbf{La tasa de participación "real":} Decido que en mi población el 70\% piensa ir a votar $\to 25\,900\,000$ "1"s.
            \item \textbf{La extracción de una muestra:} construyo un pequeño programa que extrae al azar una muestra de 3000 números dentro del conjunto grande. 
        \end{itemize}
\end{itemize}
\end{tcolorbox}
\begin{lstlisting}
poblacion <- c(rep(1, 25900000), rep(0, 11100000))
set.seed(314159)
p_muestra <- mean(sample(poblacion, 3000, replace = FALSE))
p_muestra   
\end{lstlisting}
\begin{verbatim}
## [1] 0.705667    
\end{verbatim}

Queremos descartar que haya sido suerte. Vamos a repetir muchas veces (10000 veces por ejemplo), la extracción de una muestra de 3000 personas en la población.
\begin{lstlisting}
library(tidyverse)
lista_muestras <- replicate(
    10000,
    sample(poblacion, 3000, replace = FALSE),
    simplify = FALSE
)
p_muestras <- map_dbl(lista_muestras, mean)
head(p_muestras)
\end{lstlisting}
\begin{verbatim}
## [1] 0.6970000  0.7030000  0.7036667  0.7023333  0.7013333  0.7226667
\end{verbatim}

Recogemos los valores obtenidos en un histograma.
