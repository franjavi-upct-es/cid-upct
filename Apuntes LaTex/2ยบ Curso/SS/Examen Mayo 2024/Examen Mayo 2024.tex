\documentclass{article}
\usepackage{fullpage}
\usepackage[utf8]{inputenc}
\usepackage{pict2e}
\usepackage{amsmath}
\usepackage{enumitem}
\usepackage{eurosym}
\usepackage{mathtools}
\usepackage{amssymb, amsfonts, latexsym, cancel}
\setlength{\parskip}{0.3cm}
\usepackage{graphicx}
\usepackage{fontenc}
\usepackage{slashbox}
\usepackage{setspace}
\usepackage{gensymb}
\usepackage{accents}
\usepackage{adjustbox}
\setstretch{1.35}
\usepackage{bold-extra}
\usepackage[document]{ragged2e}
\usepackage{subcaption}
\usepackage{tcolorbox}
\usepackage{xcolor, colortbl}
\usepackage{wrapfig}
\usepackage{empheq}
\usepackage{array}
\usepackage{parskip}
\usepackage{arydshln}
\graphicspath{ {images/} }
\renewcommand*\contentsname{\color{black}Índice} 
\usepackage{array, multirow, multicol}
\definecolor{lightblue}{HTML}{007AFF}
\usepackage{color}
\usepackage{etoolbox}
\usepackage{listings}
\usepackage{mdframed}
\setlength{\parindent}{0pt}
\usepackage{underscore}
\usepackage{hyperref}
\usepackage{tikz}
\usepackage{tikz-cd}
\usetikzlibrary{shapes, positioning, patterns}
\usepackage{tikz-qtree}
\usepackage{biblatex}
\usepackage{pdfpages}
\usepackage{pgfplots}
\usepackage{pgfkeys}
\addbibresource{biblatex-examples.bib}
\usepackage[a4paper, left=1cm, right=1cm, top=1cm,
bottom=1.5cm]{geometry}
\usepackage{titlesec}
\usepackage{titletoc}
\usepackage{tikz-3dplot}
\usepackage{kbordermatrix}
\usetikzlibrary{decorations.pathreplacing}
\newcommand{\Ej}{\textcolor{lightblue}{\underline{Ejemplo}}}
\setlength{\fboxrule}{1.5pt}

% Configura el formato de las secciones utilizando titlesec
\titleformat{\section}
{\color{red}\normalfont\LARGE\bfseries}
{Tema \thesection:}
{10 pt}
{}

% Ajusta el formato de las entradas de la tabla de contenidos
\addtocontents{toc}{\protect\setcounter{tocdepth}{4}}
\addtocontents{toc}{\color{black}}

\titleformat{\subsection}
{\normalfont\Large\bfseries\color{red}}{\thesubsection)}{1em}{\color{lightblue}}

\titleformat{\subsubsection}
{\normalfont\large\bfseries\color{red}}{\thesubsubsection)}{1em}{\color{lightblue}}

\newcommand{\bboxed}[1]{\fcolorbox{lightblue}{lightblue!10}{$#1$}}
\newcommand{\rboxed}[1]{\fcolorbox{red}{red!10}{$#1$}}

\DeclareMathOperator{\N}{\mathbb{N}}
\DeclareMathOperator{\Z}{\mathbb{Z}}
\DeclareMathOperator{\R}{\mathbb{R}}
\DeclareMathOperator{\Q}{\mathbb{Q}}
\DeclareMathOperator{\K}{\mathbb{K}}
\DeclareMathOperator{\im}{\imath}
\DeclareMathOperator{\jm}{\jmath}
\DeclareMathOperator{\col}{\mathrm{Col}}
\DeclareMathOperator{\fil}{\mathrm{Fil}}
\DeclareMathOperator{\rg}{\mathrm{rg}}
\DeclareMathOperator{\nuc}{\mathrm{nuc}}
\DeclareMathOperator{\dimf}{\mathrm{dimFil}}
\DeclareMathOperator{\dimc}{\mathrm{dimCol}}
\DeclareMathOperator{\dimn}{\mathrm{dimnuc}}
\DeclareMathOperator{\dimr}{\mathrm{dimrg}}
\DeclareMathOperator{\dom}{\mathrm{Dom}}
\DeclareMathOperator{\infi}{\int_{-\infty}^{+\infty}}
\newcommand{\dint}[2]{\int_{#1}^{#2}}

\newcommand{\bu}[1]{\textcolor{lightblue}{\underline{#1}}}
\newcommand{\lb}[1]{\textcolor{lightblue}{#1}}
\newcommand{\db}[1]{\textcolor{blue}{#1}}
\newcommand{\rc}[1]{\textcolor{red}{#1}}
\newcommand{\tr}{^\intercal}

\renewcommand{\CancelColor}{\color{lightblue}}

\newcommand{\dx}{\:\mathrm{d}x}
\newcommand{\dt}{\:\mathrm{d}t}
\newcommand{\dy}{\:\mathrm{d}y}
\newcommand{\dz}{\:\mathrm{d}z}
\newcommand{\dth}{\:\mathrm{d}\theta}
\newcommand{\dr}{\:\mathrm{d}\rho}
\newcommand{\du}{\:\mathrm{d}u}
\newcommand{\dv}{\:\mathrm{d}v}
\newcommand{\tozero}[1]{\cancelto{0}{#1}}
\newcommand{\lbb}[2]{\textcolor{lightblue}{\underbracket[1pt]{\textcolor{black}{#1}}_{#2}}}
\newcommand{\dbb}[2]{\textcolor{blue}{\underbracket[1pt]{\textcolor{black}{#1}}_{#2}}}
\newcommand{\rub}[2]{\textcolor{red}{\underbracket[1pt]{\textcolor{black}{#1}}_{#2}}}

\author{Francisco Javier Mercader Martínez}
\date{}
\title{Señales y Sistemas\\Exámen Mayo 2024}

\begin{document}
\maketitle
\textbf{\large Cuestiones}
\begin{enumerate}[label=\color{red}\textbf{\arabic*)}]
    \item \lb{Escriba las fórmulas para obtener las partes par e impar de un señal genérica $x(t)$.}

    \item \lb{Obtenga el periodo fundamental de las señales $x(t)=\cos\left( \dfrac{5\pi t}{27} \right) $ y de $x[n]=\cos\left( \dfrac{5\pi n}{27} \right) $.} 

    \item \lb{Dibuje las señales $\prod\left( \dfrac{t-2}{4} \right) ,\bigwedge\left( \dfrac{t-2}{4} \right) ,\prod\left( \dfrac{n-3}{5} \right) ,\bigwedge\left( \dfrac{n-2}{2} \right) $} 

    \item \lb{Explica \underline{de forma muy breve} en qué consiste la estrategia de cálculo rápido de la DFT mediante un algoritmo de diezmado en el tiempo. Indique el orden del coste computacional de este algoritmo frente al cálculo directo de la DFT, y represente de forma aproximada en una misma gráfica la evolución de ambos costes conforme aumenta $N$.} 

    \item \lb{Se desea procesar una secuencia discreta $x[n]$ de duración  $N_x=1500$ muestras mediante un filtro FIR digital con respuesta al impulso  $h[n]$ (y duración $N_h=300$). Indique \underline{la forma más eficiente} de obtener el resultado de dicho procesado, $y[n]$, especificando el número de puntos sobre el que se realizarán los cálculos.} 
\end{enumerate}
\newpage
\textbf{\large Problemas} 
\begin{enumerate}[label=\color{red}\textbf{\arabic*)}]
    \item \lb{Sea un sistema LTI continuo cuya respuesta al impulso es $h(t)=\sum_{k=0}^{5} \delta(t-k)$.}
        \begin{enumerate}[label=\color{red}\textbf{\alph*)}]
            \item \db{Estudie las propiedades de memoria, causalidad y estabilidad del sistema.}
            \item \db{Obtenga la señal de salida y represéntela cuando la señal de entradas es $x(t)=\bigwedge(t)$.} 
        \end{enumerate}
        \lb{Sea un sistema LTI causal discreto cuyas señales de entrada y salida vienen relacionadas por \[
                y[n]=-ay[n-1]+b_0x[n]+b_1x[n-1]
        \] } 
        \begin{enumerate}[label=\color{red}\textbf{\alph*)},start=3]
            \item \db{Obtenga la respuesta al impulso del sistema.}
            \item \db{Dibuje el diagrama de las formas directas I y II del sistema.} 
        \end{enumerate}
    \item \lb{Considere la señal $x(t)=3\text{sinc}^2(t)$.}
        \begin{enumerate}[label=\color{red}\textbf{\alph*)}]
            \item \db{Obtenga y represente el espectro $X(\omega)$ de la señal anterior.}

            \db{\underline{Nota:} La transformada $\dfrac{A\tau}{2\pi}\text{sinc}^2\left( \dfrac{\tau t}{2\pi} \right) \xrightarrow{\text{TF}}A\bigwedge\left( \dfrac{\omega}{\tau} \right) $ puede serle de utilidad.} 
            \item \db{Calcule la energía total de la señal $x(t)$.}

                \db{\underline{Nota:} Se recomienda abordar el cálculo anterior desde el dominio de la frecuencia.}

            \item \db{Se muestrea ahora la señal $x(t)$ del enunciado, empleando para ello una pulsación de muestreo que es \underline{el doble} del mínimo que establece el criterio de Nyquist. Represente el espectro $X_s(\omega)$ de la señal muestreada, y el espectro $X(e^{j\omega} )$ de la secuencia discreta $x[n]$ resultante de almacenar los valores muestreados.}
            \item \db{Obtenga los valores de la secuencia $x[n]$} 
        \end{enumerate}
\end{enumerate}
\end{document}
