\documentclass{article}
\usepackage{fullpage}
\usepackage[utf8]{inputenc}
\usepackage{pict2e}
\usepackage{amsmath}
\usepackage{enumitem}
\usepackage{eurosym}
\usepackage{mathtools}
\usepackage{amssymb, amsfonts, latexsym, cancel}
\setlength{\parskip}{0.3cm}
\usepackage{graphicx}
\usepackage{fontenc}
\usepackage{slashbox}
\usepackage{setspace}
\usepackage{gensymb}
\usepackage{accents}
\usepackage{adjustbox}
\setstretch{1.35}
\usepackage{bold-extra}
\usepackage[document]{ragged2e}
\usepackage{subcaption}
\usepackage{tcolorbox}
\usepackage{xcolor, colortbl}
\usepackage{wrapfig}
\usepackage{empheq}
\usepackage{array}
\usepackage{parskip}
\usepackage{arydshln}
\graphicspath{ {images/} }
\renewcommand*\contentsname{\color{black}Índice} 
\usepackage{array, multirow, multicol}
\definecolor{lightblue}{HTML}{007AFF}
\usepackage{color}
\usepackage{etoolbox}
\usepackage{listings}
\usepackage{mdframed}
\setlength{\parindent}{0pt}
\usepackage{underscore}
\usepackage{hyperref}
\usepackage{tikz}
\usepackage{tikz-cd}
\usetikzlibrary{shapes, positioning, patterns}
\usepackage{tikz-qtree}
\usepackage{biblatex}
\usepackage{pdfpages}
\usepackage{pgfplots}
\usepackage{pgfkeys}
\addbibresource{biblatex-examples.bib}
\usepackage[a4paper, left=1cm, right=1cm, top=1cm,
bottom=1.5cm]{geometry}
\usepackage{titlesec}
\usepackage{titletoc}
\usepackage{tikz-3dplot}
\usepackage{kbordermatrix}
\usetikzlibrary{decorations.pathreplacing}
\newcommand{\Ej}{\textcolor{lightblue}{\underline{Ejemplo}}}
\setlength{\fboxrule}{1.5pt}

% Configura el formato de las secciones utilizando titlesec
\titleformat{\section}
{\color{red}\normalfont\LARGE\bfseries}
{Tema \thesection:}
{10 pt}
{}

% Ajusta el formato de las entradas de la tabla de contenidos
\addtocontents{toc}{\protect\setcounter{tocdepth}{4}}
\addtocontents{toc}{\color{black}}

\titleformat{\subsection}
{\normalfont\Large\bfseries\color{red}}{\thesubsection)}{1em}{\color{lightblue}}

\titleformat{\subsubsection}
{\normalfont\large\bfseries\color{red}}{\thesubsubsection)}{1em}{\color{lightblue}}

\newcommand{\bboxed}[1]{\fcolorbox{lightblue}{lightblue!10}{$#1$}}
\newcommand{\rboxed}[1]{\fcolorbox{red}{red!10}{$#1$}}

\DeclareMathOperator{\N}{\mathbb{N}}
\DeclareMathOperator{\Z}{\mathbb{Z}}
\DeclareMathOperator{\R}{\mathbb{R}}
\DeclareMathOperator{\Q}{\mathbb{Q}}
\DeclareMathOperator{\K}{\mathbb{K}}
\DeclareMathOperator{\im}{\imath}
\DeclareMathOperator{\jm}{\jmath}
\DeclareMathOperator{\col}{\mathrm{Col}}
\DeclareMathOperator{\fil}{\mathrm{Fil}}
\DeclareMathOperator{\rg}{\mathrm{rg}}
\DeclareMathOperator{\nuc}{\mathrm{nuc}}
\DeclareMathOperator{\dimf}{\mathrm{dimFil}}
\DeclareMathOperator{\dimc}{\mathrm{dimCol}}
\DeclareMathOperator{\dimn}{\mathrm{dimnuc}}
\DeclareMathOperator{\dimr}{\mathrm{dimrg}}
\DeclareMathOperator{\dom}{\mathrm{Dom}}
\DeclareMathOperator{\infi}{\int_{-\infty}^{+\infty}}
\newcommand{\dint}[2]{\int_{#1}^{#2}}

\newcommand{\bu}[1]{\textcolor{lightblue}{\underline{#1}}}
\newcommand{\lb}[1]{\textcolor{lightblue}{#1}}
\newcommand{\db}[1]{\textcolor{blue}{#1}}
\newcommand{\rc}[1]{\textcolor{red}{#1}}
\newcommand{\tr}{^\intercal}

\renewcommand{\CancelColor}{\color{lightblue}}

\newcommand{\dx}{\:\mathrm{d}x}
\newcommand{\dt}{\:\mathrm{d}t}
\newcommand{\dy}{\:\mathrm{d}y}
\newcommand{\dz}{\:\mathrm{d}z}
\newcommand{\dth}{\:\mathrm{d}\theta}
\newcommand{\dr}{\:\mathrm{d}\rho}
\newcommand{\du}{\:\mathrm{d}u}
\newcommand{\dv}{\:\mathrm{d}v}
\newcommand{\tozero}[1]{\cancelto{0}{#1}}
\newcommand{\lbb}[2]{\textcolor{lightblue}{\underbracket[1pt]{\textcolor{black}{#1}}_{#2}}}
\newcommand{\dbb}[2]{\textcolor{blue}{\underbracket[1pt]{\textcolor{black}{#1}}_{#2}}}
\newcommand{\rub}[2]{\textcolor{red}{\underbracket[1pt]{\textcolor{black}{#1}}_{#2}}}

\author{Francisco Javier Mercader Martínez}
\date{}
\title{Señales y Sistemas\\Examen Mayo 2025}

\usetikzlibrary{shapes,arrows}

\pgfmathdeclarefunction{delta}{1}{%
    \pgfmathparse{abs(#1) < 0.001 ? 1 : 0}%
}

% definir la función rectangular
\pgfmathdeclarefunction{rect}{1}{%
    \pgfmathparse{abs(#1) <= 0.5 ? 1 : 0}%
}

% definir el pulso triangular
\pgfmathdeclarefunction{tri}{1}{%
    \pgfmathparse{abs(#1) <= 1 ? (1 - abs(#1)) : 0}%
}

% definir la función escalón unitario
\pgfmathdeclarefunction{u}{1}{%
    \pgfmathparse{#1 >= 0 ? 1 : 0}%
}

\begin{document}
\maketitle
\rc{\textbf{\underline{Problema 1} } }\lb{Dominio del tiempo.}

\lb{Sea un sistema LTI continuo cuya respuesta al impulso es $h(t)=u(t)+\delta(t-6)$.}

\begin{enumerate}[label=\color{red}\textbf{\alph*)}]
	\item \db{Represente $h(t)$ y $h(2t-2)$.}

	      \begin{center}
              \begin{tikzpicture}[declare function={
                      h(\t)=u(\t)+delta(\t-6);
                  }]
			      \begin{axis}[
					      xmin= -2, xmax= 8,
					      ymin= -0.5, ymax = 1.5,
					      xlabel={$t$}, ylabel={$h(t)$},
					      axis lines = middle,
					      xlabel style={at={(axis cs:8,0)}, anchor=west},
					      ylabel style={at={(axis cs:0,1.5)}, anchor=south},
				      ]
				      \addplot[domain=-2:8, samples=1000, lightblue, line width=1.5] {h(x)};
			      \end{axis}
		      \end{tikzpicture}\qquad
              \begin{tikzpicture}[declare function={
                      h(\t)=u(\t)+delta(\t-6);
                  }]
                  \begin{axis}[
                      xmin= -2, xmax= 8,
                      ymin= -0.5, ymax = 1.5,
                      xlabel={$t$}, ylabel={$h(2t-2)$},
                      axis lines = middle,
                      xlabel style={at={(axis cs:8,0)}, anchor=west},
                      ylabel style={at={(axis cs:0,1.5)}, anchor=south},
                  ]
                      \addplot[domain=-2:8, samples=1000, lightblue, line width=1.5] {h(2*x-2)};
                  \end{axis}
              \end{tikzpicture}
	      \end{center}

	\item \db{Estuda las propiedades de memoria, causalidad y estabilidad del sistema.}

        $\begin{array}{l}
            h(t)\neq 0\text{ para }t\neq 0\longrightarrow \text{ con memoria }\\
            h(t)= 0\text{ para }t<0\longrightarrow \text{ causal }\\
            \int_{0}^{\infty} |h(t)|\dt=\infty\longrightarrow \text{}u
        \end{array}$
	\item \db{Calcule la señal de salida cuando a la entrada se aplica la señal $x(t)=e^{-t}u(t) $.}

        La salida es: \[
            y(t)=x(t)\ast h(t)=x(t)\ast [u(t)+\delta(t-6)]=x(t)\ast u(t)+x(t)\ast \delta(t-6)
        \] 
        \begin{enumerate}[label=\arabic*)]
            \item $x(t)\ast u(t)=\int_{0}^{t} e^{-\tau}\mathrm{d}\tau=1-e^{-t}$
            \item $x(t)\ast \delta(t-6)=x(t-6)=e^{-(t-6)}u(t-6) $
        \end{enumerate}
        Resultado final: \[
        y(t)=(1-e^{-t} )+e^{-(t-6)}u(t-6) 
        \] 
	\item \db{Calcule la energía total y potencia media de la señal de entrada $x(t)$.}

        \[
        \begin{array}{c}
           E=\int_{-\infty}^{\infty} |x(t)|^2\dt=\int_{0}^{\infty} e^{-2t}\dt=\left[ -\dfrac{1}{2}e^{-2t}  \right]_0^\infty=\left( \tozero{-\dfrac{e^{-\infty} }{2}} -\left( -\dfrac{e^{0} }{2} \right)  \right) =\dfrac{1}{2}\\
           P=\lim_{T \to \infty} \dfrac{1}{2T}\int_{-T}^{T}|x(t)|^2\dt=\lim_{T \to \infty} \dfrac{E}{2T}=\dfrac{\frac{1}{2} }{\infty}=0
        \end{array}
        \] 
	\item \db{Si la señal de entrada es $3e^{-(t-1)}u(t-1)+2e^{-(t-3)}u(t-3)  $, ¿cuál será la señal de salida?}

        Usamos la linealidad de la convolución: \[
        y(t)=3e^{-(t-1)}u(t-1)\ast h(t)+2e^{-(t-3)}u(t-3)\ast h(t)  
        \] 
        Sabemos que: \[
        e^{-(t-a)}u(t-a)\ast h(t)=(1-e^{-(t-a)} )u(t-a)+e^{-(t-a-6)} u(t-a-6)
        \] 
        Entonces:
        \begin{itemize}[label=\textbullet]
            \item Para el primer término ($a=1$): \[
            3\left[ (1-e^{-(t-1)} )u(t-1)+e^{-(t-7)}u(t-7)  \right] 
            \] 
            \item Para el segundo término $(a=3)$:  \[
            2\left[ (1-e^{-(t-3)} )u(t-3)+e^{-(t-9)}u(t-9)  \right] 
            \] 
        \end{itemize}
        Resultado final: \[
        y(t)=3(1-e^{-(t-1)} )u(t-1)+3e^{-(t-7)} u(t-7)+2(1-e^{-(t-3)} )u(t-3)+2e^{-(t-9)}u(t-9) 
        \] 
\end{enumerate}
\lb{Sea el sistema LTI discreto dado por la relación salida-entrada \[
	y[n]=\dfrac{1}{2}y[n-1]+x[n]+x[n-1]
\] }
\begin{enumerate}[label=\color{red}\textbf{\alph*)}, start=6]
	\item \db{Obtenga y represente la respuesta al impulso $h[n]$.}

        La respuesta al impulso se obtiene aplicando como entrada: \[
            x[n]=\delta[n]
        \] 
        Y calculando la salida $y[n]=h[n]$ paso a paso.

        $\begin{array}{l}
            n=0\longrightarrow \tozero{\dfrac{1}{2}y[-1]}+x[0]+\tozero{x[-1]}=1\\
            n=1\longrightarrow \dfrac{1}{2}y[0]+\tozero{x[1]}+x[0]=\dfrac{1}{2}\cdot 1+1=\dfrac{3}{2}\\
            n=2\longrightarrow \dfrac{1}{2}y[1]+\tozero{x[2]}+\tozero{x[1]}=\dfrac{1}{2}\cdot \dfrac{3}{2}=\dfrac{3}{4}\\
            n=3\longrightarrow \dfrac{1}{2}y[2]+\tozero{x[3]}+\tozero{x[2]}=\dfrac{1}{2}\cdot \dfrac{3}{4}=\dfrac{3}{8}\\
            n=4\longrightarrow \dfrac{1}{2}y[3]+\tozero{x[4]}+\tozero{x[3]}=\dfrac{1}{2}\cdot \dfrac{3}{8}=\dfrac{3}{16}\\
            \vdots
        \end{array}$

        Es un sistema IIR, ya que tiene retroalimentación.
	\item \db{Dibuje el diagrama de bloques del sistema en su forma directa I.}
        \begin{center}
            \begin{tikzpicture}[ >=latex,
                	block/.style={draw, minimum width=1cm, minimum height=1cm},
                	sum/.style={draw, circle, minimum size=5mm, inner sep=0pt},
                	connector/.style={-Latex, thick},
                	node distance=1.5cm and 1.5cm,
                	]
                    \node (x) at (0,0) {$x[n]$};
                    \node [block] (delay1) at (2, -2) {\small $z^{-1}$};
                    \node [sum] (sum1) at (4, 0) {$+$};
                    \node (y) at (10,0) {$y[n]$};
                    \node [block] (delay2) at (8,-2) {\small $z^{-1}$};
                    \node [sum] (sum2) at (6,0) {$+$};

                    \draw[->] (x) -- (sum1) -- (sum2) -- (y);
                    \draw[->] (2,0) -- (delay1) -| (sum1);
                    \draw[->] (8,0) -- (delay2) -| (sum2);
                    \fill[black] (7,-1.75) -- (7,-2.25) -- (6.5,-2) -- cycle  node[above left] {\small $\tfrac{1}{2} $};
            \end{tikzpicture}
        \end{center}
	\item \db{Calcule la salida cuando la entrada es $x[n]=u[n]-u[n-2]$. Represéntela hasta $n=4$.}

        \begin{center}
            \begin{tikzpicture}
                \begin{axis}[
                    xmin= -1, xmax= 6,
                    ymin= -0.1, ymax = 1.1,
                    xlabel={$n$}, ylabel={$x[n]$},
                    axis lines = middle,
                    xlabel style={at={(axis cs:6,0)}, anchor=west},
                    ylabel style={at={(axis cs:0,1.1)}, anchor=south},
                ]
                    \pgfmathsetmacro{\mysamples}{6-(-1)+1}
                    \addplot+[ycomb, mark=*, lightblue, line width=1.5, domain=-1:6, samples=\mysamples] {u(x)-u(x-2)};
                \end{axis}
            \end{tikzpicture}
        \end{center}

        $\begin{array}{l}
            n=0\longrightarrow y[0]=\dfrac{1}{2}y[-1]+x[0]+x[-1]=1\\
            n=1\longrightarrow y[1]=\dfrac{1}{2}y[0]+x[1]+x[0]=\dfrac{1}{2}\cdot 1+1+1=\dfrac{5}{2}\\
            n=2\longrightarrow y[2]=\dfrac{1}{2}y[1]+x[2]+x[1]=\dfrac{1}{2}\cdot \dfrac{5}{2}+0+1=\dfrac{9}{4}\\
            n=3\longrightarrow y[3]=\dfrac{1}{2}y[2]+x[3]+x[2]=\dfrac{9}{8}\\
            n=4\longrightarrow y[4]=\dfrac{1}{2}y[3]+x[4]+x[3]=\dfrac{9}{16}
        \end{array}$

        \begin{center}
            \begin{tikzpicture}
                \begin{axis}[
                    xmin= -1, xmax= 4.5,
                    ymin= -0.1, ymax = 3,
                    xlabel={$n$}, ylabel={$h[n]$},
                    axis lines = middle,
                    xlabel style={at={(axis cs:4.5,0)}, anchor=west},
                    ylabel style={at={(axis cs:0,3)}, anchor=south},
                ]
                \addplot+[ycomb, mark=*, lightblue, line width=1.5] coordinates { 
                        (-1,0)
                        (0,1) 
                        (1,5/2) 
                        (2,9/4)
                        (3,9/8)
                        (4,9/16)
                    };
                \end{axis}
            \end{tikzpicture}
        \end{center}
\end{enumerate}

\rc{\textbf{\underline{Problema 2} } }\lb{Dominio de la frecuencia.}
\begin{enumerate}[label=\color{red}\textbf{\alph*)}]
	\item \db{Obtenga \underline{de 2 formas distintas} la expresión analítica del espectro $X(\omega)$ correspondiente a la señal $x(t)=\prod\left( \dfrac{t-\frac{3}{2} }{3}  \right) -\prod\left( \dfrac{t-\frac{9}{2} }{3}  \right) $.}


	\item \db{Considere ahora la señal $z(t)=\sum_{n=-\infty}^{\infty}x(t-6n) $. Represente detalladamente la señal $z(t)$, indique si es periódica y, en caso afirmativo, exprese la señal  $z(t)$ como una combinación lineal de exponenciales complejas. Obtenga la salida  $y(t)$ que resultaría de procesar la señal  $z(t)$ con el sistema LTI caracterizado por la respuesta al impulso  $h(t)=e^{-2t}u(t) $.}
	\item \db{Se nuestra ahora la sñela $x(t)$ del apartado a) tomando una muestra cada  $T_s=0.5$ seg. Escriba la expresión analítica de la secuencia  $x[n]$ resultante, e indique si se habrá producido solapamiento espectral (\textit{aliasing} ) al muestrear.}
	\item \db{Se desea filtrar los 300 primeros valores (esto es, para los índices $0\le n\le 299$) de la secuencia $x[n]$ del apartado anterior con un sistema causal cuya respuesta al impulso  $h[n]$ está comprendida entre los índices  $0\le n\le 199$. Indique \underline{de la forma más eficiente} de obtener el resultado de dicho procesado, $y[n]$, especificando el número de puntos sobre el que se realizarán los cálculos.}
\end{enumerate}
\end{document}

