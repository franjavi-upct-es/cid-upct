\documentclass{article}
\usepackage{fullpage}
\usepackage[utf8]{inputenc}
\usepackage{pict2e}
\usepackage{amsmath}
\usepackage{enumitem}
\usepackage{eurosym}
\usepackage{mathtools}
\usepackage{amssymb, amsfonts, latexsym, cancel}
\setlength{\parskip}{0.3cm}
\usepackage{graphicx}
\usepackage{fontenc}
\usepackage{slashbox}
\usepackage{setspace}
\usepackage{gensymb}
\usepackage{accents}
\usepackage{adjustbox}
\setstretch{1.35}
\usepackage{bold-extra}
\usepackage[document]{ragged2e}
\usepackage{subcaption}
\usepackage{tcolorbox}
\usepackage{xcolor, colortbl}
\usepackage{wrapfig}
\usepackage{empheq}
\usepackage{array}
\usepackage{parskip}
\usepackage{arydshln}
\graphicspath{ {images/} }
\renewcommand*\contentsname{\color{black}Índice} 
\usepackage{array, multirow, multicol}
\definecolor{lightblue}{HTML}{007AFF}
\usepackage{color}
\usepackage{etoolbox}
\usepackage{listings}
\usepackage{mdframed}
\setlength{\parindent}{0pt}
\usepackage{underscore}
\usepackage{hyperref}
\usepackage{tikz}
\usepackage{tikz-cd}
\usetikzlibrary{shapes, positioning, patterns}
\usepackage{tikz-qtree}
\usepackage{biblatex}
\usepackage{pdfpages}
\usepackage{pgfplots}
\usepackage{pgfkeys}
\addbibresource{biblatex-examples.bib}
\usepackage[a4paper, left=1cm, right=1cm, top=1cm,
bottom=1.5cm]{geometry}
\usepackage{titlesec}
\usepackage{titletoc}
\usepackage{tikz-3dplot}
\usepackage{kbordermatrix}
\usetikzlibrary{decorations.pathreplacing}
\newcommand{\Ej}{\textcolor{lightblue}{\underline{Ejemplo}}}
\setlength{\fboxrule}{1.5pt}

% Configura el formato de las secciones utilizando titlesec
\titleformat{\section}
{\color{red}\normalfont\LARGE\bfseries}
{Tema \thesection:}
{10 pt}
{}

% Ajusta el formato de las entradas de la tabla de contenidos
\addtocontents{toc}{\protect\setcounter{tocdepth}{4}}
\addtocontents{toc}{\color{black}}

\titleformat{\subsection}
{\normalfont\Large\bfseries\color{red}}{\thesubsection)}{1em}{\color{lightblue}}

\titleformat{\subsubsection}
{\normalfont\large\bfseries\color{red}}{\thesubsubsection)}{1em}{\color{lightblue}}

\newcommand{\bboxed}[1]{\fcolorbox{lightblue}{lightblue!10}{$#1$}}
\newcommand{\rboxed}[1]{\fcolorbox{red}{red!10}{$#1$}}

\DeclareMathOperator{\N}{\mathbb{N}}
\DeclareMathOperator{\Z}{\mathbb{Z}}
\DeclareMathOperator{\R}{\mathbb{R}}
\DeclareMathOperator{\Q}{\mathbb{Q}}
\DeclareMathOperator{\K}{\mathbb{K}}
\DeclareMathOperator{\im}{\imath}
\DeclareMathOperator{\jm}{\jmath}
\DeclareMathOperator{\col}{\mathrm{Col}}
\DeclareMathOperator{\fil}{\mathrm{Fil}}
\DeclareMathOperator{\rg}{\mathrm{rg}}
\DeclareMathOperator{\nuc}{\mathrm{nuc}}
\DeclareMathOperator{\dimf}{\mathrm{dimFil}}
\DeclareMathOperator{\dimc}{\mathrm{dimCol}}
\DeclareMathOperator{\dimn}{\mathrm{dimnuc}}
\DeclareMathOperator{\dimr}{\mathrm{dimrg}}
\DeclareMathOperator{\dom}{\mathrm{Dom}}
\DeclareMathOperator{\infi}{\int_{-\infty}^{+\infty}}
\newcommand{\dint}[2]{\int_{#1}^{#2}}

\newcommand{\bu}[1]{\textcolor{lightblue}{\underline{#1}}}
\newcommand{\lb}[1]{\textcolor{lightblue}{#1}}
\newcommand{\db}[1]{\textcolor{blue}{#1}}
\newcommand{\rc}[1]{\textcolor{red}{#1}}
\newcommand{\tr}{^\intercal}

\renewcommand{\CancelColor}{\color{lightblue}}

\newcommand{\dx}{\:\mathrm{d}x}
\newcommand{\dt}{\:\mathrm{d}t}
\newcommand{\dy}{\:\mathrm{d}y}
\newcommand{\dz}{\:\mathrm{d}z}
\newcommand{\dth}{\:\mathrm{d}\theta}
\newcommand{\dr}{\:\mathrm{d}\rho}
\newcommand{\du}{\:\mathrm{d}u}
\newcommand{\dv}{\:\mathrm{d}v}
\newcommand{\tozero}[1]{\cancelto{0}{#1}}
\newcommand{\lbb}[2]{\textcolor{lightblue}{\underbracket[1pt]{\textcolor{black}{#1}}_{#2}}}
\newcommand{\dbb}[2]{\textcolor{blue}{\underbracket[1pt]{\textcolor{black}{#1}}_{#2}}}
\newcommand{\rub}[2]{\textcolor{red}{\underbracket[1pt]{\textcolor{black}{#1}}_{#2}}}

\author{Francisco Javier Mercader Martínez}
\date{}
\setlistdepth{9}
\lstset{
   language=Python,
   basicstyle=\ttfamily\small,
   numberstyle=\tiny,
   keywordstyle=\color{blue},
   commentstyle=\color{olive},
   stringstyle=\color{red},
   breakatwhitespace=false, 
   breaklines=true,
   showspaces=false,
   showstringspaces=false,
   showtabs=false, 
   tabsize=2,
   literate={á}{{\'a}}1 {é}{{\'e}}1 {í}{{\'i}}1 {ó}{{\'o}}1 {ú}{{\'u}}1{ñ}{{\~n}}1 {Á}{{\'A}}1 {Í}{{\'I}}1,
   mathescape=false,
   backgroundcolor=\color{lightgray!10},   
}
\title{Análisis y Diseño de Algoritmos\\Ejercicios Tema 4}

\begin{document}
\begin{enumerate}[label=\color{red}\textbf{\arabic*)}]
  \item \lb{En el problema de la mochila (igual que en el problema del cambio de monedas) puede existir en general más de un solución óptima para unas entradas determinadas. ¿Cómo se puede comprobar si una solución óptima es única o no, suponiendo que hemos resuelto el problema utilizando programación dinámica? Dar un algoritmo para que, a paritr de las tablas resultantes del problema de la mochila, muestre todas las soluciones óptimas.}
  \item \lb{El número de combinaciones de \textbf{\texttt{m}} objetos entre un conjunto de \textbf{\texttt{n}}, denotado por $\dbinom{n}{m}$, para $n\le 1$ y $0\le m\le n$, se puede definir recurisvamente por: \[
  \begin{array}{ll}
    \dbinom{n}{m}=1 & \text{Si $m=0$ ó  $m=n$}\\
    \dbinom{n}{m}=\dbinom{n-1}{m}+\dbinom{n-1}{m-1}& \text{Si }0<m<n
  \end{array}
  \]
Conociendo el resultado puede er calculado también con la fórmula: $\dfrac{n!}{m!\cdot (n-m)!}$}
\begin{enumerate}[label=\color{red}\alph*)]
  \item \db{Dar una función recursiva para calcular $\dibnom{n}{m}$, usando la primera de las definiciones. ¿Cuál será el orden de complejidad de este algoritmo? Sugerencia: la respuesta es inmediata.}
  \item \db{Diseñar un algoritmo de programación dinámica para calcular $\dbinom{n}{m}$. Nota: la tabla construida por el algoritmo es conocida como "el triángulo de Pasca". ¿Cuál será el tiempo de ejecución en este caso?} 
\end{enumerate}
\item \lb{Considerar el problema de la mochila 0/1. En este ejercicio estamos interesados en calcular el número de formas distintas de meter o no los objetos en la mochila, pero respetando la capcaidad máxima de la mochila. Por ejemplo, si todos los \textbf{\texttt{n}} objetos cupieran en la mochila, existirían $\mathbf{\mathtt{2^n}}$ formas posibles. Pero en general, si no caben todos, habrán muchas menos. Resolver mediante programación dinámica el problema de calcular el número de formas distintas de completar total o parcialmente la mochila. Datos del problema: \textbf{\texttt{n}} objetos, \textbf{\texttt{M}} capacidad de la mochila, $\mathbf{\mathtt{p=(p1,p2,\dots,p_n)}}$ pesos de los objetos.} 
        
\item \lb{Una variante del problema de la mochila es la siguiente. Tenemos un conjunto de enteros (positivos) $A=\{a_1,a_2,\dots,a_n\} $ y un entero $K$. El objetico es encontrar si existe algún subconjunto de  $A$ cuya suma sea exactamente  $K$.}
  \begin{enumerate}[label=\color{red}\textbf{\arabic*)}]
    \item \db{Desarrollar un algoritmo para resolver este problema, utilizando programación  dinámica. ¿Cuál  es el orden de complejidad de este algoritmo?}
    \item \db{Mostrar cómo  se puede obtener el conjunto de objetos resultantes (en caso de existir solución) a partir de las tablas utilizadas por el algoritmo.}
    \item \db{Aplicar el algoritmo sobre el siguiente ejemplo $A=\{2,3,5,2\},\, K=7 $. ¿Cómo se puede comprobar que la solución no es única?} 
  \end{enumerate}
\item \lb{En el problema de la mochila 0/1 disponemos de dos mochilas, con capacidades \textbf{\texttt{M1}} y \textbf{\texttt{M2}}. El objetivo es maxmizar la suma de beneficios de los objetos transportados en ambas mochilas, respetando las capacidades de cada una. Resolver el problema mediante programación  dinámica, definiendo la ecuación  recurrente, las tablas usadas y el algoritmo para rellenarlas.\newline
    Datos del problema: \textbf{\texttt{n}} objetos, \textbf{\texttt{M1}} capacidad de la mochila 1, \textbf{\texttt{M2}} capacidad de la mochila 2, $\mathbf{\mathtt{p=(p_1,p_2,\dots,p_n)}}$ pesos de los objetos, $\mathbf{\mathtt{b=(b_1,b_2,\dots,b_n)}}$ beneficios de los objetos.}

  \item \lb{Una agencia de turismo realiza planificaciones de viajes aéreos. Para ir de una ciudad \textbf{\texttt{A}} a \textbf{\texttt{B}}  puede ser necesario coger varios vuelos distintos. El tiempo de un vuelo directo desde  \textbf{\texttt{I}} hasta \textbf{\texttt{J}} ser´á  \textbf{\texttt{T[i,j]}}. Hay que tener en cuenta que si cogemos un vuelo (de \textbf{\texttt{A}} a \textbf{\texttt{B}}) y después otro (de \textbf{\texttt{B}} a \textbf{\texttt{C}}) será  necesario esperar un tiempo de "escala" adicional en el aeropuesto (almacenado en \textbf{\texttt{E[A, B, C]}}).}
    \begin{enumerate}[label=\color{red}\alph*)]
      \item \db{Diseñar una solución para resolver este problema utilizando programación dinámica. Explicar cómo, a partir de las tablas, se puede obtener el conjunto de vuelos necesarios para hacer un viaje concreto.}
      \item \db{Mostrar la ejecución del algoritmo sobre la siguiente matriz \textbf{\texttt{T}}, suponiendo que todos los \textbf{\texttt{E[A, B , C]}} tienen valor 1. ¿Cuál es el orden de complejidad del algoritmo?

              \begin{tabular}{|c|c|c|c|c|}
              \hline
              \textbf{\texttt{T[i,j]}} & \textbf{\texttt{A}} & \textbf{\texttt{B}} & \textbf{\texttt{C}} & \textbf{\texttt{D}}\\ \hline
              \textbf{\texttt{A}} & - & 2 & 1 & 3\\ \hline
              \textbf{\texttt{B}} & 2 & 2 & - & 1\\ \hline
              \textbf{\texttt{D}} & 3 & 4 & 8 & -\\ \hline
\end{tabular}}
    \end{enumerate}

\item \lb{Supongamos una serie de \textbf{\texttt{n}} trabajos denominados \textbf{\texttt{a, b, c, \dots}} y una tabla \textbf{\texttt{B[1..n, 1..n]}}, en la que cada posición \textbf{\texttt{B[i,j]}} almacena el beneficio de ejecutar el trabajo \textbf{\texttt{i}}  y a continuación el trabajo \textbf{\texttt{j}}. Se quiere encontrar la sucesión de \textbf{\texttt{m}} trabajos que dé un beneficio óptimo. No hay límite en el número de veces que se puede ejecutar un trabajo concreto.}
        \begin{enumerate}[label=\color{red}\alph*)]
                \item \db{Idear un algoritmo por programación dinámica que resuelva el problema. Para ello, definir un subproblema (que permita realiza la combinación de problemas pequeños para resolver problemas grandes), especifica la ecuación de recurrencia para el mismo (con sus cosas base) y después describe las tablas necesarias y cómo son rellenadas.}
                \item \db{Ejecutar el algoritmo sobre la siguiente tabla, suponiendo que  \textbf{\texttt{m=5}}.
                                \begin{tabular}{|c|c|c|c|}
        \hline
        \textbf{\texttt{B[i,j]}} & \textbf{\texttt{a}} & \textbf{\texttt{b}} & \textbf{\texttt{c}} \\ \hline
        \textbf{\texttt{A}}  & 2 & 2 & 5\\ \hline
        \textbf{\texttt{B}}  & 4 & 1 & 3\\ \hline
        \textbf{\texttt{C}}  & 3 & 2 & 2\\ \hline
\end{tabular}
                        } 
                \item \db{Estimar el tiempo de ejecución del algoritmo. El tiempo estimado ¿es un orden exacto o una cota superior del peor caso?} 
        \end{enumerate}
\item \lb{Resolver el siguiente problema con programación dinámica. Tenemos un conjunto de \textbf{\texttt{n}} objetos, cada uno con un peso $\mathbf{\mathtt{p=(p_1,p_2,...,p_n)}}$. El objetivo es repartir los objetos entre dos montones diferentes, de manera que queden lo más equilibrados posible en peso. Esto es, se debe minimizar la diferencia entre los pesos totales de ambos montones. Aplciar sobre el ejemplo con \textbf{\texttt{n=4}} y \textbf{\texttt{p=(2,1,3,4)}}.} 
\end{enumerate}
\end{document}
