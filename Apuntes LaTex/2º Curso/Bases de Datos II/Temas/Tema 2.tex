\section{Introducción a los sistemas NoSQL}
\subsection{Introducción a NoSQL}
\textbf{NoSQL} $\longrightarrow$ \textit{hastag} llamativo que se eligió para una conferencia en 2009 (Johan Oskarsson de Last.fm)

Ahora se asocia a cientos de bases de datos diferentes, que se han clasificado en varios tipos (las veremos después), caracterizadas por \textbf{no usar SQL} como modelo de datos.

Más recientemente \textbf{NoSQL $\longrightarrow$ \textit{Not Only SQL}} (no sólo SQL) $\longrightarrow$ Persistencia políglota (\textit{polyglot persistence})

\subsubsection{¿Por qué se plantearon?}
\begin{enumerate}
	\item \textbf{Mayor escalabilidad horizontal}
	\begin{itemize}
		\item conjuntos de datos muy muy grandes
		\item sistemas de alto volumen de escrituras (\textbf{streaming} de eventos, aplicaciones sociales)
	\end{itemize}
	\item \textbf{Demanda de productos de software libre} (crecimiento de las \textit{start-ups})
	\item \textbf{Consultas especializadas} no eficientes en el modelo relacional (JOINs)
	\item \textbf{Expresividad, flexibilidad, dinamismo.} Frustración con \textbf{restricciones} del modelo relacional
\end{enumerate}
\subsubsection{Características}
No se basan en SQL

Modelos de datos más ricos

Orientadas a la \textbf{Escalabilidad}

Generalmente no obligan a definir un esquema
\begin{itemize}
	\item \textbf{Schemaless}
\end{itemize}
Surgidos de la comunidad para solucionar problemas
\begin{itemize}
	\item  muchas \textbf{libres/\emph{open source}}
\end{itemize}
Diseño basado en\textbf{ procesamiento distribuido}

Principios funcionales
\begin{itemize}
	\item \textbf{MapReduce}
\end{itemize}
\subsubsubsection{Categorías de NoSQL}
\begin{itemize}
\item Bases de datos \textit{key-value}\\
\item Bases de datos documentales\\
\item Bases de datos columnares (\textit{wide column})\\
\item Bases de datos de grafos\\
\item Bases de datos de arrays
\end{itemize}
\subsubsection{Evolución desde el modelo relacional}
El \textbf{modelo relacional $\Rightarrow$ predominante en los últimos ˜30 años}\\
Tiene sus raíces en el denominado \textit{business data processing}, procesamiento de
transacciones y \textit{batch}\\
Propuesto por Codd en los 70, \textbf{de alto nivel}\\
Actualmente los \textbf{sistemas SQL están muy optimizados:}
\begin{itemize}
	\item el \textbf{grado de implantación es mayoritario}
\item para el 99 \% de los problemas (que caben en un ordenador) es eficiente y adecuado
\end{itemize}
\subsection{Adopción de NoSQL}
\subsubsection{Análisis}
Dominan los grandes SGBDR\\
El \textit{Open Source} tiene una importancia crucial (PostgreSQL, MySQL, MongoDB, etc.)\\
Varias bases de datos NoSQL entre las 10 primeras. Muchas en las 20 primeras\\
La distancia entre los grandes SGBDR y el primer NoSQL (MongoDB) es de $5\times$\\
Paradigmas más "atrevidos" como el de grafos están entre los 20 primeros (Neo4j)\\
\subsection{Cambio de perspectiva: Red}
\begin{center}
	\includegraphics{"Temas/Tema 2/screenshot001"}
\end{center}
\subsubsubsection{Almacenamiento distribuido}
Desde los 90’s: Clústers/NOC/COW: procesamiento masivamente paralelo

Almacenamiento no distribuido

Ahora los nodos $\Rightarrow$ también \textbf{almacenamiento}

Minimizar el verdadero cuello de botella: \textbf{trasiego de información por la red}.

\subsubsubsection{Procesamiento distribuido}
Necesidad de \textbf{paralelización máxima}\\
\textbf{Escalabilidad}\\
Explotar de la \textbf{localidad de los datos}:
\begin{itemize}
	\item Datos producidos se utilizan localmente en siguientes iteraciones
\item Datos recibidos directamente en los \textit{hosts} (clientes simultáneos)
\end{itemize}
Vuelta al modelo funcional inherentemente paralelo: (e.g. \textbf{Map-Reduce})\\
Almacenamiento distribuido: (e.g. \textbf{HDFS})\\
Coordinación distribuida: (e.g. \textbf{Zookeeper})
\subsubsubsection{Modelo de datos}
El modelo relacional limita a tablas con valores primitivos y relaciones \textit{Primary Key/Foreign Key}\\
En programación se utilizan \textbf{listas, arrays, tipos de datos compuestos} (\textit{gap semántico})\\
ACID es \textbf{muy compleja y costosa} en ambientes distribuidos (quizá \textbf{no necesaria} en algunas aplicaciones).