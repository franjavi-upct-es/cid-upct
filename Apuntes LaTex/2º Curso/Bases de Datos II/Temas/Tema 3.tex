\section{Bases de datos basadas en documentos}
\subsection{Introducción a las bases de datos de documentos}
\subsubsection{Bases datos Documentales}
En general, organizadas como un conjunto de \textbf{Colecciones} que contienen \textbf{documentos}:
\begin{itemize}
\item Tablas relacionales $\leftrightarrow$ Colecciones
\item Filas (\textit{rows}) $\leftrightarrow$ Documentos
\item No restringen la estructura de cada documento y no tienen esquema
\end{itemize}
Cada colección es un \textit{array} donde cada documento tiene asociado un identificador

La base de datos puede ver el contenido del documento, y utilizar su información como parte de las búsquedas y actualizaciones

Documentos $\Rightarrow$ formatos jerárquicos tipo JSON o XML

No utilizan SQL y tienen su lenguaje de consulta propio

Utilizan normalmente \textbf{Map-Reduce} para cálculos distribuidos

Algunas implementan otros lenguajes de consulta y procesado, como N1QL (Couchbase) y Aggregation Framework (MongoDB)

Bases de datos de Documentos: Couchbase, MongoDB, OrientDB
\subsubsection{Modelado de bases de datos de documentos}
\subsubsubsection{Modelado de datos en NoSQL}
El modelado de datos debe ser:
\begin{itemize}
\item Realizado al mayor nivel de abstracción posible
\item Independiente de la tecnología subyacente
\end{itemize}
Sin embargo, en NoSQL:
\begin{itemize}
\item  Se tiene que tener en cuenta el diseño \textbf{distribuido}
\item \textbf{Optimización guiada por las consultas}
\end{itemize}
Con respecto al modelo de datos:
\begin{itemize}
\item Se mantienen los conceptos de entidad, relación, cardinalidades, etc.
\item El modelado relacional se centra en especificar \textbf{qué datos tenemos y podemos ofrecer}
\item El modelo NoSQL se centra en \textbf{optimizar qué consultas vamos a servir}
\item Es "barato" \textbf{duplicar (desnormalizar)} los datos si con ello se consigue \textbf{mayor eficiencia de acceso}
\end{itemize}
\subsubsubsection{Representación de CV como tablas}
\begin{center}
	\includegraphics[scale=0.7]{"Temas/Tema 3/screenshot001"}
\end{center}
\subsubsection{Representación de relaciones}
\subsubsubsection{Relaciones uno a muchos}
Las relaciones uno a muchos (por ejemplo en el CV: \texttt{positions}) en el modelo relacional:
\begin{itemize}
	\item Normalización usando varias tablas (\texttt{Positions} con \texttt{user_id})
\begin{itemize}
	\item Necesidad de más de una tabla
\item Necesidad de uso de \texttt{JOIN} $\Rightarrow$ ineficiencia
\end{itemize}
\item Algunos SGBDR ofrecen la posibilidad de tener tipos de datos estructurados y campos XML o JSON. (P. ej. PostgreSQL)
\begin{itemize}
	\item Usualmente no se pueden usar para buscar dentro
\item No son estándar
\end{itemize}
\end{itemize}
\subsubsection{CV como un documento}
\begin{lstlisting}[language=C++]

\end{lstlisting}