\documentclass[12pt]{article}
\usepackage{fullpage}
\usepackage[utf8]{inputenc}
\usepackage{pict2e}
\usepackage{amsmath}
\usepackage{enumitem}
\usepackage{eurosym}
\usepackage{pict2e}
\usepackage{mathtools}
\usepackage{amssymb, amsfonts, latexsym, cancel}
\setlength{\parskip}{0.3cm}
\usepackage{graphicx}
\usepackage{fontenc}
\usepackage{slashbox}
\usepackage{setspace}
\usepackage{gensymb}
\usepackage{accents}
\usepackage{adjustbox}
\setstretch{1.5}
\usepackage{bold-extra}
\usepackage[document]{ragged2e}
\usepackage{subcaption}
\usepackage{tcolorbox}
\usepackage{xcolor, colortbl}
\usepackage{wrapfig}
\usepackage{empheq}
\usepackage{array}
\usepackage{parskip}
\usepackage{arydshln}
\graphicspath{ {images/} }
\renewcommand*\contentsname{\color{black}Índice} 
\usepackage{array, multirow, multicol}
\definecolor{lightblue}{HTML}{007AFF}
\usepackage{color}
\usepackage{etoolbox}
\usepackage{listings}
\usepackage{mdframed}
\setlength{\parindent}{0pt}
\usepackage{underscore}
\usepackage{hyperref}
\usepackage{tikz}
\usepackage{tikz-cd}
\usetikzlibrary{shapes, positioning, patterns}
\usepackage{tikz-qtree}
\usepackage{biblatex}
\usepackage{pdfpages}
\usepackage{pgfplots}
\usepackage{pgfkeys}
\addbibresource{biblatex-examples.bib}
\usepackage[a4paper, left=1.5cm, right=1.5cm, top=1cm,
bottom=1.5cm]{geometry}
\everymath{\displaystyle}
\usetikzlibrary{decorations.pathreplacing}
\usepackage{titlesec}
\usepackage{titletoc}
\usepackage{tikz-3dplot}
\usetikzlibrary{decorations.pathreplacing}
\newcommand{\Ej}{\textcolor{lightblue}{\underline{Ejemplo}}}
\setlength{\fboxrule}{1.5pt}
\renewcommand{\arraystretch}{1.35}
\setlength{\arraycolsep}{0.3cm}

% Configura el formato de las secciones utilizando titlesec
\titleformat{\section}
{\color{red}\normalfont\LARGE\bfseries}
{Tema \thesection:}
{10 pt}
{}

% Ajusta el formato de las entradas de la tabla de contenidos
\addtocontents{toc}{\protect\setcounter{tocdepth}{4}}
\addtocontents{toc}{\color{black}}

\titleformat{\subsection}
{\normalfont\Large\bfseries\color{red}}{\thesubsection)}{1em}{\color{lightblue}}

\titleformat{\subsubsection}
{\normalfont\large\bfseries\color{red}}{\thesubsubsection)}{1em}{\color{lightblue}}

\newcommand{\bboxed}[1]{\fcolorbox{lightblue}{lightblue!10}{$#1$}}

\DeclareMathOperator{\N}{\mathbb{N}}
\DeclareMathOperator{\Z}{\mathbb{Z}}
\DeclareMathOperator{\R}{\mathbb{R}}
\DeclareMathOperator{\Q}{\mathbb{Q}}
\DeclareMathOperator{\K}{\mathbb{K}}
\DeclareMathOperator{\im}{\imath}
\DeclareMathOperator{\jm}{\jmath}
\DeclareMathOperator{\col}{\mathrm{Col}}
\DeclareMathOperator{\fil}{\mathrm{Fil}}
\DeclareMathOperator{\rg}{\mathrm{rg}}
\DeclareMathOperator{\nuc}{\mathrm{nuc}}
\DeclareMathOperator{\dimf}{\mathrm{dimFil}}
\DeclareMathOperator{\dimc}{\mathrm{dimCol}}
\DeclareMathOperator{\dimn}{\mathrm{dimnuc}}
\DeclareMathOperator{\dimr}{\mathrm{dimrg}}

\newcommand{\bu}[1]{\textcolor{lightblue}{\underline{#1}}}
\newcommand{\lb}[1]{\textcolor{lightblue}{#1}}
\newcommand{\db}[1]{\textcolor{blue}{#1}}
\newcommand{\rc}[1]{\textcolor{red}{#1}}
\newcommand{\tr}{^\intercal}

\renewcommand{\CancelColor}{\color{lightblue}}

\newcommand{\dx}{\:\mathrm{d}x}
\newcommand{\dt}{\:\mathrm{d}t}
\newcommand{\dy}{\:\mathrm{d}y}
\newcommand{\dz}{\:\mathrm{d}z}
\newcommand{\dth}{\:\mathrm{d}\theta}
\newcommand{\dr}{\:\mathrm{d}\rho}
\newcommand{\du}{\:\mathrm{d}u}
\newcommand{\dv}{\:\mathrm{d}v}
\newcommand{\tozero}[1]{\cancelto{0}{#1}}
\newcommand{\lbb}[2]{\textcolor{lightblue}{\underbracket[1pt]{\textcolor{black}{#1}}_{#2}}}
\newcommand{\dbb}[2]{\textcolor{blue}{\underbracket[1pt]{\textcolor{black}{#1}}_{#2}}}

\begin{document}
\begin{enumerate}[label=\color{red}\arabic*)]
	\item \lb{Consideremos el problema }
	\begin{enumerate}[label=\color{red}\alph*)]
		\item \db{Comprueba que $(x_1,x_2)=(2,1)$ es la solución de dicho problema}
		$\begin{array}{l}
		f(1,2)=-2+2=0\\
		f(2,1)=-4+1=-3
		\end{array}$
		\item \db{Comprueba que la función dual $\Omega(\lambda)$ viene dada por \[ \begin{cases}
		-4+5\lambda, & \lambda\le-1\\
		-8+\lambda, & -1\le \lambda\le2\\
		-3\lambda, & \lambda\ge2
		\end{cases} \]}
		$\Omega(\lambda)=\inf_{(x_1,x_2)\in X}L(x_1,x_2,\lambda)=\inf_{(x_1,x_2)}\overset{-2x_1+x_2}{f(x_1,x_2)}+\lambda\overset{x_1+x_2-3}{h(x_1,x_2)}$
		
		$\begin{array}{l}
		(x_1,x_2)=(0,0)\longrightarrow L(0,0,\lambda)=-3\lambda\\
		(x_1,x_2)=(0,4)\longrightarrow L(0,4,\lambda)=4+\lambda\\
		(x_1,x_2)=(4,4)\longrightarrow L(4,4,\lambda)=-4+5\lambda\\
		(x_1,x_2)=(4,0)\longrightarrow L(4,0,\lambda)=-8+\lambda\\
		(x_1,x_2)=(1,2)\longrightarrow L(1,2,\lambda)=0\\
		(x_1,x_2)=(2,1)\longrightarrow L(2,1,\lambda)=-3\\
		\end{array}$
		
		\begin{center}
		\begin{tikzpicture}
		\draw[->] (-4,0) -- (6,0) node[right] {$\lambda$};
		\draw[->] (0,-10) -- (0,5) node[above] {$y$};
		
		\draw[domain=-2:4, samples=100,lightblue] plot (\x,{-3*\x});
		\draw[domain=-5:1, samples=100,lightblue] plot (\x,{4+\x});
		\draw[domain=-2:2, samples=100,lightblue] plot (\x,{-4+5*\x});
		\draw[domain=-4:10, samples=100,lightblue] plot (\x,{-8+\x});
		\draw[domain=-1:2, samples=100, blue, line width=1.5] plot (\x,{-8+\x});
		\draw[domain=-2:-1, samples=100, blue, line width=1.5] plot (\x,{-4+5*\x});
		\draw[domain=2:4, samples=100, blue, line width=1.5] plot (\x,{-3*\x});
		\draw[lightblue] (-1,0) -- (5.5, 0) ;
		\draw[lightblue] (-4,-3) -- (5.5, -3);
		\foreach \x in {-4,...,5}{\draw (\x,0.1) -- (\x,-0.1);}
		\foreach \y in {-10,...,4}{\draw (0.1,\y) -- (-0.1,\y);}
		\end{tikzpicture}
		\end{center}
	\end{enumerate}
	\item \lb{Consideremos el problema \[ (Primal)\begin{cases}
	\text{Minimizar en }(x_1,x_2) & f(x_1,x_2)=x_1+x_2+\dfrac{1}{2}(x_1^2+x_2^2)\\
	\text{Sujeto a} & x_1+x_2\ge1
	\end{cases} \]Se pide:}
	\begin{enumerate}[label=\color{red}\alph*)]
		\item \db{Escribe y resuelve las condiciones necesarias de optimizalidad de Karush-Kuhn-Tucker del problema anterior.}
		Escribimos el problema en forma estándar \[ g(x_1,x_2)=1-x_1-x_2\le0 \]
		$\begin{cases}
			\text{Minimizar en }(x_1,x_2) & f(x_1,x_2)=x_1+x_2+\dfrac{1}{2}(x_1^2+x_2^2)\\
			\text{Sujeto a} & 1-x_1-x_2\le0
			\end{cases}$
			
		$\begin{array}{l}
		\nabla f=(1+x_1, 1+x_2)\\
		\nabla g=(-1, -1)\longleftarrow\mu\text{ La condición rango maximal de $\nabla g$ se cumple}\\
		(\mathrm{KKT})\begin{cases}
		\begin{rcases}
		1+x_1-\mu=0\\
		1+x_2-\mu=0\\
		\end{rcases}\longrightarrow x_1=x_2=\mu -1\longrightarrow\mu-1+ \mu-1\longrightarrow2\mu=3\longrightarrow\mu=\dfrac{3}{2}\\
		\mu(1-x_1-x_2)=0\\
		\mu\ge0\\
		x_1+x_2\ge 1
		\end{cases}
		\end{array}$
		
		\underline{Casos:}
		\begin{enumerate}[label=\arabic*º)]
			\item $\mu=0\longrightarrow x_1=x_2=-1\quad\overset{\text{No}}{\fbox{$-1-1\ge1$}}$
			\item $\mu\neq0\longrightarrow x_1+x_2=1$
		\end{enumerate}
		$x_1=x_2=\dfrac{3}{2}-1=\dfrac{1}{2}$
		
		$L(x_1,x_2,\mu)=x_1+x_2+\dfrac{1}{2}(x_1^2+x_2^2)+\mu\cdot(1-x_1-x-2)$
	\end{enumerate}
\end{enumerate}
\end{document}