\includepdf{"Tareas/Tema 1/Hoja 1"}

\begin{enumerate}[label=\color{red}\arabic*), leftmargin=*]
	\item \lb{Sea $(X,Y)$ un \vea con función de densidad conjunta \[f(x,y)=\begin{cases}
	1 & \text{si }0<x<1,\;0<y<1\\
	0 & \text{en otro caso}
	\end{cases}\] Hallar las distribuciones marginales y condicionadas}
	
	$\underset{0<x<1}{f_X(x)=}\int_{\infty}^{+\infty}f(x,y)\dy=\cancel{\int_{-\infty}^{0}0\dy}+\int_{0}^{1}1\dy+\cancel{\int_{0}^{+\infty}0\dy}=\left[y\right]_{y=0}^{y=1}=1\qquad f_X(x)=\begin{cases}
	1 & \text{si }0<x<1\\
	0 & \text{en otro caso}
	\end{cases}$
	
	$\underset{0<y<1}{f_Y(y)=}\int_{-\infty}^{+\infty}f(x,y)\dx=\int_{0}^{1}1\dx=\left[x\right]_{x=0}^{x=1}=1\qquad f_Y(y)=\begin{cases}
	1 & \text{si }0<y<1\\
	0 & \text{en otro caso}
	\end{cases}$
	
	$f(x,y)=\begin{cases}
	1 & 0<x<1,\;0<y<1\\
	0 & \text{en otro caso}
	\end{cases}$
	
	$\underset{f_X(x^*)>0}{y|x=x^*}\longrightarrow f_{\begin{subarray}{l}
	y|x=x^*\\
	0<x^*<1
	\end{subarray}}=\dfrac{f(x^*,y)}{f_X(x^*)}=\begin{cases}
	1 & 0<y<1\\
	0 & \text{en otro caso}
	\end{cases}$
	
	$f(x,y)=f_X(x)\cdot f_Y(y)$\quad $X$ e $Y$ independientes
	
	\item \lb{Obtener las distribuciones marginales y condicionadas asociadas al vector aleatorio $(X,Y)$ con función de densidad \[ f(x,y)=\begin{cases}
	2 & \text{si }0<x<1,\;0<y<x\\
	0 & \text{en otro caso}
	\end{cases} \]}
	
	$\underset{0<x<1}{f_X(x)=}\int_{-\infty}^{+\infty}f(x,y)\dy=\int_{0}^{x}2\dy=\left[2y\right]_{y=0}^{y=x}=2x\longrightarrow\begin{cases}
	2x & \text{si }0<x<1\\
	0 & \text{en otro caso}
	\end{cases}$
	
	$\underset{0<y<1}{f_Y(y)=}\int_{-\infty}^{+\infty}f(x,y)\dx=\int_{y}^{1}2\dx=[2x]_{x=y}^{x=1}=2-2y\longrightarrow\begin{cases}
	2-2y & \text{si }0<y<1\\
	0 & \text{en otro caso}
	\end{cases}$
	
	Los recintos son dependientes.
	
	$\begin{array}{l}
	y|x=x^*\\
	f_{X}(x^*)>0\\
	\underset{0<x^*<1}{f_{y|x=x^*}}(y|x^*)=\dfrac{f(x^*,y)}{f(x^*)}=\begin{cases}
	\dfrac{2}{2x^*} & 0<y<x^*\\
	0 & \text{en otro caso}
	\end{cases}=\begin{cases}
		\dfrac{1}{x^*} & 0<y<x^*\\
		0 & \text{en otro caso}
		\end{cases}
	\end{array}$
	
	$\begin{array}{l}
	x|y=y^*\\
	f_Y(y^*)>0\\
	\underset{0<y^*<1}{f_{x|y=y^*}}(x|y^*)=\dfrac{f(x,y^*)}{f_Y(y^*)}=\begin{cases}
	\dfrac{2}{2-2y^*} & \text{si } y^*<x<1\\
	0 & \text{en otro caso}
	\end{cases}=\begin{cases}
	\dfrac{1}{1-y^*} & \text{si }y^*<x<1\\
	0 & \text{en otro caso}
	\end{cases}
	\end{array}$
	
	\begin{tikzpicture}[scale=2]
	\fill[lightblue!10] (0,0) -- (1,1) -- (1,0) -- cycle;
	\draw (-0.5,0) -- (2,0);
	\draw (0,-0.5) -- (0,2);
	\draw[lightblue, domain=-0.5:2] plot (\x,\x);
	\draw[lightblue, dashed] (0,1) node[left] {1} -- (1,1) -- (1,0) node[below ] {1};
	\end{tikzpicture}
	
	\item \lb{Sea $(X,Y)$ un \vea con función de densidad \[ f(x,y)=\begin{cases}
	\dfrac{3}{4}\left[xy+\dfrac{x^2}{2}\right] & \text{si }0<x<1,\;0<y<2\\
	0 & \text{en otro caso}
	\end{cases} \]Hallar la distribución marginal de $X$ y la distribución de $Y$ condicionada a $X=\dfrac{1}{2}$.}
	
	
	$\underset{0<x<1}{f_X(x)=}\int_{-\infty}^{+\infty}f(x,y)\dy=\int_{0}^{2}\dfrac{3}{4}\left[xy+\dfrac{x^2}{2}\right]=\dfrac{3}{4}\left[\dfrac{xy^2}{2}+\dfrac{x^2}{2}\cdot y\right]_{y=0}^{y=2}=\dfrac{3}{4}\left(2x+x^2\right)\longrightarrow\begin{cases}
	\dfrac{3}{4}\left(2x+x^2\right) & \text{si }0<x<1\\
	0 & \text{en otro caso}
	\end{cases}$
	
	$\begin{array}{l}
	y|x=x^*\\
	\underset{0<x^*<1}{f_{y|x=x^*}}(y|x^*)=\dfrac{f(x^*,y)}{f(x^*)}=\dfrac{\frac{3}{4}\left(x^*y+\frac{(x^*)^2}{2}\right)}{\frac{3}{4}\left(2x^*+(x^*)^2\right)}=\dfrac{x^*y+\frac{(x^*)^2}{2}}{2x^*+\frac{(x^*)^2}{2}}=\dfrac{x^*y+(x^*)^2}{4x^*+2(x^*)^2}\xrightarrow{x^*=\frac{1}{2}}\dfrac{\frac{1}{2}y+\frac{1}{8}}{2\cdot\frac{1}{2}+\frac{1}{4}}=2\cdot\dfrac{y+\frac{1}{4}}{5}\\
	\begin{cases}
	2\cdot\dfrac{y+\frac{1}{4}}{5} & \text{si }0<y<2\\
	0 & \text{en otro caso}
	\end{cases}
	\end{array}$
	\item \lb{Sea $X=(X_1,X_2)$ un \vea con función masa de probabilidad \[ P[X_1=x_1,X_2=x_2]=\dfrac{k}{2^{x_1+x_2}},x_1,x_2\in\N \]donde $k$ es una constante. Obtener las distribuciones marginales y condicionadas.}
	
	$P[X_1=x_1,X_2=x_2]=\dfrac{k}{2^{x_1+x_2}},\:x_1,x_2\in\N$ (incluido el 0)
	
	$\underset{x_1\in\N}{P[X_1=x_1]}=\sum_{x_2\in \N}\dfrac{k}{2^{x_1+x_2}}=\dfrac{k}{2^{x_1}}\sum_{x_2\in\N}\dfrac{1}{2^{x_2}}=\dfrac{k}{2^{x_1}}\cdot\dfrac{1}{1-\frac{1}{2}}=\dfrac{2k}{2^{x_1}}$
	
	$\underset{x_1^*\in\N}{P[X_2=x_2|X_1=x_1^*]}=\dfrac{P[X_1=x_1^*,X_2=x_2]}{P[X_1=x_1]}=\begin{cases}
	\dfrac{\frac{k}{2^{x_1+x_2}}}{\frac{2k}{2^{x_1}}} = \dfrac{1}{2\cdot 2^{x_2}} & x_2\in\N\\
	0 & \text{en otro caso}
	\end{cases}$

	\item \lb{Calcular la función de densidad de una distribución normal bidimensional en $(1,1)$ si las medias son cero, las varianzas 1 y 4, y la covarianza 1.}
	
	Fórmula de la función de densidad de una distribución normal bidimensional: \[ f(x,y)=\dfrac{1}{2\pi\sigma_x\sigma_y\sqrt{1-\rho^2}}\exp\left(-\dfrac{1}{2(1-\rho^2)}\left[\dfrac{(x-\mu_x)^2}{\sigma_x^2}+\dfrac{(y-\mu_y)^2}{\sigma_y^2}-\dfrac{2\rho(x-\mu_x)(y-\mu_y)}{\sigma_x\sigma_y}\right]\right) \]
	
	$\begin{array}{l}
	\mu_x = \mu_y = 0\\
	\sigma^2_x = 1\\
	\sigma^2_y = 4\\
	\rho=\dfrac{\sigma_{xy}}{\sigma_x\cdot\sigma_y}=\dfrac{1}{\sqrt{1}\cdot\sqrt{4}}=\dfrac{1}{2}
	\end{array}\qquad \begin{aligned}
	f(1,1)&=\dfrac{1}{2\pi\cdot1\cdot2\sqrt{1-\left(\frac{1}{2}\right)^2}}\exp\left(-\dfrac{1}{2\left(1-\left(\frac{1}{2}\right)^2\right)}\cdot\left[1^2+\dfrac{1^2}{4}-\dfrac{2\cdot\frac{1}{2}}{1\cdot2}\right]\right)\\
	&=\dfrac{1}{2\pi\sqrt{3}}\exp\left(-\dfrac{2}{3}\cdot\dfrac{3}{4}\right)\\
	&=\dfrac{1}{2\pi\sqrt{3}}\exp\left(-\dfrac{1}{2}\right)\simeq\bboxed{0.0557}
	\end{aligned}$
	
	\vspace{0.5cm}
	
	\hrule
	
	$\underset{x\in\R^k}{f(x)=}\dfrac{1}{|V|(2\pi)^k}e^{-\frac{1}{2}(x-\mu)^\intercal\Sigma^{-1}(x-\mu)}$
	
	$V=\begin{pmatrix}
	1 & 1\\
	1 & 4
	\end{pmatrix}\qquad|V|=3$
	
	$\mathrm{Adj}(V^\intercal)=\begin{pmatrix}
	4 & -1\\
	-1 & 1
	\end{pmatrix}\qquad V^{-1}=\dfrac{1}{|V|}\mathrm{Adj}(V^\intercal)=\begin{pmatrix}
	\tfrac{4}{3} & -\tfrac{1}{3}\\
	-\tfrac{1}{3} & \tfrac{1}{3}
	\end{pmatrix}$
	
	$\begin{pmatrix}
	x-0 & y-0
	\end{pmatrix}\cdot\begin{pmatrix}
		\tfrac{4}{3} & -\tfrac{1}{3}\\
		-\tfrac{1}{3} & \tfrac{1}{3}
		\end{pmatrix}\cdot\begin{pmatrix}
		x-0\\
		y-0
		\end{pmatrix}=\begin{pmatrix}
		\dfrac{4}{3}x\dfrac{y}{3} & -\dfrac{x}{3}+\dfrac{y}{3}
		\end{pmatrix}\cdot\begin{pmatrix}
		x\\
		y
		\end{pmatrix}=\dfrac{4}{3}x^2-\dfrac{xy}{3}-\dfrac{x}{y}+\dfrac{y^2}{3}$
		
		$f(x,y)=\dfrac{1}{\sqrt{3(2\pi)^k}}\cdot e^{-\frac{1}{2}\left(\frac{4}{3}x^2-\frac{2xy}{3}-\frac{x}{y}+\frac{y^2}{3}\right)}\longrightarrow f(1,1)\simeq0.0557$
	\item \lb{Sea $(X,Y)$ un \vea con distribución uniforme en el cuadradado unidad, $[0,1]\times[0,1]$, con función de densidad conjunta \[ f(x,y)=\begin{cases}
	1 & \text{si }0<x<1,\;0<y<1\\
	0 & \text{en otro caso}
	\end{cases} \]Calcular el valor esperado de $g(X,Y)=XY^2$, es decir, $E[XY^2]$.}
	
	El valor esperado de una función $g(X,Y)$ para una variable aleatoria conjunta $(X,Y)$ se define como:
	
	$$E[g(X,Y)] = \int_{-\infty}^{\infty} \int_{-\infty}^{\infty} g(x,y) f(x,y) \dx \dy$$
	
	En este caso, $g(X,Y) = XY^2$ y la función de densidad conjunta $f(x,y)$ es 1 para $0<x<1$ y $0<y<1$, y 0 en otro caso. Por lo tanto, el valor esperado se convierte en:
	
	$$E[XY^2] = \int_{0}^{1} \int_{0}^{1} xy^2 \dx \dy$$
	
	Resolviendo la integral obtenemos:
	
	$$E[XY^2] = \int_{0}^{1} \left[ \frac{1}{2}x^2y^2 \right]_{0}^{1} \dy = \int_{0}^{1} \frac{1}{2}y^2 \dy = \left[ \frac{1}{6}y^3 \right]_{0}^{1} = \frac{1}{6}$$
	
	Por lo tanto, el valor esperado de $XY^2$ es $\frac{1}{6}$.
	
	\hrule
	
	$\begin{array}{l}
	E[XY^2]=E[X]\cdot E[Y^2]=\lb{(\ast)}=\dfrac{1}{2}\cdot\dfrac{1}{3}=\dfrac{1}{6}\\
	E[X]=\int_{0}^{1}x\dx=\left[\dfrac{x^2}{2}\right]_0^1=\dfrac{1}{2}\\
	E[Y^2]=\int_{0}^{1}y^2\dy=\left[\dfrac{y^3}{3}\right]_0^1=\dfrac{1}{3}
	\end{array}$
	\item \lb{$(X,Y)$ vector aleatorio discreto con función masa de probabilidad conjunta: \begin{center}
	\begin{tabular}{|c|c|c|}
	\hline
	\backslashbox{X}{Y} & 1 & 2\\ \hline
	1 & $\tfrac{1}{9}$ & $\tfrac{2}{9}$\\
	2 & $\tfrac{2}{9}$ & $\tfrac{4}{9}$ \\ \hline
	\end{tabular}
	\end{center}}
	\begin{enumerate}[label=\color{red}\alph*)]
		\item \db{Calcular $E[X+Y],E[2X+3Y]$.}
		
		$\begin{array}{l}
		E[X+Y]=E[X]+E[Y]=\dfrac{5}{3}+\dfrac{5}{3}=\dfrac{10}{3}\\
		E[X]=1\cdot P(X=1)+2\cdot P(X=2)=1\cdot\left(\dfrac{1}{9}+\dfrac{2}{9}\right)+2\cdot\left(\dfrac{2}{9}+\dfrac{4}{9}\right)=\dfrac{1}{3}+\dfrac{4}{3}=\dfrac{5}{3}\\
		E[Y]=1\cdot P(Y=1)+2\cdot P(Y=2)=1\cdot\left(\dfrac{1}{9}+\dfrac{2}{9}\right)+2\cdot\left(\dfrac{2}{9}+\dfrac{4}{9}\right)=\dfrac{1}{3}+\dfrac{4}{3}=\dfrac{5}{3}
		\end{array}$
		
		$\begin{array}{l}
		E[2X+3Y]=2E[X]+3E[Y]=2\cdot\dfrac{5}{3}+3\cdot\dfrac{5}{3}=\dfrac{25}{3}
		\end{array}$
		
		\item \db{Obtener el vector de medias, la matriz de covarianzas y la matriz de correlaciones del vector $(X,Y)$.}
		\begin{itemize}[label=\color{lightblue}\textbullet, leftmargin=*]
		\item Vector de medias: \[ \mu=\begin{bmatrix}
		E[X]\\
		E[Y]
		\end{bmatrix}=\begin{bmatrix}
		\tfrac{5}{3}\\
		\tfrac{5}{3}
		\end{bmatrix}=\dfrac{5}{3}\begin{bmatrix}
		1\\
		1
		\end{bmatrix} \]
		\item Matriz de covarianzas: \[ \Sigma=\begin{bmatrix}
		\var(X) & \cov(X,Y)\\
		\cov(Y,X) & \var(Y)
		\end{bmatrix}=\begin{bmatrix}
		\tfrac{2}{9} & 0\\
		0 & \tfrac{2}{9}
		\end{bmatrix} \]
		$\begin{array}{l}
		\var(X)=E[X^2]-(E[X])^2=3-\left(\dfrac{5}{3}\right)^2=\dfrac{2}{9}\\
		\cov(X,Y)=E[XY]-E[X]E[Y]=\dfrac{25}{9}-\dfrac{5}{3}\cdot\dfrac{5}{3}=0\\
		\var(Y)=E[Y^2]-(E[Y])^2=3-\left(\dfrac{5}{3}\right)^2=\dfrac{2}{9}\\
		E[X^2]=1^2\cdot P(X=1)+2^2\cdot P(X=2)=1^2\cdot\left(\dfrac{1}{9}+\dfrac{2}{9}\right)+2^2\cdot\left(\dfrac{2}{9}+\dfrac{4}{9}\right)=\dfrac{1}{3}+\dfrac{8}{3}=3\\
		E[Y^2]=1^2\cdot P(Y=1)+2^2\cdot P(Y=2)=1^2\cdot\left(\dfrac{1}{9}+\dfrac{2}{9}\right)+2^2\cdot\left(\dfrac{2}{9}+\dfrac{4}{9}\right)=\dfrac{1}{3}+\dfrac{8}{3}=3\\
		\end{array}$
		
		$E[XY]=1\cdot1\cdot P(X=1,Y=1)+1\cdot 2\cdot P(X=1,Y=2)+2\cdot1\cdot P(X=2,Y=1)+2\cdot2\cdot P(X=2,Y=2)=\dfrac{1}{9}+2\cdot\dfrac{2}{9}+2\cdot\dfrac{2}{9}+4\cdot\dfrac{4}{9}=\dfrac{25}{9}$
		
		\item Matriz de correlaciones: \[ R=\begin{bmatrix}
		1 & \corr(X,Y)\\
		\corr(Y,X) & 1
		\end{bmatrix}=\begin{bmatrix}
		1 & 0\\
		0 & 1
		\end{bmatrix}=I \]
		$\corr(X,Y)=\dfrac{\cov(X,Y)}{\sqrt{\var(X)\cdot\var(Y)}}=\dfrac{0}{\sqrt{\frac{2}{9}\cdot\frac{2}{9}}}=0$
		\end{itemize}
		
		\item \db{¿Son independientes? ¿Están incorreladas?}
		
		Las \vas $X$ e $Y$ serán independientes si se cumple la condición: $P(X=x,Y=y)=P(X=x)\cdot P(Y=y)\forall x,y\in\N$
		
		$P(X=1,Y=1)=P(X=1)\cdot P(Y=1)\longrightarrow \dfrac{1}{9}=\left(\dfrac{1}{9}+\dfrac{2}{9}\right)\cdot\left(\dfrac{1}{9}+\dfrac{2}{9}\right)\longrightarrow\dfrac{1}{9}=\dfrac{1}{3}\cdot\dfrac{1}{3}$
		
		Por lo tanto, son independientes.
		
		Las \vas $X$ e $Y$ están incorreladas si su covarianza vale 0. En este caso sí están incorreladas.
	\end{enumerate}
	\item \lb{Demostrar que el vector de medias muestral es el punto $\R^k$ que minimiza la suma de las distancias al cuadrado (error cuadrático medio, MSE).}
\end{enumerate}