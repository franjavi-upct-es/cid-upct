\includepdf[pages=-]{"Tareas/Tema 5/Hoja 5"}
\begin{enumerate}[label=\color{red}\textbf{\arabic*)}, leftmargin=*]
	\item \lb{Dadas tres poblaciones normales bidimensionales con medias $\mu^{(1)})(1,0)',\,\mu^{(2)}=(0,1)'$ y $\mu^{(3)}=(0,0)'$ y matrices de covarianzas iguales a \[ V=\begin{pmatrix}
			2 & 1\\
			1 & 1
		\end{pmatrix} \]se pide:}
	\begin{enumerate}[label=\color{red}\alph*)]
		\item \db{Obtener las funciones discriminantes lineales.}
		
		$\begin{array}{l}
			\mu^{(1)}=\begin{pmatrix}
				1\\
				0
			\end{pmatrix}\\
			\mu^{(2)}=\begin{pmatrix}
				0\\
				1
			\end{pmatrix}\\
			\mu^{(3)}=\begin{pmatrix}
				0\\
				0
			\end{pmatrix}\\
			V=\begin{pmatrix}
				2 & 1\\
				1 & 1
			\end{pmatrix}\\
			\left|V\right|=2-1=1\\
			V^{-1}=\begin{pmatrix}
				1 &-1\\
				-1 & 2
			\end{pmatrix}
		\end{array}$
		
		$L_i(z)=\left(\mu^{(i)}\right)'V^{-1}z-\dfrac{1}{2}\left(\mu^{(i)}\right)'V^{-1}\mu^{(i)}$
		
		$L_1(z)=\begin{pmatrix}
			1 & 0
		\end{pmatrix}\begin{pmatrix}
		1 & -1\\
		-1 & 2
		\end{pmatrix}\begin{pmatrix}
		z_1\\
		z_2
		\end{pmatrix}-\dfrac{1}{2}\begin{pmatrix}
		1 & 0
		\end{pmatrix}\begin{pmatrix}
		1 & -1\\
		-1 & 2
		\end{pmatrix}\begin{pmatrix}
		1\\
		0
		\end{pmatrix}=\begin{pmatrix}
		1 & -1
		\end{pmatrix}\cdot\begin{pmatrix}
		z_1\\
		z_2
		\end{pmatrix}-\dfrac{1}{2}\begin{pmatrix}
		1 & -1
		\end{pmatrix}\begin{pmatrix}
		1\\
		0
		\end{pmatrix}=z_1-z_2-\dfrac{1}{2}$\\
		$L_2(z)=\begin{pmatrix}
			0 & 1
		\end{pmatrix}\begin{pmatrix}
		1 & -1\\
		-1 & 2
		\end{pmatrix}\begin{pmatrix}
		z_1\\
		z_2
		\end{pmatrix}-\dfrac{1}{2}\begin{pmatrix}
		0 & 1
		\end{pmatrix}\begin{pmatrix}
		1 & -1\\
		-1 & 2
		\end{pmatrix}\begin{pmatrix}
		0\\
		1
		\end{pmatrix}=\begin{pmatrix}
		-1 & 2
		\end{pmatrix}\cdot\begin{pmatrix}
		z_1\\
		z_2
		\end{pmatrix}-\dfrac{1}{2}\begin{pmatrix}
		-1 & 2
		\end{pmatrix}\begin{pmatrix}
		0\\
		1
		\end{pmatrix}=-z_1+2z_2-1$\\
		$L_3(z)=\begin{pmatrix}
			0 & 0
		\end{pmatrix}\begin{pmatrix}
		1 & -1\\
		-1 & 2
		\end{pmatrix}\begin{pmatrix}
		z_1\\
		z_2
		\end{pmatrix}-\dfrac{1}{2}\begin{pmatrix}
		0 & 0
		\end{pmatrix}\begin{pmatrix}
		1 & -1\\
		-1 & 2
		\end{pmatrix}\begin{pmatrix}
		0\\
		0
		\end{pmatrix}=0$\\
		Criterio: $\mathbf{z}$ se clasificará en la población donde $L_i(\mathbf{z})$ sea máxima
		\item \db{Clasificar a $\mathbf{z}=(2,2)'$}
		
		Para $\mathbf{z}=(2,2)'\longrightarrow\begin{cases}
			L_1(\mathbf{z})=-\frac{1}{2}\\
			L_2(\mathbf{z})=1\\
			L_3(\mathbf{z})=0
		\end{cases}$
		
		Como $L(\mathbf{z})=-2<-\dfrac{1}{2}$, clasificamos el individuo $\mathbf{z}$ en la población $Y$.
		\item \db{Dibujar las regiones de clasificación para cada grupo}
		
		$\begin{aligned}
			L_1(z)=L_2(z)\longrightarrow& z_1-z_2-\dfrac{1}{2}=-z_1+2z_2-1\\
			&-z_2-2z_2=-z_1-z_1-1+\dfrac{1}{2}\\
			&-3z_2=-2z_1-\dfrac{1}{2}\\
			&z_2=\dfrac{2}{3}z_1+\dfrac{1}{6}
		\end{aligned}$\\
		$\begin{array}{l}
			L_2(z)=L_3(z)=0\\
			-z_1+2z_2-1=0\\
			z_2=\dfrac{z_1}{2}+\dfrac{1}{2}
		\end{array}$\\
		$\begin{array}{l}
			L_1(z)=L_3(z)=0\\
			z_1-z_2-\dfrac{1}{2}=0\\
			z_2=z_1-\dfrac{1}{2}
		\end{array}$
		
		\begin{tikzpicture}[scale=1.5]
			\draw[-latex] (-1,0) -- (4,0) node[right] {$z_1$};
			\draw[-latex] (0,-1) -- (0,4) node[above] {$z_2$};
			\foreach \x in {-1, ..., 3}{\draw (\x,0.1) -- (\x,-0.1);}
			\foreach \y in {-1, ..., 3}{\draw (0.1,\y) -- (-0.1,\y);}
			\fill[lightblue] (0,0) circle (2pt) node[below left] {$\mu^{(3)}$};
			\fill[lightblue] (0,1) circle (2pt) node[left] {$\mu^{(2)}$};
			\fill[lightblue] (1,0) circle (2pt) node[below] {$\mu^{(1)}$};
			\draw[lightblue, domain=-1:4] plot (\x,{(2/3)*\x+1/6}) node[right] {$\begin{array}{l}
					L_1=L_2\\
					z_2=\dfrac{2}{3}z_1+\dfrac{1}{6}
				\end{array}$};
			\draw[red, domain=-1:4] plot (\x,{\x-1/2}) node[above right] {$\begin{array}{l}
					L_1=L_3\\
					z_2=z_1-\dfrac{1}{2}
				\end{array}$};
			\draw[olive, domain=-1:4] plot (\x, {(\x+1)/2}) node[below right] {$\begin{array}{l}
					L_2=L_3\\
					z_2=\dfrac{z_1}{2}+\dfrac{1}{2}
				\end{array}$};
		\end{tikzpicture}
		
	
	\end{enumerate}
	\item \lb{Dadas tres poblaciones normales bidimensionales con medias $\mu^{(1)}=(0,0)',\,\mu^{(2)}=(1,1)'$ y $\mu^{(3)}=(2,0)'$ y matrices de covarianzas iguales a \[ V=\begin{pmatrix}
			5 & 2\\
			2 & 1
		\end{pmatrix}, \]se pide:}
	\begin{enumerate}[label=\color{red}\alph*)]
		\item \db{Obtener las funciones discriminantes lineales.}
		
		$\begin{array}{l}
			\mu^{(1)}=\begin{pmatrix}
				0\\
				0
			\end{pmatrix}\\
			\mu^{(2)}=\begin{pmatrix}
				1\\
				1
			\end{pmatrix}\\
			\mu^{(3)}=\begin{pmatrix}
				2\\
				0
			\end{pmatrix}\\
			V=\begin{pmatrix}
				5 & 2\\
				2 & 1
			\end{pmatrix}\\
			\left|V\right|=5-4=1\\
			V^{-1}=\begin{pmatrix}
				1 &-2\\
				-2 & 5
			\end{pmatrix}
		\end{array}$
		
		$\begin{array}{l}
			L_i(\mathbf{z})=\left(\mu^{(i)}\right)'V^{-1}\mathbf{z}-\dfrac{1}{2}\left(\mu^{(i)}\right)'V^{-1}\mu^{(i)}\qquad\text{(FDL)}\\
			L_1(\mathbf{z})=\begin{pmatrix}
				0 & 0
			\end{pmatrix}\begin{pmatrix}
			1 & -2\\
			-2  & 5
			\end{pmatrix}\mathbf{z}-\dfrac{1}{2}\begin{pmatrix}
			0 & 0
			\end{pmatrix}\begin{pmatrix}
			1 & -2\\
			-2 & 5
			\end{pmatrix}\begin{pmatrix}
			0\\
			0
			\end{pmatrix}=0\\
			L_2(\mathbf{z})=\begin{pmatrix}
				1 & 1
			\end{pmatrix}\begin{pmatrix}
			1 & -2\\
			-2  & 5
			\end{pmatrix}\mathbf{z}-\dfrac{1}{2}\begin{pmatrix}
			1 & 1
			\end{pmatrix}\begin{pmatrix}
			1 & -2\\
			-2 & 5
			\end{pmatrix}\begin{pmatrix}
			1\\
			1
			\end{pmatrix}=-z_1+3z_2=1\\
			L_3(\mathbf{z})=\begin{pmatrix}
				2 & 0
			\end{pmatrix}\begin{pmatrix}
			1 & -2\\
			-2  & 5
			\end{pmatrix}\mathbf{z}-\dfrac{1}{2}\begin{pmatrix}
			2 & 0
			\end{pmatrix}\begin{pmatrix}
			1 & -2\\
			-2 & 5
			\end{pmatrix}\begin{pmatrix}
			2\\
			0
			\end{pmatrix}=2z_1-4z_2-2\\
		\end{array}$
		
		Criterio: $z$ se clasificará en la población donde $L_i(\mathbf{z})$ sea máxima.
		
		\item \db{Clasificar a $\mathbf{z}=\left(1,\frac{1}{2}\right)'$.}
		
		Para $z=\left(1,\dfrac{1}{2}\right)'\longrightarrow\begin{cases}
			L_1(\mathbf{z})=0\\
			L_2(\mathbf{z})=-1+\dfrac{3}{2}-1=-\dfrac{1}{2}\\
			L_3(\mathbf{z})=-2-\dfrac{4}{2}-2=-2
		\end{cases}\qquad$ Se clasifica en la población (1)
		\item \db{Obtener la función discriminante de Fisher, la constante $K$ y el criterio de clasificación para distinguir entre las poblaciones 2 y 3.}
		\begin{itemize}[label=]
			\item Función discriminante de Fisher: $D=L(\mathbf{z})=\mathbf{a'Z}=(\mu^{2}-\mu^{(3)})'V^{-1}\mathbf{z}$ \[ L_(\mathbf{z})=\begin{pmatrix}
				-1 & 1
			\end{pmatrix}\begin{pmatrix}
			1 & -2\\
			-2 & 5
			\end{pmatrix}\mathbf{z}=\begin{pmatrix}
			-3 & 7
			\end{pmatrix}\begin{pmatrix}
			z_1\\
			z_2
			\end{pmatrix}=-3z_1+7z_2 \] $k=L\left(\dfrac{\mu^{(2)}+\mu^{(3)}}{2}\right)=L\left((1.5, 0.5)'\right)=-3\cdot1.5+7\cdot0.5=-1$\\
			$\begin{array}{l}
				R_{\mu^{(2)}}=\left\{\mathbf{z}:L(\mathbf{z})>k\right\}=\left\{z:-3z_1+7z_2>-1\right\}\\
				R_{\mu^{(3)}}=\left\{\mathbf{z}:L(\mathbf{z})<k\right\}=\left\{z:-3z_1+7z_2<-1\right\}\\
			\end{array}$
			
			Utilizando las funciones discriminantes lineales del apartado (a) \[ \begin{array}{lc}
				L_2(z)=L_2(z)\longrightarrow&-z_1+3z_2-1=2z_1+4z_2-2\\
				&-3z_1+7z_1=-1
			\end{array} \]
		\end{itemize}
		\item \db{Dibujar las regiones de clasificación para cada grupo.}
		
		$\begin{array}{l}
			L_1(z)=L_2(z)\longleftrightarrow z_2=\dfrac{1}{3}z_1+\dfrac{1}{3}\\
			L_1(z)=L_3(z)\longleftrightarrow z_2=\dfrac{1}{2}z_1-\dfrac{1}{2}\\
			L_2(z)=L_3(z)\longleftrightarrow z_2=\dfrac{3}{7}z_1-\dfrac{1}{7}
		\end{array}$
		
		\begin{tikzpicture}
			\draw[-latex] (-1,0) -- (10,0) node[right] {$z_1$};
			\draw[-latex] (0,-1) -- (0,4) node[above] {$z_2$};
			\foreach \x in {-1, ..., 9}{\draw (\x,0.1) -- (\x,-0.1);}
			\foreach \y in {-1, ..., 3}{\draw (0.1,\y) -- (-0.1,\y);}
			\fill[lightblue] (2,0) circle (2pt) node[below] {$\mu^{(3)}$};
			\fill[lightblue] (1,1) circle (2pt) node[left] {$\mu^{(2)}$};
			\fill[lightblue] (0,0) circle (2pt) node[below left] {$\mu^{(1)}$};
			\draw[lightblue, domain=-1:10] plot (\x,{\x/3+1/3}) node[right] {$L_1=L_2$};
			\draw[red, domain=-1:10] plot (\x,{\x/2-1/2}) node[right] {$L_1=L_3$};
			\draw[olive, domain=-1:10] plot (\x, {(3/7)*\x-1/7}) node[right] {$L_2=L_3$};
		\end{tikzpicture}
	\end{enumerate}
	\item \lb{Dadas tres poblaciones normales con matriz de covarianzas común \[ V=\begin{pmatrix}
			6 & 2\\
			2 & 1
		\end{pmatrix} \]y medias $(0,1),\,(1,0)$ y $(1,1)$, respectivamente, obtener las funciones discriminantes y el criterio de clasificación.}
		
		$\begin{array}{l}
		\mu^{(1)}=\begin{pmatrix}
		0\\
		1
		\end{pmatrix}\\
		\mu^{(2)}=\begin{pmatrix}
		1\\
		0
		\end{pmatrix}\\
		\mu^{(3)}=\begin{pmatrix}
		1\\
		1
		\end{pmatrix}\\
		V=\begin{pmatrix}
					6 & 2\\
					2 & 1
				\end{pmatrix}\longrightarrow V^{-1}=\begin{pmatrix}
				\tfrac{1}{2} & -1\\
				-1 & 3
				\end{pmatrix}\\
		L_i(\mathbf{z})=\left(\mu^{(i)}\right)'V^{-1}\mathbf{z}-\dfrac{1}{2}\left(\mu^{(i)}\right)'V^{-1}\mu^{(i)}\\
		L_1(\mathbf{z})=\begin{pmatrix}
		0 & 1
		\end{pmatrix}\begin{pmatrix}
						\tfrac{1}{2} & -1\\
						-1 & 3
						\end{pmatrix}\begin{pmatrix}
						x\\
						y
						\end{pmatrix}-\dfrac{1}{2}\begin{pmatrix}
						0 & 1
						\end{pmatrix}\begin{pmatrix}
										\tfrac{1}{2} & -1\\
										-1 & 3
										\end{pmatrix}\begin{pmatrix}
										0\\
										1
										\end{pmatrix}=\begin{pmatrix}
										-1 & 3
										\end{pmatrix}\begin{pmatrix}
										x\\
										y
										\end{pmatrix}-\dfrac{1}{2}\cdot 3=-x+3y-\dfrac{3}{2}\\
	L_2(\mathbf{z})=\begin{pmatrix}
	1 & 0
	\end{pmatrix}\begin{pmatrix}
					\tfrac{1}{2} & -1\\
					-1 & 3
					\end{pmatrix}\begin{pmatrix}
					x\\
					y
					\end{pmatrix}-\dfrac{1}{2}\begin{pmatrix}
					1 & 0
					\end{pmatrix}\begin{pmatrix}
									\tfrac{1}{2} & -1\\
									-1 & 3
									\end{pmatrix}\begin{pmatrix}
									1\\
									0
									\end{pmatrix}=\begin{pmatrix}
									\tfrac{1}{2} & -1
									\end{pmatrix}\begin{pmatrix}
									x\\
									y
									\end{pmatrix}-\dfrac{1}{2}\cdot\dfrac{1}{2}=\dfrac{1}{2}x-y-\dfrac{1}{4}\\
	L_3(\mathbf{z})=\begin{pmatrix}
	1 & 1
	\end{pmatrix}\begin{pmatrix}
					\tfrac{1}{2} & -1\\
					-1 & 3
					\end{pmatrix}\begin{pmatrix}
					x\\
					y
					\end{pmatrix}-\dfrac{1}{2}\begin{pmatrix}
					1 & 1
					\end{pmatrix}\begin{pmatrix}
									\tfrac{1}{2} & -1\\
									-1 & 3
									\end{pmatrix}\begin{pmatrix}
									1\\
									1
									\end{pmatrix}=\begin{pmatrix}
									-\tfrac{1}{2} & 2
									\end{pmatrix}\begin{pmatrix}
									x\\
									y
									\end{pmatrix}-\dfrac{1}{2}\cdot\dfrac{3}{2}=-\dfrac{1}{2}x+2y-\dfrac{3}{4}
	\end{array}$
	
	Criterio: $z$ se clasificará en la población donde $L_i(\mathbf{z})$ sea máxima.
	
	\item \lb{Dados dos vectores normales bidimensionales con medias $(0,0)$ y $(3,0)$ y matrices de covarianzas \[ V_1=\begin{pmatrix}
			1 & -1\\
			-1 & 2
		\end{pmatrix}\text{ y }V_2=\begin{pmatrix}
		1 & 2\\
		2 & 5
		\end{pmatrix}, \]se pide:}
	\begin{enumerate}[label=\color{red}\alph*)]
		\item \db{Calcular las funciones discriminantes cuadráticas}
		
		$\begin{array}{l}
		V_1^{-1}=\begin{pmatrix}
		1 & -1\\
		-1 & 2
		\end{pmatrix}^{-1}=\begin{pmatrix}
		2 & 1\\
		1 & 1
		\end{pmatrix}\\
		V_2^{-1}=\begin{pmatrix}
		1 & 2\\
		2 & 5
		\end{pmatrix}^{-1}=\begin{pmatrix}
		5 & -2\\
		-2 & 1
		\end{pmatrix}\\
		\mu^{(1)}=(0,0)'\\
		\mu^{(2)}=(3,0)'\\
		Q_i(\mathbf{z})=(\mathbf{z}-\mu^{(i)})'V_i^{-1}(\mathbf{z}-\mu^{(i)})+\log|V_i|\\
		Q_i^*(\mathbf{z})=(\mathbf{z}-\mu^{(i)})V_i^{-1}(\mathbf{z}-\mu^{(i)})
		\end{array}$
		
		Como $|V_i|=1,\:i=1,2$ las dos funciones coinciden.
		
		$Q_1(\mathbf{z})=\begin{pmatrix}
		x-0 & y-0
		\end{pmatrix}\begin{pmatrix}
		2 & 1\\
		1 & 1
		\end{pmatrix}\begin{pmatrix}
		x-0\\
		y-0
		\end{pmatrix}+\tozero{\log|V_1|}=\begin{pmatrix}
		2x+y&x+y
		\end{pmatrix}\begin{pmatrix}
		x\\
		y
		\end{pmatrix}=2x^2+2xy+y^2$\\
		$\begin{aligned}
		Q_2(\mathbf{z})&=\begin{pmatrix}
		x-3 & y-0
		\end{pmatrix}\begin{pmatrix}
		5 & -2\\
		-2 &1
		\end{pmatrix}\begin{pmatrix}
		x-3\\
		y-0
		\end{pmatrix}+\tozero{\log|V_2|}=\begin{pmatrix}
		5x-2y-15&-2x+y+6
		\end{pmatrix}\begin{pmatrix}
		x-3\\
		y
		\end{pmatrix}\\
		&=(5x-2y-15)(x-3)+2xy+y^2+6y=5x^2-15x-2xy+6y-15x+45-2xy+y^2+6y\\
		&=5x^2+y^2-30x+12y-4xy+45
		\end{aligned}$
		
		Criterio de clasificación: \textbf{z} se clasifica en la población en la que $Q_i(\mathbf{z})$ mínimo 
		\item \db{Clasificar a $\mathbf{z}=(1,-4)'$ usando dichas funciones.}
		
		$\begin{array}{ll}
		\mathbf{z}=(1,-4)' & Q_1(\mathbf{z})=2\cdot1^2+2\cdot1\cdot(-4)+(-4)^2=10\\
		&Q_2(\mathbf{z})=5\cdot1^2+(-4)^2-30\cdot1+12\cdot(-4)-4\cdot1\cdot(-4)+45=4\quad\lb{(\ast)}
		\end{array}$
		
		\textbf{z} se clasifica en la población (2)
		\item \db{Representar gráficamente las regiones de clasificación.}
		
		Se clasifica en (1) $\longleftrightarrow Q_1(\mathbf{z})<Q_2(\mathbf{z})$
		
		$2x^2+y^2+2xy<5x^2+y^2-30x+12y-4xy+45\longleftrightarrow y^2+2xy-y^2-12y+4xy<5x^2-30x+45-2x^2\linebreak\longleftrightarrow6xy-12y<3x^2-30x+45\xleftrightarrow{~~\frac{1}{3}~~}2xy-6y<x^2-10x+15\longleftrightarrow y(-4+2x)<x^2-10x+45$
		\begin{enumerate}[label=\arabic*)]
			\item Si $x=2$, como $0>-1,\:z$ se clasifica en (2).
			\item Si $x>2$, se clasifica en (1) $\longleftrightarrow y<\dfrac{x^2-10x+15}{-4+2x}$.
			\item Si $x<2$, se clasifica en (1) $\longleftrightarrow y>\dfrac{x^2-10x+15}{-4+2x}$..
		\end{enumerate}
		
		\begin{tikzpicture}[scale=1.5]
		\begin{axis}[
		    axis lines = middle,
		    domain = -15:15,
		    samples = 1000,
		    ymin = -15,
		    ymax = 15,
		    restrict y to domain=-15:15
		]
		\addplot [lightblue, line width=1.5] {(x^2-10*x+15)/(-4+2*x)};
		\fill[blue] (axis cs:0,0) circle (2pt) node[below left] {$\mu^{(1)}$};
		\fill[blue] (axis cs:3,0) circle (2pt) node[above right] {$\mu^{(2)}$};
		\end{axis}
		\end{tikzpicture}
		
	\end{enumerate}
	
	\item \lb{Dadas dos poblaciones normales bidimensionales con medias $\mu^{(1)}=(1,0)'$ y $\mu^{(2)}=(0,0)'$ y matrices de covarianzas \[ V_1=\begin{pmatrix}
			1 & 0\\
			0 & \frac{1}{2}
		\end{pmatrix}\text{ y }V_2=\begin{pmatrix}
		2 & 1\\
		1 & 1
		\end{pmatrix}, \]se pide:}
	\begin{enumerate}[label=\color{red}\alph*)]
		\item \db{Obtener las funciones discriminantes cuadráticas y clasificar a $\mathbf{z}=(1,1)'$.}
		
		$\begin{array}{l}
		V_1=\begin{pmatrix}
		1 & 0\\
		0 & \tfrac{1}{2}
		\end{pmatrix}\longrightarrow V_1^{-1}=2\cdot\begin{pmatrix}
		\tfrac{1}{2} & 0 \\
		0 & 1
		\end{pmatrix}=\begin{pmatrix}
				1 & 0\\
				0 & 2
				\end{pmatrix}\\
				V_2=\begin{pmatrix}
				2 & 1\\
				1 & 1
				\end{pmatrix}\longrightarrow V_2^{-1}=\begin{pmatrix}
				1 & -1\\
				-1 & 2
				\end{pmatrix}\\
				\mu_1=\begin{pmatrix}
				1 & 0
				\end{pmatrix}'\qquad\mu_2=\begin{pmatrix}
				0 & 0
				\end{pmatrix}'\\
				Q_i(\mathbf{z})=(\mathbf{z}-\mu^{(i)})'V_i^{-1}(\mathbf{z}-\mu^{(i)})+\log|V_i|\\
				\begin{aligned}
				Q_1(\mathbf{z})&=\begin{pmatrix}
				x-1 & y-0
				\end{pmatrix}\begin{pmatrix}
				1 & 0\\
				0 & 2
				\end{pmatrix}\begin{pmatrix}
				x-1\\
				y-0
				\end{pmatrix}+\log|0.5|=\begin{pmatrix}
				x-1 & 2y
				\end{pmatrix}\begin{pmatrix}
				x-1\\
				y
				\end{pmatrix}+\log|0.5|\\
				&=(x-1)^2+2y^2+\log|0.5|=x^2-2x+1+2y^2+\log|0.5|
				\end{aligned}\\
				Q_2(\mathbf{z})=\begin{pmatrix}
				x-0 & y-0
				\end{pmatrix}\begin{pmatrix}
				1 & -1\\
				-1 & 2
				\end{pmatrix}\begin{pmatrix}
				x-0\\
				y-0
				\end{pmatrix}+\tozero{\log(1)}=\begin{pmatrix}
				x-y & -x+2y
				\end{pmatrix}\begin{pmatrix}
				x\\
				y
				\end{pmatrix}=x^2-xy-xy+2y^2=x^2+2y^2-2xy
		\end{array} $
		
		Clasificamos al individuo en $\mathbf{z}=(1,1)'$
		
		$\begin{cases}
		Q_1(1,1)=1.31\\
		Q_2(1,1)=1^2+2\cdot1^2-2\cdot1\cdot1=\bboxed{1}
		\end{cases}$
		
		Clasificamos al individuo en la población 2.
		\item \db{Clasificar a $\mathbf{z}$ usando el criterio de mínima distancia de Mahalanobis y representar las regiones de clasificación con este criterio para cada grupo.}
		
		$\begin{array}{l}
		Q_i^*(\mathbf{z})=d_M^2(\mathbf{z},\mu^{(i)})=(\mathbf{z}-\mu^{(i)})'V_i^{-1}(\mathbf{z}-\mu^{(i)})\\
		Q_1^*(\mathbf{z})=x^2+2y^2-2x+1\\
		Q_2^*(\mathbf{z})=x^2+2y^2-2xy
		\end{array}$
		
		$z$ se clasificará en la población 1 $\longleftrightarrow Q_1^*(\mathbf{z})<Q_2^*(\mathbf{z})\longleftrightarrow\cancel{x^2}+\cancel{2y^2}-2x+1<\cancel{x^2}+\cancel{2y^2}-2xy\xleftrightarrow{\text{Despejamos } y}2xy<2x\cdot1$
		
		Tenemos tres posibilidades:
		\begin{itemize}
		\item Si $x=0:\:\mathbf{z}$ se clasificará en la población 1 $\longleftrightarrow0<-1$ 
		\lb{No se puede dar, por tanto} \fcolorbox{lightblue}{lightblue!10}{$\mathbf{z}$ se clasificará en la población 2}
		\item Si $x>0:\:\mathbf{z}$ se clasificará en la población 1 $\longleftrightarrow\bboxed{y<\dfrac{2x-1}{2x}}$
		\item Si $x<0:\:\mathbf{z}$ se clasificará en la población 1 $\longleftrightarrow\bboxed{y>\dfrac{2x-1}{x}}$
		\end{itemize}
		
		\begin{center}
		\begin{tikzpicture}
		\begin{axis}[axis lines=center, xmin=-3, xmax=3, ymin=-3, ymax=3, yticklabels=\empty, xticklabels={}, ylabel=$y$, xlabel=$x$, x label style={right}, y label style={above}, scale=1.5]
		\addplot[line width=1.5, samples=100, domain=-3:-0.01, lightblue] {1-1/(2*x)};
		\addplot[line width=1.5, samples=100, domain=0.01:3, lightblue] {1-1/(2*x)};
		\fill[blue] (axis cs:0,0) circle (2pt) node[below left] {$\mu^{(2)}$};
		\fill[blue] (axis cs:1,0) circle (2pt) node[below left] {$\mu^{(1)}$};
		\fill[blue] (axis cs:1,1) circle (2pt) node[above] {\small \textbf{z} (individuo)};
		\end{axis}
		\end{tikzpicture}
		\end{center}
		
		Si $x>0\qquad y<\dfrac{2x-1}{2x}\longrightarrow\dfrac{2x}{2x}-\dfrac{1}{2x}=1-\dfrac{1}{2x}$
		
		Como $x>0$, si $x$ converge a $0\longrightarrow1-\infty=\bboxed{-\oo}$
		
		Si $x<0\qquad y>\dfrac{2x-1}{2x}\longrightarrow\dfrac{2x}{2x}-\dfrac{1}{2x}=1-\dfrac{1}{2x}$
		
		Como $x>0$, si $x$ converge a $0\longrightarrow1+\infty=\bboxed{+\oo}$
		
		\begin{itemize}[label=\color{red}\textbullet, leftmargin=*]
			\item \color{lightblue}Interpretación de la gráfica
		\end{itemize}
		\begin{enumerate}[label=\arabic*)]
			\item Coordenada $x$ donde habrá una recta vertical donde se produce la convergencia de las exponenciales.
			\item Determina en que lado del gráfico se dibuja la exponencial $\begin{cases}
			x>\longrightarrow\text{ derecha}\\
			x<\longrightarrow\text{ izquierda}\\
			\end{cases}$
			\item Marcará donde converge la exponencial en la \lb{recta vertical} establecida$\begin{cases}
						y>\longrightarrow\text{ plano positivo}\\
						y<\longrightarrow\text{ plano negativo}\\
						\end{cases}$
		\end{enumerate}
	\end{enumerate}
	\item \lb{Obtener un criterio de clasificación para dos poblaciones exponenciales unidimensionales con medias distintas usando máxima verosimilitud. Clasificar a $z=1.5$ entre dos poblaciones exponenciales con medias 2 y 1. (Indicación: La función de densidad de la distribución exponencial es $f(x)=\dfrac{e^{-\frac{x}{\mu}}}{\mu}$ para $x\ge0$).}
	
		$f(x)=\begin{cases}
		\frac{1}{\mu}e^{-\frac{x}{\mu}} & \text{si }x>0\\
		0 & \text{en otro caso}
		\end{cases}$
		
		Población (1): media$=E[X^{(1)}]=\mu_1;\, f_1$ función de densidad\\
		Población (2): media$=E[X^{(2)}]=\mu_2;\, f_2$ función de densidad\\
		
		Criterio de máxima verosimilitud:
		
		Se clasificará \textbf{z} en (1)$\longleftrightarrow f_1(\mathbf{z})>f_2(\mathbf{z})\longleftrightarrow\dfrac{1}{\mu_1}e^{-\frac{\mathbf{z}}{\mu_1}}>\dfrac{1}{\mu_2}e^{-\frac{\mathbf{z}}{\mu_2}}\longleftrightarrow e^{-\frac{\mathbf{z}}{\mu_1}}e^{\frac{\mathbf{z}}{\mu_2}}>\dfrac{\mu_1}{\mu_2}\longleftrightarrow e^{\mathbf{z}\left(\frac{1}{\mu_2}-\frac{1}{\mu_1}\right)}>\dfrac{\mu_1}{\mu_2}\longleftrightarrow\mathbf{z}\left(\dfrac{1}{\mu_2}-\dfrac{1}{\mu_1}\right)>\ln\left(\dfrac{\mu_1}{\mu_2}\right)$
		
		Si $\mu_1>\mu_2$, se clasifica en (1) $\longleftrightarrow \mathbf{z}>\dfrac{1}{\left(\frac{1}{\mu_2}-\frac{1}{\mu_1}\right)}\ln\left(\dfrac{\mu_1}{\mu_2}\right)$\\
		Si $\mu_1<\mu_2$, se clasifica en (1) $\longleftrightarrow \mathbf{z}<\dfrac{1}{\left(\frac{1}{\mu_2}-\frac{1}{\mu_1}\right)}\ln\left(\dfrac{\mu_1}{\mu_2}\right)$\\
		$\begin{array}{l}
		\mu_1=2,\:\mu_2=1\\
		\mathbf{z}=1.5
		\end{array}~~\begin{cases}
		f_1(1.5)=0.2361\quad\lb{(\ast)\text{ Se clasificará en la población (1)}}\\
		f_2(1.5)=0.2231
		\end{cases}$
\end{enumerate}

\rc{Ejercicio de clase (22/04/2024):} \lb{Sea $(X,Y)$ un \vea con función de densidad \[ f(x,y)=\begin{cases}
\frac{2x+2y}{5} & 0<x+y<2,0<y<1\\
0 & \text{en otro caso}
\end{cases} \]}
\begin{enumerate}[label=\color{red}\alph*)]
	\item \db{Calcular la recta de regresión de $X$ sobre $Y$ y obtener una medida de la bondad del ajuste realizado.}
	
	Recta de regresión de $X$ sobre $Y$:\[ X-E[X]=\dfrac{\cov(X,Y)}{\var(Y)}(Y-E[Y]) \]
	$\begin{array}{l}
	E[X]=\iint xf(x,y)\dx\dy\\
	E[Y]=\iint y f(x,y)\dx\dy\\
	\var[Y]=E[Y^2]-(E[Y])^2\\
	E[Y^2]=\iint y^2f(x,y)\dx\dy\\
	\cov(X,Y)=E[X\cdot Y]-(E[X])\cdot(E[Y])\\
	E[X\cdot Y]=\iint x\cdot y\cdot f(x,y)\dx\dy
	\end{array}$
	
	\begin{center}
	\begin{tikzpicture}[scale=1.5]
	\begin{axis}[axis lines=center, xmin=-1.5, xmax=3, ymin=-1.5, ymax=3, x label style={right}, y label style=above, xlabel=$x$, ylabel=$y$]
	\fill[lightblue, opacity=0.2] (axis cs:-1,1) -- (axis cs:0,0) -- (axis cs:2,0) -- (axis cs:1,1) -- cycle;
	\addplot[lightblue, line width=1.5, domain=-1.5:3] {-x};
	\addplot[blue, line width=1.5, domain=-1.5:3] {2-x};
	\addplot[dashed, red] {1};
	\draw[red, line width=1.2] (axis cs:0,1) -- (axis cs:1,1) -- (axis cs:1,0);
	\addlegendentry{$y=-x$}
	\addlegendentry{$y=2-x$}
	\end{axis}
	\end{tikzpicture}
	\end{center}
	
	\item \db{Calcular la curva de regresión de $X$ sobre $Y$.}
	\item \db{Predecir el valor de $X$ cuando $Y$ toma el valor $\dfrac{1}{2}$ mediante la curva y la recta, y dar un valor del error de predicción en cada caso.}
\end{enumerate}