\includepdf[pages=-]{"Tareas/Tema 2/Hoja 2"}
\begin{enumerate}[label=\color{red}\arabic*),leftmargin=*]
	\item \lb{Dado el \vea $(X,Y)$ con función de densidad \[ f(x,y)=\begin{cases}
	2 & \text{si }0<x<1,\:0<y<x\\
	0 & \text{en otro caso}
	\end{cases} \]}
	\begin{enumerate}[label=\color{red}\alph*)]
		\item \db{Obtener la curva de regresión para predecir $Y$ en función de valores de la variable $X$.}
		
		Primero saco las distribución marginal $f_X$:
		
		$\underset{0<x<1}{f_X(x)=}\infi f(x,y)\dy=\int_0^12\dy=\left[2y\right]_0^1=2\longrightarrow\begin{cases}
		2 & \text{si }0<y<1\\
		0 & \text{en caso contrario}
		\end{cases}$
		
		$\underset{0<x<1}{f_{Y|X}(y|x)}=\dfrac{f(x,y)}{f(x)}=\begin{cases}
		\dfrac{1}{x} & \text{si }0<y<x\\
		0 & \text{en caso contrario}
		\end{cases}$
		
		Curva de regresión: $h_{\mathrm{opt}}(X)=E[Y|X=x]=\infi y\cdot f_{Y|X}(y|x)\dy=\int_0^xy\cdot\dfrac{1}{x}\dy=\dfrac{1}{x}\left[\dfrac{y^2}{2}\right]_{y=0}^{y=x}=\dfrac{1}{x}\cdot\dfrac{x^2}{2}=\dfrac{x}{2}$

		\item \db{¿Coincide con la recta de regresión?}
		
		Recta de regresión de $Y|X=x$: $Y-\mu_Y=\dfrac{\cos(X,Y)}{\sigma_X^2}(x-\mu_X)\longrightarrow y-\dfrac{1}{3}-\dfrac{\frac{1}{36}}{\frac{1}{18}}\left(x-\dfrac{2}{3}\right)\longrightarrow \bboxed{y=\dfrac{x}{2}}
		$
		
		$\begin{array}{l}
		E[X]=\int_{0}^{1}2x\dx=\left[2\cdot\dfrac{x^3}{3}\right]_{x=0}^{x=1}=\dfrac{2}{3}\\
		E[Y]=\int_{0}^{1}y\cdot2(1-y)\dy=2\int_0^1(y-y^2)\dy=2\left[\dfrac{y^2}{2}-\dfrac{y^3}{3}\right]_{y=0}^{y=1}=2\cdot\dfrac{1}{6}=\dfrac{1}{3}\\
		\sigma_X^2=E[X^2]-(E[X])^2=\dfrac{1}{2}-\left(\dfrac{2}{3}\right)^2=\dfrac{1}{2}-\dfrac{4}{9}=\dfrac{1}{18}\\
		E[X^2]=\int_0^1x^2\cdot2x\dx=2\int_0^1x^3\dx=2\left[\dfrac{x^4}{4}\right]_{x=0}^{x=1}=\dfrac{1}{2}\\
		\cov(X,Y)=E[X\cdot Y]-E[X]\cdot E[Y]=\dfrac{1}{4}-\dfrac{2}{3}\cdot\dfrac{1}{3}=\dfrac{1}{36}\\
		E[X\cdot Y]=\int_0^1\int_0^x2xy\dy\dx=2\int_0^1x\left[\dfrac{y^2}{2}\right]_{y=0}^{y=x}\dx=\cancel{2}\int_{0}^{1}x\cdot\dfrac{x^2}{\cancel{2}}\dx=2\cdot\left[\dfrac{x^4}{4}\right]_{x=0}^{x=1}=\dfrac{1}{4}
		\end{array}$
		
		\item \db{$\rho_{X,Y}^2,\:\var(R),\:ECM$}
	\end{enumerate}
	\item \lb{Sea $(X,Y)$ un vector aleatorio con función de densidad \[ f(x,y)=\begin{cases}
	\frac{3}{4}\left[xy+\frac{x^2}{2}\right] & \text{si }0<x<1,\:0<y<2\\
	0 & \text{en otro caso}
	\end{cases} \]A partir de la distribución de $Y$ condicionada a $X=x$, obtener la curva de regresión para predecir $Y$ a partir de los valores de $X$ y proporcionar una predicción para $X=\dfrac{2}{3}$}
	
	$h_{\mathrm{opt}}(x)=E[Y|X=x]=\infi y\cdot f_{Y|X}(y|x)\dy=\int_{0}^{2}y\cdot\dfrac{y+\frac{x}{2}}{x+2}\dy=\dfrac{1}{x+2}\int_0^2y\cdot(y+\dfrac{x}{2})\dy=\dfrac{1}{x+2}\cdot\left[\dfrac{y^3}{3}-\dfrac{y^2x}{4}\right]_0^2=\dfrac{1}{x+2}\left(\dfrac{8}{3}-x\right)$
	
	$f_X(x)=\begin{cases}
	\frac{3}{4}(x^2+2x) & \text{si }0<x<1\\
	0 & \text{en otro caso}
	\end{cases}$
	
	$\underset{0<x<1}{f_{Y|X}(y|x)=}\begin{cases}
	\dfrac{y+\frac{x}{2}}{x+2} & \text{si }0<y<2\\
	0 & \text{en otro caso}
	\end{cases}$
	
	Para $x=\dfrac{2}{3},\:h_{\mathrm{opt}}\left(\dfrac{2}{3}\right)=\dfrac{1}{\frac{2}{3}+2}\cdot\left(\dfrac{8}{3}+\dfrac{2}{3}\right)=\bboxed{1.25}$
	
	\item \lb{Sabiendo que el vector $(X,Y)$ tiene una distribución normal con medias 1 y 2, varianzas 2 y correlación $-\dfrac{1}{2}$, calcular la curva de regresión para predecir $Y$ a partir de valores de $X$ y obtener una predicción para $X=1.5$}
	
	$(X,Y)\rightsquigarrow \mathcal{N}_2(\mu, V)$
	
	$\begin{array}{l}
	\mu=(1,2)\\
	V=\begin{pmatrix}
	2 & -1\\
	-1 & 2
	\end{pmatrix}
	\end{array}\qquad f_{X,Y}=-\dfrac{1}{2}=\dfrac{\cov(X,Y)}{\sqrt{\sigma_X^2\cdot\sigma_Y^2}}\longrightarrow\cov(X,Y)=-\dfrac{1}{2}\cdot\sqrt{2\cdot2}=-1$
	
	$h_{\mathrm{top}}=\mu_Y+\dfrac{\cov(X,Y)}{\sigma_X^2}(x-\mu_X)=2+\dfrac{-1}{2}(x-1)=-\dfrac{x}{2}+\dfrac{5}{2}\longrightarrow h_{\mathrm{opt}}(1.5)=2-\dfrac{1}{2}\cdot\dfrac{1}{2}=\bboxed{\dfrac{7}{4}}$
	
	\item \lb{Encontrar la recta de regresión para el conjunto de datos: \[ \{(x_i,y_i)\}=\{(1,4),(2,2),(1,5),(5,3),(6,2)\} \]Estimar $y$ para $x=3$. ¿Será fiable esa aproximación?}
	
	$\begin{aligned}
		\text{Recta de regresión: } & y=\hat{\theta}_0+\hat{\theta}_1x\\
		&\hat{\theta}_0=\overline{y}-\hat{\theta}_1\overline{x}=3.2-(-0.\overline{36})\cdot3=4.29\\
		&\hat{\theta}_1=\dfrac{S_{xy}}{S_x^2}=-\dfrac{1.6}{4.4}=-0.\overline{36}
	\end{aligned}$
	
	$\begin{array}{l}
		n=5\\
		\overline{x}=3\\
		\overline{y}=3.2\\
		S_x^2=\overline{x^2}-\overline{x}^2=\dfrac{67}{3}-3^2=13.4-9=4.4\\
		S_{xy}=\overline{xy}-\overline{x}\cdot\overline{y}=8-3*3.2=-1.6\\
	\end{array}$
	
	Recta de regresión: $y=4.29-0.\overline{36}x\xrightarrow{x=3}y=4.29-0.\overline{36}\cdot3=3.2$
	
	Coeficiente de correlación lineal al cuadrado: \[ r_{xy}^2=\dfrac{S_{xy}^2}{s_x^2\cdot s_y^2}=\dfrac{(-1.6)^2}{4.4\cdot1.36}=0.428\text{ Ajuste malo} \]
	
	$s_y^2=\overline{y^2}-\overline{y}^2=\dfrac{58}{5}-3.2^2=1.36$
	\item \lb{En el rendimiento de una reacción química depende de la concetración del reactivo y de la temperatura de la operación. \begin{center}
			\begin{tabular}{ccc}
				\hline
				Rendimiento & Concentración & Temperatura \\
				\hline
				81 & 1.00 & 150 \\
				89 & 1.00 & 180 \\
				83 & 2.00 & 150 \\
				91 & 2.00 & 180 \\
				79 & 1.00 & 150 \\
				87 & 1.00 & 180 \\
				84 & 2.00 & 150 \\
				90 & 2.00 & 180 \\ \hline
			\end{tabular}
	\end{center}En el modelo de regresión del rendimiento sobre la temperatura y la concentración, los valores ajustados son \[ \hat{y}=\begin{pmatrix}
	80.25, & 87.75, & 83.25, & 90.75, & 87.75, & 83.25, & 90.75
	\end{pmatrix}' \]Calcular el error cuadrático medio y el coeficiente de determinación e interpretar su valor.}

$\begin{array}{c|c|c}
	\mathrm{Rend.}(y)& \hat{y} & y-\hat{y} \\ \hline
	81 & 80.25 & 0.75 \\
	89 & 87.75 & 1.25 \\
	83 & 83.25 & -0.25 \\
	91 & 90.75 & 0.25 \\
	79 & 80.25 & -1.25 \\
	87 & 87.75 & -0.75 \\
	84 & 83.25 & 0.75 \\
	90 & 90.75 & -0.75 \\ \hline
\end{array}$

$\begin{array}{l}
	ECM=\dfrac{1}{n}\sum_{i=1}^{n}(y_i-\hat{y}_i)^2=\dfrac{5.5}{8}=\bboxed{0.6875}\\
	R^2=1-\dfrac{SCR}{SCT}=1-\dfrac{\sum_{i=1}^{n}(y_i-\hat{y}_i)^2}{\sum_{i=1}^{n}(y_i-\overline{y})^2}=1-\dfrac{5.5}{136}=\bboxed{0.9595}\text{ Ajuste muy bueno}\\\\
	\overline{y}=\dfrac{1}{n}\sum_{i=1}^{n}y_i=85.5\\
	R^2=\dfrac{SCE}{SCT}=\dfrac{\sum_{i=1}^{n}(\hat{y}_i-\overline{y}_i)^2}{\sum_{i=1}^{n}(y_i-\overline{y})^2}
\end{array}$
\item \lb{Un modelo ajustado para predecir la extracción de manganeso en \% ($y$) a partir del tamaño de partícula en mm $(x_1)$, la cantidad de dióxido de azufre en múltiplos}

\begin{enumerate}[label=\color{red}\alph*)]
	\item 
	
	$\hat{y}_{x_1=3,x_2=1.5,x_3=20}=56.145-9.0469\cdot3-33.421\cdot1.5+0.243\cdot20-0.5963\cdot3\cdot1.5-0.0394\cdot3\cdot20+0.60022\cdot1.5\cdot20+0.6901\cdot3^2+11.7244\cdot(1.5)^2-0.0097\cdot20^2=25.4028$
	\item 
	
	$x_3\longrightarrow x_3+1$
	
	$\begin{aligned}
		\hat{y}_{x_1,x_2,x_3+1}-\hat{y}_{x_1,x_2,x_3}&=56.145-9.0496\cdot x_1-\cdots-0.0097\cdot(x_3+1)^2-56.145-9.0496\cdot x_1-\cdots-0.0097\cdot(x_3)^2\\
		&=0.243\cdot(x_3+1)-0.0394\cdot x_1(x_3+1)+0.60022\cdot x_2\cdot(x_3+1)-0.0097(x_3+1)^2-0.243x_3\\
		&+0.0394\cdot x_1\cdot x_3-0.60022\cdot x_2\cdot x_3+0.0097 x_3^2\\
		&=0.243-0.0394\cdot x_1+0.60022\cdot x_2-0.0097\cdot 2\cdot x_3-0.0097
	\end{aligned}$
	\item 
	
	$R^2=1-\dfrac{SCR}{SCT}=1-\dfrac{\sum_{i=1}^{n}(y_i-\hat{y}_i)^2}{\sum_{i=1}^{n}(y_i-\overline{y})^2}=1-\dfrac{209.55}{6777.5}=0.969$
	
	Ajuste bueno.
\end{enumerate}
\item 

\begin{enumerate}[label=\color{red}\alph*)]
	\item 
	
	$\hat{\sigma}^2=\dfrac{SCR}{n},\tilde{\sigma}^2=\dfrac{SCR}{n-k-1}$
	
	$\begin{array}{l}
		\hat{\sigma}^2=\dfrac{SCR}{n}=\dfrac{1950.7292}{6}=325.0705\\
		\tilde{\sigma}^2=\dfrac{SCR}{n-k-1}=\dfrac{\sum_{i=1}^{n}(y_i-\hat{y}_i)^2}{n-k-1}=650.141\\
		R^2=1-\dfrac{\sum_{i=1}^{n}(y_i-\hat{y}_i)^2}{\sum_{i=1}^{n}(y_i-\overline{y})^2}=1-\dfrac{1950.423}{14112}=1-0.1382=\bboxed{0.8618}
	\end{array}$
	\item 
	
	$\hat{y}_{x_1=25,x_2=1000}=350.994-1.2799\cdot25-0.15390\cdot1000=\bboxed{165.29}$
\end{enumerate}
\end{enumerate}