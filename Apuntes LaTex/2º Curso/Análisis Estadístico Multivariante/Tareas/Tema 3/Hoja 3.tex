\includepdf[pages=-]{"Tareas/Tema 3/Hoja 3"}

\begin{enumerate}[label=\color{red}\textbf{\arabic*)}, leftmargin=*]
	\item \lb{En la siguiente tabla se ha recopilado una serie de 20 datos que relacionan las horas de estudio de cada alumno y si han aprobado o suspendido un examen de estadística. \begin{center}
			\begin{tabular}{c|c}
				Horas de estudio & Aprobado (1 sí, 0 no) \\
				\hline
				0.5 & 0 \\
				0.75 & 0 \\
				1 & 0 \\
				1.25 & 0 \\
				1.5 & 0 \\
				1.75 & 0 \\
				1.75 & 1 \\
				2 & 0 \\
				2.25 & 1 \\
				2.5 & 0 \\
			\end{tabular}\qquad\begin{tabular}{c|c}
			Horas de estudio & Aprobado (1 sí, 0 no) \\
			\hline
			2.75 & 1 \\
			3 & 0 \\
			3.25 & 1 \\
			3.5 & 0 \\
			4 & 1 \\
			4.25 & 1 \\
			4.5 & 1 \\
			4.75 & 1 \\
			5 & 1 \\
			5.5 & 1 \\
			\end{tabular}
	\end{center} Se ha ajustado un modelo de regresión y los parámetros estimados han sido $\hat{\theta}_0=-4.077$ y $\hat{\theta}_1=1.5046$}
	
	\begin{enumerate}[label=\color{red}\alph*)]
		\item \db{¿Como se interpreta el valor de $\hat{\theta}_1$?}
		
		$\begin{array}{l}
			\log\dfrac{p}{1-p}=\hat{\theta}_0+\hat{\theta}_1\cdot x\\
			P\equiv Pr[Y=1]\\
			\log\dfrac{p}{1-p}=\exp(\hat{\theta}_0+\hat{\theta}_1x)\\
			p=(1-p)\exp(\hat{\theta}_0+\hat{\theta}_1x)
		\end{array}$
		\item \db{A partir del modelo ajustado, obtener una predicción para la probabilidad de que un alumno apruebe si ha estudiado 2 horas. ¿Cuál sería dicha probabilidad si dedicara una hora más al estudio? ¿Cómo ha variado la razón de estas probabilidades?}
		
		$Pr[Y=1/X=2]=\dfrac{1}{1+\exp(-(-40.77+1.5046\cdot2))}\simeq0.25$\\
		$Pr[Y=1/X=3]=\dfrac{1}{1+\exp(-(-40.77+1.5046\cdot3))}\simeq0.607$\\
		$\dfrac{Pr[Y=1/X=3]}{Pr[Y=1/X=2]}=\dfrac{0.607}{0.2556}=2.428$
		
		La razón de ambas propiedades presenta una gran diferencia.
	\end{enumerate}
	\item \lb{En un estudio clínico se desea predecir la probabilidad de padecer una enfermedad coronaria ($Y$, con valores 1 sí, 0 no) a partir de las covarianzas siguientes: Nivel de colesterol ($X_1$, 1 alto, 0 bajo), Edad ($X_2$) y Resultado del electrocardiograma ($X_3$, 1 anormal, 0 normal). Para ello, se analizaron 750 casos y se propuso un modelo logístico para estimar el riesgo de padecer una enfermedad coronaria, obteniendo las siguientes estimaciones para los parámetros: $\hat{\theta}_0=-3.912,\hat{\theta}_1=0.852,\hat{\theta}=0.025$ y $\hat{\theta}_3=0.441$.}
	\begin{enumerate}[label=\color{red}\alph*)]
		\item \db{Interpretar el significado de los coeficientes $\hth_1,\hth_2$ y $\hth_3$.}
		
		\item \db{Obtener una predicción para la probabilidad de padecer una enfermedad coronaria para una persona con 40 años, electrocardiograma normal y nivel de colesterol bajo. ¿Cuál sería dicha probabilidad si tuviera un nivel de colesterol alto?}
		
		\item \db{Para una persona con 40 años y electrocardiograma normal, ¿cómo influye el nivel de colesterol en el riesgo de padecer una enfermedad coronaria?}
	\end{enumerate}
\end{enumerate}