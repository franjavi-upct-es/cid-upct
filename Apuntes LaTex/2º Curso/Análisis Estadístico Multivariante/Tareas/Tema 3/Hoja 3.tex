\includepdf[pages=-]{"Tareas/Tema 3/Hoja 3"}

\begin{enumerate}[label=\color{red}\textbf{\arabic*)}, leftmargin=*]
	\item \lb{En la siguiente tabla se ha recopilado una serie de 20 datos que relacionan las horas de estudio de cada alumno y si han aprobado o suspendido un examen de estadística. \begin{center}
			\begin{tabular}{c|c}
				Horas de estudio & Aprobado (1 sí, 0 no) \\
				\hline
				0.5 & 0 \\
				0.75 & 0 \\
				1 & 0 \\
				1.25 & 0 \\
				1.5 & 0 \\
				1.75 & 0 \\
				1.75 & 1 \\
				2 & 0 \\
				2.25 & 1 \\
				2.5 & 0 \\
			\end{tabular}\qquad\begin{tabular}{c|c}
			Horas de estudio & Aprobado (1 sí, 0 no) \\
			\hline
			2.75 & 1 \\
			3 & 0 \\
			3.25 & 1 \\
			3.5 & 0 \\
			4 & 1 \\
			4.25 & 1 \\
			4.5 & 1 \\
			4.75 & 1 \\
			5 & 1 \\
			5.5 & 1 \\
			\end{tabular}
	\end{center} Se ha ajustado un modelo de regresión y los parámetros estimados han sido $\hat{\theta}_0=-4.077$ y $\hat{\theta}_1=1.5046$}
	
	\begin{enumerate}[label=\color{red}\alph*)]
		\item \db{¿Como se interpreta el valor de $\hat{\theta}_1$?}
		
		$\begin{array}{l}
			\log\dfrac{p}{1-p}=\hat{\theta}_0+\hat{\theta}_1\cdot x\\
			P\equiv Pr[Y=1]\\
			\log\dfrac{p}{1-p}=\exp(\hat{\theta}_0+\hat{\theta}_1x)\\
			p=(1-p)\exp(\hat{\theta}_0+\hat{\theta}_1x)
		\end{array}$
		\item \db{A partir del modelo ajustado, obtener una predicción para la probabilidad de que un alumno apruebe si ha estudiado 2 horas. ¿Cuál sería dicha probabilidad si dedicara una hora más al estudio? ¿Cómo ha variado la razón de estas probabilidades?}
		
		$Pr[Y=1/X=2]=\dfrac{1}{1+\exp(-(-40.77+1.5046\cdot2))}\simeq0.25$\\
		$Pr[Y=1/X=3]=\dfrac{1}{1+\exp(-(-40.77+1.5046\cdot3))}\simeq0.607$\\
		$\dfrac{Pr[Y=1/X=3]}{Pr[Y=1/X=2]}=\dfrac{0.607}{0.2556}=2.428$
		
		La razón de ambas propiedades presenta una gran diferencia.
	\end{enumerate}
	\item \lb{En un estudio clínico se desea predecir la probabilidad de padecer una enfermedad coronaria ($Y$, con valores 1 sí, 0 no) a partir de las covarianzas siguientes: Nivel de colesterol ($X_1$, 1 alto, 0 bajo), Edad ($X_2$) y Resultado del electrocardiograma ($X_3$, 1 anormal, 0 normal). Para ello, se analizaron 750 casos y se propuso un modelo logístico para estimar el riesgo de padecer una enfermedad coronaria, obteniendo las siguientes estimaciones para los parámetros: $\hat{\theta}_0=-3.912,\hat{\theta}_1=0.852,\hat{\theta}=0.025$ y $\hat{\theta}_3=0.441$.}
	\begin{enumerate}[label=\color{red}\alph*)]
		\item \db{Interpretar el significado de los coeficientes $\hth_1,\hth_2$ y $\hth_3$.}
		
		\item \db{Obtener una predicción para la probabilidad de padecer una enfermedad coronaria para una persona con 40 años, electrocardiograma normal y nivel de colesterol bajo. ¿Cuál sería dicha probabilidad si tuviera un nivel de colesterol alto?}
		
		\item \db{Para una persona con 40 años y electrocardiograma normal, ¿cómo influye el nivel de colesterol en el riesgo de padecer una enfermedad coronaria?}
	\end{enumerate}
	
	\item \lb{Se desea evaluar la satisfacción con la enseñanza pública de 1500 estudiantes mediante la variable \textit{Satisfecho}($Y$, con valores 1 sí, 0 no) y tres variables periódicas: Nacionalidad (España=1, Ecuador=2, Colombia=3), Género (Hombre=0, Mujer=1) y Estudios (ESO=0, Primaria=1). Se ajusta al siguiente modelo logístico: \[ \log\dfrac{p}{1-p}=-0.877-0.052\cdot\text{Nacionalidad2} + 1.72\cdot\text{Nacionalidad3}+0.256\cdot\text{Género}-0.008\cdots\text{Estudios} \] donde $p=Pr[Y=1]$. Las variables \textit{Nacionalidad2} y \textit{Nacionalidad3} son variables dicotómicas ficticias (\textit{dummy}) que toman el valor 1 si el valor de nacionalidad se corresponde con su índice y valen cero en caso contrario. Por ejemplo, Si Nacionalidad = 3, entonces Nacionalidad2 = 0 y Nacionalidad3 = 1.}
	\begin{enumerate}[label=\color{red}\alph*)]
		\item \db{Predecir la probabilidad de que una alumna colombiana de primaria no esté satisfecha con la enseñanza pública. ¿Cuál sería dicha probabilidad si la alumna tuviera nacionalidad española y estudiara primaria? ¿Y si fuera un alumno de secundaria con nacionalidad española?}
		
		\item \db{Comparar el grado de satisfacción con la enseñanza pública de los alumnos de primaria con nacionalidad española según el género.}
	\end{enumerate}
	
	\item \lb{Una determinada compañía desea mejorar el marketing de cinco variedades de cereales para el desayuno. Para ello planifica un estudio encuestando a 900 personas, registrando su edad, género y si tiene o no un estilo de vida activo. Cada participante degustó los cinco tipos de cereales y se le preguntó sobre su preferencia. En la tabla adjunta se presenta las definiciones de las variables
	\begin{center}
		\begin{tabular}{ll}
			Variable & Categoría \\ \hline
			\hline
			Tipo de cereal preferido& 1: Cebada\\
			& 2: Centeno\\
			& 3: Avena\\
			& 4: Esbelta\\
			& 5: Trigo\\ \hline
			Edad & 1: Menor de 30 años\\
			 & 2: 31 a 50 años\\
			& 3: Más de 50 años \\ \hline
			Género & 1: Hombre\\
			& 2: Mujer \\ \hline
			Estilo de vida & 0: No activo\\
			(realiza o no actividad física) & 1: Activo \\ \hline
		\end{tabular}
		\end{center}}
	\begin{enumerate}[label=\color{red}\alph*)]
		\item \db{Con el objetivo de explicar la preferencia del tipo de cereal en función de la edad, el género y el estilo de vida se desea ajustar un modelo logístico multinomial. ¿Qué variables ficticias (\textit{dummy}) debemos crear para la formulación del modelo?}
		\item \db{Especificar el modelo tomando el trigo como la categoría de referencia para la variable respuesta $(g=5)$.}
		\item \db{¿Cómo se interpretan las estimaciones de los coeficientes del modelo $\hth_{ij}$ que verifican que $\exp(\hth_{ij})>1$? ¿Y si $\exp(\hth_{ij})<1$?}
	\end{enumerate}
\end{enumerate}