\includepdf[pages=-]{"Tareas/Tema 4/Hoja 4"}
\begin{enumerate}[label=\color{red}\textbf{\arabic*)}, leftmargin=*]
	\item \lb{Calcular las componentes principales para una variable bidimensional con matriz de covarianzas \[ V=\begin{pmatrix}
			1 & 0.8\\
			0.8 & 1
		\end{pmatrix}. \] ¿Qué información contiene cada componente? Calcular la matriz de saturaciones e interpretar sus valores.}
	
	$\left|V-\lambda I \right|=\begin{vmatrix}
		1-\lambda & 0.8\\
		0.8 & 1-\lambda
	\end{vmatrix}=(1-\lambda)^2-0.8^2=\lambda^2-2\lambda+1-0.64=\lambda^2-2\lambda+0.36$
	
	$\lambda=\dfrac{2\pm\sqrt{(-2)^2}-4\cdot 1\cdot 0.36}{2\cdot 1}=\dfrac{2\pm1.6}{2}=1\pm0.8$
	
	$\bboxed{\begin{array}{l}
			\lambda_1 = 1.8\\
			\lambda_2=0.2
	\end{array}}$

1ª Componente: $Vx=\lambda_1x$

$\begin{pmatrix}
	1 & 0.8\\
	0.8 & 1
\end{pmatrix}\cdot\begin{pmatrix}
x_1\\
x_2
\end{pmatrix}=1.8\cdot\begin{pmatrix}
x_1\\
x_2
\end{pmatrix}\longrightarrow\begin{pmatrix}
x_1+0.8x_2\\
0.8x_1+x_2
\end{pmatrix}=\begin{pmatrix}
1.8x_1\\
1.8x_2
\end{pmatrix}\longrightarrow\begin{pmatrix}
-0.8x_1+0.8x_2\\
0.8x_1-0.8x_2
\end{pmatrix}=\begin{pmatrix}
0\\
0
\end{pmatrix}\longrightarrow x_1=x_2\longrightarrow v=\alpha(1,1)'$

$t_1=\left(\dfrac{1}{\sqrt{2}},\dfrac{1}{\sqrt{2}}\right)'$
	
	$Y_1=\dfrac{1}{\sqrt{2}}X_1+\dfrac{1}{\sqrt{2}}X_2$
	
2ª Componente: $Vx=\lambda_2x$

$\begin{pmatrix}
	1 & 0.8\\
	0.8 & 1
\end{pmatrix}\cdot\begin{pmatrix}
x_1\\
x_2
\end{pmatrix}=0.2\cdot\begin{pmatrix}
x_1\\
x_2
\end{pmatrix}\longrightarrow\begin{pmatrix}
x_1+0.8x_2\\
0.8x_1+x_2
\end{pmatrix}=\begin{pmatrix}
0.2x_1\\
0.2x_2
\end{pmatrix}\longrightarrow\begin{pmatrix}
0.8x_1+0.8x_2\\
0.8x_1+0.8x_2
\end{pmatrix}=\begin{pmatrix}
0\\
0
\end{pmatrix}\longrightarrow x_1=-x_2\longrightarrow v=\alpha(1,-1)'$

$t_2=\left(\dfrac{1}{\sqrt{2}},-\dfrac{1}{\sqrt{2}}\right)'$
	
	$Y_2=\dfrac{1}{\sqrt{2}}X_1-\dfrac{1}{\sqrt{2}}X_2$
	\item \lb{Calcular las componentes principales para una variable bidimensional con matriz de correlaciones \[ \Pi=\begin{pmatrix}
			1 & r\\
			r & 1
		\end{pmatrix}. \]¿Qué condiciones debe verificar $r$? Calcular la información que contiene cada componente.}
	
	$\left|\Pi-\lambda I\right|=\begin{vmatrix}
		1-\lambda & r\\
		r & 1-\lambda
	\end{vmatrix}=(1-\lambda)^2-r^2=\lambda^2-2r\lambda+1-r^2\longrightarrow \lambda_{1,2}=1\pm r$
	
	Como $r$ es un coeficiente de correlación, la condición que debe cumplir es que $-1<r<1$.
	
	\item \lb{Calcular las componentes principales para una variable bidimensional con matriz de covarianzas \[ \begin{pmatrix}
			10 & -3\\
			-3 & 2
		\end{pmatrix}. \]Calcular la matriz de saturaciones e interpretar sus valores.}
	
	$\left|V-\lambda I\right|=\begin{vmatrix}
		10-\lambda & -3\\
		-3 & 2-\lambda
	\end{vmatrix}=(10-\lambda)\cdot(2-\lambda)-(-3)^2=20-10\lambda-2\lambda+\lambda^2-9=\lambda^2-12\lambda-11$
	
	$\lambda=\dfrac{12\pm\sqrt{(-12)^2-4\cdot1\cdot11}}{2\cdot1}=\dfrac{12\pm10}{2}=\begin{cases}
		\dfrac{12+10}{2}=11\\
		\dfrac{12-10}{2}=1
	\end{cases}$\\
	$\bboxed{\begin{array}{l}
		\lambda_1=11\\
		\lambda_2=1
	\end{array}}$
	
	1º Componente: $Vx=\lambda_1x$
	
	$\begin{pmatrix}
		10 & -3\\
		-3 & 2
	\end{pmatrix}\cdot\begin{pmatrix}
	x_1\\
	x_2
	\end{pmatrix}=11\cdot\begin{pmatrix}
	x_1\\
	x_2
	\end{pmatrix}\longrightarrow\begin{pmatrix}
	10x_1-3x_2\\
	-3x_1+2x_2
	\end{pmatrix}=\begin{pmatrix}
	11x_1\\
	11x_2
	\end{pmatrix}\longrightarrow\begin{pmatrix}
	-x_1-3x_2\\
	-3x_1-9x_2
	\end{pmatrix}=\begin{pmatrix}
	0\\
	0
	\end{pmatrix}\longrightarrow x_1=-3x_2\longrightarrow v=\alpha(1,-3)'$
	
	$t_1=\left(\dfrac{1}{\sqrt{10}},-\dfrac{3}{\sqrt{10}}\right)'$
	
	2ª Componente: $Vx=\lambda_2x$
	
	$\begin{pmatrix}
		10 & -3\\
		-3 & 2
	\end{pmatrix}\cdot\begin{pmatrix}
	x_1\\
	x_2
	\end{pmatrix}=1\cdot\begin{pmatrix}
	x_1\\
	x_1
	\end{pmatrix}\longrightarrow\begin{pmatrix}
	10x_1-3x_2\\
	-3x_1+2x_2
	\end{pmatrix}=\begin{pmatrix}
	x_1\\
	x_2
	\end{pmatrix}\longrightarrow\begin{pmatrix}
	9x_1-3x_2\\
	-3x_1+x_2
	\end{pmatrix}\longrightarrow x_2=3x_1\longrightarrow v=\alpha(3,1)'$
	
	$t_2=\left(\dfrac{3}{\sqrt{10}},\dfrac{1}{\sqrt{10}}\right)'$
	
	$T=(t_1|t_2)=\begin{pmatrix}
		\frac{1}{\sqrt{10}} & \dfrac{3}{\sqrt{10}}\\
		-\dfrac{3}{\sqrt{10}} & \dfrac{1}{\sqrt{10}}
	\end{pmatrix}$
	
	$A=\mathrm{diag}(V)^{-\frac{1}{2}}TD^{-\frac{1}{2}}=\begin{pmatrix}
		10 & 2\\
	\end{pmatrix}^{-\frac{1}{2}}\cdot\begin{pmatrix}
	\frac{1}{\sqrt{10}} & \dfrac{3}{\sqrt{10}}\\
	-\dfrac{3}{\sqrt{10}} & \dfrac{1}{\sqrt{10}}
	\end{pmatrix}\cdot\begin{pmatrix}
	11 & 0\\
	0 & 1
	\end{pmatrix}^{\frac{1}{2}}=\begin{pmatrix}
	-0.0764 & -0.7708
	\end{pmatrix}\cdot\begin{pmatrix}
	11 & 0\\
	0 & 1
	\end{pmatrix}^{\frac{1}{2}}=\begin{pmatrix}
	-0.2534 & -0.7708
	\end{pmatrix}$
	\item \lb{Calcular la primera componente principal para una variable tridimensional con media cero y matriz de correlaciones \[ \begin{pmatrix}
			1 & 0.8 & 0.8\\
			0.8 & 1 & 0.8\\
			0.8 & 0.8 & 1
		\end{pmatrix}. \]}
	
	$|V-\lambda I|=\begin{vmatrix}
		1-\lambda & 0.8 & 0.8\\
		0.8 & 1-\lambda & 0.8\\
		0.8 & 0.8 & 1-\lambda
	\end{vmatrix}=(\lambda-1)^3+0.8^3+0.8^3-(0.8^2\cdot(1-\lambda)+0.8^2\cdot(1-\lambda)+0.8^3\cdot(1-\lambda))=-\lambda^3+3\lambda^2-3\lambda+1+0.512+0.512-0.64+0.64\lambda-0.64+0.64\lambda-0.64+0.64\lambda=-\lambda^3+3\lambda^2-1.08\lambda+0.104$
	
	$\bboxed{\begin{array}{l}
			\lambda_1=2.6\\
			\lambda_2=0.2
	\end{array}}$

$\lambda=0.2$ con multiplicidad 2.
	\item \lb{Calcular las componentes principales para una variable tridimensional con media cero y matriz de covarianzas \[ \Sigma=\begin{pmatrix}
			\beta^2+\delta & \beta & \beta\\
			\beta & 1+\delta & 1\\
			\beta & 1 & 1+\delta
		\end{pmatrix}. \](Indicación: $\Sigma-\delta I=(\beta,1,1)'(\beta,1,1)$).}
	
	\item \lb{Demostrar que si las varianzas iniciales son iguales entonces las componentes principales que se obtienen con la matriz de covarianzas son iguales a las que se obtienen con la matriz de correlaciones.}
	
	\item \lb{Calcular las componentes principales de $k$ variables con media cero, varianza uno y correlaciones iguales a $r$. ¿Qué condiciones debe verificar $r$? Calcular la información que contiene cada componente.}
	
	\item \lb{Demostrar que las componentes principales no son invariantes por cambio de escala.}
	
\end{enumerate}