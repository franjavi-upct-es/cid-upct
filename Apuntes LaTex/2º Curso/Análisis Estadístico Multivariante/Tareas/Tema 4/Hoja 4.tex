\includepdf[pages=-]{"Tareas/Tema 4/Hoja 4"}
\begin{enumerate}[label=\color{red}\textbf{\arabic*)}, leftmargin=*]
	\item \lb{Calcular las componentes principales para una variable bidimensional con matriz de covarianzas \[ V=\begin{pmatrix}
			1 & 0.8\\
			0.8 & 1
		\end{pmatrix}. \] ¿Qué información contiene cada componente? Calcular la matriz de saturaciones e interpretar sus valores.}

	\item \lb{Calcular las componentes principales para una variable bidimensional con matriz de correlaciones \[ \Pi=\begin{pmatrix}
			1 & r\\
			r & 1
		\end{pmatrix}. \]¿Qué condiciones debe verificar $r$? Calcular la información que contiene cada componente.}
	
	\item \lb{Calcular las componentes principales para una variable bidimensional con matriz de covarianzas \[ \begin{pmatrix}
			10 & -3\\
			-3 & 2
		\end{pmatrix}. \]Calcular la matriz de saturaciones e interpretar sus valores.}
	
	\item \lb{Calcular la primera componente principal para una variable tridimensional con media cero y matriz de correlaciones \[ \begin{pmatrix}
			1 & 0.8 & 0.8\\
			0.8 & 1 & 0.8\\
			0.8 & 0.8 & 1
		\end{pmatrix}. \]}
	
	\item \lb{Calcular las componentes principales para una variable tridimensional con media cero y matriz de covarianzas \[ \Sigma=\begin{pmatrix}
			\beta^2+\delta & \beta & \beta\\
			\beta & 1+\delta & 1\\
			\beta & 1 & 1+\delta
		\end{pmatrix}. \](Indicación: $\Sigma-\delta I=(\beta,1,1)'(\beta,1,1)$).}
	
	\item \lb{Demostrar que si las varianzas iniciales son iguales entonces las componentes principales que se obtienen con la matriz de covarianzas son iguales a las que se obtienen con la matriz de correlaciones.}
	
	\item \lb{Calcular las componentes principales de $k$ variables con media cero, varianza uno y correlaciones iguales a $r$. ¿Qué condiciones debe verificar $r$? Calcular la información que contiene cada componente.}
	
	\item \lb{Demostrar que las componentes principales no son invariantes por cambio de escala.}
	
\end{enumerate}