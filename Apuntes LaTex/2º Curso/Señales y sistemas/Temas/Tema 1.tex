\section{Conceptos básicos de señales y sistemas}
\subsection{Introducción al procesado de señales}
\lb{Comunicación:} Transmisión de información entre dos o más puntos. Intervienen:
\begin{itemize}[label=$-$]
\item Señal: transporta información.
\item Sistema: actúa sobre la señal, modificándola.
\end{itemize}
\lb{Concepto de señal:} representación matemática (analítica) o gráfica de alguna magnitud física o dato.

Información: almacenada en $M$ componentes (señales) dependientes de $N$ variables independientes.

Ejemplos: tensión, corriente, audio, vídeo \dots

\lb{Concepto de sistema:} proceso que realiza una transformación sobre la señal.

\begin{center}
\begin{tikzpicture}
\node (s1) at (0,0) {$\begin{array}{r}
x\\
\text{señal}
\end{array}$};
\node[draw=lightblue, fill=lightblue!10, rectangle, rounded corners=5pt, inner sep=8pt, right=of s1, line width=1.2] (S)  {Sistema};
\node[right=of S] (s2) {$\begin{array}{l}
y\\
\text{señal modificada}
\end{array}$};
\draw[-latex, lightblue, line width=1.2] (s1) -- (S);
\draw[-latex, lightblue, line width=1.2] (S) -- (s2);
\end{tikzpicture}
\end{center}
\subsection{Transformaciones de la variable independiente}
\subsection{Señales exponenciales y sinusoidales}
\subsection{Señales elementales}
\subsection{Sistemas continuos y discreto}
\subsection{Propiedades básicas de los sistemas}