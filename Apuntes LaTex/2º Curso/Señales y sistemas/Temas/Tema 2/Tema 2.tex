\includepdf[pages=-]{"Temas/Tema 2/Hoja 2"}

\begin{enumerate}[label=\color{red}\textbf{\arabic*)}, leftmargin=*]
	\item \lb{Obtenga la convolución de las señales $x(t)=\prod\left(\dfrac{t-\frac{T_2}{2}}{T_2}\right)$ y $h(t)=t\prod\left(\dfrac{t-T}{2T}\right)$}
	
	$y(t)=\begin{cases}
		0 & t<0\\
		\dfrac{t^2}{2} & 0<t<T\\
		tT-\dfrac{t^2}{2} & T<2T\\
		\dfrac{t^2}{2}+tT+\dfrac{3T^2}{2} & 2T<t<3T\\
		0 & t>3T
	\end{cases}$
	
	\item \lb{Calcule $\left(\dfrac{t}{T_1}+1\right)\prod\left(\dfrac{t-\frac{T_1}{2}}{T_1}\right)\cdot\prod\left(\dfrac{t-\frac{T_2}{2}}{T_2}\right)$, con $T_2>T_1$.}
	
	$\lbb{\left(\dfrac{t}{T_1}+1\right)\prod\left(\dfrac{t-\frac{T_1}{2}}{T_1}\right)}{x(t)}\cdot\lbb{\prod\left(\dfrac{t-\frac{T_2}{2}}{T_2}\right)}{h(t)}$
	
	\begin{enumerate}[label=\color{lightblue}\arabic*)]
		\item $t<0\longrightarrow y(t)=0$
		\item $0<t<T_1\longrightarrow y(t)=\int_0^t\left(\dfrac{\tau}{T_1}+1\right)\dtau=\left[\dfrac{\tau^2}{2T_1}+\tau\right]_0^t=\dfrac{t^2}{2T_1}+t$
		\item $t-T_2<0,\:t>T_1\longrightarrow T_1<t<T_2\longrightarrow y(t)=\int_0^{T_1}\left(\dfrac{\tau}{T_1}+1\right)\dtau=\left[\dfrac{\tau^2}{2T_1}+\tau\right]_0^{T_1}=\dfrac{T_1^2}{2T_1}+T_1=\dfrac{3T_1}{2}$
		\item $0<t-T_2<T_1\longrightarrow T_2<t<T_1+T_2$
		
		$\begin{aligned}
			y(t) &=\int_{t-T_2}^{T_1}\left(\dfrac{\tau}{2T_1}+2\right)\dtau=\left[\dfrac{\tau^2}{2T_1}+2\right]_{t-T_2}^{T_1}=\left[\dfrac{T_1^2}{2T_1}+T_1\right]-\left[\dfrac{(t-T_2)^2}{2T_1}+(t-T_2)\right]\\
			&=-\dfrac{t^2}{2T_1}-\left(1-\dfrac{T_2}{T_1}\right)t+\dfrac{3T_1}{2}+T_2+\dfrac{T_2^2}{2T_1}
		\end{aligned}$
		\item $t-T_2>T_1\longrightarrow t>T_1+T_2\longrightarrow y(t)=0$
	\end{enumerate}
	
	\begin{center}
		\includegraphics[width=\linewidth]{"Temas/Tema 2/ScreenShot001.drawio"}
	\end{center}
	
	\item \lb{Calcule la convolución de $x(t)=e^{2t}u(-t)$ con $h(t)=u(t-3)$.}
	
	\item \lb{Sea $x[n]=\delta[n]+2\delta[n-1]-\delta[n-3]$ y $h[n]=2\delta[n+1]+2\delta[n-1]$.} Calcule y dibuje cada una de las siguientes convoluciones:
	\begin{enumerate}[label=\color{red}\alph*)]
		\item $\db{y_1[n]=x[n]\cdot h[n]}$
		\item $\db{y_2[n]=x[n+2]\cdot h[n]}$
		\item $\db{y_3[n]=x[n]\cdot h[n+2]}$
	\end{enumerate}
	\item \lb{Un sistema lineal $S$ relaciona su entrada $x[n]$ y su salida $y[n]$ como \[ y[n]=\sum_{k=-\infty}^{\infty}x[k]g[n-2k] \] donde $g[n]=u[n]-u[n-4]$.}
	\begin{enumerate}[label=\color{red}\alph*)]
		\item \db{Determine $y[n]$ cuando $x[n]=\delta[n-1]$}
		\item \db{Determine $y[n]$ cuando $x[n]=\delta[n-2]$}
		\item \db{¿Es $S$ un sistema LTI?}
		\item \db{Determine $y[n]$ cuando $x[n]=u[n]$}
	\end{enumerate}
	\item 
	\item 
	\item 
	\item \lb{Sean \[ x(t)=u(t-3)-u(t-5)\qquad h(t)=e^{-3t}u(t) \]}
	\begin{enumerate}[label=\color{lightblue}\arabic*)]
		\item \db{Calcule $y(t)=x(t)\cdot h(t)$}
		
		\item \db{Calcule $g(t)=\dfrac{\dx(t)}{\dt}\cdot h(t)$}
		
		\item \db{Establezca una relación entre $g(t)$ e $y(t)$.} 
	\end{enumerate}
	\item 
	\item 
	\item \lb{Para las siguientes respuestas al impulso de sistemas LTI, determine si cada sistema es causal  y/o estable, justificando la repuesta.}
	\begin{enumerate}[label=\color{red}\alph*)]
		\item $\db{h[n]=\left(\dfrac{1}{5}\right)^nu[n]}$
		
		Causal $(h[n]=0\:\forall n<0)$
		
		$\sum_{k=-\infty}^{\infty}\left|h[k]\right|=\sum_{k=0}^{\infty}\left(\dfrac{1}{5}\right)^n\longrightarrow$ Converge a un valor, por lo tanto es un sistema estable.
		
		Como es un sistema casual y estable $\longrightarrow$ es realizable
		\item $\db{h[n]=0.8^nu[n+2]}$
		
		$h[n]\neq0$ para $-1,-2\longrightarrow$ No es causal
		
		$\sum_{k=-\infty}^{\infty}\left|h[k]\right|=\sum_{k=-2}^{\infty}0.8^n=\dfrac{0.8^{-2}-\tozero{0.8^{\infty}}}{1-0.8}\simeq7.8\longrightarrow$ Estable
		
		Como es un sistema estable pero no es causal, entonces diremos que no es realizable
		\item $\db{h[n]=\left(\dfrac{1}{2}\right)^nu[-n]}$
		
		No es causal
		
		$\sum_{k=-\infty}^{\infty}\left|h[k]\right|=\sum_{k=-\infty}^{0}\cancelto{\infty}{\left(\dfrac{1}{2}\right)^n}\longrightarrow$ No es estable
	\end{enumerate}
	\item \lb{Considere un sistema LTI que se encuentra inicialmente en reposo y cuya entrada $x(t)$ y salida $y(t)$ se relacionan por la ecuación diferencial \[ \dfrac{\mathrm{d}}{\dt}y(t)+4y(t)=x(t) \]}
	\begin{enumerate}[label=\color{red}\alph*)]
		\item \db{Obtenga la respuesta al impulso del sistema.}
		
		Solución homogénea normal:
		
		$\begin{rcases}
			\dfrac{\dy_h(t)}{\dt}+4y_h(t)\\
			y_h(t)=C_h\cdot e^{st}
		\end{rcases}s\cdot C_h e^{st}+4C_he^{st}=0\longrightarrow C_h(s+4)=0\longrightarrow s=4$
		
		$\begin{array}{l}
			y_h(t)=C_he^{-4t}\\
			h(t)=b_0\cdot e^{-4t}\cdot u(t)\\
			\bboxed{h(t)=e^{-4t}\cdot u(t)}
		\end{array}$
		
		\item \db{Si $x(t)=e^{(-1+3j)t}u(t)$, calcule $y(t)$.}
		Solución particular:
		
		$y_p(t)=A\cdot x(t)=Ae^{(-1+3j)t}\:\forall t>0$
		
		$\lbb{(-1+3j)A}{}\cdot e^{(-1+3j)t}+\lbb{4A}{}\cdot e^{(-1+3j)t}=e^{(-1+3j)t}$
		
		$(-1+3j)A+4A=1\longrightarrow A=\dfrac{1}{3(1+j)}=\dfrac{1-j}{6}$
		
		$y_p(t)=\dfrac{1-j}{6}e^{(-1+3j)t}$
		
		$y(t)=y_h(t)+y_p(t)=C_he^{-4t}\cdot u(t)+\dfrac{1-j}{6}e^{(-1+3j)t}\cdot u(t)$
		
		$y(0)=0\longrightarrow C_h\cdot 1+\dfrac{1-j}{6}\cdot 1=0\longrightarrow C_h=\dfrac{j-1}{6}$
		
		$\bboxed{y(t)=1-j\left[e^{(-1+3j)t}-e^{-4t}\right]\cdot u(t)}$
	\end{enumerate}
	\item \lb{Considere un sistema LTI que se encuetra inicialmente en reposo y cuya entrada $x(t)$ y salida $y(t)$ se relacionan por la ecuación diferencial \[ \dfrac{\mathrm{d}}{\dt}y(t)+3y(t)=2x(t) \]}
	\begin{enumerate}[label=\color{red}\alph*)]
		\item \db{Si $x(t)=\cos(2t)u(t)$, calcule $y(t)$}
		
		Solución partida
		
		$\begin{array}{l}
			x_1=\dfrac{e^{2tj}}{2}\longrightarrow y_{p_1}(t)=C_{p_1}e^{2tj}\longrightarrow \lbb{j2C_{p_1}}{}e^{2tj}+\lbb{3C_{p_1}}{}e^{2tj}=e^{2tj}\longrightarrow j2C_{p_1}+3C_{p_1}=1\longrightarrow C_{p_1}=\dfrac{1}{3+2j}\\
			x_2=\dfrac{e^{-2tj}}{2}\longrightarrow  \lbb{-j2C_{p_2}}{}e^{-2tj}+\lbb{3C_{p_2}}{}e^{-2tj}=e^{-2tj}\longrightarrow -j2C_{p_2}+3C_{p_2}=1\longrightarrow C_{p_2}=\dfrac{1}{3-2j}\\
			x(t)=\dfrac{1}{2}\left[e^{j2t}+e^{-j2t}\right]\longrightarrow y_p=\dfrac{1}{3+2j}e^{j2t}+\dfrac{1}{3-2j}e^{-j2t}
		\end{array}$
		
		Solución homogénea:
		
		$y_h(t)=C_he^{st}\longrightarrow sC_he^{st}+3C_he^{st}=0\longrightarrow s+3=0\longrightarrow s=-3\longrightarrow y_h(t)=C_h\cdot e^{-3t}\cdot u(t)$
		
		$y(t)=C_he^{-st}+\dfrac{1}{3+2j}e^{2tj}+\dfrac{1}{3-2j}e^{-2tj}$
		
		$y(0)=0\longrightarrow C_h+\dfrac{1}{3+2j}+\dfrac{1}{3-2j}=0\longrightarrow C_h=\dfrac{3-\cancel{2j}+3+\cancel{2j}}{13}=-\dfrac{6}{13}$
		
		$y(t)=-\dfrac{6}{13}e^{-3t}+\lbb{\dfrac{3-2j}{13}e^{2jt}}{a}+\lbb{\dfrac{3+2j}{13}e^{-2jt}}{a^*}$
		
		$a+a^*=2\cdot\mathrm{Re}\{a\}$
		
		$\begin{aligned}
			a&=\dfrac{3-2j}{13}e^{2tj}=\dfrac{3-2j}{13}\cdot\left(\cos(2t)+j\cdot\sin(2t)\right)\\
			&=\dfrac{1}{13}\left[\underbrace{3\cos(2t)}+j\cdot3\sin(2t)-j\cdot2\cos(2t)+\underbrace{2\sin(2t)}\right]\\
		\end{aligned}\\
		2\cdot\mathrm{Re}\{a\}=\dfrac{1}{13}\left[6\cos(2t)+4\sin(2t)\right]\\
		\bboxed{y(t)=\dfrac{1}{13}\left[-6e^{-3t}+6\cos(2t)+4\sin(2t)\right]\cdot u(t)}$
		\item \db{Obtenga la respuesta al impulso del sistema.}
		
		$\begin{rcases}
			y_h(t)=\dfrac{-6}{13}e^{-3t}\\
			b_0=2
		\end{rcases}h(t)=b_0\cdot e^{-3t}\cdot u(t)\longrightarrow\bboxed{h(t)=2\cdot e^{-3t}\cdot u(t)}$
	\end{enumerate}
	
	\item \lb{Obtenga la respuesta al impulso, así como las propiedades de memoria, causalidad, estabilidad, invarianza en el tiempo y linealidad de los siguientes sitemas:}
	\begin{enumerate}[label=\color{red}\alph*)]
		\item $\db{y(t)=\int_{-\infty}^{t}x(\tau)\dtau}$
		
		$x(t)=\delta(t)\longrightarrow y(t)=h(t)$
		
		$h(t)=\int_{-\infty}^{t}\delta(\tau)\dtau=\left\{\begin{array}{l}
			t<0,\:h(t)=0\\
			t>0,\:h(t)=1
		\end{array}\right\}h(t)=u(t)$
		\begin{itemize}
			\item Memoria:
			
			$h(t)\neq0,t\neq0\longrightarrow$ con memoria\\
			$h(t)=0,t<0\longrightarrow$ casual\\
			\item Estabilidad
			
			$\int_{-\infty}^{+\infty}|h(t)|\dt=\int_0^{\infty}1\dt=\left[t\right]_0^{\infty}=\infty\longrightarrow$ no estable
			\item Invarianza en el tiempo
			
			$x_1(t)=x(t-t_0)\longrightarrow y_1(t)=\int_{-\infty}^tx(\lbb{\tau-t_0}{s})\dtau=\left\{\begin{array}{l}
				s=\tau-t_0\\
				\mathrm{d}s=\dtau
			\end{array}\right\}=\int_{-\infty}^{t-t_0}x(s)\mathrm{d}s=y(t-t_0)\longrightarrow$ Invariante
			\item Linealidad del sistema:
			
			$ax_1(t)+bx_2(t)\longrightarrow y(t)=\int_{-\infty}^{t}[ax_1(\tau)+bx_2(\tau)]\dtau=a\lbb{\int_{-\infty}^{t}x_1(\tau)\dtau}{y_1(t)}+b\lbb{\int_{-\infty}^{t}x_2(\tau)\dtau}{y_2(t)}=ay_1(t)+by_2(t)\longrightarrow$ lineal
		\end{itemize}
		\begin{tikzpicture}[baseline=(current bounding box.center)]
			\draw[-latex] (-3,0) -- (3,0) node[right] {$\tau$};
			\draw[-latex] (0,-1) -- (0,5);
			\draw[-*, lightblue, line width=1.2] (-3,0) -- (-1,0) node[below] {$t$};
			\draw[-*, blue, line width=1.2] (-3,0.5) -- (1,0.5);
			\node[below] at (1,0) {$t$};
			\draw[-latex, lightblue, line width=1.2] (0,0) -- (0,4) node[right] {$\delta(\tau)$};
		\end{tikzpicture}\qquad\begin{tikzpicture}[baseline=(current bounding box.center), scale=1.5]
		\draw[-latex] (-2,0) -- (4,0) node[right] {$t$};
		\draw[-latex] (0,-1) -- (0,2);
		\draw[lightblue, line width=1.5] (-2,0) -- (0,0) node[below left] {$0$} -- (0,1) node[left] {$1$} -- (1,1);
		\end{tikzpicture}
		\item $\db{y(t)=\int_{-\infty}^{t}3x(\tau-5)\dtau}$
		
		$x(t)\longrightarrow\delta(t)=h(t)=\int_{-\infty}^{t}3\delta(\tau-5)\dtau=\left\{\begin{array}{l}
			0,\:t<5\\
			3,\:t>5
		\end{array}\right\}h(t)=3\cdot u(t-5)$
		\begin{itemize}
			\item Memoria:
			
			Con memoria
			\item Estabilidad
			
			$h(t)=0\forall t<0\longrightarrow$ Casual
			
			
		\end{itemize}
		\item $\db{y(t)=\int_{-\infty}^{t}e^{-(t-\tau)}3x(\tau-2)\dtau}$
		
		$y(t)=\int_{-\infty}^{t}e^{-(t-\tau)}3x(\tau-2)\dtau$
		
		$h(t)=\int_{-\infty}^{t}\lbb{e^{-(t-\tau)}}{}\cdot3\cdot\delta(\tau-2)\dtau=3e^{-(t-2)}\int_{-\infty}^{t}\delta(\tau-2)\dtau=3e^{-(t-2)}\cdot u(t-2)$
		
		\begin{tikzpicture}[yscale=1.5]
			\draw[-latex] (-1,0) -- (4,0) node[right] {$t$};
			\draw[-latex] (0,-1) -- (0,2) node[above] {$h(t)$};
			\draw[lightblue, line width=1.5] (-1,0) -- (2,0) -- (2,1) -- plot[domain=2:3.85, samples=100] (\x, {exp(-3*(\x-2))});
			\node[below left] at (0,0) {$0$};
			\node[left] at (0,1) {$1$};
			\draw[lightblue, dashed] (0,1) -- (2,1);
		\end{tikzpicture}
		
		Con memoria
		
		Causal
		
		$\int_{-\infty}^{+\infty}\left|h(t)\right|\dt=\int_{2}^{+\infty}\left|h(t)\right|\dt=\int_{2}^{\infty}e^{-3(t-2)}\dt<\infty\longrightarrow$ sistema estable
		
		$x_1(t)=x(t-t_0)\longrightarrow y_1(t)=\int_{-\infty}^{t}e^{-(t-\tau)}3x(\lbb{\tau-t_0}{s}-2)\dtau=\left\{\begin{array}{l}
			s=\tau-t_0\longrightarrow\tau=s+t_0\\
			\mathrm{d}s=\dtau
		\end{array}\right\}=\linebreak\int_{-\infty}^{t-t_0}e^{-(\overbrace{s+t_0}^{-s-t_0})}3x(s-2)\mathrm{d}s=y(t-t_0)\longrightarrow$ Invariante
		\item $\db{y(t)=\int_{t}^{\infty}e^{t-\tau}x(\tau+5)\dtau}$
		
		$h(t)=\int_{t}^{\infty}e^{t-\tau}\cdot \delta(\tau+5)\:\mathrm{d}\tau=e^{t+5}\int_{t}^{\infty}\delta(\tau+5)\:\mathrm{d}\tau=\left\{\begin{array}{ll}
			e^{t+5}\cdot1 & t<-5\\
			0 & t>-5
		\end{array}\right\}=e^{t+5}\cdot u(-t-5)$
		
		\begin{tikzpicture}[xscale=0.5, baseline=(current bounding box.center)]
			\draw[-latex] (-6, 0) -- (2,0) node[right] {$\tau$};
			\draw[-latex] (0,-1) -- (0,2);
			\draw[-latex, lightblue, line width=1.2] (-5,0) node[below] {\color{black}$-5$} -- (-5,1.5);
			\draw[lightblue, -*, line width=1.1] (-6,0) node[below] {$t$} -- (-6,0.25) -- (1.5,0.25);
			\draw[blue, -*, line width=1.1] (-4,0) node[below] {$t$} -- (-4,0.5) -- (1.5,0.5);
		\end{tikzpicture}\qquad\begin{tikzpicture}[baseline=(current bounding box.center), xscale=0.5]
		\draw[-latex] (-7, 0) -- (7,0);
		\draw[-latex] (0,-1) -- (0,2);
		\draw[lightblue, line width=1.5] (-7,1) -- (-5,1) -- (-5,0) node[below] {$-5$}-- (5,0)node[below] {$5$} -- (5,1)  -- (7,1);
		\end{tikzpicture}
		
		$\int_{-\infty}^{+\infty}\left| h(t) \right|\:\mathrm{d}t=\int_{-\infty}^{-5}e^{t+5}\:\mathrm{d}t=e^5\int_{-\infty}^{-5}e^t\:\mathrm{d}t=e^t\left[ e^t \right]_{-\infty}^{-5}=e^{5}\cdot(e^{-5}-0)=1<\infty\longrightarrow$ Estable
		\item $\db{y(t)=\int_{-\infty}^{t}e^{-(t-\tau)}x(\tau-2)\dtau}$
	\end{enumerate}
\end{enumerate}