\begin{center}
    \textbf{\large Hoja de ejercicios Tema 3: Intervalos de confianza} 
\end{center}
\begin{enumerate}[label=\color{red}\textbf{\arabic*)}]
    \item \lb{Sea $X\sim \mathcal{N}(\mu,\sigma^2)$ y se considera una m.a.s. de tamaño $n$ de $X$.}
        \begin{enumerate}[label=\color{red}\textbf{\alph*)}]
            \item \db{Suponiendo $\sigma^2$ conocida obtener el intervalo de confianza para $\mu$ con nivel de confianza $1-\alpha$. Determinar el tamaño de muestra necesario para estimar $\mu$ con una precisión $\pm\delta$.}

                Como queremos estimar $\mu$ con $\sigma^2$ conocida: 
                
                $\begin{array}{l}
                    T=\dfrac{\overline{X}-\mu}{\sigma /\sqrt{n} }\leadsto \mathcal{N}(0,1)\\
                    P(-Z_{1-\frac{1}{\alpha} }<T<Z_{1-\frac{\alpha}{2} })=1-\alpha\\
                    P\left( -Z_{1-\frac{\alpha}{2} }<\dfrac{\overline{X}-\mu}{\sigma /\sqrt{n} }<Z_{1-\frac{\alpha}{2} } \right)=1-\alpha 
                \end{array}\qquad$ 
                \begin{tikzpicture}[baseline=(current bounding box.center), scale=0.7]
                    \begin{axis}[
                        domain=-3.5:3.5,
                        samples=200,
                        hide y axis,
                        axis x line=middle,
                        axis line style={-latex},
                        xmin=-3.5, xmax=3.5,
                        ymin=0, ymax=0.5,
                        clip=false,
                        ticks=none
                        ]
                        \addplot[thick, lightblue, name path=Ndist] {1/sqrt(2*pi)*exp(-x^2/2)};
                        \def\Zalph{1.96}
                        \path[name path=xaxis] (axis cs:-3.5,0) -- (axis cs:3.5,0);
                        \addplot[lightblue!20] fill between [of=Ndist and xaxis, soft clip={domain=-\Zalph:\Zalph}];
                        \draw[dashed, lightblue] (axis cs: -\Zalph, 0) -- (axis cs:-\Zalph,{1/sqrt(2*pi)*exp(-(\Zalph)^2/2)});
                        \draw[dashed, lightblue] (axis cs: \Zalph, 0) -- (axis cs:\Zalph,{1/sqrt(2*pi)*exp(-(\Zalph)^2/2)});
                        \draw[lightblue] (axis cs: 0, 0) node[below] {$0$} -- (axis cs: 0, {1/sqrt(2*pi)});
                        \node[below, color=lightblue] at (axis cs: -\Zalph, 0) {$-Z_{1-\frac{\alpha}{2} }$};
                        \node[below, color=lightblue] at (axis cs: \Zalph, 0) {$Z_{1-\frac{\alpha}{2} }$};
                        \draw[-latex, lightblue] (axis cs:-2.5,0.01) -- (axis cs: -3.5, 0.04) node[above left] {$-\dfrac{\alpha}{2}$};
                        \draw[-latex, lightblue] (axis cs:2.5,0.01) -- (axis cs: 3.5, 0.04) node[above right] {$\dfrac{\alpha}{2}$};
                    \end{axis}
                \end{tikzpicture}       

                Despejamos $\mu$: \[
                \begin{array}{l}
                    P\left( -\dfrac{\sigma}{\sqrt{n} }Z_{1-\frac{\alpha}{2} }<\overline{X}-\mu<\dfrac{\sigma}{\sqrt{n} }Z_{1-\frac{\alpha}{2} } \right) =1-\alpha\\
                    -\left( P\left( -\overline{X}-\dfrac{\sigma}{\sqrt{n} }Z_{1-\frac{\alpha}{2} }<-\mu<-\overline{X}+\dfrac{\sigma}{\sqrt{n} }Z_{1-\frac{\alpha}{2} } \right)  \right) =1-\alpha\\
                    P\left( \overline{X}-\dfrac{\sigma}{\sqrt{n} }Z_{1-\frac{\alpha}{2} }<\mu<\overline{X}+\dfrac{\sigma}{\sqrt{n} }Z_{1-\frac{\alpha}{2} } \right) =1-\alpha
                \end{array}
                \] 
                Para determinar el tamaño de la muestra que nos asegure una precisión de $\pm\delta$. \[
                \dfrac{\sigma}{\sqrt{n} }Z_{1-\frac{\alpha}{2} }\le \delta\longrightarrow \dfrac{\sigma}{\delta}Z_{1-\frac{\alpha}{2} }\le \sqrt{n}\longrightarrow \left( \dfrac{\sigma Z_{1-\frac{\alpha}{2} }}{\delta} \right)  ^2\le n.
                \] 
            \item \db{Suponiendo $\sigma^2$ desconocida, obtener un intervalo de confianza óptimo para $\mu$ con nivel de confianza $1-\alpha$. Determinar el tamaño de muestra necesario para estimar $\mu$ con una precisión $\pm\delta$.} 

                Como queremos estimar $\mu$ con $\sigma^2$ desconocida:

                $\begin{array}{l}
                    T=\dfrac{\overline{X}-\mu}{S /\sqrt{n} }\leadsto t_{n-1}\\
                    P\left( -t_{n-1,1-\frac{\alpha}{2} }<T<t_{n-1,1-\frac{\alpha}{2} } \right) \\
                    P\left( -t_{n-1,1-\frac{\alpha}{2} }<\dfrac{\overline{X}-\mu}{S /\sqrt{n} }<t_{n-1,1-\frac{\alpha}{2} } \right) 
                \end{array}\qquad$
                \begin{tikzpicture}[baseline=(current bounding box.center), scale=0.7]
                    \begin{axis}[
                        domain=-3.5:3.5,
                        samples=200,
                        hide y axis,
                        axis x line=middle,
                        axis line style={-latex},
                        xmin=-3.5, xmax=3.5,
                        ymin=0, ymax=0.5,
                        clip=false,
                        ticks=none
                        ]
                        \addplot[thick, lightblue, name path=Ndist] {1/sqrt(2*pi)*exp(-x^2/2)};
                        \def\Zalph{1.96}
                        \path[name path=xaxis] (axis cs:-3.5,0) -- (axis cs:3.5,0);
                        \addplot[lightblue!20] fill between [of=Ndist and xaxis, soft clip={domain=-\Zalph:\Zalph}];
                        \draw[dashed, lightblue] (axis cs: -\Zalph, 0) -- (axis cs:-\Zalph,{1/sqrt(2*pi)*exp(-(\Zalph)^2/2)});
                        \draw[dashed, lightblue] (axis cs: \Zalph, 0) -- (axis cs:\Zalph,{1/sqrt(2*pi)*exp(-(\Zalph)^2/2)});
                        \draw[lightblue] (axis cs: 0, 0) node[below] {$0$} -- (axis cs: 0, {1/sqrt(2*pi)}) node[above] {$t_{n-1}$};
                        \node[below, color=lightblue] at (axis cs: -\Zalph, 0) {$-t_{n-1,1-\frac{\alpha}{2} }$};
                        \node[below, color=lightblue] at (axis cs: \Zalph, 0) {$t_{n-1,1-\frac{\alpha}{2} }$};
                        \draw[-latex, lightblue] (axis cs:-2.5,0.01) -- (axis cs: -3.5, 0.04) node[above left] {$-\dfrac{\alpha}{2}$};
                        \draw[-latex, lightblue] (axis cs:2.5,0.01) -- (axis cs: 3.5, 0.04) node[above right] {$\dfrac{\alpha}{2}$};
                    \end{axis}
                \end{tikzpicture}       

                Despejamos $\mu$: \[
                \begin{array}{l}
                    P\left( -\dfrac{S}{\sqrt{n} }t_{n-1,1-\frac{\alpha}{2} }<\overline{X}-\mu<\dfrac{S}{\sqrt{n} }t_{n-1,1-\frac{\alpha}{2} } \right) =1-\alpha\\
                    (-1)\cdot \left( P\left( -\overline{X}-\dfrac{S}{\sqrt{n} }t_{n-1,1-\frac{\alpha}{2}}<-\mu<-\overline{X}-\dfrac{S}{\sqrt{n} }t_{n-1,1-\frac{\alpha}{2} } \right)  \right) =1-\alpha\\
                    P\left( \overline{X}-\dfrac{S}{\sqrt{n} }t_{n-1,1-\frac{\alpha}{2} }<\mu<\overline{X}+\dfrac{S}{\sqrt{n} }t_{n-1,1-\frac{\alpha}{2} } \right) =1-\alpha\longrightarrow \mathrm{I.C}=\left( \overline{X}-\dfrac{S}{\sqrt{n} }t_{n-1,1-\frac{\alpha}{2} },\overline{X}-\dfrac{S}{\sqrt{n} }t_{n-1,1-\frac{\alpha}{2} } \right) 
                \end{array}
                \] 
                Para estimar el tamaño de la muestra: \[
                \dfrac{S}{\sqrt{n} }t_{n-1,1-\frac{\alpha}{2} }\le \delta\quad \rboxed{\text{¡No podemos despejar $n$ porque aparece en $t_{n-1}$ !}}
                \] 
                Supondremos que $S\approx 0$ o tomando un valor  $\sigma$ obtenido de estudios anteriores  \[
                \left( \dfrac{\sigma\cdot Z_{1-\frac{\alpha}{2} }}{\delta} \right) ^2\le n.
                \] 
        \end{enumerate}
    \item \lb{Sean $X\sim \mathcal{N}(\mu_1,\sigma_1^2)$ e $Y\sim \mathcal{N}(\mu_2,\sigma_2^2)$ independientes. Se considera una m.a.s. de tamaño $n_1$ de $X$ y una m.a.s. de tamaño  $n_2$ de $Y$.}
        \begin{enumerate}[label=\color{red}\textbf{\alph*)}]
            \item \db{Suponiendo $\sigma_1^2$ y $\sigma_2^2$ conocidas obtener un intervalo de confianza para $\mu_1-\mu_2$ con nivel de confianza $1-\alpha$.}

                Como $\sigma_1^2$ y $\sigma_2^2$ son conocidas, podemos asegurar: 

                Por el teorema central del límite: $\left\{ \begin{array}{l}
                    \overline{X}_1\sim \mathcal{N}\left(\mu_1,\dfrac{\sigma_1^2}{n}\right)\\
                    \overline{X}_2\sim \mathcal{N}\left( \mu_2,\dfrac{\sigma_2^2}{n} \right) 
                \end{array} \right\} $. Si planteamos la variable diferencia: \[
                \overline{X}_1-\overline{X}_2\sim \mathcal{N}\left( \mu_1-\mu_2,\dfrac{\sigma_1^2}{n_1}+\dfrac{\sigma_2^2}{n_2} \right) \begin{cases}
                    E(\overline{X}_1-\overline{X}_2)=E(\overline{X}_1)-E(\overline{X}_2)=\mu_1-\mu_2\\
                    \mathrm{Var}(\overline{X}_1-\overline{X}_2)=\mathrm{Var}(\overline{X}_1)+(-1)^2\cdot \mathrm{Var}(\overline{X}_2)=\dfrac{\sigma_1^2}{n_1}+\dfrac{\sigma_2^2}{n_2}
                \end{cases}
                \] 
                Por lo tanto, tipificando $\left( \dfrac{x-\mu}{\sigma} \right) $: \[
                z=\dfrac{\overline{X}_1-\overline{X}_2-(\mu_1-\mu_2)}{\sqrt{\dfrac{\sigma_1^2}{n_1}+\dfrac{\sigma_2^2}{n}} }\sim \mathcal{N}(0,1)
                \] 
                Para construir el intervalo de confianza, debemos:
                \begin{tikzpicture}[baseline=(current bounding box.center), scale=0.7]
                    \begin{axis}[
                        domain=-3.5:3.5,
                        samples=200,
                        hide y axis,
                        axis x line=middle,
                        axis line style={-latex},
                        xmin=-3.5, xmax=3.5,
                        ymin=0, ymax=0.5,
                        clip=false,
                        ticks=none
                        ]
                        \addplot[thick, lightblue, name path=Ndist] {1/sqrt(2*pi)*exp(-x^2/2)};
                        \def\Zalph{1.96}
                        \path[name path=xaxis] (axis cs:-3.5,0) -- (axis cs:3.5,0);
                        \addplot[blue!20] fill between [of=Ndist and xaxis, soft clip={domain=-3.5:-\Zalph}];
                        \draw[dashed, lightblue] (axis cs: -\Zalph, 0) -- (axis cs:-\Zalph,{1/sqrt(2*pi)*exp(-(\Zalph)^2/2)});
                        \draw[dashed, lightblue] (axis cs: \Zalph, 0) -- (axis cs:\Zalph,{1/sqrt(2*pi)*exp(-(\Zalph)^2/2)});
                        \draw[lightblue] (axis cs: 0, 0) node[below] {$0$} -- (axis cs: 0, {1/sqrt(2*pi)});
                        \node[below, color=lightblue] at (axis cs: -\Zalph, 0) {$-Z_{1-\frac{\alpha}{2} }$};
                        \node[below, color=lightblue] at (axis cs: \Zalph, 0) {$Z_{1-\frac{\alpha}{2} }$};
                        \draw[-latex, lightblue] (axis cs:-2.5,0.01) -- (axis cs: -3.5, 0.04) node[above left] {$-\dfrac{\alpha}{2}$};
                        \draw[-latex, lightblue] (axis cs:2.5,0.01) -- (axis cs: 3.5, 0.04) node[above right] {$\dfrac{\alpha}{2}$};
                    \end{axis}
                \end{tikzpicture}\quad$\lb{P\left( Z<Z_{1-\frac{\alpha}{2} } \right) =1-\dfrac{\alpha}{2}} $
                \[
                \begin{array}{l}
                    P\left( -Z_{1-\frac{\alpha}{2} }<Z<Z_{1-\frac{\alpha}{2} } \right) =1-\alpha\qquad P\left( -Z_{1-\frac{\alpha}{2} }<\dfrac{\overline{X}_1-\overline{X}_2-(\mu_1-\mu_2)}{\sqrt{\dfrac{\sigma_1^2}{n_1}+\dfrac{\sigma_2^2}{n_2}} }<Z_{1-\frac{\alpha}{2} } \right) =1-\alpha\\
                    P\left( -\sqrt{\dfrac{\sigma_1^2}{n_1}+\dfrac{\sigma_2^2}{n_2}}Z_{1-\frac{\alpha}{2} }< \overline{X}_1-\overline{X}_2-(\mu_1-\mu_2)<\sqrt{\dfrac{\sigma_1^2}{n_1}+\dfrac{\sigma_2^2}{n_2}} Z_{1-\frac{\alpha}{2} } \right) =1-\alpha\\
                    P\left( -(\overline{X}_1-\overline{X}_2)-\sqrt{\dfrac{\sigma_1^2}{n_1}+\dfrac{\sigma_2^2}{n_2}}Z_{1-\frac{\alpha}{2} }<-(\mu_1-\mu_2)<-(X_1-X_2)\sqrt{\dfrac{\sigma_1^2}{n_1}+\dfrac{\sigma_2^2}{n_2}}Z_{1-\frac{\alpha}{2} }   \right) =1-\alpha\\
                    P\left( (\overline{X}_1-\overline{X}_2)+\sqrt{\dfrac{\sigma_1^2}{n_1}+\dfrac{\sigma_2^2}{n_2}}Z_{1-\frac{\alpha}{2} }<\mu_1-\mu_2<(X_1-X_2)\sqrt{\dfrac{\sigma_1^2}{n_1}+\dfrac{\sigma_2^2}{n_2}}Z_{1-\frac{\alpha}{2} }   \right) =1-\alpha
                \end{array}
                \]  
                Por lo tanto, el intervalo de confianza es: $\left( (\overline{X}_1-\overline{X}_2)+\sqrt{\dfrac{\sigma_1^2}{n_1}+\dfrac{\sigma_2^2}{n_2}} Z_{1-\frac{\alpha}{2} },(\overline{X}_1-\overline{X}_2)\sqrt{\dfrac{\sigma_1^2}{n_1}+\dfrac{\sigma_2^2}{n_2}}Z_{1-\frac{\alpha}{2} }  \right) $.
            \item \db{Suponiendo $\sigma_1^2$ y $\sigma_2^2$ desconocidas pero iguales, obtener un intervalo de confianza para $\mu_1-\mu_2$ con nivel de confianza $1-\alpha$.} 

                Suponiendo que $\sigma_1^2$ y $\sigma_2^2$ son desconocidas pero igales, podemos estimar es valor único de la varianza como la media de las dos varianzas muestrales: \[
                \begin{rcases}
                    S_1^2=\dfrac{\sum(x_i-\overline{X}_1)}{n_1-1}\\
                    S_2^2=\dfrac{\sum(x_i-\overline{X}_2)}{n_2-1}
                \end{rcases}S_T^2=\dfrac{(n_1-1)S_1^2+(n_2-1)S_2^2}{n_1+n_2-2}\longrightarrow \begin{cases}
                    \dfrac{\sigma_1^2}{n_1}\leadsto \dfrac{S_T^2}{n_1}\\
                    \dfrac{\sigma_2^2}{n_2}\leadsto \dfrac{S_T^2}{n_2}
                \end{cases}\longrightarrow \dfrac{\sigma_1^2}{n_1}+\dfrac{\sigma_2^2}{n_2}\leadsto \dfrac{S_T^2}{n_1}+\dfrac{S_T^2}{n_2}=S_T^2\left( \dfrac{1}{n_1}+\dfrac{1}{n_2} \right) 
                \] 
                $Z=\dfrac{\overline{X}_1-\overline{X}_2-(\mu_1-\mu_2)}{S_T\sqrt{\dfrac{1}{n_1}+\dfrac{1}{n_2}} }\leadsto t_{n_1+n_2-2}\qquad$
                \begin{tikzpicture}[baseline=(current bounding box.center), scale=0.7]
                    \begin{axis}[
                        domain=-3.5:3.5,
                        samples=200,
                        hide y axis,
                        axis x line=middle,
                        axis line style={-latex},
                        xmin=-3.5, xmax=3.5,
                        ymin=0, ymax=0.5,
                        clip=false,
                        ticks=none
                        ]
                        \addplot[thick, lightblue, name path=Ndist] {1/sqrt(2*pi)*exp(-x^2/2)};
                        \def\Zalph{1.96}
                        \path[name path=xaxis] (axis cs:-3.5,0) -- (axis cs:3.5,0);
                        \addplot[blue!20] fill between [of=Ndist and xaxis, soft clip={domain=-3.5:-\Zalph}];
                        \draw[dashed, lightblue] (axis cs: -\Zalph, 0) -- (axis cs:-\Zalph,{1/sqrt(2*pi)*exp(-(\Zalph)^2/2)});
                        \draw[dashed, lightblue] (axis cs: \Zalph, 0) -- (axis cs:\Zalph,{1/sqrt(2*pi)*exp(-(\Zalph)^2/2)});
                        \draw[lightblue] (axis cs: 0, 0) node[below] {$0$} -- (axis cs: 0, {1/sqrt(2*pi)});
                        \node[below, color=lightblue] at (axis cs: -\Zalph, 0) {$-t_{n_1+n_2-2}$};
                        \node[below, color=lightblue] at (axis cs: \Zalph, 0) {$t_{n_1+n_2-2}$};
                        \draw[-latex, lightblue] (axis cs:-2.5,0.01) -- (axis cs: -3.5, 0.04) node[above left] {$-\dfrac{\alpha}{2}$};
                        \draw[-latex, lightblue] (axis cs:2.5,0.01) -- (axis cs: 3.5, 0.04) node[above right] {$\dfrac{\alpha}{2}$};
                    \end{axis}
                \end{tikzpicture}       

                La condición para plantear el intervalo de confianza es:
                \[
                \begin{array}{l}
                    P\left( -t_{n_1+n_2-2,1-\frac{\alpha}{2} }<\dfrac{\overline{X}_1-\overline{X}_2-(\mu_1-\mu_2)}{S_T\sqrt{\dfrac{1}{n_1}+\dfrac{1}{n_2}} }<t_{n_1+n_2-2,1-\frac{\alpha}{2} } \right) =1-\alpha\\
                    P\left( -S_T\sqrt{\dfrac{1}{n_1}+\dfrac{1}{n_2}}t_{n_1+n_2-2,1-\frac{\alpha}{2} }<\overline{X}_1-\overline{X}_2-(\mu_1-\mu_2)<S_T\sqrt{\dfrac{1}{n_1}+\dfrac{1}{n_2}}t_{n_1+n_2-2,1-\frac{\alpha}{2} }   \right) =1-\alpha\\
                    (-1)\cdot P\left( -(\overline{X}_1-\overline{X}_2)-S_T\sqrt{\dfrac{1}{n_1}+\dfrac{1}{n_2}}t_{n_1+n_2-2}<-(\mu_1-\mu_2)<-(\overline{X}_1-\overline{X}_2)S_T\sqrt{\dfrac{1}{n_1}+\dfrac{1}{n_2}}t_{n_1+n_2-2,1-\frac{\alpha}{2} }   \right) =1-\alpha\\
                    P\left( (\overline{X}_1+\overline{X}_2)-S_T\sqrt{\dfrac{1}{n_1}+\dfrac{1}{n_2}}t_{n_1+n_2-2,1-\frac{\alpha}{2} }<(\mu_1-\mu_2)<(\overline{X}_1-\overline{X}_2)+S_T\sqrt{\dfrac{1}{n_1}+\dfrac{1}{n_2}}t_{n_1+n_2-2,1-\frac{\alpha}{2}}   \right) =1-\alpha\\
                    P\left( (\overline{X}_1+\overline{X}_2)-S_T\sqrt{\dfrac{1}{n_1}+\dfrac{1}{n_2}}t_{n_1+n_2-2,1-\frac{\alpha}{2} }<(\mu_1-\mu_2)<(\overline{X}_1-\overline{X}_2)+S_T\sqrt{\dfrac{1}{n_1}+\dfrac{1}{n_2}}t_{n_1+n_2-2,1-\frac{\alpha}{2}}   \right) =1-\alpha\\
                    \mathrm{IC}=\left((\overline{X}_1+\overline{X}_2)-S_T\sqrt{\dfrac{1}{n_1}+\dfrac{1}{n_2}}t_{n_1+n_2-2,1-\frac{\alpha}{2} },\overline{X}_1-\overline{X}_2)+S_T\sqrt{\dfrac{1}{n_1}+\dfrac{1}{n_2}}t_{n_1+n_2-2,1-\frac{\alpha}{2}}\right) 
                \end{array}
                \] 
        \end{enumerate}
    \item \lb{Sea $X\sim \mathcal{N}(\mu,\sigma^2)$. Se considera una muestra aleatoria simple de tamaño $n$ de  $X$. Suponiendo  $\mu$ desconocida obtener un intervalo de confianza para $\sigma^2$ con nivel de confianza $1-\alpha$.}
        \newpage
    \item \lb{Sea $X$ una población con función de densidad  \[
    f(x,\theta)=\dfrac{2x}{\theta^2}\text{ para $x \in (0,\theta)$},
    \]siendo $\theta$ un parámetro positivo desconocido. Se considera una muestra aleatoria simple de tamaño $n$ de $X$. Obtener el intervalo de confianza óptimo para  $\theta$ con nivel de confianza $1-\alpha$.}
    \newpage
\item \lb{Sea $X$ una población con función de densidad  \[
f(x,\theta)=\exp(-(x-\theta)),\text{ para $x \in (\theta,+\infty)$,}
\]siendo $\theta$ un parámetro positivo desconocido. Se considera una muestra aleatoria simple de tamaño $n$ de  $X$. Obtener el intervalo de confianza óptimo para  $\theta$ con nivel de confianza $1-\alpha$.}
    \newpage
\item \lb{En un laboratorio, se estudia la velocidad de combustión de dos tipos de combustibles sólidos A y B. Se pretende determinar qué combustibles presentan una velocidad de combustión en promedio mayor de 50 cm/s. Los resultados obtenidos fueron:}

\begin{center}
    \color{lightblue}
    \begin{tabular}{cccc}
        \hline
        Combustible & Tamaño muestral & Media muestral & Varianza muestral \\ \hline
        A & 25 & 51.3 & 3.9 \\
        B & 30 & 51.5 & 3.6 \\ \hline
\end{tabular}
\end{center}
    \lb{Suponiendo que la distribución de las variables es normal, se pide:}
    \begin{enumerate}[label=\color{red}\textbf{\alph*)}]
        \item \db{Obtener un intervalo de confianza al 95\% para la velocidad esperada para el combustible A y otro para el combustible B.}
        \item \db{Obtener un intervalo de confianza al 95\% para la varianza de la velocidad de combustión del combustible A.}
        \item \db{Obtener un intervalo de confianza al 95\% para el cociente de las varianzas $\sigma_A^2 / \sigma_B^2$.}
        \item \db{Suponiendo que las varianzas $\sigma_A^2$ y $\sigma_B^2$ son iguales, construir un intervalo de confianza para la diferencia entre la velocidad esperada para el combustible A y la velocidad esperada para el combustible B.} 
    \end{enumerate}
\end{enumerate}
