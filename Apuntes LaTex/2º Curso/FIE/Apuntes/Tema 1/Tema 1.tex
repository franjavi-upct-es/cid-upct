\section{Muestreo y distribuciones muestrales}
\subsection{Introducción}
\begin{tcolorbox}[colback=blue!5!white, colframe=blue!75!black, title=\textbf{El contexto}]
\begin{itemize}[label=\textbullet]
    \item Tenemos una pregunta acerca de un fenómenos aleatorio.
    \item Formulamos un modelo para la varaible de interés $X$.
    \item Traducimos la pregunta de interés en términos de uno o varios parámetros del modelo.
    \item Repetimos el experimento varias veces, apuntamos los valores de  $X$.
    \item ¿Cómo usar estos valores para extraer información sobre el parámetro?
\end{itemize}
\end{tcolorbox}

\subsection{Ejemplos}

\begin{tcolorbox}[colback=blue!5!white, colframe=blue!75!black, title=\textbf{¿Está la moneda trucada?}]
\begin{itemize}[label=\textbullet]
    \item Experimento: tirar la modena. $X=$ resultado obtenido.

         $P(X=+)=p,P(X=c)=1-p$  \[
         \text{¿} p = \dfrac{1}{2}?
         \] 
\end{itemize}
\end{tcolorbox}

\begin{tcolorbox}[colback=blue!5!white, colframe=blue!75!black, title=\textbf{Sondeo sobre intención de participación en unas elecciones}]
\begin{itemize}[label=\textbullet]
    \item Queremos estima la tasa de participación antes de unas elecciones generales.
    \item Formulamos un modelo:
        \begin{itemize}[label=\textrightarrow]
            \item Experimento: "escoger una persona al azar en el censo".
            \item $X$: participación, variable dicotómica ("Sí" o "No").  $p=P(X=\text{Si})$.
        \end{itemize}
    \item ¿Cuánto vale $p$?
    \item Censo: aproximadamente 37 000 000. Escogemos aproximadamente 3000 personas.
\end{itemize}
\end{tcolorbox}

\begin{tcolorbox}[colback=blue!5!white, colframe=blue!75!black, title=\textbf{Determinación de la concentración de un producto}]
\begin{itemize}[label=\textbullet]
    \item Quiero determinar la concentración de un producto.
    \item Formulo el modelo:
        \begin{itemize}[label=\textrightarrow]
            \item Experimento: "llevar a cabo una medición".
            \item $X$: "valor proporcionado por el aparato".
            \item  $X\sim \mathcal{N}(\mu,\sigma^2)$.
        \end{itemize}
    \item ¿Qué vale $\mu$?
\end{itemize}
\end{tcolorbox}
\subsection{Surge una pregunta}
En todas estas situaciones donde nos basamos en la repetición de un experimento simple\dots
\begin{itemize}[label=\textbullet]
    \item ¿Cómo sabemos que nuestra estimación es fiable?
    \item ¿Qué confianza tenemos al extrapolar los resultados de una muestra de 3000 personas a una población de 37 millones de personas?
\end{itemize}
\subsection{Esbozo de respuesta: tasa de participación}
\begin{tcolorbox}[colback=blue!5!white, colframe=blue!75!black, title=\textbf{Para convenceros, un experimento de simulación}]
\begin{itemize}[label=\textbullet]
    \item Voy a simular el proceso de extracción de una muestra de 3000 personas en una población de 37 millones de personas.
    \item Construyo a mi antojo los distintos componentes:
        \begin{itemize}[label=\textrightarrow]
            \item \textbf{La población:} defino en mi ordenador un conjunto de 37 000 000 de ceros y unos. ($\Leftrightarrow$ el censo electoral)
                \begin{itemize}[label=\textbullet]
                    \item "1" $\Leftrightarrow$ "la persona piensa ir a votar".
                    \item "0" $\Leftrightarrow$ "la persona \textbf{no} peinsa ir a votar"
                \end{itemize}
            \item \textbf{La tasa de participación "real":} Decido que en mi población el 70\% piensa ir a votar $\to 25\,900\,000$ "1"s.
            \item \textbf{La extracción de una muestra:} construyo un pequeño programa que extrae al azar una muestra de 3000 números dentro del conjunto grande. 
        \end{itemize}
\end{itemize}
\end{tcolorbox}
\begin{lstlisting}
poblacion <- c(rep(1, 25900000), rep(0, 11100000))
set.seed(314159)
p_muestra <- mean(sample(poblacion, 3000, replace = FALSE))
p_muestra   
\end{lstlisting}
\begin{verbatim}
## [1] 0.705667    
\end{verbatim}

Queremos descartar que haya sido suerte. Vamos a repetir muchas veces (10000 veces por ejemplo), la extracción de una muestra de 3000 personas en la población.
\begin{lstlisting}
library(tidyverse)
lista_muestras <- replicate(
    10000,
    sample(poblacion, 3000, replace = FALSE),
    simplify = FALSE
)
p_muestras <- map_dbl(lista_muestras, mean)
head(p_muestras)
\end{lstlisting}
\begin{verbatim}
## [1] 0.6970000  0.7030000  0.7036667  0.7023333  0.7013333  0.7226667
\end{verbatim}

Recogemos los valores obtenidos en un histograma.

\begin{center}
\includegraphics[width=0.7\linewidth]{"Tema 1/figures/Figure 1"}
\end{center}

\subsection{Realización del experimento: conclusiones}
\begin{itemize}[label=\textbullet]
  \item La enorme mayoría de las muestras de 3000 individuos proporcionan una tasa de partición muy próxima a la de la población.
    \begin{itemize}[label=\textrightarrow]
      \item \textbf{\rc{El riesgo}} de cometer un error superior a $\pm 2$ puntos, al coger \textbf{\rc{una}} muestra de 3000 individuos es muy pequeño (y asumible\dots)
    \end{itemize}
    \item Si nos limitamos a muestras de 300 individuos, ¿qué esperáis?
\end{itemize}
\begin{center}
  \includegraphics[width=0.7\linewidth]{"Tema 1/figures/Figure 2"}
\end{center}
\subsection{En la práctica}
\begin{tcolorbox}[colback=blue!5!white, colframe=blue!75!black, title=\textbf{Usamos las distribuciones muestrales}]
\begin{itemize}[label=\textbullet]
  \item Las empresas de sondeos no se basan en simulaciones sino en cálculos teóricos.
  \item Experimento aleatorio: escoger al azar una muestra de 3000 personas dentro de una población de 37 000 000, con una tasa de participación $p$.
  \item Llamamos a  $\hat{p}$ la variables aleatoria: proporción de "1"s en la muestra escogida.
  \item ¿Cuál es la distribución de valores de $\hat{p}$? \[
  \hat{p}\sim \mathcal{N}\left( p,\dfrac{p(1-p)}{n} \right) 
  \] 
  Es lo qe llamamos la \lb{distribución muestral} de $\hat{p}$.
\end{itemize}
\end{tcolorbox}
\subsection{Uso de la distribución muestral}
\begin{tcolorbox}[colback=blue!5!white, colframe=blue!75!black, title=\textbf{La distribución muestral de $\hat{p}$:}]
Es la distribución esperada de los valores de $\hat{p}$ respecto a todas las muestras de ese tamaño que podría extraer.
\end{tcolorbox}
\begin{center}
  \includegraphics[width=0.7\textwidth]{"Tema 1/figures/Figure 3"}
\end{center}
\subsection{Antes de extraer una muestra:}
\begin{itemize}[label=\textbullet]
  \item ¿Es suficiente el tamaño de la muestra para el riesgo asumible y la precisión requerida?
  \item Una vez extraida la muestra:
    \begin{itemize}[label=\textrightarrow]
      \item ¿Puedo dar un margen de error?
      \item ¿Puedo de decidir si $p$ poblacional es, por ejemplo, mayor que un valor dado?
    \end{itemize}
\end{itemize}
\subsection{Otro ejemplo: valores muestrales de una distribución normal}
\begin{center}
  \includegraphics[width=\textwidth]{"Tema 1/figures/Figure 4"}\\
  \includegraphics[width=\textwidth]{"Tema 1/figures/Figure 5"}\\
\end{center}
\subsection{Un resultado importante}
\begin{tcolorbox}[colback=blue!5!white, colframe=blue!75!black, title=\textbf{Ley (débil) de los grandes números}]
Sea $X$ una variable aletoria y $g(X)$ una variable aleatoria transformada de  $X$, con esperanza y momento de orden 2 finitos. Supongamos $X_1,X_2,\dots,X_n,\dots$ una sucesión de variables aleatorias (vv.aa) independientes con la misma distribución que $X$, entonces  \[
  \lim_{n \to +\infty} P\left[ \left| \dfrac{\sum_{i=1}^{n} g(X_i)}{n}-E[g(X)] \right| < \varepsilon \right] =1,\text{ para todo }\varepsilon>0.
\] 
\end{tcolorbox}
\subsection{Algunos términos}
\begin{tcolorbox}[colback=blue!5!white, colframe=blue!75!black, title=\textbf{Definición}]
\begin{itemize}[label=\textbullet]
  \item Sea una variable aleatoria $X$. Consideramos $n$ variables aleatorias independientes e idénticamente distribuidas  $X_1,X_2,\dots,X_n$, que se distribuyen como $X$. La variable aleatoria multidimensional  $(X_1,X_2,\dots,X_n)$ es una \lb{muestra aleatoria simple} (m.a.s) de $X$.
  \item Cualquier cantidad calculada a partir de las observaciones de un muestra: \lb{estadístico}.
  \item Experimento aleatorio: extraer una muestra. Consideramos un estadístico como una variable aleatoria. Nos interesa conocer la distribución del estadístico: \lb{distribución muestral}. 
\end{itemize}
\end{tcolorbox}
\subsection{Ejemplos de estadísticos}
\begin{itemize}[label=\textbullet]
  \item Proporción muestral: $\hat{p}$
  \item Media muestral: $\overline{X}=\dfrac{1}{n}\sum_{i=1}^{n} X_i$.
  \item Desviación típica muestral: $S_X=\sqrt{\dfrac{1}{n+1}\sum_{i=1}^{n} (X_i-\overline{X})^2} $
\end{itemize}
\subsection{La media muestral}
\begin{tcolorbox}[colback=blue!5!white, colframe=blue!75!black, title=\textbf{Contexto}]
Estudiamos una variable $X$ cuantitativa.
\begin{itemize}[label=\textbullet]
  \item Estamos interesados en $\mu$, el centro de la distribución de $X$.
  \item Extraemos una muestra de tamaño  $n$:  \[
  x_1,x_2,\dots,x_{n}
  \] 
\item Calculamos su media $\overline{x}$ para aproximar  $\mu$.
\item ¿Cuál es la distribución muestral de $\overline{X}$?
\end{itemize}
\end{tcolorbox}
\Ej
\begin{itemize}[label=\textbullet]
  \item Quiero medir una cantidad. Hay variabilidad en las mediciones.
  \item Introduzco una variable aleatoria $X$="valor proporcionado por el aparato".
  \item  $\mu$ representa el centro de los valores.
  \item Extraigo una muestra de tamaño 5 del valor de $X$
\end{itemize}
\subsubsection{Esperanza y varianza de la media muestral}
Llamamos $\mu=E[X]$ y $\sigma^2=\mathrm{Var}(X)$.
\begin{itemize}[label=\textbullet]
  \item Tenemos \[
      \bboxed{E[\overline{X}]=\mu.} 
  \] 
  \begin{itemize}[label=\textrightarrow]
    \item Es decir que el centro de la distribución muestral de $\overline{X}$ coincide con el centro de la distribución  $X$.
  \end{itemize}
\item Tenemos $\mathrm{Var}(\overline{X})=\dfrac{\sigma^2}{n}$, es decir, la dispersión de la distribución muestral de $\overline{X}$ es  $\sqrt{n} $ veces más pequeña que la dispersión inicial de $X$.
\end{itemize}
\textbf{Ilustración: $X$ inicial,  $\overline{X}$ con  $n=3, \overline{X} $ con $n=10$.}
\begin{center}
  \includegraphics[width=\textwidth]{"Tema 1/figures/Figure 6"}
\end{center}
\subsection{Consecuencia práctica}
\begin{tcolorbox}[colback=blue!5!white, colframe=blue!75!black, title=\textbf{Aparato de medición}]
\begin{itemize}[label=\textbullet]
  \item Experimento: llevar a cabo una medición con un aparato.
  \item Variable aleatoria $X$: "valor propocionado por el aparato".
  \item  $E[X]$: centro de la distribución de los valores proporcionados por el aparato.
    \begin{itemize}[label=\textrightarrow]
      \item Lo deseable: $E[X]$=valor exacto de la cantidad que buscamos medir.
      \item En este caso, decimos: el aparato es  \lb{exacto}. 
    \end{itemize}
  \item $\sigma_X$: dispersión de la distribución de los valores proporcionados por el aparato.
    \begin{itemize}[label=\textrightarrow]
      \item Lo deseable: $\sigma_X$ pequeño.
      \item En este caso, decimos: el aparato es \lb{preciso}. 
    \end{itemize}
\end{itemize}
\end{tcolorbox}
\subsubsection{Analogía con una diana}
\begin{center}
  \includegraphics[width=\textwidth]{"Tema 1/figures/Figure 7"}
\end{center}
\subsection{Varianza muestral}
Si $(X_1,X_2,\dots,X_n)$ es una muestra aleatoria simple de $X$, definimos la  \lb{varianza muestral} $S_n^2$ como \[
  S_n^2=\dfrac{1}{n-1}\sum_{i=1}^{n} \left( X_i-\overline{X}_n \right)^2. 
\] 
\begin{tcolorbox}[colback=blue!5!white, colframe=blue!75!black, title=\textbf{Fórmula alternativa para $S_n^2$:}]
\[
  S_n^2=\dfrac{n}{n-1}\left( \overline{X^2}_n-(\overline{X}_n)^2 \right) ,
\] donde $\overline{X^2}_n=\dfrac{1}{n}\sum_{i=1}^{n}X_i^2$.
\end{tcolorbox}
\subsubsection{Dos apuntes}
\begin{tcolorbox}[colback=blue!5!white, colframe=blue!75!black, title=\textbf{En algunos textos en castellano:}]
  Se suele llama $S_n^2$ \lb{cuasi-varianza muestra}, reservando el término varianza muestral para la cantidad $\dfrac{1}{n}\sum_{i=1}^{n} \left( X_i-\overline{X}_n \right) ^2$. 
\end{tcolorbox}
\begin{tcolorbox}[colback=blue!5!white, colframe=blue!75!black, title=\textbf{En estas fórmulas:}]
  Omitimos, si no hay confusión posible, el subíndice $n$, escribiendo  $S^2,\, \overline{X}=\sum_{i=1}^{n} X_i$ y $\overline{X^2}=\dfrac{1}{n}\sum_{i=1}^{n} X_i^2$.
\end{tcolorbox}
\subsection{Esperanza de la varianza muestral}
\begin{tcolorbox}[colback=blue!5!white, colframe=blue!75!black, title=\textbf{Proposición}]
Si $(X_1,X_2,\dots,X_n)$ es una muestra aleatoria simple de $X$ con varianza  $\sigma_X^2$, \[
  E[S_n^2]=\sigma_X^2.
\] 
\end{tcolorbox}
\subsection{Distribuciones muestrales de $\overline{X}$ y  $S^2$}
\begin{tcolorbox}[colback=olive!5!white, colframe=olive!75!black, title=\textbf{Tened en cuenta}]
\begin{itemize}[label=\textbullet]
  \item Los resultados anteriores sobre $E[\overline{X}]$ y  $\sigma_{\overline{X}}$ son válidos sea cual sea el modelo escogido para la distribución de $X$.
  \item Si queremos decir algo más preciso sobre la distribución de  $\overline{X}$ (densidad, etc\dots) necesitamos especificar la distribución de $X$.
  \item En el caso en que la variable  $X$ siga una distribución normal, el  \textbf{teorema de Fisher} analiza cómo se comportan los estadísticos anteriores y nos permiten establecer una serie de consecuencias que serán utilizadas posteriormente en los temas de intervalos de confianza y de constrastes de hipótesis. 
\end{itemize}
\end{tcolorbox}
\subsection{Distribución de $\overline{X}$ y  $S^2$ para una m.a.s. de una distribución normal}
\begin{tcolorbox}[colback=blue!5!white, colframe=blue!75!black, title=\textbf{Teorema de Fisher}]
Consideramos una muestra aleatoria simple de una variable aleatoria $X$ con distribución normal  $\mathcal{N}(\mu,\sigma^2)$, entonces se verifica:
\begin{enumerate}[label=\arabic*)]
  \item $\overline{X}_n$ y  $S_n^2$ son dos variables aleatorias independientes.
  \item $\dfrac{\overline{X}_n-\mu}{\sigma / \sqrt{n} }\sim \mathcal{N}(0,1)$
  \item $\dfrac{(n-1)S_n^2}{\sigma^2}\sim \chi_{n-1}^2$.
\end{enumerate}
\end{tcolorbox}
\subsection{Recordatorio: distribución $\chi^2$ con $p$ grados de libertad}
\begin{tcolorbox}[colback=blue!5!white, colframe=blue!75!black, title=\textbf{La distribución $\chi^2$.}]
Para $p\in \N^{+}$, la función de densidad de la distribución $\chi^2$ es igual a \[
\dfrac{1}{\Gamma\left( \frac{p}{2}  \right) 2^{\frac{p}{2} }}\cdot x^{\frac{p}{2} -1}e^{\frac{x}{2} },\quad \text{si }x>0,
\] donde $\Gamma$ denota la función Gamma (Nota: para cualquier real $\alpha>0,\Gamma(\alpha)=\int_{0}^{+\infty} t^{\alpha-1}e^{-t}\dt$).
\end{tcolorbox}
\begin{tcolorbox}[colback=blue!5!white, colframe=blue!75!black, title=\textbf{Caracterización de la $\chi^2$}]
Si $ Z_1,\dots,Z_p$ son $p$ variables aleatorias independientes, con $Z_i\sim \mathcal{N}(0,1)$, entonces la variable aleatoria $X$ definida como \[
X=Z_1^2+\cdots+Z_p^2=\sum_{i=1}^{p} Z_i^2
\] tiene una distribución $\chi^2$ con $p$ grados de libertad.
\end{tcolorbox}
\begin{itemize}[label=\color{red}\textbullet, leftmargin=*]
  \item \lb{¿Cómo es su función de densidad?}
\end{itemize}
Depende de los grados de libertad
\begin{center}
  \includegraphics[width=\textwidth]{"Tema 1/figures/Figure 8"}
\end{center}
\subsection{Distribución t-Student}
\begin{tcolorbox}[colback=blue!5!white, colframe=blue!75!black, title=\textbf{Hemos visto, si $X$ es Normal:}]
\[
  \dfrac{\overline{X}_n-\mu}{\sigma / \sqrt{n} }\sim \mathcal{N}(0,1).
\] Si queremos centrarnos en $\mu$ es natural sustituir en ella $\sigma$ por $S_n$.
\end{tcolorbox}
\begin{tcolorbox}[colback=blue!5!white, colframe=blue!75!black, title=\textbf{Proposición}]
Sea $(X_1,\dots,X_n)$ una muestra aleatoria simple de una población $\mathcal{N}(\mu,\sigma^2)$, \[
  T=\dfrac{\overline{X}-\mu}{S / \sqrt{n} }
\] tiene por densidad 
\begin{equation}
  f_{n-1}(t)\propto \dfrac{1}{\left( \frac{1+t^2}{n-1}  \right)^{\frac{n}{2}  }},\quad-\infty<t<\infty,
\end{equation}
La distribución que admite esta densidad se llama \lb{distribución t-Student} con $n-1$ grados de libertad. Escribimos  $T\sim t_{n-1}$.
\end{tcolorbox}
\begin{tcolorbox}[colback=red!5!white, colframe=red!75!black, title=\textbf{Su densidad}]
La función de densidad de un t-Student con $k$ grados de libertad: \[
f_k(t)=\dfrac{\Gamma\left( \frac{k+1}{2}  \right) }{\Gamma\left( \frac{k}{2}  \right) }\cdot \dfrac{1}{\sqrt{k\pi} }\cdot \dfrac{1}{\left( \frac{1\tau^2}{k}  \right) ^{\frac{k+1}{2} }},\quad-\infty<t<\infty,
\] donde $\Gamma$ denota la función Gamma.
\end{tcolorbox}
\begin{tcolorbox}[colback=blue!5!white, colframe=blue!75!black, title=\textbf{Caracterización de la t-Student como cociente}]
Si $Z$ e $Y$ son dos variables aleatorias independientes, con $Z\sim \mathcal{N}(0,1)$ e $Y\sim \chi_p^2$, el cociente \[
T=\dfrac{Z}{\sqrt{\frac{Y}{p} } }\sim t_p,
\] donde $t_p$ denota la t-Student con  $p$ grados de libertad.
\end{tcolorbox}
\begin{itemize}[label=\color{red}\textbullet, leftmargin=*]
  \item \lb{¿Cuál es la forma de la densidad de una t-Student?}
\end{itemize}
Tiene colas más pesadas que una normal
\begin{center}
  \includegraphics[width=0.7\textwidth]{"Tema 1/figures/Figure 9"}
\end{center}
\subsection{Distribución F de Snedecor para el cociente de varianzas}
\begin{tcolorbox}[colback=blue!5!white, colframe=blue!75!black, title=\textbf{Proposición}]
Consideremos $U_1$ y $U_2$ dos variables aleatorias independientes con distribución $\chi^2$ con $p_1$ y $p_2$ grados de libertad, respectivamente.

El cociente $F=\dfrac{\frac{U_1}{p_1} }{\frac{U_2}{p_2}}$ admite la densidad \[
  f_F(x)=\dfrac{\Gamma\left( \frac{p_1+p_2}{2}  \right) }{\Gamma(p_1)\Gamma(p_2)}\left( \dfrac{p_1}{p_2} \right) ^{p_1}\dfrac{x^{\frac{p_{1}}{2 }-1}}{\left( 1+\frac{p_1}{p_2} x \right) ^{\frac{p_1+p_2}{2} }}.
\] Esta distribución se llama F de Snedecor $p_1$ y $p_2$ grados de libertad y escribimos $F\sim F_{p_1,p_2}$.
\end{tcolorbox}
\begin{tcolorbox}[colback=blue!5!white, colframe=blue!75!black, title=\textbf{Consecuencia}]
Consideremos $X$ e $Y$ variables aleatorias normales independientes con varianzas $\sigma_X^2$ y $\sigma_Y^2$, así como $X_1,\dots,X_{n_x}$ e $Y_1,\dots,Y_{n_Y}$ dos muestras aleatorias simples de $X$ e $Y$, respectivamente. Deducimos que \[
  \dfrac{\frac{S_X^2}{\sigma_X^2}}{\frac{S_Y^2}{\sigma_Y^2} }\sim F_{n_X-1,n_Y-1}.
\] 
\end{tcolorbox}
\begin{itemize}[label=\color{red}\textbullet, leftmargin=*]
  \item \lb{¿Cuál es la forma de la densidad de una F de Snedecor?}
\end{itemize}
Depende mucho de los grados de libertad

\begin{center}
  \includegraphics[width=0.85\textwidth]{"Tema 1/figures/Figure 10"}
\end{center}
\subsection{Si la distribución de $X$ no es Normal}
No podemos decir nada en general, \lb{excepto} si $n$ es grande\dots
\begin{tcolorbox}[colback=blue!5!white, colframe=blue!75!black, title=\textbf{Teorema Central del Límite}]
  Si $n$ es \textit{"suficientemente"} grande, se puede aproximar la distribución de $\overline{X}$ por una Normal con media  $\mu$ y varianza $\dfrac{\sigma^2}{n}$: \[
    \overline{X}\sim \mathcal{N}\left( \mu,\dfrac{\sigma^2}{n} \right) \text{ aproximadamente. }
  \]  
\end{tcolorbox}
\begin{tcolorbox}[colback=blue!5!white, colframe=blue!75!black, title=\textbf{Formulación matemática}]
  El resultado anterior se taduce por una convergencia de la sucesión de las variables aleatorias $\left( \overline{X}_n \right)_n $ en distribución cuando $n\to \infty$.
\end{tcolorbox}
\begin{itemize}[label=\color{red}\textbullet, leftmargin=*]
  \item \lb{¿Cuándo considerar que $n$ es grande?}
\end{itemize}
Depende de la forma de la distribución de $X$:
 \begin{itemize}[label=\textbullet]
  \item Si $X$ casi Normal: $n$ pequeño es suficiente.
  \item Si $X$ es muy asimétrico: $n$ mucho más grande necesario.
\end{itemize}
En general, se suele considerar $n\ge 30$ suficiente\dots

\textbf{Ilustración, $X$ inicial $\sim \mathrm{Exp}(\lambda=0.5),\,\overline{X}$ con $n=3,10$ y  $n=30$}
\begin{center}
  \includegraphics[width=\textwidth]{"Tema 1/figures/Figure 11"}
\end{center}
\subsection{Distribución muestral de la proporción muestral}
\begin{tcolorbox}[colback=blue!5!white, colframe=blue!75!black, title=\textbf{Contexto}]
\begin{itemize}[label=\textbullet]
    \item Hay situaciones donde $X$ toma el valor 0 o 1, con probabilidades $1-p$ y $p$, respectivamente.
    \item Por ejemplo, el siguiente experimento: escoger una pieza en la producción. $X=1$ si es defectuosa,  $X=0$ si es correcta.
    \item Repetimos  $n$ veces el experimento. Obtenemos $$x_1=1,x_2=0,x_3=0,x_4=1,x_5=0,\dots,x_n$$ Contamos $N$ el número de "1"s.
    \item La proporción muestral es: \[
    \hat{p}=\dfrac{N}{n}.
    \] 
\end{itemize}
\end{tcolorbox}
\begin{tcolorbox}[colback=blue!5!white, colframe=blue!75!black, title=\textbf{Distribución exacta de $\hat{p}$}]
¿Cuál es la distribución de $N$?
\begin{itemize}[label=\textbullet]
    \item Experimento simple con situación dicotómica, repetimos $n$ veces \dots \[
    N=\mathcal{B}(n,p).
    \] 
\item POdemos usar esta distribución para hacer cálculos exactos\dots
\end{itemize}
\end{tcolorbox}
\begin{tcolorbox}[colback=blue!5!white, colframe=blue!75!black, title=\textbf{Distribución aproximada de $\hat{p}$}]
Tenemos que $N=X_1+X_2+\cdots+X_n$, por lo tanto \[
    \hat{p}=\dfrac{N}{n}=\dfrac{X_1+X_2+\cdots+X_n}{n}=\overline{X}.
\] 
Por el Teorema Central del Límite: \[
\hat{p}\sim \mathcal{N}\left( p, \dfrac{p(1-p)}{n} \right) \:\text{aproximadamente.}
\] Podemos usar esta distribución para hacer cálculos aproximados.
\end{tcolorbox}
\subsection{Simulación y método de Monte-Carlo}
\begin{tcolorbox}[colback=blue!5!white, colframe=blue!75!black, title=\textbf{Motivación}]
En muchas situaciones, la capacidad de simular valores de las distribuciones de interés puede resultar útil calcular o estimar cantidades relevantes para la inferencia sobre la distribución de $X_1,X_2,\dots,X_n$.
\end{tcolorbox}
\begin{tcolorbox}[colback=blue!5!white, colframe=blue!75!black, title=\textbf{¿Qué es ser capaz de simular valores de una distribución $f$?}]
\begin{itemize}[label=\textbullet]
    \item Se refiere a la posibilidad de producir, para cualquier tamaño $k$, conjuntos de valores $u_1,u_2,\dots,u_k$, cuyo comportamiento imita el de $k$ realizaciones aleatorias independientes de la distribución $f$.
    \item Quiere ecir que las propiedades del conjunto generado lo hacer indistinguible, si le aplicamos tests de independencia o de bondad de ajuste, de  $k$ realizaciones independientes de $f$.
\end{itemize}
\end{tcolorbox}
\begin{tcolorbox}[colback=blue!5!white, colframe=blue!75!black, title=\textbf{Simulación y método de Monte-Carlo}]
\begin{itemize}[label=\textbullet]
    \item Como hemos visto en los primeros ejemplos, gráficos como el histograma de frecuencias se comportan como la función de densidad de la variable de la que provienen las observaciones. También se pueden utilizar gráficos como la función de distribución empírica que veremos más adelante.
    \item Como consecuencia, dado un estadístico, si podemos obtener un número grande de observaciones del mismo podemos a través de alunos gráficos obtener información sobre su distribución.
    \item Podemos generar esas observaciones a través de lo que se conoce como simulaciones, observaciones generadas mediante algún algoritmo.
    \item Esta metodología que se puede aplicar en muchas otras situaciones se conoce como el \lb{método de Monte-Carlo}. 
\end{itemize}
\end{tcolorbox}
\subsubsection{Muestreo de Monte-Carlo para aproximar esperanzas}
\begin{tcolorbox}[colback=blue!5!white, colframe=blue!75!black, title=\textbf{Ley de los grandes números}]
    Consideremos una muestra aleatoria simple $X_1,X_2,\dots,X_n$ de una distribución $X$. Para cualquier función  $g$ que cumple  $E[g^2(X)]<+\infty$, tenemos que, con probabilidad 1, \[
        \lim_{n \to \infty} \dfrac{1}{n}\sum_{i=1}^{n} g(X_i)=E[g(X)].
    \] 
\end{tcolorbox}
\subsubsection*{Ejemplos}
\begin{itemize}[label=\textbullet]
    \item $g(x)=x: \lim_{n \to \infty} \dfrac{1}{n}\sum_{i=1}^{n} X_i=E[X]$, es decir $\overline{X}_n\longrightarrow E[X]=\mu_X$.
    \item $g(x)=(x-\mu_X)^2: \lim_{n \to \infty} \dfrac{1}{n}\sum_{i=1}^{n} (X_i-\mu_X)^2=E[(X-\mu_X)^2]=\mathrm{Var}(X)$.
    \item Si combinamos las dos convergencias anteriores: $\lim_{n \to \infty} \dfrac{1}{n}(X_i-\overline{X}_n)^2=\mathrm{Var}(X)$, es decir, $S_n^2\longrightarrow \mathrm{Var}(X)$.
    \item $g(x)=1_{x\le q}: \lim_{n \to \infty} \dfrac{1}{n}\sum_{i=1}^{n} 1_{X_i\le q}=P(X\le q)$, es decir, que las frecuencias acumuladas relativas convergen hacia la probabilidad asociada.
\end{itemize}
\subsubsection*{Aplicaciones}
\begin{tcolorbox}[colback=blue!5!white, colframe=blue!75!black, title=\textbf{Ejemplo: el movimiento Browniano}]
    \begin{itemize}[label=\textbullet]
        \item Es proceso muy usado en la predicción de precios (opciones) en matemáticas financieras.
        \item Una caracterización simplificada: es la suma infinitesimal de pequeñas contribuciones normales independientes.
        \item Para cualquier $t,W_t\sim \sum_{i=1}^{\frac{t}{h} }\sqrt{h} \cdot Z_i $, donde $h$ es el paso infinitesimal y $Z_i$ son normales estándares independientes e idénticamente distribuidos 
    \end{itemize}
\end{tcolorbox}
\subsection{Movimiento Browniano}
\begin{center}
    \includegraphics[width=0.8\textwidth]{"Tema 1/figures/Figure 12"}
\end{center}
\subsection{En finanzas, el modelo de Black-Scholes}
\begin{tcolorbox}[colback=blue!5!white, colframe=blue!75!black, title=\textbf{El movimiento Browniano Geométrico:}]
Lo propusieron Merton, Scholes y Black como modelo teórico para precios de acciones: \[
S_t=S_0\exp\left( \left( \mu-\dfrac{1}{2}\sigma^2 \right) t+\sigma W_t \right) .
\] 
\begin{itemize}[label=\textbullet]
    \item $S_0$: precio inicial de la acción.
    \item  $\mu$: el drift.
    \item $\sigma^2$: la volatilidad
\end{itemize}
\end{tcolorbox}
\begin{center}
    \includegraphics[width=0.8\textwidth]{"Tema 1/figures/Figure 13"}
\end{center}
\subsection*{¿Cuán observaremos $S(t)\ge 110$€?}
\begin{center}
    \includegraphics[width=0.8\textwidth]{"Tema 1/figures/Figure 14"}
\end{center}
\begin{itemize}[label=\textbullet]
    \item Podemos simular muchas trayectorias del Movimiento Browniano Geométrico y observa qué pasa con el tiempo en el que supera el umbral 110.
    \item Esto es Monte-Carlo.
    \item Luego podremos obtener indicadores de la distribución de este tiempo.
\end{itemize}
\begin{itemize}[label=\color{red}\textbullet, leftmargin=*]
    \item \lb{Representamos las 8 primeras}
        \begin{center}
            \includegraphics[width=0.8\textwidth]{"Tema 1/figures/Figure 15"}
        \end{center}
    \item \lb{¿Cuál es la probabilidad de que alcance el umbral antes de un año?}

        \begin{tcolorbox}[colback=blue!5!white, colframe=blue!75!black, title=\textbf{Por Monte-Carlo:}]
        Simulamos 1000 trayectorias hata el año y aproximamos la probabilidad que nos interesa por la porporción de trayectorias que alcanzan los 110 euros.
        \end{tcolorbox}
\end{itemize}
\subsection{Simulación y método de Monte-Carlo}
\begin{tcolorbox}[colback=blue!5!white, colframe=blue!75!black, title=\textbf{Simulación de variables aleatorias}]
    \begin{itemize}[label=\textbullet, leftmargin=*]
        \item Lo que nos planteamos ahora es qué algoritmo podemos utilizar para generar observaciones (simulaciones) de una variable aleatoria.
        \item Para un gran número de varaibles aleatorias y modelos probabilísticos, estas simulaciones ya están incorporadas en los paquetes y lenguajes de programación con enfoque matemático y estadístico, como \textbf{\texttt{R}}.
        \item Aquí describiremo uno de los métodos más básicos, basado en la inversa de la función de distribución.
        \item El \lb{método de la función inversa} es uno de los principales métodos de simulación de variables aleatorias.
        \item Está basado en la inversa de la función de distribución de una variable aleatoria.
        \item No todas las funciones de distribucion admiten inversa, como la conocemos usualmente. Para ello es necesaria que sea continua y estrictamente creciente.
        \end{itemize} 
\end{tcolorbox}
\begin{tcolorbox}[colback=blue!5!white, colframe=blue!75!black, title=\textbf{Inversa generalizada de la función de distribución o función cuantil}]
 Dada una función de distribución $F$, su \lb{función cuantil (inversa generalizada)}  se define como \[
 F^{-1}(p)=\inf \{x|F(x)\ge p\}\text{, para todo }p\in (0,1). 
 \] 
\end{tcolorbox}
\begin{tcolorbox}[colback=blue!5!white, colframe=blue!75!black, title=\textbf{Propiedades de la función cuantil}]
\begin{enumerate}[label=\arabic*)]
    \item $F(F^{-1}(p))\ge p$, para todo $p \in (0,1)$.
    \item $F^{-1}(F(x))\le x$, para todo $x$ donde  $F(x)>0$.
    \item Si $U\sim U(0,1)$ entonces $F^{-1}(U)$ tiene como distribución $F$.
    \item Dada una variable aleatoria  $X$, con media finita y función de distribución  $F$ se verifica que \[
            E[X]=\int_{0}^{1} F^{-1}(u) \du .
    \] 
\end{enumerate}
\end{tcolorbox}
\begin{tcolorbox}[colback=blue!5!white, colframe=blue!75!black, title=\textbf{Método de la función inversa}]
\begin{itemize}[label=\textbullet]
    \item Dada una variable aleatoria $X$ con función de distribución $F$, si generamos observaciones independientes $U\sim U(0,1),\,u_1,u_2,\dots,u_n$, entonces los valores $$x_1=F^{-1}(u_1),x_2=F^{-1}(u_2),\dots,x_n=F^{-1}(u_n)$$ constituyen una observación de la muestra aleatoria simple de tamaño $n$ de la variable aleatoria $X$.
\end{itemize}
\end{tcolorbox}

\subsubsection*{Generación de valores de una distribución uniforme}
\begin{center}
    \includegraphics[width=0.7\textwidth]{"Tema 1/figures/Figure 16"}
\end{center}
\subsubsection*{Generación de valores de una distribución gamma}
\begin{center}
    \includegraphics[width=0.7\textwidth]{"Tema 1/figures/Figure 17"}
\end{center}
\subsection*{Transformación de una variable aleatoria}
\begin{tcolorbox}[colback=blue!5!white, colframe=blue!75!black, title=\textbf{Planteamiento del problema}]
En muchas ocasiones, tenemos una variable aleatoria continua $X$, de la que conocemos la función de densidad, pero nos interesa conocer cómo se comporta una transformación de $X,\,Y=\varphi(X)$.

\end{tcolorbox}
\begin{tcolorbox}[colback=blue!5!white, colframe=blue!75!black, title=\textbf{Teorema}]
Sea $X$ una viarble aleatoria con función de densidad $f_X$ definida en un intervalo abierto  $(a,b)\subseteq \R$. Sea $\varphi:(a,b)\to \R$ tal que:
\begin{itemize}[label=\textbullet]
    \item Es continua.
    \item Es estrictamente creciente o decreciente.
    \item $\varphi^{-1}$ es diferenciable.
\end{itemize}
Entonces, la variable aleatoria $Y=\varphi(X)$ tiene función de densidad \[
f_Y(y)=\begin{cases}
    f_X(\varphi^{-1}(y))\left| \dfrac{\mathrm{d}}{\mathrm{d}x}(\varphi^{-1}(y)) \right| , & \text{si }y\in \varphi(a,b)\\
    0 & \text{en otro caso}

\end{cases}
\] 
\end{tcolorbox}
\subsection*{Función característica}
\begin{tcolorbox}[colback=blue!5!white, colframe=blue!75!black, title=\textbf{Definición}]
Sea $X$ una variable aleatoria cualquiera. La \lb{función característica} de  $X$ se define como \[
    \phi(t)=E[e^{itX} ]=\int_{-\infty}^{\infty} e^{itx}\mathrm{d}F(x)  
\] en donde $i=\sqrt{-1} $
\end{tcolorbox}
\begin{tcolorbox}[colback=red!5!white, colframe=red!75!black]
Nota: $\phi(t)=\int_{-\infty}^{\infty} xos(tx)\mathrm{d}F(x)+i \int_{-\infty}^{\infty} \sin(tx)\mathrm{d}F(x).  $
\end{tcolorbox}
\begin{tcolorbox}[colback=blue!5!white, colframe=blue!75!black, title=\textbf{Propiedades}]
\begin{itemize}[label=\textbullet]
    \item \lb{Existencia:} $|\phi(t)|\le 1$, para todo $t\in \R$.
    \item Si $X$ e  $Y$ son variables aleatorias \lb{independientes}, entonces \[
    \phi_{X+Y}(t)=\phi_X(t)\phi_Y(t),
    \]para todo $t\in \R$.
\end{itemize}
\end{tcolorbox}
\subsection*{Desigualdades}
\begin{tcolorbox}[colback=blue!5!white, colframe=blue!75!black, title=\textbf{Desigualdad de Markov}]
    Si $Z$ es una variable aleatoria no negativa con medi finita  $E[Z]$ y  $\varepsilon>0 $, entonces \[
        \varepsilon\mathrm{Pr}[Z\ge\varepsilon ]=\varepsilon\int_{[\varepsilon,\infty)}\mathrm{d}F_Z(x)\le \int_{[\varepsilon,\infty)}x\mathrm{d}F_Z(x)\le \int_{[0,\infty)}x\mathrm{d}F_Z(x)=E(Z)
    \] (donde $F_Z(x)=\mathrm{Pr}[Z\le x]$ es su función de distribución), es decir \[
    \mathrm{Pr}[Z\ge \varepsilon]\le \dfrac{E[Z]}{\varepsilon}.
    \] 
\end{tcolorbox}
\begin{tcolorbox}[colback=blue!5!white, colframe=blue!75!black, title=\textbf{Desigualdad de Chebyshev}]
    Si $X$ es una variable aleatoria con media finita $\mu=E[X]$ y varianza $\sigma^2=\mathrm{Var}(X)>0$, entonces tomando $Z=\dfrac{(X-\mu)^2}{\sigma^2}\ge 0$ y aplicando la desigualdad de Markov, tenemos \[
    \mathrm{Pr}\left[ \dfrac{(X-\mu)^2}{\sigma^2}\ge \varepsilon \right] \le \dfrac{1}{\varepsilon}
    \] para todo $\varepsilon>0$.

    También se puede escribir como \[
        \mathrm{Pr}[(X-\mu)^2<\varepsilon\sigma^2]\ge 1-\dfrac{1}{\varepsilon}
    \] o como \[
    \mathrm{Pr}[|X-\mu|<r]\ge 1-\dfrac{\sigma^2}{r^2},
    \] para todo $r>0$.
\end{tcolorbox}
\newpage
\begin{center}
\textbf{\large Hoja de ejercicios Tema 1: Muestreo y distribuciones muestrales}
\end{center}
\begin{enumerate}[label=\color{red}\textbf{\arabic*)}]
    \item \lb{Sea $X$ variable aleatoria con distribución Bernoulli, de parámetro  $p(X\sim b(p))$ y sea $X_1,X_2,X_3$ una muestra aleatoria simple (m.a.s) de $X$. Se pide:}
        \begin{enumerate}[label=\color{red}\textbf{\alph*)}]
            \item \db{Estudiar la distribución del vector $(X_1,X_2,X_3)$.}

                Dado que $X_1,X_2,X_3$ son una muestra aleatoria simple de $X\sim \mathrm{Bernoulli}(p)$, entonces las siguientes propiedades son ciertas:
                \begin{enumerate}[label=\arabic*)]
                    \item $X_1,X_2,X_3$ son independientes e idénticamente distribuidas.
                    \item Cada $X_i\sim \mathrm{Bernoulli}(p)$, es decir, la probabilidad de éxito $P(X_i=1)=p$ y  $P(X_i=0)=1-p$.
                \end{enumerate}
                El vector $(X_1,X_2,X_3)$ tiene una \textbf{distribución multinomial} con 2 posibles resultados (0 o 1) para cada componente. La distribución conjunta es: \[
                P(X_1=x_1,X_2=x_2,X_3=x_3)=p^{x_1+x_2+x_3}(1-p)^{3-(x_1+x_2+x_3)},
                \]donde $x_1,x_2,x_3\in \{0,1\} $.

                Esto corresponde a la \textbf{distribución conjunta} de 3 variables Bernoulli independientes. 
            \item \db{Estudiar la distribución en el muestreo del estadístico $\dfrac{X_1+X_2+X_3}{3}$.} 

                Es estidístico $\overline{X}=\dfrac{X_1+X_2+X_3}{3}$ es el \textbf{promedio muestral} de las 3 variables. Para analizar su distribución:
                \begin{enumerate}[label=\arabic*)]
                    \item $S=X_1+X_2+X_3$ sigue una \textbf{distribución binomial} porque es la suma de $n=3$ variables Bernoulli independientes:  \[
                    S\sim B(n=3,p).
                    \] 
                    La función de probabilidad de $S$ es: \[
                    P(S=k)=\binom{3}{k} p^k(1-p)^{3-k},\quad k=0,1,2,3.
                    \] 
                \item El estadístico $\overline{X}=\dfrac{S}{3}$ simplemente escala los valores posibles de $S$ dividiéndolos por 3. Los valores posibles de $\overline{X}$ son: \[
                        \overline{X}\in \left\{ 0,\dfrac{1}{3}, \dfrac{2}{3},1 \right\}.
                \] 
            \item La probabilidad de cada valor de $\overline{X}$ es proporcional a la probabilidad de los valores correspondientes de $S$:  \[
                    P\left( \overline{X}=\dfrac{k}{3} \right) =P(S=k)=\binom{3}{k} p^k(1-p)^{3-k},\quad k=0,1,2,3.
            \] 
            Por lo tanto, la distribución de $\overline{X}$ es discreta y está determinada por la distribución binomial de $S$.
                \end{enumerate}
        \end{enumerate}
    \item \lb{Sea $X$ variable aleatoria con distribución $\mathrm{Exp}(\lambda)$. Sea $(X_1,\dots,X_n)$ una m.a.s de $X$, estudiar la distribución en el muestre de  $S=\sum_{j=1}^{n} X_j$.} 

        Dado que $X\sim \mathrm{Exp}(\lambda)$ (exponencial con parámetro $\lambda>0$), y que $(X_1,\dots,X_n)$ es una muestra aleatoria simple (m.a.s) de $X$, podemos analizar la distribución del estadístico  $S=\sum_{j=1}^{n} X_j$.

        \textbf{Propiedades relevantes:}
        \begin{enumerate}[label=\arabic*)]
            \item \textbf{Distribución de la suma de variables exponenciales independientes:} Si $X_1,\dots,X_n$ son variables aleatorias independientes e idénticamente distribuidas $(X_i\sim \mathrm{Exp}(\lambda))$, entonces la suma: \[
            S=\sum_{j=1}^{n} X_j
            \] sigue una distribución \textbf{Gamma} con parámetros $n$ y  $\lambda$. Esto se denota como: \[
            S\sim \mathrm{Gamma}(n,\lambda),
            \]  donde:
            \begin{itemize}[label=\textbullet]
                \item $n$ es el parámetro de forma.
                \item $\lambda$ es el parámetro de escala.
            \end{itemize}
        \end{enumerate}
        \textbf{Distribución Gamma:}

        La función de densidad de probabilidad de una variables aleatoria $S\sim \mathrm{Gamma}(n,\lambda)$ está dada por: \[
        f_S(s)=\begin{cases}
            \dfrac{\lambda^n}{\Gamma(n)}s^{n-1}e^{-\lambda s} & s>0\\
            0 & s\le 0
        \end{cases}
        \] donde:
        \begin{itemize}[label=\textbullet]
            \item $\Gamma(n)$ es la función gamma (para $n\in \N,\Gamma(n)=(n-1)!$).
            \item $s^{n-1}$ y $e^{-\lambda s} $ controlan la forma y decaimiento de la densidad.
        \end{itemize}
        \textbf{Propiedades del estadístico $S$:} 
        \begin{enumerate}[label=\arabic*)]
            \item \textbf{Esperanza ($E[S] $):}

                Si $S\sim \mathrm{Gamma}(n,\lambda)$, entonces: \[
                    E[S]=\dfrac{n}{\lambda}.
                \] 
            \item \textbf{Varianza $(\mathrm{Var}(S))$:}

                La varianza de $S$ está dada por:  \[
                \mathrm{Var}(S)=\dfrac{n}{\lambda^2}.
                \] 
            \item \textbf{Caso especial $(n=1)$:} 

                Cuando $n=1$, la distribución Gamma coincide con la distribución exponencial. Es decir: \[
                \mathrm{Gamma}(1,\lambda)=\mathrm{Exp}(\lambda).
                \] 
        \end{enumerate}
    \item \lb{Sea $X$ variable aleatoria con distribución  $\mathcal{N}(\mu,\sigma^2)$. Sea $(X_1,\dots,X_n)$ una m.a.s de $X$, estudiar la distribución en el muestreo de  $S=\sum_{j=1}^{n} X_j$.}

        Dado que $X\sim \mathcal{N}(\mu,\sigma^2)$, y que $(X_1,\dots,X_n)$ es una muestra aleatoria simple (m.a.s) de $X$, las  $X_i$ son independientes e idénticamente distribuidas con  $X_i\sim N(\mu,\sigma^2)$. Queremos analizar la distribución en el muestreo de $S=\sum_{j=1}^{n} X_j$.

        \textbf{Propiedades relevantes:}
        \begin{enumerate}[label=\arabic*)]
            \item \textbf{Suma de variables normales independientes:} Si $X_1,\dots,X_n$ son independientes y $X_i\sim \mathcal{N}(\mu,\sigma^2)$, entonces la suma: \[
            S=\sum_{j=1}^{n} X_j
            \] sigue una distribución normal: \[
            S\sim N(n\mu,n\sigma^2).
            \] 
        \end{enumerate}
        \textbf{Derivación:}
        \begin{enumerate}[label=\arabic*)]
            \item \textbf{Esperanza ($E[S]$):} La esperanza de $S$ es la suma de las esperanzas de las  $X_i$:  \[
                    E[S]=E\left[ \sum_{j=1}^{n} X_j \right] =\sum_{j=1}^{n} E[X_j]=\sum_{j=1}^{n} \mu=n\mu.
            \] 
        \item \textbf{Varianza ($\mathrm{Var}(S)$):} La varianza de $S$ es la suma de las varianzas de las  $X_i$, ya que son independientes:  \[
        \mathrm{Var}(S)=\mathrm{Var}\left( \sum_{j=1}^{n} X_j \right) =\sum_{j=1}^{n} \mathrm{Var}(X_j)=\sum_{j=1}^{n} \sigma^2=n\sigma^2.
        \] 
    \item \textbf{Distribución:} Dado que una combinación lineal de variables normales independientes también sigue una distribución normal, se concluye que: \[
    S\sim N(n\mu,n\sigma^2).
    \]  
        \end{enumerate}
    \item \lb{Sea $X$ una variable aleatoria con función de densidad: \[
    f(x,\theta)=\dfrac{2x}{\theta}\exp\left( -\dfrac{x^2}{\theta} \right)\chi_{(0,+\infty)}(x). 
    \]Obtener la distribución en el muestreo estadístico: \[
    T(X_1,X_2,\dots,X_n)=\sum_{j=1}^{n} X_j^2.
    \]Obtener su media y su varianza.}

\begin{enumerate}[label=Paso \arabic*:]
    \item Verificar la distribución de $X$

        La función de densidad de  $X$ es:  \[
        f(x;\theta)=\dfrac{2x}{\theta}\exp\left( -\dfrac{x^2}{\theta} \right) \chi_{(0,\infty)}(x).
        \] 
        Esta densidad corresponde a una \textbf{distribución Rayleigh generalizada} con parámetro de escala $\theta$. Para esta distribución:
        \begin{itemize}[label=\textbullet]
            \item $X^2$ sigue una distribución exponencial con parámetro $\lambda=\dfrac{1}{\theta}$.
        \end{itemize}
        Entonces: \[
        Y=X^2\sim \mathrm{Exp}\left( \lambda=\dfrac{1}{\theta} \right) .
        \] 
    \item Distribución del estadístico $T=\sum_{j=1}^{n} X_j^2$ 

        Dado que $Y_j=X_j^2\sim \mathrm{Exp}\left(\dfrac{1}{\theta}\right)$, y las $Y_j$ son independientes, la suma de $n$ variables exponenciales independientes sigue una distribución \textbf{Gamma}.

        Por lo tanto, el estadístico: \[
        T=\sum_{j=1}^{n} X_j^2=\sum_{j=1}^{n} Y_j
        \] sigue la distribución: \[
        T\sim \mathrm{Gamma}\left( n,\lambda=\dfrac{1}{\theta} \right) ,
        \]  donde:
        \begin{itemize}[label=\textbullet]
            \item $n$ es el parámetro de forma.
            \item $\lambda=\dfrac{1}{\theta}$ es el parámetro de escala.
        \end{itemize}
        La densidad de la distribución Gamma es: \[
        f_T(t;n,\lambda)=\dfrac{\lambda^nt^{n-1}e^{-\lambda t} }{\Gamma(n)},\quad t>0.
        \] 
    \item Esperanza y Varianza del estadístico $T$

        Para una distribución Gamma con parámetros  $(n,\lambda)$, las propiedades son:
        \begin{enumerate}[label=\arabic*)]
            \item Esperanza: \[
                    E[T]=\dfrac{n}{\lambda}.
            \] 
        \item Varianza: \[
        \mathrm{Var}(T)=\dfrac{n}{\lambda^2}.
        \] 
        \end{enumerate}
        En este caso, como $\lambda=\dfrac{1}{\theta}$: \[
        \begin{array}{c}
            E[T]=\dfrac{n}{\frac{1}{\theta} }=n\theta,\\
            \mathrm{Var}(T)=\dfrac{n}{\left( \frac{1}{\theta}  \right) ^2}=n\theta^2.
        \end{array}
        \] 
\end{enumerate}

\item \lb{Sea $X$ una variable aleatoria con función de densidad: \[
f(x,\theta)=\dfrac{\theta}{(1+x)^{1+\theta}}\chi_{(0,+\infty)}(x).
\]Obtener la distribución en el muestreo del estadístico: \[
T(X_1,X_2,\dots,X_n)=\dfrac{\sum_{j=1}^{n} \ln(1+X_j)}{n}.
\]Obtener su media y su varianza. }

\begin{enumerate}[label=\arabic*)]
    \item Identificación de la distribución de $X$

        La función de densidad de  $X$ viene dada por:  \[
        f(x,\theta)=\dfrac{\theta}{(1+x)^{1+\theta}}\chi_{(0,+\infty)}(x),\quad \theta>0.
        \] 
        Observemos que, para $x>0$, la forma  $\dfrac{1}{(1+x)^{1+\theta}}$ sugiere una transformación logarítmica conveniente: \[
        Y=\ln(1+X).
        \] 
        Vamos a encontrar la distribución de $Y$.
    \item Transformación  $Y=\ln(1+X)$
        \begin{enumerate}[label=\arabic*)]
            \item Relación entre $X$ y $Y$: \[
            Y=\ln(1+X)\longleftrightarrow X=e^{Y}-1. 
            \] 
        \item Soporte:

            Dado que $x>0$, entonces  $1+x>1$ y por ende  $Y>\ln(1)=0$. De modo que $Y\in (0,+\infty)$.
        \item Derivada $\dfrac{\dx}{\dy } $: \[
        \dfrac{\dx }{\dy }=\dfrac{\mathrm{d}}{\mathrm{d}y}(e^{Y}-1 )=e^{Y}. 
        \] 
    \item Función de densidad de $Y$: 

        Partiendo de que $f_X(x,\theta)$ es la densidad de $X$, la densidad de  $Y$ se obtiene mediante: \[
        f_X(y)=F_X(x(y))\cdot \left| \dfrac{\dx }{\dy } \right| .
        \] 
        Sustituyendo $x=e^{Y} -1$, obtenemos: \[
        f_X(e^{Y}-1,\theta )=\dfrac{\theta}{(1+(e^{Y} -1))^{1+\theta}}=\dfrac{\theta}{(e^{Y})^{1+\theta} }=\theta e^{-(1+\theta)Y}. 
        \] 
        Por último, multiplicamos por $\dfrac{\dx }{\dy }=e^{Y} $: \[
        f_Y(y)=\left[ \theta e^{-(1+\theta)Y}  \right] \cdot e^{Y}=\theta e^{-(1+\theta)Y} e^{Y}=\theta e^{-\theta Y}, \quad y>0.
        \] 
        Esta es precisamente la \textbf{densidad de una Exponencial} con parámetro $\theta$. Por tanto, \[
        Y=\ln(1+X)\sim \mathrm{Exp}(\theta).
        \] 
        \end{enumerate}
    \item Distribución del estadístico
        \[
        T(X_1,\dots,X_n)=\dfrac{1}{n}\sum_{j=1}^{n} \ln(1+X_j).
        \] 
        Definamos \[
        Y_j=\ln(1+X_j).
        \] 
        Cada $Y_j$ es  $\mathrm{Exp}(\theta)$ y son independientes e idénticamente distribuidas al provenir de una muestra aleatoria simple de $X$. Entonces  \[
        T=\dfrac{1}{n}\sum_{j=1}^{n} Y_j.
        \] 
        \begin{itemize}[label=\textbullet]
            \item La suma $S=\sum_{j=1}^{n} Y_j$ sigue una $\mathrm{Gamma}(n,\theta)$.
            \item El estadístico $T=\dfrac{1}{n}S$ es simplemente la suma escalada por $\dfrac{1}{n}$.
        \end{itemize}
        \begin{enumerate}[label=\arabic*), leftmargin=*]
            \item Si $S\sim \Gamma(n,\theta )$, entonces $T=\dfrac{1}{n}S$ tiene parámetro de forma $n$ y de \textbf{tasa} $n\theta$. En otras palabras, \[
            T\sim \Gamma(n, n\theta).
            \]  
            Explícitamente, su función de densidad es: \[
            f_T(t)=\dfrac{(n\theta)^n}{\Gamma(n)}t^{n-1}e^{-n\theta t},\quad t>0. 
            \] 
        \end{enumerate}
    \item Media y varianza de $T$

        Para una  $\Gamma(k,\lambda)$, se sabe que: \[
            E[W]=\dfrac{k}{\lambda},\qquad \mathrm{Var}(W)=\dfrac{k}{\lambda^2}.
        \] 
        En nuestro caso, $T\sim \Gamma(n,n\theta)$. Luego: 
        \begin{enumerate}[label=\arabic*)]
            \item Media de $T$:  \[
                    E[T]=\dfrac{n}{n\theta}=\dfrac{1}{\theta}
            \] 
        \item Varianza de $T$:  \[
        \mathrm{Var}(T)=\dfrac{n}{(n\theta)^2}=\dfrac{1}{n\theta^2}.
        \] 
        \end{enumerate}
\end{enumerate}

\item \lb{Sea $X$ una variable aleatoria con función de densidad en todo $\R$: \[
f(x,\theta)=\exp(-(x-\theta))\exp(-\exp(-(x-\theta))).
\]Obtener la distribución en el muestreo del estadístico: \[
T(X_1,X_2,\dots,X_n)=\dfrac{\sum_{j=1}^{n} \exp(-X_j)}{n}.
\]Obtener su media y su varianza.}

\begin{enumerate}[label=\arabic*)]
    \item Identificar la distribución de $X$

        La función de densidad que se nos da es, para todo  $x\in \R$, \[
        f(x,\theta)=\exp(-(x-\theta))\exp\left( -\exp(-(x-\theta)) \right) .
        \] 
        Obsérvese que si definimos \[
        Z=X-\theta, 
        \] 
        la densidad $Z$ queda  \[
            f_Z(z)=\exp(-z)\exp(-e^{-z} ),\quad z \in \R.
        \] 
        Esta es la \textbf{distribución Gumbel} estándar (con parámetro de localización 0 y escala 1) pa el máximo. Por lo tanto, $X$ se distribuye como una $\mathrm{Gumble}(\theta,1)$  con localización $\theta$ y escala $1$.

        En resumen:  \[
        X\sim \mathrm{Gumble}(\theta,1)\longleftrightarrow Z=X-\theta\sim \mathrm{Gumble(0,1)} .
        \] 
    \item Analizar el estadístico
        \[
        T(X_1,\dots,X_n)=\dfrac{1}{n}\sum_{j=1}^{n} \exp(-X_j).
        \] 
        Definamos \[
        Y_j=\exp(-X_j).
        \] 
        El objetivo es estudiar la distribución de \[
        T=\dfrac{1}{n}\sum_{j=1}^{n} Y_j=\dfrac{1}{n}\sum_{j=1}^{n} e^{-X_j}. 
        \] 
        \begin{enumerate}[label=2.\arabic*)]
            \item Distribución de $Y_j=e^{-X_j} $ 

                Usaremos el hecho de que $Z_j=X_j-\theta$ es Gumbel estándar. Entonces \[
                X_j=Z_j+\theta\longrightarrow e^{-X_j}=e^{-\theta}e^{-Z_j}.   
                \] 
                Definamos $W_j=e^{-Z_j} $. Veamos la distribución de $W_j$:
                 \begin{itemize}[label=\textbullet]
                    \item  Si $Z_j\sim \mathrm{Gumbel}(0,1)$, la función de distribución acumulativa (CDF) de $Z_j$ es  \[
                    F_{Z_j}(z)=e^{-e^{-z} } , z\in \R.
                    \] 
                \item Entonces \[
                W_j=e^{-Z_j} >0.
                \] 
                Para $w>0$,  \[
                \{W_j\le w\} \equiv \{e^{-Z_j} \le w\} \equiv \{Z_j\ge -\ln w\} .
                \] 
                Por tanto, \[
                F_{W_j}(w)=\mathrm{Pr}(Z_j\ge -\ln w)=1-\mathrm{Pr}(Z_j<-\ln w)=1-F_{Z_j}(-\ln w).
                \] 
                Usando $F_{Z_j}(z)=e^{-e^{-z} } $, se obtiene \[
                F_{Z_j}(-\ln w)=e^{-\exp(-(-\ln w))} =e^{-w}. 
                \] 
                Por lo tanto, \[
                F_{W_j}(-\ln w)=e^{-w},\quad w> 0,  
                \] 
                que es la función de distribución acumulativa de una \textbf{Exponencial}. 
                \end{itemize}
                En consecuencia, \[
                W_j\sim \mathrm{Exp}(1)
                \] 
                Como \[
                Y_j=e^{-\theta}W_j, 
                \] 
                entonces $Y-j$ es simplemente  $W_j$ escalada por  $e^{-\theta} $.
                \begin{itemize}[label=\textbullet]
                \item Si $W_j\sim \mathrm{Exp}(1)$, entonces la variable $c\cdot W_j$ con $c>0 $ es $\mathrm{Exp}\left( \dfrac{1}{c} \right) $.
                \item Aquí $c=e^{-\theta}\longrightarrow \dfrac{1}{c}=e^{\theta}  $.
                \end{itemize}
                Por tanto: \[
                Y_j=e^{-\theta}W_j\sim \mathrm{Exp}(e^{\theta} ) .
                \] 
                Es decir, cada $Y_j$ tiene tasa  $e^{\theta} $.
            \item Distribución de la media muestral $T$

                Dado que los  $Y_j$ son independientes e idénticamente distribuidos $\mathrm{Exp}(e^{\theta} )$, la suma \[
                S=\sum_{j=1}^{n} Y_j
                \] 
                sigue una distribución \textbf{Gamma} con forma $n$ y tasa $e^{\theta} $; escribimos \[
                S\sim \Gamma(n,e^{\theta} ).
                \] 
                El estadístico \[
                T=\dfrac{S}{n}
                \] es simplemente la suma $S$ escalada por $\dfrac{1}{n}$. Se conoce la siguiente propiedad de la distribución Gamma:
                \begin{itemize}[label=\textbullet]
                    \item Si $S\sim \Gamma(n,\lambda)$, entonces $\alpha S\sim \Gamma\left( n,\dfrac{\lambda}{\alpha} \right) $.
                \end{itemize}
                En nuestro caso, $\alpha=\dfrac{1}{n}$. Por tanto, \[
                T=\dfrac{S}{n}\sim \Gamma\left( n,ne^{\theta}  \right) .
                \] 
        \end{enumerate}
    \item Media y varianza de $T$

        Sea  $T\sim \Gamma(k,\lambda)$ con $k=n$ y $\lambda=ne^{\theta} $. Recordemos que: \[
            E[\Gamma(k,\lambda)]=\dfrac{k}{\lambda}, \quad \mathrm{Var}[\Gamma(k,\lambda)]=\dfrac{k}{\lambda^2}.
        \] 
        Por tanto:
        \begin{enumerate}[label=\arabic*)]
            \item Media de $T$:  \[
                    E[T]=\dfrac{n}{ne^{\theta} }=\dfrac{1}{e^{\theta} }.
            \] 
        \item Varianza de $T$:  \[
        \mathrm{Var}(T)=\dfrac{n}{\left( ne^{\theta}  \right) ^2}=\dfrac{n}{n^2e^{2\theta} }=\dfrac{1}{ne^{2\theta} }.
        \] 
        \end{enumerate}
\end{enumerate}
\item \lb{Sea $X_1,X_2,\dots,X_n$ una m.a.s de una variable $X$ con distribución  $\mathcal{N}(\mu,\sigma^2)$. Sean $\overline{X}$ y  $S^2$ su media y cuasi-varianzas muestrales, respectivamente. Sea $X_{n+1}$ una nueva observación de $X$ independiente de $X_1,X_2,\dots,X_n$. Obtener la distribución en el muestreo del estadístico: \[
            \dfrac{X_{n+1}-\overline{X}}{S}\sqrt{\dfrac{n}{n+1}} .
\] } 

\begin{enumerate}[label=\arabic*)]
    \item Distribución de $X_{n+1}-\overline{X}$ 
        \begin{enumerate}[label=\arabic*)]
            \item La media muestral $\overline{X}=\dfrac{1}{n}\sum_{j=1}^{n} X_j$ es independiente de $X_{n+1}$ (porque $X_{n+1}$ es una nueva observación independiente).
            \item Cada $X_i$ (incluyendo $X_{n+1}$) se distribuye como $\mathcal{N}(\mu,\sigma^2)$.
            \item Se sabe que \[
            \overline{X}\sim \mathcal{N}\left( \mu,\dfrac{\sigma^2}{n} \right) ,\quad X_{n+1}\sim \mathcal{N}(\mu,\sigma^2),
            \] 
            y son independientes. Por ende, \[
            X_{n+1}-\overline{X}\sim \mathcal{N}\left( 0,\sigma^2+\dfrac{\sigma^2}{n} \right) =\mathcal{N}\left( 0,\sigma^2\cdot \dfrac{n+1}{n} \right) .
            \] 
            Es decir, \[
            \mathrm{Var}(X_{n+1}-\overline{X})=\sigma^2\cdot \dfrac{n+1}{n}.
            \] 
        \end{enumerate}
    \item Relación con el cociente Normal-Chi-cuadrado

        Sabemos además que la cuasi-varianza $S^2$ satisface \[
            (n-1) \dfrac{S^2}{\sigma^2}\sim ~\chi_{n-1}^2,
        \] y es independiente de $\overline{X}$ (y por tanto también independiente de $X_{n+1}$).

        Para simplicar la notación, definamos \[
        Y=X_{n+1}-\overline{X}.
        \] 
        Entonces $Y\sim \mathcal{N}\left( 0,\sigma^2\cdot \dfrac{n+1}{n} \right) $. Podemos escribir \[
            T=\dfrac{Y}{S}\sqrt{\dfrac{n}{n+1}} =\underbrace{\dfrac{\frac{Y}{\sigma} }{\sqrt{\frac{n+1}{n} } }}_{\displaystyle =U\sim \mathcal{N}(0,1)} \bigg/ \underbrace{\dfrac{S}{\sigma}}_{\displaystyle \sqrt{\dfrac{\chi_{n-1}^2}{n-1}} }.
        \] 
        \begin{itemize}[label=\textbullet]
            \item La variable $U=\dfrac{\frac{Y}{\sigma} }{\sqrt{\frac{n+1}{n} } }\sim \mathcal{N}(0,1)$.
            \item $\dfrac{S^2}{\sigma^2}$ es $\dfrac{1}{n-1}$ veces una $\chi^2$ con $n-1$ grados de libertad.
            \item  $U$ y $\dfrac{S^2}{\sigma^2}$ son independientes.
        \end{itemize}
        La razón \[
        \dfrac{U}{\sqrt{\dfrac{\chi_{n-1}^2}{n-1}} }
        \] sigue una distribución \textbf{t de Stuent} con $n-1$ grados de libertad.

        Por lo tanto, \[
        T=\dfrac{Y}{S}\sqrt{\dfrac{n}{n+1}}=\dfrac{U}{\frac{S}{\sigma} } =\dfrac{U}{\sqrt{\frac{\chi_{n-1}^2}{n-1} } }\sim t_{n-1}.
        \] 
\end{enumerate}
\item \lb{Sean $X_1,X_2,\dots,X_{n_1}$ e $Y_1,Y_2,\dots,Y_{n_2}$ muestras aleatorias simples independientes de dos poblaciones $X\sim \mathcal{N}(\mu_1,\sigma^2)$ e $Y\sim \mathcal{N}(\mu_2,\sigma^2)$, respectivamente. Obtener la distribución en el muestreo estadístico: \[
            \dfrac{\alpha(\overline{X}-\mu_1)+\beta(\overline{Y}-\mu_2)}{\sqrt{\frac{(n_1-1)S_1^2+(n_2+1)S_2^2}{n_1+n_2-2} }\sqrt{\frac{\alpha^2}{n_1} +\frac{\beta^2}{n_2}}  },
\]siendo $\alpha$ y $\beta$ dos números reales fijos.} 

\item \lb{Una empresa de agua produce botellas que deberían contener 300ml pero que presentan en la práctica una variabilidad modelada por una distribución Normal con media $\mu=298\mathrm{ml}$ y desviación típica $\sigma=3\mathrm{ml}$.}
    \begin{enumerate}[label=\color{red}\textbf{\alph*)}]
        \item \db{¿Cuál es la probabilidad de que una botella elegida al azar de la producción contenga menos de 295ml?}
        \item \db{¿Cuál es la probabilidad de que el contenido medio de un paquete de seis botellas sea inferior a 295ml?} 
    \end{enumerate}
\item \lb{Se realiza una medición de peso en un laboratorio, sabiendo que la desviación típica de las mediciones es $\sigma=10\mathrm{mg}$. La medición se repite 3 veces, se calcula la media $\overline{x}$, y este es el resultado proporcionado como estimación del peso.}
    \begin{enumerate}[label=\color{red}\textbf{\alph*)}]
        \item \db{¿Cuál es la desviación típica del resultado calculado?}
        \item \db{¿Cuántas veces se debe repetir la medición para que la desviación típica del valor medio se reduzca a 5?} 
    \end{enumerate}
\item \lb{El resultado de una encuesta fue que el 59\% de la población española opina que el contexto económico es bueno o muy bueno. Supongamos que, extrapolando al conjunto de la población, efectivamente la proporción de todos los españoles que piensan que la situación es buena o muy buena es del 0.59.}
    \begin{enumerate}[label=\color{red}\textbf{\alph*)}]
        \item \db{Sabemos que las encuentas incluyen márgenes de error que son aproximadamente $\pm 3$ puntos. ¿Cuál es la probabilidad de que una muestra aleatoria de 300 españoles presente una proporción muestral que caiga dentro del intervalo $0.59\pm 0.03$?}
        \item \db{Contesta a la pregunta anterior en el caso en que la muestra consta de 600 personas y cuando la muestra consta de 1200 personas. ¿Cuál es el efecto de aumentar el tamaño de la muestra?} 
    \end{enumerate}
\item \lb{Un aparato de medición es exacto (el valor proporcionado medio es el valor auténtico de la señal) y la desviación típica del valor medido es 0.1 unidades. La distribución del valor medido es aproximadamente normal. ¿Cuál es la probabilidad de que el valor de una medición se aleje de la señal auténtica en más de 0.1 unidades? ¿Y si se repite la medición 5 veces y se toma la media de los 5 valores obtenidos?}
\item \lb{En condiciones normales, una máquina produce piezas con una tasa de defectuosas del 1\%. Para controlar que la máquina sigue bien ajustada, se escogen al azar cada día 100 piezas en la producción y se somete a un test. ¿Cuál es la probabilidad de que, si la máquina está bien ajustada, haya, en una de esas muestras, más del 2\% de piezas defectuosas? Si un día, 3 piezas resultan defectuosas, ¿cuáles son las conclusiones que sacaríamos sobre el funcionamiento de la máquina?} 
\end{enumerate}

