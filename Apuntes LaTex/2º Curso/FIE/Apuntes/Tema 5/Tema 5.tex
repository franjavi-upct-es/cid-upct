\section{Inferencia Bayesiana}
\subsection{Introducción}
\begin{tcolorbox}[colback=blue!5!white, colframe=blue!75!black, title=\textbf{Lo que hemos visto hasta el momento:}]
\begin{itemize}[label=\textbullet]
    \item Se llama el enfoque frecuentista, o clásico de la inferencia.
    \item Hay otro enfoque, llamado la \textbf{inferencia bayesiana}. 
\end{itemize}
\end{tcolorbox}
\begin{tcolorbox}[colback=blue!5!white, colframe=blue!75!black, title=\textbf{La inferencia bayesiana}]
\begin{itemize}[label=\textbullet]
    \item Asume que modelizamos la información que podemos tener sobre el parámetro a través de una distribución de probabilidad.
    \item Formulamos como parte del modelo una \textbf{distribución a priori} sobre $\theta$. Su f.p.p o función de densidad se denota por  $\pi(\theta)$.
    \item La inferencia bayesiana actualiza la información a priori, usando para ello la muestra $\mathbf{x}$ observada, para obtener la \textbf{distribución a posteriori $\pi(\theta|\mathbf{x})$}.
\end{itemize}
\end{tcolorbox}
\subsection{La inferencia bayesiana}
\begin{center}
    \includegraphics[width=\textwidth]{Tema 5/figures/Figure 1}
\end{center}
\begin{tcolorbox}[colback=olive!5!white, colframe=olive!75!black, title=\textbf{Fórmula de Bayes}]
\[
P(B|A)=\dfrac{P(A|B)P(B)}{P(A)}
\] 
\end{tcolorbox}
\begin{tcolorbox}[colback=blue!5!white, colframe=blue!75!black, title=\textbf{Para distribuciones condicionadas:}]
\[
\pi(\theta|\mathbf{x})=\dfrac{f(\mathbf{x}|\theta)\pi(\theta)}{m(\mathbf{x})},
\] 
\begin{itemize}[label=\textbullet]
    \item $m(\mathbf{x})$ es la marginal de $\mathbf{x}$.
    \item Si $\pi$ es una distribución continua: $m(\mathbf{x})=\int f(\mathbf{x}|u)\pi(u)\du $, la densidad marginal de $\mathbf{x}$ calculada a partir de la conjunta.
    \item Esta integral se realiza sobre el conjunto de valores posibles de $\theta$.
\end{itemize}
\end{tcolorbox}
\subsection{La distribución a posteriori}
En la expresión \[
\pi(\theta|\mathbf{x})=\dfrac{f(\mathbf{x}|\theta)\pi(\theta)}{m(\mathbf{x})},
\] 
$\mathbf{x}$ es el input, una constante. Nos interesa la variable $\theta$, y el denominador solo depende de  $\mathbf{x}$. Escribimos \[
\pi(\theta|\mathbf{x})\propto f(\mathbf{x}|\theta)\pi(\theta),
\] donde $\propto$ quiere decir "es proporcional a".
 \begin{tcolorbox}[colback=olive!5!white, colframe=olive!75!black, title=\textbf{Consejo}]
A veces, será suficiente reconocer el modelo de distribución que sigue $\pi(\theta|\mathbf{x})$ sin tener que calcular el denominador.
\end{tcolorbox}
\subsection*{Ejemplo}
\begin{tcolorbox}[colback=blue!5!white, colframe=blue!75!black, title=\textbf{Ingredientes}]
Lanzamos un dado $n$ veces y apuntamos en cada tirada si ha salido par o impar.
\begin{itemize}[label=\textbullet]
    \item $X\sim \mathrm{Bernoulli}(p),X=1$ si el resultado es par.
    \item muestra: $(x_1,\dots,x_n)$, tenemos que \[
    f(x_1,\dots,x_n|p)=p^{n\overline{x}}(1-p)^{n(1-\overline{x})}.
    \] 
\item Información a priori sobre $p$: puede tomar cualquier valor en $(0,1)$ con igual probabilidad:  \[
\pi(p)=\begin{cases}
    1 & \text{si }p \in (0,1)\\
    0 & \text{en otro caso}
\end{cases}
\] 
\end{itemize}
\end{tcolorbox}
\[
\pi(p|\mathbf{x})\propto p^{n\overline{x}}(1-p)^{n(1-\overline{x})}.
\] 
\begin{tcolorbox}[colback=blue!5!white, colframe=blue!75!black, title=\textbf{Reconocemos una distribución beta}]
Sean dos parámetros positivos $a$ y $b,\,X\sim \mathrm{Be}(a,b)$, si \[
f(x)=\dfrac{\Gamma(a+b)}{\Gamma(a)\Gamma(b)}x^{a-1}(1-x)^{b-1},\text{ para todo $x \in (0,1)$, }
\] siendo $\Gamma$ la función gamma.
 \begin{itemize}[label=\textbullet]
     \item Su esperanza: $E[X]=\dfrac{a}{a+b}$ 
     \item Su varianza: $\mathrm{Var}[X]=\dfrac{ab}{(a+b)^2(a+b+1)}$.
     \item Incluye la distribución uniforme en $(0,1)$, si  $a=1$ y  $b=1$.
     \item En  \lb{\textbf{\texttt{R}}}, \lb{\textbf{\texttt{dbeta}} }, \lb{\textbf{\texttt{pbeta}} }, \lb{\textbf{\texttt{qbeta}} }.
\end{itemize}
\end{tcolorbox}
\begin{tcolorbox}[colback=blue!5!white, colframe=blue!75!black, title=\textbf{Tiramos 10 veces el dado, obtenemos 6 vecer un número par}]
\[
\begin{array}{c}
    \pi(p|\mathbf{x})\propto p^{n\overline{x}}(1-p)^{n(1-\overline{x})}.\\
    \pi(p|\mathbf{x})\sim \mathrm{Be}(1+n\overline{x},1+n(1-\overline{x}))
\end{array}
\] 
¿Qué vale $n\overline{x}$? \[
\pi(p|\mathbf{x})\sim \mathrm{Be}(7,5)
\] 
\end{tcolorbox}
\begin{center}
    \includegraphics[width=0.7\textwidth]{Tema 5/figures/Figure 2}
\end{center}
Seguimos tirando el dado, hasta 50 veces, obtenemos otras 18 veces un número par, en total 24 veces: $\pi(p)=1,\pi(p|\mathbf{x})\sim \mathrm{Be}(25,27)$ 
\begin{center}
    \includegraphics[width=0.7\textwidth]{Tema 5/figures/Figure 3}
\end{center}
\subsection{Distribución a posteriori}
Con el enfoque bayesiando, el resultado de la inferencia, una vez observada una muestra, es una distribución entera, la distribución a posteriori.
\begin{itemize}[label=\textbullet]
    \item Podemos representarla, contiene toda la información que hemos inferido sobre $\theta$, a partir de la muestra y nuestra especificación de la distribución a priori.
    \item Podemos resumirla, por ejemplo con la media, la mediana.
    \item Podemos incluso dar un intervalo que contiene el 95\% de los valores de $\theta$. (\textit{"intervalo de credibilidad al 95\%"} ).
\end{itemize}
\begin{tcolorbox}[colback=blue!5!white, colframe=blue!75!black, title=\textbf{Estimador Bayes}]
Si consideramos una función de pérdida $L(\theta,a)$ asociada a estimar  $\theta$ por  $a$, la pérdida esperada a partir de la distribución a posteriori sería:  \[
    E[L(\theta,a)][\mathbf{x}]=\int_\Omega L(\theta,a)\pi(\theta|\mathbf{x})\dth 
\] para cada valor posible $\mathbf{x}$ del vector aleatorio $\mathbf{X}$, sea $\delta^*(\mathbf{x})$ un valor de a cuya pérdida esperada, dada por la expresión anterior sea mínima. La función $\delta^*(\mathbf{X})$ define un estimado llamado estimador Bayes: \[
\delta^*(\mathbf{X})=\min_{a\in \Theta}E[L(\theta,a)|\mathbf{x}]
\] 
\begin{itemize}[label=\textbullet]
    \item Si $L(\theta,a)=(\theta-a)^2\longrightarrow \delta^*(\mathbf{X})=E(\theta|\mathbf{X})$.
    \item Si $L(\theta,a)=|\theta-a|\longrightarrow \delta^*(\mathbf{X})=$ mediana.
\end{itemize}
\end{tcolorbox}
\subsection*{En el ejemplo de las tiradas de dados}
\begin{minipage}{0.45\textwidth}
\includegraphics[width=\textwidth]{Tema 5/figures/Figure 4}
\end{minipage} \begin{minipage}{0.59\textwidth}
\begin{tcolorbox}[colback=blue!5!white, colframe=blue!75!black, title=\textbf{Cuando usamos las 10 primeras observaciones }]
\begin{itemize}[label=\textbullet]
    \item $E[p|\mathbf{x}]=\dfrac{7}{12}=0.58$ 
    \item Intervalo credibilidad al 95\%:

        \textbf{\texttt{(qbeta(0.025,7,5),qbeta(0.975,7,5))=(0.31,0.83)}}
\end{itemize}
\end{tcolorbox}
\begin{tcolorbox}[colback=blue!5!white, colframe=blue!75!black, title=\textbf{Cuando usamos las 50 observaciones}]
\begin{itemize}[label=\textbullet]
    \item $E[p|\mathbf{x}]=\dfrac{25}{52}=0.48$ 
    \item Intervalo credibilidad al 95\%:

        \textbf{\texttt{(qbeta(0.025,25,27),qbeta(0.975,25,27))=(0.35,0.62)}}
\end{itemize}
\end{tcolorbox}
\end{minipage}
\subsection{Cambiamos la distribución a priori}
Si $\pi(p)\sim \mathrm{Be}(a_0,b_0)$, y $X|p\sim \mathrm{Bernoulli}(p)$ entonces \[
\pi(p|\mathbf{x})\sim \mathrm{Be}(a'+n\overline{x},b_0+n(1-\overline{x})).
\] 
\begin{tcolorbox}[colback=olive!5!white, colframe=olive!75!black, title=\textbf{Clases conjugadas}]
Decimos que las clase Bernoulli (para $X$) y Beta (para la priori) son \textbf{conjugadas:} la posteriori se queda en la misma clase de distribuciones que la a priori.
\end{tcolorbox}
\subsubsection{La distribución $\mathrm{Be}(a,b)$ es bastante flexible como a priori}
\includegraphics[width=\textwidth]{Tema 5/figures/Figure 5}
\subsection{Aplicaciones de la inferencia Bayesiana}
\begin{tcolorbox}[colback=blue!5!white, colframe=blue!75!black, title=\textbf{La inferencia bayesiana se utiliza en varias áreas del aprendizaje máquina}]
\begin{itemize}[label=\textbullet]
    \item \textbf{Redes Bayesianas} (Belief Networks): Modelos gráficos que permiten representar las dependencias condicionales entre variables aleatorias de forma compacta.
    \item \textbf{Optimización Bayesiana:} Se utilzia para optimizar funciones objetivas que son costosas de evaluar, útil en el ajuste de hiperparámetros para modelos de aprendizaje máquina.
    \item \textbf{Procesamiento de Lenguaje Natural} (PLN): por ejemplo para la clasificiación de texto y el análisis de sentimientos.
    \item \textbf{Sistemas de Recomendación:} Los modelos bayesianos se pueden utilziar para predecir las preferencias de los usuarios y hacer recomendaciones.
\end{itemize}
\end{tcolorbox}
\subsubsection{Ilustración para un ejemplo simple de sistema de recomendación:}
\begin{tcolorbox}[colback=blue!5!white, colframe=blue!75!black, title=\textbf{Contexto:}]
\begin{itemize}[label=\textbullet]
    \item Queremos predecir la probabilidad de que un usuario aprecie una nueva pelicula, para incluirla en las recomendaciones. Sabemos que, entre las 10 películas de este mismo tipo que ha visto, siete le han gustado.
    \item $X\sim \mathrm{Bernoulli}(p):X=1\longleftrightarrow$ "el usuario aprecia una película de este tipo".
    \item Observamos $x_1,x_2,\dots,x_{10}$.
    \item Usamos uan a priori no informativa para $p:\, \pi(p)=\begin{cases}
            1 & \text{si }p\in (0,1)\\
            0 & \text{en caso contrario}
    \end{cases}$
\end{itemize}
\end{tcolorbox}
\[
\pi(p|\mathbf{x})\sim \mathrm{Be}(1+n\overline{x},1+n(1-\overline{x}))\sim \mathrm{Be}(8,4).
\] 
\begin{tcolorbox}[colback=blue!5!white, colframe=blue!75!black, title=\textbf{Inferencia bayesiana predictiva:}]
\begin{itemize}[label=\textbullet]
    \item Tenemos la posteriori $\pi(p|\mathbf{x})$.
    \item Nos interesa $P(X=1|\mathbf{x})$, que es la llamada probabilidad predictiva.
\end{itemize}
\end{tcolorbox}
Para calcularla la obtenemos como marginal usando la posteriori: \[
P(X=1|\mathbf{x})=\int P(X=1|p)\pi(p|\mathbf{x})\mathrm{d}p=\int p\pi(p|\mathbf{x})\mathrm{d}p.
\] 
En este caso: $P(X=1|\mathbf{x})=E[p|\mathbf{x}]=\dfrac{8}{12}=\dfrac{2}{3}$.
\newpage
\documentclass[12pt]{article}
\usepackage{fullpage}
\usepackage[utf8]{inputenc}
\usepackage{pict2e}
\usepackage{amsmath}
\usepackage{enumitem}
\usepackage{eurosym}
\usepackage{pict2e}
\usepackage{mathtools}
\usepackage{amssymb, amsfonts, latexsym, cancel}
\setlength{\parskip}{0.3cm}
\usepackage{graphicx}
\usepackage{fontenc}
\usepackage{slashbox}
\usepackage{setspace}
\usepackage{gensymb}
\usepackage{accents}
\usepackage{adjustbox}
\setstretch{1.5}
\usepackage{bold-extra}
\usepackage[document]{ragged2e}
\usepackage{subcaption}
\usepackage{tcolorbox}
\usepackage{xcolor, colortbl}
\usepackage{wrapfig}
\usepackage{empheq}
\usepackage{array}
\usepackage{parskip}
\usepackage{arydshln}
\graphicspath{ {images/} }
\renewcommand*\contentsname{\color{black}Índice} 
\usepackage{array, multirow, multicol}
\definecolor{lightblue}{HTML}{007AFF}
\usepackage{color}
\usepackage{etoolbox}
\usepackage{listings}
\usepackage{mdframed}
\setlength{\parindent}{0pt}
\usepackage{underscore}
\usepackage{hyperref}
\usepackage{tikz}
\usepackage{tikz-cd}
\usetikzlibrary{shapes, positioning, patterns}
\usepackage{tikz-qtree}
\usepackage{biblatex}
\usepackage{pdfpages}
\usepackage{pgfplots}
\usepackage{pgfkeys}
\addbibresource{biblatex-examples.bib}
\usepackage[a4paper, left=1.5cm, right=1.5cm, top=1cm,
bottom=1.5cm]{geometry}
\everymath{\displaystyle}
\usetikzlibrary{decorations.pathreplacing}
\usepackage{titlesec}
\usepackage{titletoc}
\usepackage{tikz-3dplot}
\usetikzlibrary{decorations.pathreplacing}
\newcommand{\Ej}{\textcolor{lightblue}{\underline{Ejemplo}}}
\setlength{\fboxrule}{1.5pt}
\renewcommand{\arraystretch}{1.35}
\setlength{\arraycolsep}{0.3cm}

% Configura el formato de las secciones utilizando titlesec
\titleformat{\section}
{\color{red}\normalfont\LARGE\bfseries}
{Tema \thesection:}
{10 pt}
{}

% Ajusta el formato de las entradas de la tabla de contenidos
\addtocontents{toc}{\protect\setcounter{tocdepth}{4}}
\addtocontents{toc}{\color{black}}

\titleformat{\subsection}
{\normalfont\Large\bfseries\color{red}}{\thesubsection)}{1em}{\color{lightblue}}

\titleformat{\subsubsection}
{\normalfont\large\bfseries\color{red}}{\thesubsubsection)}{1em}{\color{lightblue}}

\newcommand{\bboxed}[1]{\fcolorbox{lightblue}{lightblue!10}{$#1$}}

\DeclareMathOperator{\N}{\mathbb{N}}
\DeclareMathOperator{\Z}{\mathbb{Z}}
\DeclareMathOperator{\R}{\mathbb{R}}
\DeclareMathOperator{\Q}{\mathbb{Q}}
\DeclareMathOperator{\K}{\mathbb{K}}
\DeclareMathOperator{\im}{\imath}
\DeclareMathOperator{\jm}{\jmath}
\DeclareMathOperator{\col}{\mathrm{Col}}
\DeclareMathOperator{\fil}{\mathrm{Fil}}
\DeclareMathOperator{\rg}{\mathrm{rg}}
\DeclareMathOperator{\nuc}{\mathrm{nuc}}
\DeclareMathOperator{\dimf}{\mathrm{dimFil}}
\DeclareMathOperator{\dimc}{\mathrm{dimCol}}
\DeclareMathOperator{\dimn}{\mathrm{dimnuc}}
\DeclareMathOperator{\dimr}{\mathrm{dimrg}}

\newcommand{\bu}[1]{\textcolor{lightblue}{\underline{#1}}}
\newcommand{\lb}[1]{\textcolor{lightblue}{#1}}
\newcommand{\db}[1]{\textcolor{blue}{#1}}
\newcommand{\rc}[1]{\textcolor{red}{#1}}
\newcommand{\tr}{^\intercal}

\renewcommand{\CancelColor}{\color{lightblue}}

\newcommand{\dx}{\:\mathrm{d}x}
\newcommand{\dt}{\:\mathrm{d}t}
\newcommand{\dy}{\:\mathrm{d}y}
\newcommand{\dz}{\:\mathrm{d}z}
\newcommand{\dth}{\:\mathrm{d}\theta}
\newcommand{\dr}{\:\mathrm{d}\rho}
\newcommand{\du}{\:\mathrm{d}u}
\newcommand{\dv}{\:\mathrm{d}v}
\newcommand{\tozero}[1]{\cancelto{0}{#1}}
\newcommand{\lbb}[2]{\textcolor{lightblue}{\underbracket[1pt]{\textcolor{black}{#1}}_{#2}}}
\newcommand{\dbb}[2]{\textcolor{blue}{\underbracket[1pt]{\textcolor{black}{#1}}_{#2}}}
\setlistdepth{9}
\lstset{
   language=Python,
   basicstyle=\ttfamily\small,
   numberstyle=\tiny,
   keywordstyle=\color{blue},
   commentstyle=\color{olive},
   stringstyle=\color{red},
   breakatwhitespace=false, 
   breaklines=true,
   showspaces=false,
   showstringspaces=false,
   showtabs=false, 
   tabsize=2,
   literate={á}{{\'a}}1 {é}{{\'e}}1 {í}{{\'i}}1 {ó}{{\'o}}1 {ú}{{\'u}}1{ñ}{{\~n}}1 {Á}{{\'A}}1 {Í}{{\'I}}1 {Ú}{{\'U}}1 {¿}{{\textexclamdown}}1,
   mathescape=false,
   backgroundcolor=\color{lightgray!10},   
}
\title{Análisis y Diseño de Algoritmos\\Ejercicios Tema 5}

\begin{document}
\maketitle
\begin{enumerate}[label=\color{red}\textbf{\arabic*)}]
  \item \lb{En el problema de la mochila (igual que en el problema del cambio de monedas) puede existir en general más de un solución óptima para unas entradas determinadas. ¿Cómo se puede comprobar si una solución óptima es única o no, suponiendo que hemos resuelto el problema utilizando programación dinámica? Dar un algoritmo para que, a paritr de las tablas resultantes del problema de la mochila, muestre todas las soluciones óptimas.}

Para determinar si una solución óptima al problema de la mochila es única y mostrar toas las soluciones óptimas, se puede utilziar el siguiente engoque basado en tablas generadas por el algoritmo de programación dinámica.

\textbf{Algoritmo}

\begin{enumerate}[label=\arabic*)]
\item Resuelve el problema de la mochila usando programación dinámica:
\begin{itemize}[label=\textbullet]
\item Construye la tabla \textbf{\texttt{dp[i][w]}}, donde \textbf{\texttt{dp[i][w]}} es el valor máximo que se puede obtener utilizando los primeros \textbf{\texttt{i}} elementos y un peso máximo \textbf{\texttt{w}}.
\end{itemize}
\item Encuentra el valor óptimo:
\begin{itemize}[label=\textbullet]
\item El valor óptimo se encuentra en \textbf{\texttt{dp[n][W]}}, donde $n$ es el número de elementos y $W$ es la capacidad máxima de la mochila.
\end{itemize}
\item Rastrea todas la soluciones posibles:
 \begin{itemize}[label=\textbullet]
   \item  A partir de \textbf{\texttt{dp[n][W]}}, rastrea todas las combinaciones de elementos que generan el valor \textbf{\texttt{dp[n][W]}}.
   \item Para cada elemento \textbf{\texttt{i}}, verifica si está incluido en la solución utilizando la condición: 
     \begin{center}
       \textbf{\texttt{dp[i][w]=dp[i-1][w]}} o \textbf{\texttt{dp[i][w]=dp[i-1][w-peso[i]] + valor[i]}}.  
     \end{center}
   \item Si el elemento puede estar incluido, explora ambas posibilidades (incluirlo y no incluirlo).
 \end{itemize}
  \item Construye todas las combinaciones:
    \begin{itemize}[label=\textbullet]
      \item Usa una búsqueda recursiva o iterativa para explorar todas las combinaciones posibles.
      \item Si llegas a un peso \textbf{\texttt{w=0}} o \textbf{\texttt{i=0}}, almacena la combinación como una solución válida.
    \end{itemize}
  \item Verifica unicidad:
    \begin{itemize}[label=\textbullet]
      \item Si encuentras más de una solución válida, la solución no es única. Si solo encuentras una solución, es única.
    \end{itemize}
\textbf{Implementación en psudocódigo:}
\lstinputlisting{"Ejercicio 1.py"}
\textbf{Explicación}
\begin{enumerate}[label=\arabic*)]
  \item Tabla \textbf{\texttt{dp}}:
    \begin{itemize}[label=\textbullet]
      \item Esta tabla se generó al resolver el problema de la mochila mediante programación dinámica.
    \end{itemize}
  \item Búsqueda de todas las soluciones:
    \begin{itemize}[label=\textbullet]
      \item La función \textbf{\texttt{backtrack}} recorre la tabla \textbf{\texttt{dp}} desde \textbf{\texttt{dp[n][W]}} hacia atrás, explorando todas las combinaciones posibles de inclusión y exclusión de elementos que generan el valor óptimo. 
    \end{itemize}
  \item Verificación de unicidad:
    \begin{itemize}[label=\textbullet]
      \item La lista \textbf{\texttt{solutions}} contendrá todas las combinaciones óptimas. Si su longitud es mayor a 1, significa que hay múltiples soluciones.
    \end{itemize}
\end{enumerate}
    
    \textbf{Complejidad:}
    \begin{itemize}[label=\textbullet]
      \item La complejidad temporal adicional para encontrar todas las soluciones es proporcional al número de combinaciones óptimas, que puede ser exponencial en el peor caso.
      \item Sin embargo, este proceso es manejable para entradas pequeñas o moderadas.
    \end{itemize}
\end{enumerate}
  \item \lb{El número de combinaciones de \textbf{\texttt{m}} objetos entre un conjunto de \textbf{\texttt{n}}, denotado por $\dbinom{n}{m}$, para $n\le 1$ y $0\le m\le n$, se puede definir recurisvamente por: \[
  \begin{array}{ll}
    \dbinom{n}{m}=1 & \text{Si $m=0$ ó  $m=n$}\\
    \dbinom{n}{m}=\dbinom{n-1}{m}+\dbinom{n-1}{m-1}& \text{Si }0<m<n
  \end{array}
  \]
Conociendo el resultado puede er calculado también con la fórmula: $\dfrac{n!}{m!\cdot (n-m)!}$}
\begin{enumerate}[label=\color{red}\alph*)]
  \item \db{Dar una función recursiva para calcular $\dbinom{n}{m}$, usando la primera de las definiciones. ¿Cuál será el orden de complejidad de este algoritmo? Sugerencia: la respuesta es inmediata.}

    La función recursiva se basa directametne en la definición dada: 
    \begin{enumerate}[label=\arabic*)]
      \item Definición recursiva: \[
          \dbinom{n}{m}=\begin{cases}
            1 & \text{si }m=0\: m=n\\
            \binom{n-1}{m} +\binom{n-1}{m-1} & \text{si }0<m<n 
          \end{cases}
      \] 
    \item Implementación en Python:
\lstinputlisting[firstline=1, lastline=4]{"Ejercicio 2.py"}
\item Complejidad:

  La complejidad de este algoritmo es \textbf{exponencial}, $O(2^n)$, porque calcula repetidamente los mismos valores para diferentes combinaciones de  $n$ y $m$. Por ejemplo, calcular  $\dbinom{n-1}{m} $, se vuelve a calcular los subproblemas de $\binom{n-2}{m} $ y $\binom{n-2}{m-1} $, causando redundancia.
    \end{enumerate}

  \item \db{Diseñar un algoritmo de programación dinámica para calcular $\dbinom{n}{m}$. Nota: la tabla construida por el algoritmo es conocida como "el triángulo de Pasca". ¿Cuál será el tiempo de ejecución en este caso?} 

    Para evitar cálculos repetidos, podemos usar un enfoque de \textbf{programación dinámica} que construye una tabla que almacena los valores de $\binom{n}{m} $. Esta tabla es conocida como el "Triángulo de Pascal".
    \begin{enumerate}[label=\arabic*)]
      \item Idea básica:
        \begin{itemize}[label=\textbullet]
          \item Construimos: una tabla \textbf{\texttt{dp}} de dimensiones $(n+1)\times (m+1)$.
          \item Inicializamos las condiciones base:
            \begin{itemize}[label=\textbullet]
              \item \textbf{\texttt{dp[i][0]=1}} para todo \textbf{\texttt{i}}, porque $\binom{\mathtt{i} }{0}=1 $.
              \item \textbf{\texttt{dp[i][i]=1}} para todo \textbf{\texttt{i}}, porque $\binom{\mathtt{i}}{\mathtt{i}}=1$.
            \end{itemize}
          \item Rellemanos la tabla usando la fórmula recursiva:
            \begin{center}
              \textbf{\texttt{dp[i][j]=dp[i-1][j]+dp[i-1][j-1]}}. 
            \end{center}
        \end{itemize}
    \end{enumerate}
  \item Implementación en Python
    \lstinputlisting[firstline=6, lastline=21]{"Ejercicio 2.py"}
  \item Complejidad:
    \begin{itemize}[label=\textbullet]
      \item \textbf{Tiempo:} $O(n\cdot m)$. Cada celda en la tabla se calcula una vez y depende de dos valores previos. 
    \end{itemize}
  \item Ejemplo de uso:
    \lstinputlisting[firstline=24, lastline=26]{"Ejercicio 2.py"}
    \begin{verbatim}
## Resultado con recursión: 10
## Resultado con DP: 10
    \end{verbatim}
\end{enumerate}
\textbf{Comparación de enfoques:}
\begin{center}
  \begin{tabular}{|l|c|p{5cm}|}
    \hline
    \textbf{Enfoque} & \textbf{Complejidad temporal} & \textbf{Ventajas/Desventajas}\\ \hline
    Recursivo & $O(2^n)$ & Simple de implementar pero extremadamente ineficiente.\\ \hline
    Programación dinámica & $O(n\cdot m)$ & Eficiente, especialmente para valores grandes de $n$ y  $m$.\\ \hline
  \end{tabular}
\end{center}
\item \lb{Considerar el problema de la mochila 0/1. En este ejercicio estamos interesados en calcular el número de formas distintas de meter o no los objetos en la mochila, pero respetando la capcaidad máxima de la mochila. Por ejemplo, si todos los \textbf{\texttt{n}} objetos cupieran en la mochila, existirían $\mathbf{\mathtt{2^n}}$ formas posibles. Pero en general, si no caben todos, habrán muchas menos. Resolver mediante programación dinámica el problema de calcular el número de formas distintas de completar total o parcialmente la mochila. Datos del problema: \textbf{\texttt{n}} objetos, \textbf{\texttt{M}} capacidad de la mochila, $\mathbf{\mathtt{p=(p1,p2,\dots,p_n)}}$ pesos de los objetos.} 

  \textbf{Definición del subproblema}

  Sea \textbf{\texttt{dp[i][w]}} el número de formas de llenar la mochila con un subconjunto de los primeros \textbf{\texttt{i}} objetos, de manera que el peso total sea \textbf{\texttt{w}}.

  \textbf{Casos base:}
  \begin{enumerate}[label=\arabic*)]
    \item \textbf{\texttt{dp[0][w]=1}} para todo \textbf{\texttt{w$\ge$ 0}}: hay exactamente una forma de no seleccionar ningún objeto (la mochila vacía).
    \item \textbf{\texttt{dp[i][0]=1}} para todo \textbf{\texttt{i}}: hay exactamente una forma de obtener un peso de $0$ (no seleccionar ningún objeto).
  \end{enumerate}
  \textbf{Relación de recurrencia:} 
        
  Para cada objeto \textbf{\texttt{i}} con el peso \textbf{\texttt{p[i]}}:
  \begin{itemize}[label=\textbullet]
    \item Si no seleccionados el objeto \textbf{\texttt{i}}, las formas posibles son \textbf{\texttt{dp[i-1][w]}} (todas las formas de llenar la mochila con los \textbf{\texttt{i-1}} objetos y el mismo peso \textbf{\texttt{w}} ).
    \item Si seleccionamos el objeto \textbf{\texttt{i}}, las formas posibles son \textbf{\texttt{dp[i-1][w-p[i]]}} (todas las formas de llenar la mochila con \textbf{\texttt{i-1}} objetos y un peso \textbf{\texttt{w-p[i]}}).
  \end{itemize}
  Por lo tanto:
  \begin{center}
    \textbf{\texttt{dp[i][w]=dp[i-1][w]+dp[i-1][w-p[i]]}} si \textbf{\texttt{w $\ge$ p[i]}}. 
  \end{center}
  \textbf{Algoritmo}
  \begin{enumerate}[label=\arabic*)]
    \item \textbf{Inicializa la tabla} \textbf{\texttt{dp}} con dimensiones $(n+1)\times (M+1)$, donde $n$ es el número de objetos y $M$ es la capacidad de la mochila.
    \item Usa la relación de recurrencia para llenar la tabla.
    \item La suma de todos los valores en la última fila (\textbf{\texttt{dp[n][w]}} para \textbf{\texttt{w=0}} hasta \textbf{\texttt{w=W}}) será el número total de formas válidas.
  \end{enumerate}
  \textbf{Implementación en Python}

  \lstinputlisting{"Ejercicio 3.py"}
  \begin{verbatim}
## Número de formas de llenar la mochila: 6
  \end{verbatim}
  \textbf{Ejemplo de ejecución}
  
  Supongamos:
  \begin{itemize}[label=\textbullet]
    \item $n=3$
    \item  $M=4$
    \item  $p = [1,2,3]$
  \end{itemize}
  \begin{enumerate}[label=Paso \arabic*:]
    \item Inicialización

      La tabla \textbf{\texttt{dp}} se inicializa con: 
      \begin{center}
        \textbf{\texttt{dp[0][w]=1}} para todo \textbf{\texttt{w}}. 
      \end{center}
    \item Llenado de la tabla
      \begin{itemize}[label=\textbullet]
        \item Para \textbf{\texttt{i=1 (p[1]=1)}:}
          \begin{center}
            \textbf{\texttt{dp[1][w]=dp[0][w] + dp[0][w-1]}} si \textbf{\texttt{w $\ge $ 1}}.  
          \end{center}
        \item Para \textbf{\texttt{i=2 (p[2]=2)}:}
          \begin{center}
            \textbf{\texttt{dp[2][w]=dp[1][w] + dp[1][w-2]}} si \textbf{\texttt{w $\ge $ 2}}.  
          \end{center}
        \item Para \textbf{\texttt{i=3 (p[3]=3)}:}
          \begin{center}
            \textbf{\texttt{dp[3][w]=dp[2][w] + dp[2][w-3]}} si \textbf{\texttt{w $\ge $ 3}}.  
          \end{center}
      \end{itemize}
  \end{enumerate}
  \textbf{Resultado final}

  La suma de \textbf{\texttt{dp[3][w]}} para \textbf{\texttt{w=0}} a \textbf{\texttt{w=4}} da el número total de formas válida.
\item \lb{Una variante del problema de la mochila es la siguiente. Tenemos un conjunto de enteros (positivos) $A=\{a_1,a_2,\dots,a_n\} $ y un entero $K$. El objetico es encontrar si existe algún subconjunto de  $A$ cuya suma sea exactamente  $K$.}
  \begin{enumerate}[label=\color{red}\textbf{\alph*)}]
    \item \db{Desarrollar un algoritmo para resolver este problema, utilizando programación dinámica. ¿Cuál es el orden de complejidad de este algoritmo?}

      Este problema se puede resolver utilizando \textbf{programación dinámica} con un enfoque similar al problema de la mochila, La idea es contruir una tabla \textbf{\texttt{dp[i][j]}}, donde:
      \begin{itemize}[label=\textbullet]
        \item \textbf{\texttt{dp[i][j]}} es \textbf{\texttt{True}}  si es posible obtener una suma \textbf{\texttt{j}} utilizando los primeros \textbf{\texttt{i}} elementos del conjunto $A$.
        \item \textbf{\texttt{dp[i][j]}} es \textbf{\texttt{False}} en caso contrario.
      \end{itemize}
      \begin{enumerate}[label=\arabic*)]
        \item Casos base
          \begin{enumerate}[label=\arabic*)]
            \item \textbf{\texttt{dp[0][0]=True}:} Es posible obtener una suma de $0$ utilizando un subconjunto vacío.
            \item \textbf{\texttt{dp[i][0]=True} para todo \texttt{i}:} También es posible obtener $0$ sin incluir ningún elemento.
            \item \textbf{\texttt{dp[0][j]=False} para todo $\mathbf{\mathtt{j>0}}$:} No es posible obtener una suma mayor que $0$ con un subconjunto vacío.
          \end{enumerate}
        \item Relación de recurrencia

          Para cada \textbf{\texttt{i}} y \textbf{\texttt{j}}, el estado de \textbf{\texttt{dp[i][j]}} depende de:
          \begin{enumerate}[label=\arabic*)]
            \item No incluir el elemento actual $\mathbf{\mathtt{a_i}}$: Entonces \textbf{\texttt{dp[i][j]=dp[i-1][j]}}.
            \item Incluir el elemento actual $\mathbf{\mathtt{a_i}}$ (si $\mathbf{\mathtt{j\ge a_i}}$): Entonces \textbf{\texttt{dp[i][j]=dp[i-1][j-$\mathbf{\mathtt{a_i}}$]}}.
          \end{enumerate}
          Por lo tanto: 
          \begin{center}
            \textbf{\texttt{dp[i][j]=dp[i-1][j]} OR \texttt{dp[i-1][j-$\mathbf{\mathtt{a_i}}$]} }\quad (si $\mathbf{\mathtt{j\ge  a_i}}$).
          \end{center}
          \item Complejidad temporal: $O(n\cdot K)$, donde $n$ es el número de elementos en $A$ y $K$ es la suma objetivo.

            \textbf{Implementación}

            \lstinputlisting[firstline=1, lastline=24]{"Ejercicio 4.py"}
            \begin{verbatim}
## ¿Existe una solución? True
            \end{verbatim}
      \end{enumerate}
    \item \db{Mostrar cómo se puede obtener el conjunto de objetos resultantes (en caso de existir solución) a partir de las tablas utilizadas por el algoritmo.}

      Para obtener el subconjunto que produce la suma $K$, podemos rastrear la tabla  \textbf{\texttt{dp}} desde \textbf{\texttt{dp[n][K]}} hacia atrás:
      \begin{enumerate}[label=\arabic*)]
        \item Comenzamos desde \textbf{\texttt{dp[n][K]}} y verificamos si el elemento \textbf{\texttt{A[i]}} está incluido:
          \begin{itemize}[label=\textbullet]
            \item Si \textbf{\texttt{dp[i][K]$\neq $dp[i-1][K]}}, significa que el elemento \textbf{\texttt{A[i-1]}} está incluido.
            \item Restamos \textbf{\texttt{A[i-1]}} de $K$ y continuamos con \textbf{\texttt{dp[i-1][K - A[i-1]]}}.
          \end{itemize}
        \item Si \textbf{\texttt{dp[i][K]=dp[i-1][K]}}, el elemento no está incluido.
        \item Detenemos cuando $K=0$.
      \end{enumerate}
      \textbf{Implementación}

      \lstinputlisting[firstline=26]{"Ejercicio 4.py"}

      \begin{verbatim}
## Subconjunto encontrado: [5, 2]
      \end{verbatim}
    \item \db{Aplicar el algoritmo sobre el siguiente ejemplo $A=\{2,3,5,2\},\, K=7 $. ¿Cómo se puede comprobar que la solución no es única?} 

      Para $A=\{2,3,5,2\} $ y $K=7$, construimos la tabla  \textbf{\texttt{dp}}:

      \begin{center}
      \begin{tabular}{|l|c|c|c|c|c|c|l|l|}
        \hline
        \textbf{i/j} & \textbf{0} & \textbf{1} & \textbf{2} & \textbf{3} & \textbf{4} & \textbf{5} & \textbf{6} & \textbf{7} \\ \hline
        0 & T & F & F & F & F & F & F & F \\ \hline
        1(2) & T & F & T & F & F & F & F & F \\ \hline
        2(3) & T & F & T & T & F & T & F & F \\ \hline
        3(5) & T & F & T & T & F & T & T & T \\ \hline
        4(2) & T & F & T & T & T & T & T & T \\ \hline
      \end{tabular}
      \end{center}
      \begin{itemize}[label=\textbullet]
        \item \textbf{Solución:} \textbf{\texttt{dp[4][7]=True}}, por lo que existe al menos un subconjunto. 
      \end{itemize}
      \textbf{Obtener subconjunto:}

      Siguiendo la tabla:
      \begin{itemize}[label=\textbullet]
        \item \textbf{\texttt{A[3]=2}} está incluido $(K=7\to K=5)$.
        \item \textbf{\texttt{A[2]=5}} está incluido $(K=5\to K=0)$.
      \end{itemize}
      \textbf{Subconjunto:} $\{5,2\}$.

      \textbf{Comprobación de unicidad:}

      Para verificar si hay múltiples soluciones, buscamos otras formas de descomponer $K=7$:
      \begin{enumerate}[label=\arabic*)]
        \item Si \textbf{\texttt{dp[i][K - A[i-1]]=True}} para diferentes \textbf{\texttt{i}}, hay más soluciones.
        \item En este caso, otra solución es $\{3,2,2\} $.
      \end{enumerate}
      \textbf{Conclusión:} La solución no es única.
  \end{enumerate}
\item \lb{En el problema de la mochila 0/1 disponemos de dos mochilas, con capacidades \textbf{\texttt{M1}} y \textbf{\texttt{M2}}. El objetivo es maxmizar la suma de beneficios de los objetos transportados en ambas mochilas, respetando las capacidades de cada una. Resolver el problema mediante programación dinámica, definiendo la ecuación recurrente, las tablas usadas y el algoritmo para rellenarlas.\newline
    Datos del problema: \textbf{\texttt{n}} objetos, \textbf{\texttt{M1}} capacidad de la mochila 1, \textbf{\texttt{M2}} capacidad de la mochila 2, $\mathbf{\mathtt{p=(p_1,p_2,\dots,p_n)}}$ pesos de los objetos, $\mathbf{\mathtt{b=(b_1,b_2,\dots,b_n)}}$ beneficios de los objetos.}
    
    \textbf{Definición del subproblema} 

    Sea \textbf{\texttt{dp[i][$\mathtt{c_1}$][$\mathtt{c_2}$]}} es el beneficio máximo que se puede obtener utilizando los primeros \textbf{\texttt{i}} objetos, con capacidad disponible $\mathbf{\mathtt{c_1}}$ en la mochila 1 y $\mathbf{\mathtt{c_2}}$ en la mochila 2.

    \textbf{Casos base:}
    \begin{enumerate}[label=\arabic*)]
      \item $\mathbf{\mathtt{dp[0][c_1][c_2]=0}}$: Si no consideramos ningún objeto $(\mathbf{\mathtt{i=0}})$, el beneficio es $0$, sin importar las capacidades de $\mathbf{\mathtt{c_1}}$ y $\mathbf{\mathtt{c_2}}$.
      \item $\mathbf{\mathtt{cp[i][0][c_2]=dp[i][c_1][0]=0}}$: Si alguna mochila tiene capacidad $0$, no se puede incluir objetos en ella.
    \end{enumerate}
    \textbf{Relación de recurrencia:}

    Para cada objeto \textbf{\texttt{i}} con peso $\mathbf{\mathtt{p[i]}}$ y beneficio $\mathbf{\mathtt{b[i]}}$, tenemos tres opciones:
    \begin{enumerate}[label=\arabic*)]
      \item No incluir el objeto \textbf{\texttt{i}} en ninguna mochila: \[
          \mathbf{\mathtt{dp[i][c_1][c_2]=dp[i-1][c_1][c_2]}}.
      \] 
    \item Incluir el objeto \textbf{\texttt{i}} en la mochila 1 (si cabe): \[
        \mathbf{\mathtt{dp[i][c_1][c_2]= \max(dp[i][c_1][c_2], dp[i-1][c_1-p[i]][c_2]+b[i])}}\quad \text{si }\mathbf{\mathtt{c_1\ge p[i]}}.
    \] 
    \item Incluir el objeto \textbf{\texttt{i}} en la mochila 2 (si cabe): \[
        \mathbf{\mathtt{dp[i][c_1][c_2]= \max(dp[i][c_1][c_2], dp[i-1][c_1][c_2-p[i]]+b[i])}}\quad \text{si }\mathbf{\mathtt{c_2\ge p[i]}}.
    \] 
    \end{enumerate}
En resumen: \[
  \mathbf{\mathtt{dp[i][c_1][c_2]=\max(dp[i-1][c_1][c_2], dp[i-1][c_1-p[i]][c_2]+b[i], dp[i-1][c_1][c_2-p[i]]+b[i])}}.
\]
\textbf{Algoritmo}
\begin{enumerate}[label=\arabic*)]
  \item Inicialización:
    \begin{itemize}[label=\textbullet]
      \item Crear una tabla $\mathbf{\mathtt{dp[i][c_1][c_2]}}$ de dimensiones $(n+1)\times (M_1+1)\times (M_2+1)$.
      \item Inicializar $\mathbf{\mathtt{dp[0][c_1][c_2]}}=0$ para todas las combinaciones de $\mathbf{\mathtt{c_1}}$ y $\mathbf{\mathtt{c_2}}$.
    \end{itemize}
  \item Llenado de la tabla:
    \begin{itemize}[label=\textbullet]
      \item Iterrar sobre los objetos (\textbf{\texttt{1}} de $1$ a  $n$).
      \item Para cada combinación de capacidades $\mathbf{\mathtt{c_1}}$ y $\mathbf{\mathtt{c_2}}$, calcular el valor de $\mathbf{\mathtt{dp[i][c_1][c_2]}}$ usando la relación de recurrencia.
    \end{itemize}
  \item Resultado:
    \begin{itemize}[label=\textbullet]
      \item El beneficio máximo está en $\mathbf{\mathtt{dp[n][M_1][M_2]}}$.
    \end{itemize}
  \item Reconstrucción de la solución:
    \begin{itemize}[label=\textbullet]
      \item Para determinar qué objetos se seleccionaron y en qué mochila, rastrear las decisiones tomadas al llenar la tabla  \textbf{\texttt{dp}}.
    \end{itemize}
\end{enumerate}
\textbf{Implementación en Python} 

\lstinputlisting{"Ejercicio 5.py"}
\begin{verbatim}
## Beneficio máximo: 18
## Objetos en la mochila 1: [3, 2]
## Objetos en la mochila 2: [1, 0]
\end{verbatim}
\textbf{Complejidad temporal:} $O(n\cdot M_1\cdot M_2)$, ya que iteramos sobre $n,M_1$ y $M_2$. 
  \item \lb{Una agencia de turismo realiza planificaciones de viajes aéreos. Para ir de una ciudad \textbf{\texttt{A}} a \textbf{\texttt{B}}  puede ser necesario coger varios vuelos distintos. El tiempo de un vuelo directo desde  \textbf{\texttt{I}} hasta \textbf{\texttt{J}} será \textbf{\texttt{T[i,j]}}. Hay que tener en cuenta que si cogemos un vuelo (de \textbf{\texttt{A}} a \textbf{\texttt{B}}) y después otro (de \textbf{\texttt{B}} a \textbf{\texttt{C}}) será necesario esperar un tiempo de "escala" adicional en el aeropuesto (almacenado en \textbf{\texttt{E[A, B, C]}}).}
    \begin{enumerate}[label=\color{red}\alph*)]
      \item \db{Diseñar una solución para resolver este problema utilizando programación dinámica. Explicar cómo, a partir de las tablas, se puede obtener el conjunto de vuelos necesarios para hacer un viaje concreto.}

        El problema se puede resolver utilizando una variante del algoritmo de \textbf{Floyd-Warshall} para calcular el tiempo mínimo necesario para viajar entre dos ciudades teniendo en cuenta tiempos de escala adicionales.
        \begin{enumerate}[label=\arabic*)]
          \item Representación del problema
            \begin{itemize}[label=\textbullet]
              \item Supongamos que tenemos  \textbf{\texttt{n}} ciudades, y una matriz \textbf{\texttt{T[i,j]}} que almacena el tiempo de vuelo directo entre las ciudades \textbf{\texttt{i}} y \textbf{\texttt{j}}. Si no hay vuelo directo entre dos ciudades, representaremos esto como infinito $(\infty)$.
              \item Además, la matriz \textbf{\texttt{E[i,j,k]}} indica el tiempo de escala adicional necesario si tomamos el vuelo de \textbf{\texttt{i}} a \textbf{\texttt{j}} y luego de \textbf{\texttt{j}} a\textbf{\texttt{k}}.
            \end{itemize}
          \item Idea general
            \begin{enumerate}[label=\arabic*)]
              \item Usa programación dinámica para calcular la tabla \textbf{\texttt{dp[k][i][j]}}, que representa el tiempo mínimo necesario para viajar de \textbf{\texttt{i}} a \textbf{\texttt{j}} utilizando como máximo \textbf{\texttt{k}} ciudades intermedias.
              \item Define:
                \begin{itemize}[label=\textbullet]
                  \item Si no se usa ninguna ciudad intermedia  \textbf{\texttt{(k=0)}}, el tiempo es simplemente tiempo de vuelo directo \textbf{\texttt{T[i][j]}}.
                  \item Para $\mathbf{\mathtt{k>0}}$, el tiempo mínimo se calcular como:
                    \[
                      \mathbf{\mathtt{dp[k][i][j]=\min(dp[k-1][i][j], dp[k-1][i][m]+E[i,m,j]+dp[k-1][m][j])}}
                    \] donde \textbf{\texttt{m}} es una ciudad intermedia.
                \end{itemize}
              \item El tiempo óptimo para ir de \textbf{\texttt{A}} a \textbf{\texttt{B}} se encuentra en \textbf{\texttt{dp[n][A][B]}}, donde \textbf{\texttt{n}} es el número total de ciudades.
            \end{enumerate}
          \item Algoritmo paso a paso
            \begin{enumerate}[label=\arabic*)]
              \item Inicializa \textbf{\texttt{dp[0][i][j]}} con los valores de \textbf{\texttt{T[i][j]}}. Si no hay vuelo directo, establece $\mathbf{\mathtt{dp[0][i][j]=\infty}}$.
              \item Para cada \textbf{\texttt{k}}, calcula \textbf{\texttt{dp[k][i][j]}} utilizando la fórmula recursiva.
              \item Almacena en una tabla auxiliar \textbf{\texttt{next[i][j]}} el vuelo intermedio que minimiza el tiempo. Esto permite reconstruir el conjunto de vuelos necesarios.
            \end{enumerate}
          \item Obtención del conjunto de vuelos
            \begin{enumerate}[label=\arabic*)]
              \item A partir de la tabla \textbf{\texttt{next}}, se puede reconstruir el conjunto de vuelos necesaros de la siguiente manera:
                \begin{itemize}[label=\textbullet]
                  \item Si \textbf{\texttt{next[A][B]=C}}, significa que la ruta óptima de \textbf{\texttt{A}} a \textbf{\texttt{B}} pasa por \textbf{\texttt{C}}.
                  \item Recursivamente, descompín \textbf{\texttt{A\textrightarrow B}} en \textbf{\texttt{A\textrightarrow C}} y \textbf{\texttt{C\textrightarrow B}}.   
                \end{itemize}
            \end{enumerate}
        \end{enumerate}
      \item \db{Mostrar la ejecución del algoritmo sobre la siguiente matriz \textbf{\texttt{T}}, suponiendo que todos los \textbf{\texttt{E[A, B , C]}} tienen valor 1. ¿Cuál es el orden de complejidad del algoritmo?}

\begin{center}
\color{blue}
              \begin{tabular}{|c|c|c|c|c|}
              \hline
              \textbf{\texttt{T[i,j]}} & \textbf{\texttt{A}} & \textbf{\texttt{B}} & \textbf{\texttt{C}} & \textbf{\texttt{D}}\\ \hline
              \textbf{\texttt{A}} & - & 2 & 1 & 3\\ \hline
              \textbf{\texttt{B}} & 2 & 2 & - & 1\\ \hline
              \textbf{\texttt{D}} & 3 & 4 & 8 & -\\ \hline

\end{tabular}
\end{center}

        Dado: \[
        \mathbf{\mathtt{T=\begin{bmatrix} 
              - & 2 & 1 & 3\\
              2 & - & - & 1\\
              3 & 4 & 8 & -
        \end{bmatrix},\quad E[i,j,k]=1. }}
        \] 

        \begin{enumerate}[label=Paso \arabic*:]
          \item Inicialización
            \begin{itemize}[label=\textbullet]
              \item La matriz inicial (\textbf{\texttt{dp[0][i][j]}}) es: \[
                  \mathbf{\mathtt{dp[0]=}}\begin{bmatrix} 
                    \infty & 2 & 1 & 3\\
                    2 & \infty & \infty & 1\\
                    3 & 4 & 8 & \infty
                  \end{bmatrix}. 
              \] 
            \end{itemize}
          \item Iteración para \textbf{\texttt{k=1}} 
            \begin{itemize}[label=\textbullet]
              \item Para cada par de ciudades \textbf{\texttt{i}} y \textbf{\texttt{j}}, actualizamos: \[
                  \mathbf{\mathtt{dp[1][i][j]=\min(dp[0][i][j],dp[0][i][m]+E[i,m,j]+dp[0][m][j]),}}
              \] donde \textbf{\texttt{m}} es la ciudad intermedia.
            \end{itemize}
        \end{enumerate}
        Por ejemplo:
        \begin{enumerate}[label=\arabic*)]
          \item Para $\mathbf{\mathtt{A\to D}}$: \[
              \mathbf{\mathtt{dp[1][A][D]=\min(dp[0][A][D], dp[0][A][B]+E[A,B,D]+dp[0][B][D]).}}
          \] 
          Sustituyendo: \[
            \mathbf{\mathtt{dp[1][A][D]=\min(3,2+1+1)=\min(3,4)=3.}}
          \] 
          \begin{itemize}[label=\textbullet, leftmargin=*]
            \item Repite este cálculo para todos los pares \textbf{\texttt{(i,j)}}.
          \end{itemize}
        \end{enumerate}
      \item Resultado final

        La matriz \textbf{\texttt{dp[n]}} después de iterar por todas las ciudades intermedias será: \[
        \mathbf{\mathtt{dp=\begin{bmatrix} 
              \infty & 2 & 1 & 3\\
              2 & \infty & 3 & 1\\
              3 & 4 & 8 & 4
        \end{bmatrix}. }}
        \] 
    \end{enumerate}
    \textbf{Implementación en Python}

    \lstinputlisting{"Ejercicio 6.py"}
\begin{verbatim}
## Matriz de tiempos mínimos (dp):
## [[5. 2. 1.]
## [2. 5. 4.]
## [3. 4. 5.]]
## 
## Matriz next_city:
## [[1, 1, 2], [0, 0, 0], [0, 1, 0]
## 
## Ruta óptima de 0 a 2: [0, 2]]]]]
\end{verbatim}
    \textbf{Complejidad temporal:} $O(n^3)$, esto se debe a que para cada par de ciudades \textbf{\texttt{(i,j)}}, iteramos sobre todas las ciudades intermedias \textbf{\texttt{m}}.

\item \lb{Supongamos una serie de \textbf{\texttt{n}} trabajos denominados \textbf{\texttt{a, b, c, \dots}} y una tabla \textbf{\texttt{B[1..n, 1..n]}}, en la que cada posición \textbf{\texttt{B[i,j]}} almacena el beneficio de ejecutar el trabajo \textbf{\texttt{i}}  y a continuación el trabajo \textbf{\texttt{j}}. Se quiere encontrar la sucesión de \textbf{\texttt{m}} trabajos que dé un beneficio óptimo. No hay límite en el número de veces que se puede ejecutar un trabajo concreto.}
        \begin{enumerate}[label=\color{red}\alph*)]
                \item \db{Idear un algoritmo por programación dinámica que resuelva el problema. Para ello, definir un subproblema (que permita realiza la combinación de problemas pequeños para resolver problemas grandes), especifica la ecuación de recurrencia para el mismo (con sus cosas base) y después describe las tablas necesarias y cómo son rellenadas.}

                  \begin{enumerate}[label=\arabic*)]
                    \item Definición del subproblema:

                      Sea \textbf{\texttt{dp[k][j]}} el beneficio máximo que se pued eobtener ejeutando  \textbf{\texttt{k}} trabajos, donde el último trabajo es \textbf{\texttt{j}}. El objetivo es calcular: \[
                        \mathbf{\mathtt{\underset{j}{\max}\,dp[m][j]}}
                      \] donde \textbf{\texttt{m}} es el número total de trabajos en la secuencia.
                    \item Ecuación de recurrencia:

                      Para calcular \textbf{\texttt{dp[k][j]}}, consideramos que el último trabajo de la secuencia es \textbf{\texttt{j}} y el penúltimo trabajo es \textbf{\texttt{i}}. La relación es: \[
                        \mathbf{\mathtt{dp[k][j]=\underset{i}{\max}(dp[k-1][i]+B[i][j])}}
                      \] donde:
                      \begin{itemize}[label=\textbullet]
                        \item \textbf{\texttt{dp[k-1][i]}} es el beneficio para \textbf{\texttt{k-1}} trabajos terminando en \textbf{\texttt{i}}.
                        \item \textbf{\texttt{B[i][j]}} es el beneficio de ejecutar el trabajo \textbf{\texttt{i}} seguido del trabajo \textbf{\texttt{j}}.
                      \end{itemize}
                    \item Condiciones base: \[
                        \mathbf{\mathtt{dp[1][j]=0\quad \forall j}}
                    \] 
                    No hay beneficio inicial para \textbf{\texttt{k=1}} porque no hay trabajo anterior.
                  \item Tablas necesarias:
                    \begin{itemize}[label=\textbullet]
                      \item Una tabla \textbf{\texttt{dp[k][j]}} de dimensiones $m\times n$, donde \textbf{\texttt{k}} es el número de trabajos y \textbf{\texttt{j}} es el índice del último trabajo.
                      \item Una tabla auxiliar \textbf{\texttt{prev[k][j]}} para reconstruir la secuencia óptima.
                    \end{itemize}
                  \item Proceso de llenado:
                    \begin{enumerate}[label=\arabic*)]
                      \item Inicializar \textbf{\texttt{dp[1][j]=0}}.
                      \item Para cada \textbf{\texttt{k}} de 2 a \textbf{\texttt{m}}, y para cada \textbf{\texttt{j}}, calcular: \[
                          \mathbf{\mathtt{dp[k][j]=\underset{i}{\max}(dp[k-1][i]+B[i][j])}}
                        \] y almacenar en \textbf{\texttt{prev[k][j]}} el índice \textbf{\texttt{i}} que maximiza el beneficio.
                    \end{enumerate}
                  \item Reconstrucción de la solución:

                    Comenzando desde el índice \textbf{\texttt{j}} que maximizar \textbf{\texttt{dp[m][j]}}, usar \textbf{\texttt{prev[k][j]}} para retroceder y encontrar la secuencia de trabajos óptima.
                  \end{enumerate}
                \item \db{Ejecutar el algoritmo sobre la siguiente tabla, suponiendo que  \textbf{\texttt{m=5}}.}
                
\begin{center}
\color{blue}
                                \begin{tabular}{|c|c|c|c|}
        \hline
        \textbf{\texttt{B[i,j]}} & \textbf{\texttt{a}} & \textbf{\texttt{b}} & \textbf{\texttt{c}} \\ \hline
        \textbf{\texttt{A}}  & 2 & 2 & 5\\ \hline
        \textbf{\texttt{B}}  & 4 & 1 & 3\\ \hline
        \textbf{\texttt{C}}  & 3 & 2 & 2\\ \hline
\end{tabular}
\end{center}

Tabla \textbf{\texttt{B}}: \[
\mathbf{\mathtt{B=\begin{bmatrix} 
      2 & 2 & 5\\
      4 & 1 & 3\\
      3 & 2 & 2
\end{bmatrix} }}
\]  
\textbf{Parámetros:}
\begin{itemize}[label=\textbullet]
  \item \textbf{\texttt{n=3}} (número de trabajos).
  \item \textbf{\texttt{m=5}} (longitud de la secuencia).
\end{itemize}

\textbf{Algoritmo de Python:}

\lstinputlisting{"Ejercicio 7.py"}
\begin{verbatim}
## Beneficio máximo: 17.0
## Secuencia óptima: [1, 0, 2, 0, 2]
\end{verbatim}
                \item \db{Estimar el tiempo de ejecución del algoritmo. El tiempo estimado ¿es un orden exacto o una cota superior del peor caso?} 

                  \begin{enumerate}[label=\arabic*)]
                    \item Tiempo de ejecución:
                      \begin{itemize}[label=\textbullet]
                        \item La fórmula \textbf{\texttt{dp[k][j]}} requiere calcular $\mathbf{\mathtt{\max_i(dp[k-1][i]+B[i][j])}}$, lo cual tiene un complejidad de $O(n)$.
                        \item LLenar toda la tabla  \textbf{\texttt{dp}} implica $O(m\cdot n^2)$
                      \end{itemize}
                    \item Cota:
                      \begin{itemize}[label=\textbullet]
                        \item El tiempo estimado es una \textbf{cota superior del peor caso} porque considera todas las combinaciones posibles de trabajos. 
                      \end{itemize}
                  \end{enumerate}
        \end{enumerate}
        
\item \lb{Resolver el siguiente problema con programación dinámica. Tenemos un conjunto de \textbf{\texttt{n}} objetos, cada uno con un peso $\mathbf{\mathtt{p=(p_1,p_2,...,p_n)}}$. El objetivo es repartir los objetos entre dos montones diferentes, de manera que queden lo más equilibrados posible en peso. Esto es, se debe minimizar la diferencia entre los pesos totales de ambos montones. Aplciar sobre el ejemplo con \textbf{\texttt{n=4}} y \textbf{\texttt{p=(2,1,3,4)}}.} 


\item \lb{Nos vamos de compras al mercado. Tenemos \textbf{\texttt{K}} euros en el bolsillo y una lista de \textbf{\texttt{m}} productos que podemos comprar. Cada producto tiene un precio $\mathbf{\mathtt{p_i}}$ (que será simpre un número entero), y una utilidad, $\mathbf{\mathtt{u_i}}$. De cada producto podemos comprar como máximo 3 unidades. Además, tenemos una oferta según según la cual la segunda unidad nos cuesta 1 euro menos, y la tercera 2 euros menos. Queremos elegir los productos a comprar, maximizando la utilidad de los productos comprados. Resolver el problema por programación dinámica, indicando la ecuación recurrene, con sus casos base, las tablas, el algoritmo para rellenarlas y la forma de componer la solución a partir de las tablas.}

\item \lb{Resolver el siguiente problema usando programación dinámica. Dada una secuencia de enteros positivos $\mathbf{\mathtt{(a_1,a_2,a_3,\dots,a_n)}}$, encontrar la subsecuencia creciente más larga de elementos no necesariamente consecutivos. Es decir, encontrar una subseuencia $\mathbf{\mathtt{(a_{i1}, a_{i2}, \dots,a_{ik})}}$, con $\mathbf{\mathtt{(a_{i1}< a_{i2}< \cdots< a_{ik})}}$ y $\mathbf{\mathtt{(1\le i1<i2<\cdots<ik\le n)}}$. Por ejemplo, para la siguiente secuencia la solución sería longitud 6 (formada por los números señalados en negrita): 3, \textbf{1}, 3, \textbf{2}, \textbf{3}, 8, \textbf{4}, 7, \textbf{5}, 4, \textbf{6}.}

\end{enumerate}
\end{document}

