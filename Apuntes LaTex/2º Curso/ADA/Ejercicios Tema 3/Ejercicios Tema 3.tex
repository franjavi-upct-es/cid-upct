\documentclass[12pt]{article}
\usepackage{fullpage}
\usepackage[utf8]{inputenc}
\usepackage{pict2e}
\usepackage{amsmath}
\usepackage{enumitem}
\usepackage{eurosym}
\usepackage{pict2e}
\usepackage{mathtools}
\usepackage{amssymb, amsfonts, latexsym, cancel}
\setlength{\parskip}{0.3cm}
\usepackage{graphicx}
\usepackage{fontenc}
\usepackage{slashbox}
\usepackage{setspace}
\usepackage{gensymb}
\usepackage{accents}
\usepackage{adjustbox}
\setstretch{1.5}
\usepackage{bold-extra}
\usepackage[document]{ragged2e}
\usepackage{subcaption}
\usepackage{tcolorbox}
\usepackage{xcolor, colortbl}
\usepackage{wrapfig}
\usepackage{empheq}
\usepackage{array}
\usepackage{parskip}
\usepackage{arydshln}
\graphicspath{ {images/} }
\renewcommand*\contentsname{\color{black}Índice} 
\usepackage{array, multirow, multicol}
\definecolor{lightblue}{HTML}{007AFF}
\usepackage{color}
\usepackage{etoolbox}
\usepackage{listings}
\usepackage{mdframed}
\setlength{\parindent}{0pt}
\usepackage{underscore}
\usepackage{hyperref}
\usepackage{tikz}
\usepackage{tikz-cd}
\usetikzlibrary{shapes, positioning, patterns}
\usepackage{tikz-qtree}
\usepackage{biblatex}
\usepackage{pdfpages}
\usepackage{pgfplots}
\usepackage{pgfkeys}
\addbibresource{biblatex-examples.bib}
\usepackage[a4paper, left=1.5cm, right=1.5cm, top=1cm,
bottom=1.5cm]{geometry}
\everymath{\displaystyle}
\usetikzlibrary{decorations.pathreplacing}
\usepackage{titlesec}
\usepackage{titletoc}
\usepackage{tikz-3dplot}
\usetikzlibrary{decorations.pathreplacing}
\newcommand{\Ej}{\textcolor{lightblue}{\underline{Ejemplo}}}
\setlength{\fboxrule}{1.5pt}
\renewcommand{\arraystretch}{1.35}
\setlength{\arraycolsep}{0.3cm}

% Configura el formato de las secciones utilizando titlesec
\titleformat{\section}
{\color{red}\normalfont\LARGE\bfseries}
{Tema \thesection:}
{10 pt}
{}

% Ajusta el formato de las entradas de la tabla de contenidos
\addtocontents{toc}{\protect\setcounter{tocdepth}{4}}
\addtocontents{toc}{\color{black}}

\titleformat{\subsection}
{\normalfont\Large\bfseries\color{red}}{\thesubsection)}{1em}{\color{lightblue}}

\titleformat{\subsubsection}
{\normalfont\large\bfseries\color{red}}{\thesubsubsection)}{1em}{\color{lightblue}}

\newcommand{\bboxed}[1]{\fcolorbox{lightblue}{lightblue!10}{$#1$}}

\DeclareMathOperator{\N}{\mathbb{N}}
\DeclareMathOperator{\Z}{\mathbb{Z}}
\DeclareMathOperator{\R}{\mathbb{R}}
\DeclareMathOperator{\Q}{\mathbb{Q}}
\DeclareMathOperator{\K}{\mathbb{K}}
\DeclareMathOperator{\im}{\imath}
\DeclareMathOperator{\jm}{\jmath}
\DeclareMathOperator{\col}{\mathrm{Col}}
\DeclareMathOperator{\fil}{\mathrm{Fil}}
\DeclareMathOperator{\rg}{\mathrm{rg}}
\DeclareMathOperator{\nuc}{\mathrm{nuc}}
\DeclareMathOperator{\dimf}{\mathrm{dimFil}}
\DeclareMathOperator{\dimc}{\mathrm{dimCol}}
\DeclareMathOperator{\dimn}{\mathrm{dimnuc}}
\DeclareMathOperator{\dimr}{\mathrm{dimrg}}

\newcommand{\bu}[1]{\textcolor{lightblue}{\underline{#1}}}
\newcommand{\lb}[1]{\textcolor{lightblue}{#1}}
\newcommand{\db}[1]{\textcolor{blue}{#1}}
\newcommand{\rc}[1]{\textcolor{red}{#1}}
\newcommand{\tr}{^\intercal}

\renewcommand{\CancelColor}{\color{lightblue}}

\newcommand{\dx}{\:\mathrm{d}x}
\newcommand{\dt}{\:\mathrm{d}t}
\newcommand{\dy}{\:\mathrm{d}y}
\newcommand{\dz}{\:\mathrm{d}z}
\newcommand{\dth}{\:\mathrm{d}\theta}
\newcommand{\dr}{\:\mathrm{d}\rho}
\newcommand{\du}{\:\mathrm{d}u}
\newcommand{\dv}{\:\mathrm{d}v}
\newcommand{\tozero}[1]{\cancelto{0}{#1}}
\newcommand{\lbb}[2]{\textcolor{lightblue}{\underbracket[1pt]{\textcolor{black}{#1}}_{#2}}}
\newcommand{\dbb}[2]{\textcolor{blue}{\underbracket[1pt]{\textcolor{black}{#1}}_{#2}}}
\lstset{
language=Python,
basicstyle=\ttfamily\small,
numberstyle=\tiny,
keywordstyle=\color{blue},
commentstyle=\color{olive},
stringstyle=\color{red},
breakatwhitespace=false, 
breaklines=true,
showspaces=false,
showstringspaces=false,
showtabs=false, 
tabsize=2,
literate={á}{{\'a}}1 {é}{{\'e}}1 {í}{{\'i}}1 {ó}{{\'o}}1 {ú}{{\'u}}1{ñ}{{\~n}}1 {Á}{{\'A}}1 {Í}{{\'I}}1,
mathescape=false,
backgroundcolor=\color{lightgray!10},
}

\begin{document}
  \begin{enumerate}[label=\color{red}\textbf{\arabic*)}]
    \item \lb{Diseñar una solución para el problema de colaración de grafos utilizando \textit{backtracking}. Se supondrá que el grafo consta de \textbf{\texttt{n}} nodos y que utilizamos una representación  con matriz de adyacencia \textbf{\texttt{A[1..n][1..n]}} de booleanos. En lugar de colores se asignarán etiquetas numéricas a los nodos $1,2,3\dots$ Una solución será representada como una tupla $S=(s_1,s_2,\dots,sn)$ donde $s_i$, que puede tomar valores  $1,2,3\dots$ representa el color asignado al nodo \textbf{\texttt{i}}.}

\begin{enumerate}[label=\arabic*)]
  \item Descripción del algoritmo

    El algoritmo explora todas las posibles asignaciones de colores a los nodos del grafo siguiendo estas reglas:
    \begin{itemize}[label=\textbullet]
      \item Se asigna un color a cada nodo en orden secuencial.
      \item Antes de asignar un color a un nodo, se verifica si este es \textbf{compatible} (es decir, no se asigna un color que ya esté presente en los nodos adyacentes).
      \item Si se alcanza una asignación completa y válida para todos los nodos, se reporta como solución.
    \end{itemize}
    El \textbf{backtracking} asegura que si se asigna un color no válido en algún caso, se retrocede y se intenta otro color.
  \item Representación de los datos
    \begin{enumerate}[label=\arabic*)]
      \item La matriz de adyacencia \textbf{\texttt{A[1..n][1..n] }} representará el grafo.
         \begin{itemize}[label=\textbullet]
           \item \textbf{\texttt{A[i][j]=True}} si hay una arista entre \textbf{\texttt{i}} y \textbf{\texttt{j}}, y \textbf{\texttt{False}} en caso contrario.
        \end{itemize}
      \item Un arreglo $S=[s_1,s_2,\dots,s_n]$ almacenará el color (número) asignado a cada nodo.
    \end{enumerate}
  \item Funciones clave del algoritmo
    \begin{enumerate}[label=\arabic*)]
      \item \textbf{\texttt{es\_valido}:} Verifica si asignar un color \textbf{\texttt{c}} al nodo \textbf{\texttt{v}} es válido (no está en conflicto con los nodos adyacentes).
      \item \textbf{\texttt{backtracking}:} Explora las asignaciones de colores a los nodos de manera recursiva.
      \item \textbf{\texttt{solucionar\_coloreo}:} Inicia el proceso y muestra el resultado final. 
    \end{enumerate}
  \item Implementación en Python
    \lstinputlisting{"Ejercicio 1.py"}
  \item Explicación del código
    \begin{enumerate}[label=\arabic*)]
      \item \textbf{\texttt{es\_valido}:}
        \begin{itemize}[label=\textbullet]
          \item Comprueba si asignar un color al nodo actual no entra en conflicto con los colores de los nodos adyacentes.
        \end{itemize}
      \item \textbf{\texttt{backtracking}:}
        \begin{itemize}[label=\textbullet]
          \item Recursivamente asigna colores a los nodos.
          \item Si encuentra una asignación válida para todos los nodos, termina.
          \item De lo contrario, realiza un \textbf{backtrack} (deshace una asignación y prueba otra opción).
        \end{itemize}
      \item \textbf{\texttt{solucionar\_coloreo}:}
        \begin{itemize}[label=\textbullet]
          \item Es la función principal que inicializa el proceso y muestra el resultado.
        \end{itemize}
    \end{enumerate}
  \item Ejemplo de salida

    Para la matriz de adyacencia del ejemplo y con un máximo de \textbf{3 colores}, la salida es:
    \begin{verbatim}
## Solución encontrada:
## [1,2,3,2]
    \end{verbatim}
  \item Complejidad Temporal
    La complejidad del algoritmo es $O(m^{n})$, donde $n$ es el número de nodos y  $m$ el número máximo de colores. Aunque esta complejidad es exponencial en el peor caso, el backtracking con poda reduce el tiempo considerablemente en la práctica.
\end{enumerate}

    \item \lb{Diseñar una solución para el problema del ciclo hamiltoniano utilizando \textit{backtracking}.}

      \begin{enumerate}[label=\arabic*)]
        \item Problema del Ciclo Hamiltoniano

          Un \textbf{ciclo hamiltoniano} es un recorrido que pasa \textbf{exactamente una vez} por todos los vértices de un grafo y regresa al vértice inicial. El objetivo del algoritmo es determinar si existe tal ciclo en un grafo dado y mostrar la secuencia de vértices que forma el ciclo.

        \item Implementación en Python

          \lstinputlisting{"Ejercicio 2.py"}

        \item Representación del grafo

          El grafo se representa mediante una \textbf{matriz de adyacencia}, donde:
          \begin{itemize}[label=\textbullet]
            \item \textbf{\texttt{grafo[i][j]=1}} significa que hay una arista entre los vértices \textbf{\texttt{i}} y \textbf{\texttt{j}}.
            \item \textbf{\texttt{grafo[i][j]=0}} indica que no hay aristas entre ellos.
          \end{itemize}
        \item Explicación de las funciones
          \begin{enumerate}[label=\arabic*)]
            \item \textbf{\texttt{es\_valido}:}
              \lstinputlisting[firstline=1, lastline=18]{"Ejercicio 2.py"}
              \begin{itemize}[label=\textbullet]
                \item \textbf{Propósito:} Verifica si es válido añadir un vértice \textbf{\texttt{v}} al camino actual.
                \item \textbf{Chequeos:}
                  \begin{enumerate}[label=\arabic*)]
                    \item Comprueba si existe una arista entre el último vértice del camino \textbf{\texttt{v}}.
                    \item Verifica que \textbf{\texttt{v}} no haya sido visitado antes.
                  \end{enumerate}
              \end{itemize}
            \item \textbf{\texttt{backtracking\_hamiltoniano}:}
            \lstinputlisting[firstline=20, lastline=48]{"Ejercicio 2.py"}
            \begin{itemize}[label=\textbullet]
              \item \textbf{Propósito:} Aplica \textbf{backtracking} para construir el camino hamiltoniano paso a paso.
              \item \textbf{Lógica:}
                \begin{enumerate}[label=\arabic*)]
                  \item Si se han añadido todos los vértices (\textbf{\texttt{pos==n}}), verifica si existe una arista que conecte el último vértice con el inicial.
                  \item Prueba todos los vértices desde \textbf{\texttt{1}} hasta \textbf{\texttt{n-1}}:
                    \begin{itemize}[label=\textbullet]
                      \item Si el vértice es válido (\textbf{\texttt{es\_valido}}), lo añade al camino y llama recursivamente.
                      \item Si no encuentra una solución, elimina el vértice (\textbf{backtrack}).
                    \end{itemize}
                \end{enumerate}
            \end{itemize}
          \item \textbf{\texttt{encontrar\_ciclo\_hamiltoniado}:}
            \lstinputlisting[firstline=50,lastline=64]{"Ejercicio 2.py"}
            \begin{itemize}[label=\textbullet]
              \item \textbf{Propósito:} Es la función principal que incializa el proceso.
              \item \textbf{Lógica:}
                \begin{enumerate}[label=\arabic*)]
                  \item El camino empieza desde el vértice $0$.
                  \item LLama a \textbf{\texttt{backtracking\_hamiltoniano}} para encontrar el ciclo.
                  \item Si se encuentra una solución, muestra el camino junto con el vértice inicial al cerrar el ciclo.
                \end{enumerate}
            \end{itemize}
          \end{enumerate}
        \item Ejemplo de uso

          Para el siguiente grafo de 5 vértices:
          \lstinputlisting[firstline=67, lastline=75]{"Ejercicio 2.py"}
          \begin{itemize}[label=\textbullet]
            \item El grafo tiene las siguientes aristas:
              \begin{itemize}[label=\textbullet]
                \item $0\longleftrightarrow 1, 0 \longleftrightarrow 3$
                \item $0\longleftrightarrow 2, 1\longleftrightarrow 3, 1\longleftrightarrow 4$
                \item $2\longleftrightarrow 4$
                \item $3\longleftrightarrow 4$
              \end{itemize}
          \end{itemize}
          Salida del programa
          \begin{verbatim}
## Ciclo Hamiltoniano encontrado:
## [0, 1, 2, 4, 3, 0]
          \end{verbatim}
          El ciclo recorre los vértice en el orden: $0\to 1\to 2\to 4\to 3\to 0$, cumpliendo las condiciones del ciclo hamiltoniano.
        \item Complejidad del algoritmo

          \begin{itemize}[label=\textbullet]
            \item \textbf{Tiempo:} $O(n!)$, ya que prueban todas las permutaciones posibles de los  $n$ vértices.
          \end{itemize}
      \end{enumerate}

    \item \lb{En un matriz cuadrada \textbf{\texttt{M}}, de tamaño $n\times n$ representamos un laberinto. Partimos de la posición $(1,1)$ y el objetivo es moverse en la posición  $(n,n)$. Podemos pasar por la casilla  $(i,j)$ si y sólo si  \textbf{\texttt{M[i,j]=A}} (abierta). Si \textbf{\texttt{M[i,j]=C}} (cerrada) entonces no podemos pasar por esa casilla. Desde cada casilla existen 4 posibles movimientos: arriba, abajo, izquierda y derecha.
      \begin{center}
        \begin{tabular}{|c|c|c|c|}
          \hline
          A & C & A & A \\ \hline
          A & A & A & C \\ \hline
          A & C & A & A \\ \hline
          C & A & A & A \\ \hline
        \end{tabular}
      \end{center}
Describe la forma de resolver el problema utilizando \textit{backtracking}. Pista: ten en cuenta que en cada movimiento sólo necesitamos conocer la posición actual.}

\begin{enumerate}[label=\arabic*)]
  \item Descripción del problema

    Dado un \textbf{laberinto} representado por una matriz cuadrada \textbf{\texttt{M}} de tamaño $n\times n$:
    \begin{itemize}[label=\textbullet]
      \item Cada celda puede estar \textbf{abierta} (\textbf{\texttt{A}}) o \textbf{cerrada} (\textbf{\texttt{C}}).
      \item Se parte de la celda $(0,0)$ (esquina superior izquierda).
      \item El objetivo es llega a la celda  \textbf{\texttt{(n-1,n-1)}} (esquina inferior derecha) moviéndose en las \textbf{4 direcciones posibles}:
        \begin{itemize}[label=\textbullet]
          \item \textbf{Abajo:} (\textbf{\texttt{x+1,y}} ) 
          \item \textbf{Derecha: \texttt{(x,y+1)}} 
          \item \textbf{Arriba: \texttt{(x-1,y)}} 
          \item \textbf{Izquierda: \texttt{(x,y-1)}} 
        \end{itemize}
    \end{itemize}
  \item Implementación en Python

    \lstinputlisting{"Ejercicio 3.py"}
  \item Estrategia del algoritmo
    \begin{enumerate}[label=\arabic*)]
      \item Empezamos desde la posición inicial $(0,0)$ y seguimos probando movimientos válidos.
      \item En cada celda, realizamos los siguientes pasos:
         \begin{itemize}[label=\textbullet]
          \item Marcamos la celda como \textbf{visitada}.
          \item Intentamos movernos a una de las \textbf{4 direcciones} posibles.
          \item Si llegamos a la celda destino \textbf{\texttt{(n-1, n-1)}}, hemos encontrado un camino válido.
        \end{itemize}
      \item Si un movimiento no lleva a un solución (camino sin salida), retrocedemos (\textbf{backtrack}) y probamos con otra dirección.
      \item El proceso se repite hasta que se encuentra un camino válido o se verifica que no hay solución.
    \end{enumerate}
  \item Explicación de las funciones
    \begin{enumerate}[label=\arabic*)]
      \item \textbf{\texttt{es\_valido}:}
        \lstinputlisting[firstline=1, lastline=9]{"Ejercicio 3.py"}
        \begin{itemize}[label=\textbullet]
          \item \textbf{Propósito:} Verificar si una celda \textbf{\texttt{(x, y)}}  es válida para moverse.
          \item \textbf{Condiciones:}
            \begin{enumerate}[label=\arabic*)]
              \item La celda está dentro de los límites de la matriz.
              \item La celda está \textbf{abierta \texttt{(M[x][y] == "A")}}.
              \item La celda \textbf{no ha sido visitada} aún. 
            \end{enumerate}
        \end{itemize}
      \item \textbf{\texttt{backtracking\_laberinto}:}
        \lstinputlisting[firstline=11, lastline=42]{"Ejercicio 3.py"}
        \begin{itemize}[label=\textbullet]
          \item \textbf{Lógica:}
            \begin{enumerate}[label=\arabic*)]
              \item Si estamos en la posición final \textbf{\texttt{(n-1, n-1)}}, añadimos la celda al camino y retornamos \textbf{\texttt{True}}.
              \item Marcamos la celda actual como visitada y la añadimos al camino.
              \item Probamos los \textbf{4 movimientos posibles}:
                \begin{itemize}[label=\textbullet]
                  \item Si el movimiento es válido, hacemos una llamada recursiva.
                  \item Si encontramos un camino válido, retornamos \textbf{\texttt{True}}. 
                \end{itemize}
              \item Si ningún movimiento es válido, \textbf{retrocedemos}:
                \begin{itemize}[label=\textbullet]
                  \item Desmarcamos la celda como visitada.
                  \item Eliminamos la celda del camino.
                \end{itemize}
            \end{enumerate}
        \end{itemize}
      \item \textbf{\texttt{resolver\_laberinto}:}
        \lstinputlisting[firstline=44, lastline=58]{"Ejercicio 3.py"}
        \begin{itemize}[label=\textbullet]
          \item \textbf{Propósito:} Inicia el proceso resolviendo el laberinto desde $(0,0)$.
          \item \textbf{Resultado:}
            \begin{itemize}[label=\textbullet]
              \item Si se encuentra un camino válido, se imprime el camino.
              \item Si no hay solución, se muestra un mensaje de error.
            \end{itemize}
        \end{itemize}
    \end{enumerate}
  \item Ejemplo de uso:

    \textbf{Entrada:}
    \lstinputlisting[firstline=62, lastline=67]{"Ejercicio 3.py"}
    \textbf{Salida:}
    \begin{verbatim}
## Camino encontrado:
## (0, 0)
## (1, 0)
## (1, 1)
## (1, 2)
## (2, 2)
## (3, 2)
## (3, 3)
    \end{verbatim}
  \item \textbf{Tiempo:} $O(4^{n^2})$ en el peor caso, porque en cada celda probamos 4 direcciones posibles. 
\end{enumerate}
\item \lb{En una liga de fútbol participan \textbf{\texttt{n}} equipos (suponemos que \textbf{\texttt{n}} es par). En cada jornada se juegan $\mathbf{\mathtt{\dfrac{n}{2}}}$ partidos, que enfrentan a dos equipos, dirigidos por un árbitro. Existen \textbf{\texttt{m}} árbitros disponibles, siendo $m>\dfrac{n}{2}$. Cada equipo \textbf{\texttt{i}} valora cada árbitro \textbf{\texttt{j}} con una puntuación \textbf{\texttt{P[i,j]}} entre $0$ y $10$, indicando su preferencia por ese árbitro. un valor alto indica que le gusta el árbitro y un valor bajo que no le gusta. Si el árbitro y el equipo son de la misma región entonces \textbf{\texttt{P[i,j]=$-\infty$}}.\newline
  El objetivo es (para cada jornada contreta) asignar un árbitro distinto a cada partido, de manera que se maximice la puntuación total de los árbitros asignados, teniendo en cuenta las preferencias de todos los equipos.\newline
  Dar una solución óptima para el problema usando \textit{backtracking}. Exponer cómo es la representación de la solución, cuál es el esquema que habría que utilizar, cuál la condición de fin y cómo son las soluciones del esquema (\textbf{\texttt{Generar, MasHermanos, Criterio, Solucion\dots}}). 
  } 
  \begin{enumerate}[label=\arabic*)]
    \item Representación del problema
      \begin{itemize}[label=\textbullet]
        \item \textbf{\texttt{n}:} Número de equipo (par).
        \item \textbf{\texttt{m}:} Número de árbitros disponibles $\mathbf{\mathtt{\left( m>\dfrac{n}{2} \right) }}$.
        \item \textbf{\texttt{P[i][j]}:} Matriz $n\times m$ que representa la puntuación de preferencia del equipo \textbf{\texttt{i}} por el árbitro \textbf{\texttt{j}}:
          \begin{itemize}[label=\textbullet]
            \item \textbf{\texttt{P[i][j]}} alto: el equipo \textbf{\texttt{i}} prefiere al árbitro \textbf{\texttt{j}}.
            \item \textbf{\texttt{P[i][j]=$-\infty$}:} el árbitro \textbf{\texttt{j}} no puede arbitrar al equipo \textbf{\texttt{j}} (misma región). 
          \end{itemize}
      \end{itemize}
    \item Representación de la solución
      \begin{itemize}[label=\textbullet]
        \item La solución se representa como una \textbf{tupla}:
          \[
          S=(a_1,a_2,\dots,a_{k})
          \] 
          donde $a_k$ es el árbitro asignado al partido  $k$, y  $k=\dfrac{n}{2}$ es el número total de partidos por jornada.
      \end{itemize}
    \item Esquema de Backtracking
      \begin{itemize}[label=\textbullet]
        \item \textbf{\texttt{Generar}:}
          \begin{itemize}[label=\textbullet]
            \item En la posición $k$-ésima (el partido $k$), se intenta asignar un de los \textbf{\texttt{m} árbitros} disponibles que no haya sido asignado aún.
          \end{itemize}
        \item \textbf{\texttt{MasHermanos}:}
          \begin{itemize}[label=\textbullet]
            \item Un árbitro \textbf{\texttt{j}} tiene \textbf{más hermanos} si existe otro árbitro candidato válido que no haya sido asignado.
          \end{itemize}
        \item \textbf{\texttt{Criterio}:}
          \begin{itemize}[label=\textbullet]
            \item Se verifica que:
              \begin{enumerate}[label=\arabic*)]
                \item El árbitro \textbf{\texttt{j}} no haya sido asignado a ningún partido anterior.
                \item El árbitro \textbf{\texttt{j}} no tiene preferencia $-\infty$ con los equipos que juegan en el partido \textbf{\texttt{k}}.
              \end{enumerate}
          \end{itemize}
        \item \textbf{\texttt{Solucion}:}
          \begin{itemize}[label=\textbullet]
            \item La asignación se completa cuando se asignan árbitros a los \textbf{$\mathbf{\dfrac{n}{2}}$ partidos}. 
          \end{itemize}
        \item Función objetivo:
          \begin{itemize}[label=\textbullet]
            \item El objetivo es \textbf{maximizar la suma de puntuaciones} de los árbitros asignados según las preferencias de los equipos.
          \end{itemize}
      \end{itemize}
    \item Condición de fin
      \begin{itemize}[label=\textbullet]
        \item La condición de fin se alcanza cuando:
          \begin{itemize}[label=\textbullet]
            \item Todos los partidos $k=1,\dots,\dfrac{n}{2}$ tienen asignado un árbitro válido.
          \end{itemize}
      \end{itemize}
    \item Implementación del algoritmo
      \lstinputlisting{"Ejercicio 4.py"}
    \item Explicación del algoritmo
      \begin{enumerate}[label=\arabic*)]
        \item Entrada:
          \begin{itemize}[label=\textbullet]
            \item \textbf{\texttt{P}:} Matriz de puntuaciones con preferencias de cada equipo para cada árbitro.
            \item \textbf{\texttt{partidos}:} Lista con los equipos enfrenteados en cada partido. 
          \end{itemize}
        \item Backtracking:
          \begin{itemize}[label=\textbullet]
            \item El algoritmo intenta asignar árbitros a cada partido de manera secuencial.
            \item Verifica si un árbitro es válido con la función \textbf{\texttt{es\_valido}}:
              \begin{itemize}[label=\textbullet]
                \item No debe estar asignado aún.
                \item La puntuación no debe ser $-\infty$ para ninguno de los equipos en el partido.
              \end{itemize}
            \item Se actualiza con la puntuación total conforme se asignan árbitros.
          \end{itemize}
        \item Condición de Fin:
          \begin{itemize}[label=\textbullet]
            \item Si todos los partidos tienen asignado un árbitro, se compara la puntuación total obtenida con la máxima registrada.
          \end{itemize}
        \item Solución Óptima:
          \begin{itemize}[label=\textbullet]
            \item  Se guarda la asignación que maximiza la puntuación total.
          \end{itemize}
      \end{enumerate}
    \item Ejemplo de salida

      Para la matriz de preferencias y los partidos dados:
      \begin{verbatim}
## Mejor puntuación total: 29
## Asignación de árbitros a partidos: [0, 1]
      \end{verbatim}
  \end{enumerate}
  \end{enumerate}
\end{document}
