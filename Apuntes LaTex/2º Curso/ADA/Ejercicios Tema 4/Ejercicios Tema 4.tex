\documentclass[12pt]{article}
\usepackage{fullpage}
\usepackage[utf8]{inputenc}
\usepackage{pict2e}
\usepackage{amsmath}
\usepackage{enumitem}
\usepackage{eurosym}
\usepackage{pict2e}
\usepackage{mathtools}
\usepackage{amssymb, amsfonts, latexsym, cancel}
\setlength{\parskip}{0.3cm}
\usepackage{graphicx}
\usepackage{fontenc}
\usepackage{slashbox}
\usepackage{setspace}
\usepackage{gensymb}
\usepackage{accents}
\usepackage{adjustbox}
\setstretch{1.5}
\usepackage{bold-extra}
\usepackage[document]{ragged2e}
\usepackage{subcaption}
\usepackage{tcolorbox}
\usepackage{xcolor, colortbl}
\usepackage{wrapfig}
\usepackage{empheq}
\usepackage{array}
\usepackage{parskip}
\usepackage{arydshln}
\graphicspath{ {images/} }
\renewcommand*\contentsname{\color{black}Índice} 
\usepackage{array, multirow, multicol}
\definecolor{lightblue}{HTML}{007AFF}
\usepackage{color}
\usepackage{etoolbox}
\usepackage{listings}
\usepackage{mdframed}
\setlength{\parindent}{0pt}
\usepackage{underscore}
\usepackage{hyperref}
\usepackage{tikz}
\usepackage{tikz-cd}
\usetikzlibrary{shapes, positioning, patterns}
\usepackage{tikz-qtree}
\usepackage{biblatex}
\usepackage{pdfpages}
\usepackage{pgfplots}
\usepackage{pgfkeys}
\addbibresource{biblatex-examples.bib}
\usepackage[a4paper, left=1.5cm, right=1.5cm, top=1cm,
bottom=1.5cm]{geometry}
\everymath{\displaystyle}
\usetikzlibrary{decorations.pathreplacing}
\usepackage{titlesec}
\usepackage{titletoc}
\usepackage{tikz-3dplot}
\usetikzlibrary{decorations.pathreplacing}
\newcommand{\Ej}{\textcolor{lightblue}{\underline{Ejemplo}}}
\setlength{\fboxrule}{1.5pt}
\renewcommand{\arraystretch}{1.35}
\setlength{\arraycolsep}{0.3cm}

% Configura el formato de las secciones utilizando titlesec
\titleformat{\section}
{\color{red}\normalfont\LARGE\bfseries}
{Tema \thesection:}
{10 pt}
{}

% Ajusta el formato de las entradas de la tabla de contenidos
\addtocontents{toc}{\protect\setcounter{tocdepth}{4}}
\addtocontents{toc}{\color{black}}

\titleformat{\subsection}
{\normalfont\Large\bfseries\color{red}}{\thesubsection)}{1em}{\color{lightblue}}

\titleformat{\subsubsection}
{\normalfont\large\bfseries\color{red}}{\thesubsubsection)}{1em}{\color{lightblue}}

\newcommand{\bboxed}[1]{\fcolorbox{lightblue}{lightblue!10}{$#1$}}

\DeclareMathOperator{\N}{\mathbb{N}}
\DeclareMathOperator{\Z}{\mathbb{Z}}
\DeclareMathOperator{\R}{\mathbb{R}}
\DeclareMathOperator{\Q}{\mathbb{Q}}
\DeclareMathOperator{\K}{\mathbb{K}}
\DeclareMathOperator{\im}{\imath}
\DeclareMathOperator{\jm}{\jmath}
\DeclareMathOperator{\col}{\mathrm{Col}}
\DeclareMathOperator{\fil}{\mathrm{Fil}}
\DeclareMathOperator{\rg}{\mathrm{rg}}
\DeclareMathOperator{\nuc}{\mathrm{nuc}}
\DeclareMathOperator{\dimf}{\mathrm{dimFil}}
\DeclareMathOperator{\dimc}{\mathrm{dimCol}}
\DeclareMathOperator{\dimn}{\mathrm{dimnuc}}
\DeclareMathOperator{\dimr}{\mathrm{dimrg}}

\newcommand{\bu}[1]{\textcolor{lightblue}{\underline{#1}}}
\newcommand{\lb}[1]{\textcolor{lightblue}{#1}}
\newcommand{\db}[1]{\textcolor{blue}{#1}}
\newcommand{\rc}[1]{\textcolor{red}{#1}}
\newcommand{\tr}{^\intercal}

\renewcommand{\CancelColor}{\color{lightblue}}

\newcommand{\dx}{\:\mathrm{d}x}
\newcommand{\dt}{\:\mathrm{d}t}
\newcommand{\dy}{\:\mathrm{d}y}
\newcommand{\dz}{\:\mathrm{d}z}
\newcommand{\dth}{\:\mathrm{d}\theta}
\newcommand{\dr}{\:\mathrm{d}\rho}
\newcommand{\du}{\:\mathrm{d}u}
\newcommand{\dv}{\:\mathrm{d}v}
\newcommand{\tozero}[1]{\cancelto{0}{#1}}
\newcommand{\lbb}[2]{\textcolor{lightblue}{\underbracket[1pt]{\textcolor{black}{#1}}_{#2}}}
\newcommand{\dbb}[2]{\textcolor{blue}{\underbracket[1pt]{\textcolor{black}{#1}}_{#2}}}

\lstset{
language=Python,
basicstyle=\ttfamily\small,
numberstyle=\tiny,
keywordstyle=\color{blue},
commentstyle=\color{olive},
stringstyle=\color{red},
breakatwhitespace=false, 
breaklines=true,
showspaces=false,
showstringspaces=false,
showtabs=false, 
tabsize=2,
literate={á}{{\'a}}1 {é}{{\'e}}1 {í}{{\'i}}1 {ó}{{\'o}}1 {ú}{{\'u}}1{ñ}{{\~n}}1 {Á}{{\'A}}1 {Í}{{\'I}}1,
mathescape=false,
backgroundcolor=\color{lightgray!10},
}
\title{Análisis y Diseño de Algoritmos\\Ejercicios Tema 4}

\begin{document}
\maketitle
\begin{enumerate}[label=\color{red}\textbf{\arabic*)}]
  \item \lb{Diseñar un algoritmo para calcular el mayor y el segundo mayor elemento de un array de enteros utilizando la técnica divide y venceras.\newline
    Calcular el número de comparaciones realizadas en el peor y el mejor caso suponenido \textbf{\texttt{n}} potencia de 2.\newline
  ¿Sería el orden obtenido extrapolable a un \textbf{\texttt{n}} que no sea potencia de 2?} 
  \begin{enumerate}[label=\arabic*)]
        \item  Algoritmo de Divide y Vencerás

          La idea principal es dividir el problema en subproblemas más pequeños, resolverlos y luego cominar las soluciones.

          \textbf{Esquema General:}
          \begin{enumerate}[label=\arabic*)]
            \item División:
              \begin{itemize}[label=\textbullet]
                \item Divide el arreglo en dos mitades iguales.
              \end{itemize}
            \item Conquista:
              \begin{itemize}[label=\textbullet]
                \item Encuentra el mayor y el segundo mayor en cada mitad de forma recursiva.
              \end{itemize}
              \begin{itemize}[label=\textbullet]
                \item Combinación:
                  \begin{itemize}[label=\textbullet]
                    \item Combina las soluciones de las dos mitades:
                      \begin{itemize}[label=\textbullet]
                        \item El mayor de las dos mitades es el mayor global.
                        \item El segundo mayor será el mayor de:
                          \begin{itemize}[label=\textbullet]
                            \item El segundo mayor de la mitad que contiene el mayor global.
                            \item El mayor de la otra mitad
                          \end{itemize}
                      \end{itemize}
                  \end{itemize}
              \end{itemize}
          \end{enumerate}
              \textbf{Implementación} 
    \lstinputlisting{"Ejercicio 1.py"}
  \item Análisis del número de comparaciones

    \textbf{Peor caso y mejor caso}

    El número de comparaciones es el mismo en el peor y el mejor caso, ya que cada división implica el mismo número de subproblemas y combinaciones.
    \begin{enumerate}[label=\arabic*)]
      \item Caso base:
        \begin{itemize}[label=\textbullet]
          \item Para $n=2$: Se realiza una única comparación.
        \end{itemize}
      \item Generalización para $n=2^{k}$:
        \begin{itemize}[label=\textbullet]
          \item Hay $n-1$ comparaciones para encontrar el mayor (esto se debe a que cada elemento pierde contra el mayor exactamente una vez).
          \item Hay $\log_2(n)-1$ comparaciones adicionales para encontrar el segundo mayor, porque el segundo mayor es el que pierde contra el mayor en el "torneo".

            \textbf{Total de comparaciones:} \[
            T(n)=(n-1)(log_2(n)-1)=n\log_2(n)-2.
            \]  
        \end{itemize}
    \end{enumerate}
  \item Orden para $n$ no potencia de 2

    Cuando  $n$ no es potencia de 2, el algoritmo sigue siendo aplicable:
     \begin{enumerate}[label=\arabic*)]
      \item Se rellena el arreglo con elementos adiciones $(-\infty)$ para completar una potencia de 2.
      \item Las comparaciones adicionales no afectan significativamente el orden del algoritmo, ya que $T(n)=O(n+\log_2(n))$ sigue siendo válido.
    \end{enumerate}
  \end{enumerate}
\item 
\end{enumerate}
\end{document}
