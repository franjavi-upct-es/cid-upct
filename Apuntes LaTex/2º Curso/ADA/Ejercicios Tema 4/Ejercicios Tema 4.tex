\documentclass[12pt]{article}
\usepackage{fullpage}
\usepackage[utf8]{inputenc}
\usepackage{pict2e}
\usepackage{amsmath}
\usepackage{enumitem}
\usepackage{eurosym}
\usepackage{pict2e}
\usepackage{mathtools}
\usepackage{amssymb, amsfonts, latexsym, cancel}
\setlength{\parskip}{0.3cm}
\usepackage{graphicx}
\usepackage{fontenc}
\usepackage{slashbox}
\usepackage{setspace}
\usepackage{gensymb}
\usepackage{accents}
\usepackage{adjustbox}
\setstretch{1.5}
\usepackage{bold-extra}
\usepackage[document]{ragged2e}
\usepackage{subcaption}
\usepackage{tcolorbox}
\usepackage{xcolor, colortbl}
\usepackage{wrapfig}
\usepackage{empheq}
\usepackage{array}
\usepackage{parskip}
\usepackage{arydshln}
\graphicspath{ {images/} }
\renewcommand*\contentsname{\color{black}Índice} 
\usepackage{array, multirow, multicol}
\definecolor{lightblue}{HTML}{007AFF}
\usepackage{color}
\usepackage{etoolbox}
\usepackage{listings}
\usepackage{mdframed}
\setlength{\parindent}{0pt}
\usepackage{underscore}
\usepackage{hyperref}
\usepackage{tikz}
\usepackage{tikz-cd}
\usetikzlibrary{shapes, positioning, patterns}
\usepackage{tikz-qtree}
\usepackage{biblatex}
\usepackage{pdfpages}
\usepackage{pgfplots}
\usepackage{pgfkeys}
\addbibresource{biblatex-examples.bib}
\usepackage[a4paper, left=1.5cm, right=1.5cm, top=1cm,
bottom=1.5cm]{geometry}
\everymath{\displaystyle}
\usetikzlibrary{decorations.pathreplacing}
\usepackage{titlesec}
\usepackage{titletoc}
\usepackage{tikz-3dplot}
\usetikzlibrary{decorations.pathreplacing}
\newcommand{\Ej}{\textcolor{lightblue}{\underline{Ejemplo}}}
\setlength{\fboxrule}{1.5pt}
\renewcommand{\arraystretch}{1.35}
\setlength{\arraycolsep}{0.3cm}

% Configura el formato de las secciones utilizando titlesec
\titleformat{\section}
{\color{red}\normalfont\LARGE\bfseries}
{Tema \thesection:}
{10 pt}
{}

% Ajusta el formato de las entradas de la tabla de contenidos
\addtocontents{toc}{\protect\setcounter{tocdepth}{4}}
\addtocontents{toc}{\color{black}}

\titleformat{\subsection}
{\normalfont\Large\bfseries\color{red}}{\thesubsection)}{1em}{\color{lightblue}}

\titleformat{\subsubsection}
{\normalfont\large\bfseries\color{red}}{\thesubsubsection)}{1em}{\color{lightblue}}

\newcommand{\bboxed}[1]{\fcolorbox{lightblue}{lightblue!10}{$#1$}}

\DeclareMathOperator{\N}{\mathbb{N}}
\DeclareMathOperator{\Z}{\mathbb{Z}}
\DeclareMathOperator{\R}{\mathbb{R}}
\DeclareMathOperator{\Q}{\mathbb{Q}}
\DeclareMathOperator{\K}{\mathbb{K}}
\DeclareMathOperator{\im}{\imath}
\DeclareMathOperator{\jm}{\jmath}
\DeclareMathOperator{\col}{\mathrm{Col}}
\DeclareMathOperator{\fil}{\mathrm{Fil}}
\DeclareMathOperator{\rg}{\mathrm{rg}}
\DeclareMathOperator{\nuc}{\mathrm{nuc}}
\DeclareMathOperator{\dimf}{\mathrm{dimFil}}
\DeclareMathOperator{\dimc}{\mathrm{dimCol}}
\DeclareMathOperator{\dimn}{\mathrm{dimnuc}}
\DeclareMathOperator{\dimr}{\mathrm{dimrg}}

\newcommand{\bu}[1]{\textcolor{lightblue}{\underline{#1}}}
\newcommand{\lb}[1]{\textcolor{lightblue}{#1}}
\newcommand{\db}[1]{\textcolor{blue}{#1}}
\newcommand{\rc}[1]{\textcolor{red}{#1}}
\newcommand{\tr}{^\intercal}

\renewcommand{\CancelColor}{\color{lightblue}}

\newcommand{\dx}{\:\mathrm{d}x}
\newcommand{\dt}{\:\mathrm{d}t}
\newcommand{\dy}{\:\mathrm{d}y}
\newcommand{\dz}{\:\mathrm{d}z}
\newcommand{\dth}{\:\mathrm{d}\theta}
\newcommand{\dr}{\:\mathrm{d}\rho}
\newcommand{\du}{\:\mathrm{d}u}
\newcommand{\dv}{\:\mathrm{d}v}
\newcommand{\tozero}[1]{\cancelto{0}{#1}}
\newcommand{\lbb}[2]{\textcolor{lightblue}{\underbracket[1pt]{\textcolor{black}{#1}}_{#2}}}
\newcommand{\dbb}[2]{\textcolor{blue}{\underbracket[1pt]{\textcolor{black}{#1}}_{#2}}}
\setlistdepth{9}
\lstset{
language=Python,
basicstyle=\ttfamily\small,
numberstyle=\tiny,
keywordstyle=\color{blue},
commentstyle=\color{olive},
stringstyle=\color{red},
breakatwhitespace=false, 
breaklines=true,
showspaces=false,
showstringspaces=false,
showtabs=false, 
tabsize=2,
literate={á}{{\'a}}1 {é}{{\'e}}1 {í}{{\'i}}1 {ó}{{\'o}}1 {ú}{{\'u}}1{ñ}{{\~n}}1 {Á}{{\'A}}1 {Í}{{\'I}}1,
mathescape=false,
backgroundcolor=\color{lightgray!10},
}
\title{Análisis y Diseño de Algoritmos\\Ejercicios Tema 4}

\begin{document}
\maketitle
\begin{enumerate}[label=\color{red}\textbf{\arabic*)}]
  \item \lb{Diseñar un algoritmo para calcular el mayor y el segundo mayor elemento de un array de enteros utilizando la técnica divide y venceras.\newline
    Calcular el número de comparaciones realizadas en el peor y el mejor caso suponenido \textbf{\texttt{n}} potencia de 2.\newline
  ¿Sería el orden obtenido extrapolable a un \textbf{\texttt{n}} que no sea potencia de 2?} 
  \begin{enumerate}[label=\arabic*)]
        \item  Algoritmo de Divide y Vencerás

          La idea principal es dividir el problema en subproblemas más pequeños, resolverlos y luego cominar las soluciones.

          \textbf{Esquema General:}
          \begin{enumerate}[label=\arabic*)]
            \item División:
              \begin{itemize}[label=\textbullet]
                \item Divide el arreglo en dos mitades iguales.
              \end{itemize}
            \item Conquista:
              \begin{itemize}[label=\textbullet]
                \item Encuentra el mayor y el segundo mayor en cada mitad de forma recursiva.
              \end{itemize}
              \begin{itemize}[label=\textbullet]
                \item Combinación:
                  \begin{itemize}[label=\textbullet]
                    \item Combina las soluciones de las dos mitades:
                      \begin{itemize}[label=\textbullet]
                        \item El mayor de las dos mitades es el mayor global.
                        \item El segundo mayor será el mayor de:
                          \begin{itemize}[label=\textbullet]
                            \item El segundo mayor de la mitad que contiene el mayor global.
                            \item El mayor de la otra mitad
                          \end{itemize}
                      \end{itemize}
                  \end{itemize}
              \end{itemize}
          \end{enumerate}
              \textbf{Implementación} 
    \lstinputlisting{"Ejercicio 1.py"}
  \item Análisis del número de comparaciones

    \textbf{Peor caso y mejor caso}

    El número de comparaciones es el mismo en el peor y el mejor caso, ya que cada división implica el mismo número de subproblemas y combinaciones.
    \begin{enumerate}[label=\arabic*)]
      \item Caso base:
        \begin{itemize}[label=\textbullet]
          \item Para $n=2$: Se realiza una única comparación.
        \end{itemize}
      \item Generalización para $n=2^{k}$:
        \begin{itemize}[label=\textbullet]
          \item Hay $n-1$ comparaciones para encontrar el mayor (esto se debe a que cada elemento pierde contra el mayor exactamente una vez).
          \item Hay $\log_2(n)-1$ comparaciones adicionales para encontrar el segundo mayor, porque el segundo mayor es el que pierde contra el mayor en el "torneo".

            \textbf{Total de comparaciones:} \[
            T(n)=(n-1)(log_2(n)-1)=n\log_2(n)-2.
            \]  
        \end{itemize}
    \end{enumerate}
  \item Orden para $n$ no potencia de 2

    Cuando  $n$ no es potencia de 2, el algoritmo sigue siendo aplicable:
     \begin{enumerate}[label=\arabic*)]
      \item Se rellena el arreglo con elementos adiciones $(-\infty)$ para completar una potencia de 2.
      \item Las comparaciones adicionales no afectan significativamente el orden del algoritmo, ya que $T(n)=O(n+\log_2(n))$ sigue siendo válido.
    \end{enumerate}
  \end{enumerate}
\item \lb{Sea \textbf{\texttt{T[1..n]}} un array ordenado pormado por eventos diferentes, algunos de los cuales pueden ser negativos. Dar un algoritmo DyV que pueda hallar \text    bf{\texttt{i}} tal que $1\le i\le n$ y \textbf{\texttt{T[i]=1}}, siempre que este índice exista. El algoritmo debe tener un $O(log n)$.} 

  El problema es encontrar un índice \textbf{\texttt{i}} tal que \textbf{\texttt{T[i]=1}} en un arreglo ordenado \textbf{\texttt{T}} tamaño $n$, donde los elementos son únicos y pueden incluir números negativos. Este arreglo está ordeado, lo que permite usar una búsqeda binaria para alcanzar una complejidad  $O(\log n)$.

  \textbf{Algoritmo} 

  Usaremos el enfoque de \textbf{Divide y Vencerás} a través de una implementaación de búsqueda binaria para encontrar el índice donde \textbf{\texttt{T[i]=1}}.

  \textbf{Esquema} 
  \begin{enumerate}[label=\arabic*)]
    \item División:
      \begin{itemize}[label=\textbullet]
        \item Divide el arreglo en dos mitades en cada paso, seleccionando el elemento medio.
      \end{itemize}
    \item Conquista:
      \begin{itemize}[label=\textbullet]
        \item Si \textbf{\texttt{T[mid]=1}}, se ha encontrado el índice esperado. 
        \item Si \textbf{\texttt{T[mid]>1}}, busca en la mitad izquierda del arreglo. 
        \item Si \textbf{\texttt{T[mid]<1}}, busca en la mitad derecha del arreglo. 
      \end{itemize}
    \item Combinación:
      \begin{itemize}[label=\textbullet]
        \item En este caso, no es necesario combinar resultados ya que solo buscamos uun único índice.
      \end{itemize}
      \textbf{Implementación}

      \lstinputlisting{"Ejercicio 2.py"}

      \textbf{Explicación del código}

      \begin{enumerate}[label=\arabic*)]
        \item Caso base:
          \begin{itemize}[label=\textbullet]
            \item Si el inventario de búsqeuda es inválido \textbf{\texttt{(inicio > fin)}}, significa que $1$ no está en el arreglo, y retornamos $-1$.
          \end{itemize}
        \item Cállculo del índice medio:
          \begin{itemize}[label=\textbullet]
            \item Dividimos el arreglo por la mitad \textbf{\texttt{(mid = (inicio + fin)//2)}}.
          \end{itemize}
        \item Condiciones:
          \begin{itemize}[label=\textbullet]
            \item Si \textbf{\texttt{T[mid]=1}}, hemos encontrado el índice deseado y lo retornamos.
            \item Si \textbf{\texttt{T[mid]>1}}, buscamos en la mitad izquierda \textbf{\texttt{(inicio} a \texttt{mid-1)}}.
            \item Si \textbf{\texttt{T[mid]<1}}, vuscamos en la mitad derecha \textbf{\texttt{(mid+1} a \texttt{fin)}}.  
          \end{itemize}
        \item Llamada inicial:
          \begin{itemize}[label=\textbullet]
            \item La función \textbf{\texttt{buscar\_uno}} inicia la búsqueda con los límites completos del arreglo. 
          \end{itemize}
      \end{enumerate}
      \textbf{Análisis de complejidad}

      El algoritmo realiza $\log(n)$ comparaciones en el peor caso, ya que el tamaño del arreglo se reduce a la mitad en cada llamada recursiva.

      \textbf{Ejemplo de salida}

      Dado el arreglo \textbf{\texttt{T=[-10, -5, -2, 0, 1, 3, 5, 8]}}:
\begin{verbatim}
## El índice donde T[i] = 1 es: 4
\end{verbatim}
Si \textbf{\texttt{T}} con contiene el valor $1$, por ejemplo, \textbf{\texttt{T=[-10, -5, -2, 0, 2, 3, 5, 8]}}:
\begin{verbatim}
## No se encontró ningún índice donde T[i] = 1.
\end{verbatim}
\end{enumerate}
\item \lb{Resuelve por DyV este problema. Dado un array de \textbf{\texttt{N}} números enteros, buscar la cadena de \textbf{\texttt{n}} celdas consecutivas $(n\le N)$ cuya suma sea máxima. \newline
    ¿Sería conveniente aplicar aquí DyV? ¿Cómo sería una resolución directa? Calcula el orden de ambos algoritmos.} 

    \begin{enumerate}[label=\arabic*)]
      \item Descomposición del problema con Divide y Vencerás
        \begin{enumerate}[label=\arabic*)]
          \item División:
            \begin{itemize}[label=\textbullet]
              \item  Divide el arreglo de dos mitades: izquierda y derecha.
            \end{itemize}
          \item Conquista:
            \begin{itemize}[label=\textbullet]
              \item  Encuentra la subcadena de suma máxima en la mitad izquierda.
              \item Encuentra la subcadena de suma máxima en la mitad derecha.
              \item Encuentra la subcadena de suma máxima que cruza las dos mitades.
            \end{itemize}
          \item Combinación:
            \begin{itemize}[label=\textbullet]
              \item Compara las tres sumas obtenidas (izquierda, derecha y cruzada) y selecciona la máxima.
            \end{itemize}
        \end{enumerate}
      \item Implementación

        \lstinputlisting{"Ejercicio 3.py"}

      \item Explicación del algoritmo
        \begin{enumerate}[label=\arabic*)]
          \item \textbf{\texttt{suma\_maxima\_cruzada}:}
            \begin{itemize}[label=\textbullet]
              \item Calcula la suma máxima de una subcadena de longitud $n$ que cruza el punto medio del arreglo.
              \item Se consideran:
                \begin{itemize}[label=\textbullet]
                  \item Las subcadenas a la izquierda que terminan en el punto medio.
                  \item Las subcadenas a la derecha que comienzan justo después del punto medio.
                \end{itemize}
              \item La suma máxima cruzada es la combinación de ambas partes.
            \end{itemize}
          \item \textbf{\texttt{suma\_maxima\_dyv}:}
            \begin{itemize}[label=\textbullet]
              \item Implementa la estrategia de divide y vencerás:
                \begin{itemize}[label=\textbullet]
                  \item  Resuelve recursivamente el problema en las mitades izquierda y derecha.
                  \item Calcula la suma máxima cruzada entre las ods mitades.
                  \item Retorna la mayor de estas tres sumas.
                \end{itemize}
            \end{itemize}
          \item \textbf{\texttt{buscar\_suma\_maxima}:}
            \begin{itemize}[label=\textbullet]
              \item Valida que $n$ sea menor o igual al tamaño del arreglo.
              \item Llama a la función recursiva para encontrar la suma máxima.
            \end{itemize}
        \end{enumerate}
      \item Complejidad

        \textbf{Tiempo}
        \begin{itemize}[label=\textbullet]
          \item  El algoritmo divide el arreglo en dos mitades en cada paso, lo que prodce una recursión de $log N$ niveles.
          \item En cada nivel, calcular la suma cruzada tiene un costo lineal $O(n)$.
          \item \textbf{Complejidad total:} $O(N\cdot \log N)$.
        \end{itemize}
      \item Ejemplo de salida

        Con el arreglo \textbf{\texttt{arr = [2, -1, 3, 5, -2, 8, -2, 4]}} y \textbf{\texttt{n = 3}}:
        \begin{verbatim}
## La suma máxima de una subcadena de longitud 3 es: 14
        \end{verbatim}

      \item Análisis

        \textbf{Ventajas de Divide y Vencerás:}
        \begin{itemize}[label=\textbullet]
          \item Divide el problema en partes más pequeñas y combina soluciones.
          \item Útil cuando el problema puede descomponerse de manera eficiente.
        \end{itemize}
        \textbf{Desventajas en este caso:}
        \begin{enumerate}[label=\arabic*)]
          \item El problema no se divide de manera uniforme porque la longitud $n$ es fija. La combinación de resultados de subarreglos de tamaño  $n$ no es directa.
          \item La combinación requere considerar subarreglos que cruzan las divisiones, lo que introduce sobrecarga adicional.
        \end{enumerate}
        Por lo tanto, \textbf{no es ideal usar Divide y Vencerás para este problema.} Una resolución directa (lienal) es más eficiente.
    \end{enumerate}
\end{enumerate}
\end{document}
