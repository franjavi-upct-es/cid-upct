\begin{center}
	\large\textbf{\rc{Hoja de ejercicios Tema 6: Transformaciones Lineales}}
\end{center}
\begin{enumerate}[label=\color{red}\textbf{\arabic*)}, leftmargin=*]
	\item \lb{Comprueba que si $v=(a_1,a_2,a_3)$ es un vector fijo, entonces la aplicación $T_v:\R^3\longrightarrow\R^3$ definida por $T_v(x)=x+v$ \underline{no} es lineal. Este tipo de aplicación se llama una \textbf{traslación de vector} $v$ y no se puede representar por medio de una matriz $3\times 3$. Sin embargo, esta dificultad se soluciona aumentando en uno el tamaño de la matriz de modo que la traslación de vector $v=(a_1,a_2,a_3)$ se calcula por medio de la matriz \[ T=\begin{bmatrix}
            1 & 0 & 0 & a_1 \\
            0 & 1 & 0 & a_2 \\
            0 & 0 & 1 & a_3 \\
            0 & 0 & 0 & 1
        \end{bmatrix} \]es decir, se tiene que \[ \begin{bmatrix}
        1 & 0 & 0 & a_1 \\
        0 & 1 & 0 & a_2 \\
        0 & 0 & 1 & a_3 \\
        0 & 0 & 0 & 1
        \end{bmatrix}\cdot\begin{bmatrix}
        x_1\\
        x_2\\
        x_3\\
        1
        \end{bmatrix}=\begin{bmatrix}
        a_1+x_1\\
        a_2+x_2\\
        a_3+x_3\\
        1
        \end{bmatrix} \]Las 4 coordenadas $(x_1,x_2,X-3,1)$ del vector tridiemensional $x$ se llaman coordenadas homogéneas de $x$.}
  
$\begin{aligned}
	T_v:&\R^3\longrightarrow\R^3\\
	x\longmapsto T_v(x)=v+x
\end{aligned}\quad$ no es lineal, $v\neq0$.

$T_v(0)=v\neq0$

$T=\begin{bmatrix}
	1 & 0 & 0 & a_1 \\
	0 & 1 & 0 & a_2 \\
	0 & 0 & 1 & a_3 \\
	0 & 0 & 0 & 1
\end{bmatrix}\qquad v=(a_1,a_2,a_3)$

$T_v=\begin{bmatrix}
	1 & 0 & 0 & a_1 \\
	0 & 1 & 0 & a_2 \\
	0 & 0 & 1 & a_3 \\
	0 & 0 & 0 & 1
\end{bmatrix}\cdot\begin{bmatrix}
x_1\\
x_2\\
x_3\\
1
\end{bmatrix}=\begin{bmatrix}
a_1+x_1\\
a_2+x_2\\
a_3+x_3\\
1
\end{bmatrix}$

$x=(x_1,x_2,x_3)\equiv(x_1,x_2,x_3,1)\equiv$ coordenadas homogéneas.
    \item \lb{Sea $\mathcal{B}=\{v_1,v_2,v_3\}$ con $v_1=(-2,3,-2),\:v_2=(-4,5,-3)$ y $v_3=(5,-6,4)$ una base de $\R^3$. Calcula la matriz en la base $\mathcal{B}$ de la transformación lineal cuya matriz en las bases canónicas es \[ A=\begin{bmatrix}
            6 & 26 & 34 \\
            -6 & -31 & -42 \\
            4 & 20 & 27
        \end{bmatrix} \]}
  
  $M_{B\to B}(f)=M_{C\to B}\cdot M_{C\to C}(f)\cdot M_{B\to C}=\begin{bmatrix}
        -2 & -4 & 5 \\
        3 & 5 & -6 \\
        -2 & -3 & 4
  \end{bmatrix}^{-1}\cdot\begin{bmatrix}
  3 & 26 & 34 \\
  -6 & -31 & -42 \\
  4 & 20 & 27
  \end{bmatrix}\cdot\begin{bmatrix}
  -2 & -4 & 5 \\
  3 & 5 & -6 \\
  -2 & -3 & 4
  \end{bmatrix}$
  
  $M_{C\to C}(f)=\begin{bmatrix}
        3 & 26 & 34 \\
        -6 & -31 & -42 \\
        4 & 20 & 27
  \end{bmatrix},\:\underset{M_{C\to B}=\left(M_{B\to C}\right)^{-1}}{M_{B\to C}=\begin{bmatrix}
        -2 & -4 & 5 \\
        3 & 5 & -6 \\
        -2 & -3 & 4
        \end{bmatrix}}$
  
  \begin{tikzcd}
        ~_Cf:\mathbb{R}^3 \arrow[rr, "M_{C\to C}(f)"]        &  & \mathbb{R}^3_C \arrow[dd, "M_{C\to B}"] \\
        &  &                                         \\
        f:\mathbb{R}^3_B \arrow[uu, "M_{B\to C}"] \arrow[rr] &  & \mathbb{R}^3_B                         
  \end{tikzcd}
    \item \lb{Las llamadas \textbf{rotaciones de Givens} son las transformaciones lineales que vienes dadas por las matrices \[ R-x(\alpha)=\begin{bmatrix}
            1 & 0 & 0 \\
            0 & c & -s \\
            0 & s & c
        \end{bmatrix},\qquad R_y(\alpha)=\begin{bmatrix}
        c & 0 & -s \\
        0 & 1 & 0 \\
        s & 0 & c
        \end{bmatrix},\qquad R_z(\alpha)=\begin{bmatrix}
        c & -s & 0 \\
        s & c & 0 \\
        0 & 0 & 1
        \end{bmatrix}\]donde $c=\cos\alpha$ y $s=\sin\alpha$ para un cierto ángulo $\alpha$. ¿Qué interpretación geométrica tienen?\\
        Dada la transformación lineal que tiene por matriz en las bases canónicas \[ A=\begin{bmatrix}
            1 & -2 & 2 \\
            0 & -1 & 2 \\
            -1 & 0 & 1
        \end{bmatrix} \]calcula la matriz en la base $\mathcal{B}=\{u_1,u_2,u_3\}$, donde \[ u_1=(1,1,1),\qquad u_2=(0,1,1),\qquad u_3=(1,1,0) \] A la vista del resultado, ¿cuál es la interpretación geométrica de la transformación lineal dada por la matriz $A$?
    }
    
    Rotaciones de Givens
    
    $\begin{array}{l}
          R_x(\alpha)=\begin{bmatrix}
          1 & 0 & 0\\
          0 & c & -s\\
          0 & s & c
    \end{bmatrix}\\
    x=\cos(\alpha)\\
    s=\sin(\alpha)
    \end{array}\qquad$\begin{tikzpicture}[baseline=(current bounding box.center),scale=2]
    \draw (0,0,0) -- (1,0,0);
    \draw (0,0,0) -- (0,1,0);
    \draw (0,0,0) -- (0,0,1);
    \draw[lightblue, line width=1.5, -latex] (0,0,0) -- (0.5,0,0);
    \draw[lightblue, line width=1.5, -latex] (0,0,0) -- (0,0.5,0);
    \draw[lightblue, line width=1.5, -latex] (0,0,0) -- (0,0,0.5);
    \end{tikzpicture}\begin{tikzpicture}[baseline=(current bounding box.center),scale=2]
    \draw[blue, dashed] (135:1) |- (0,0);
    \draw[blue,dashed] (125:1) -| (0,0);
    \draw (-1.2,0) -- (1.2,0) node[right] {$x_2$};
    \draw (0,-1.2) -- (0,1.2) node[above] {$x_3$};
    \draw[lightblue,-latex, line width=1.5] (0,0) -- (1,0) node[below] {$e_2=(1,0)$};
    \draw[lightblue,-latex, line width=1.5] (0,0) -- (0,1) node[right] {$e_1=(0,1)$};
    \draw[lightblue] (0,0) circle (1);
    \draw[blue,-latex] (0,0) -- (45:1) node[right] {$\left(\cos(\alpha),\sin(\alpha)\right)$};
    \draw[blue,-latex] (0,0) -- (135:1);
    \draw[blue] (0.2,0) arc (0:45:0.2) node[midway, right] {$\alpha$};
    \draw[blue] (0,0.2) arc (0:45:0.2) node[midway, above] {$\alpha$};
    \draw[-latex] (0.5,0) -- (-2,-1.5) node[below] {$(-\sin(\alpha),\cos(\alpha))$};
    \end{tikzpicture}
    
    $\begin{array}{l}
          \begin{bmatrix}
          1 & 0 & 0 \\
          0 & c & -s \\
          0 & s & c
    \end{bmatrix}\cdot\overset{e_1}{\begin{bmatrix}
    1\\
    0\\
    0
    \end{bmatrix}}=\begin{bmatrix}
    1\\
    0\\
    0
\end{bmatrix}\\
\begin{bmatrix}
      1 & 0 & 0 \\
      0 & c & -s \\
      0 & s & c
\end{bmatrix}\cdot\overset{e_2}{\begin{bmatrix}
            0\\
            1\\
            0
\end{bmatrix}}=\begin{bmatrix}
      0\\
      \cos(\alpha)\\
      \sin(\alpha)
\end{bmatrix}\qquad\text{giro de ángulo $\alpha$ en el plano $x_2x_3$}\\
\begin{bmatrix}
      1 & 0 & 0 \\
      0 & c & -s \\
      0 & s & c
\end{bmatrix}\cdot\overset{e_3}{\begin{bmatrix}
            0\\
            0\\
            1
\end{bmatrix}}=\begin{bmatrix}
      1\\
      -\sin(\alpha)\\
      \cos(\alpha)
\end{bmatrix}
    \end{array}$
    
    $\underset{\begin{subarray}{c}
                \rotatebox{-90}{=}\\
                M_{C\to C}(f)
    \end{subarray}}{A}=\begin{bmatrix}
          1 & -2 & 2\\
          0 & -1 & 2\\
          -1 & 0 & 1
    \end{bmatrix}\qquad B=\{u_1=(1,1,1),\: u_2=(0,1,1),\:u_3=(1,1,0)\}$
    
    $M_{B\to B}(f)=M_{C\to B}\underbrace{M_{C\to C}(f)}_A\underbrace{M_{B\to C}}_{\begin{subarray}{c}
                \rotatebox{-90}{=}\\
                \begin{bmatrix}
                      1 & 0 & 1 \\
                      1 & 1 & 1 \\
                      1 & 1 & 0
                \end{bmatrix}
    \end{subarray}}$
    \item \lb{Consideremos un plano de $\R^3$ que pasa por el origen, es decir, un subespacio y recibe el nombre de \textbf{reflexión de Householder}. Para verlo, prueba que si $x$ es un vector arbitrario, entonces $x+Hx$ es ortogonal a $v$ (es decir, $x+Hx$ pertenece al plano) y $x-Hx$ es proporcional a $v$ (estas dos cosas prueban de $Hx$ es el simétrico de $x$ respecto del plano dado).}
    
    $U=\{(x,y,z)\in\R^3:ax+by+cz=0\}\qquad v\cdot v^\intercal=\begin{bmatrix}
          a\\
          b\\
          c
    \end{bmatrix}\cdot\begin{bmatrix}
          a & b &c
    \end{bmatrix}=\begin{bmatrix}
          a^2 & ab & ac \\
          ba & b^2 & bc \\
          ca & cb & c^2
    \end{bmatrix}$
    
    $H=I-\dfrac{2v\cdot v^\intercal}{v^\intercal\cdot v}$
    
    Simetría ortogonal respecto del plano (reflexión de Householder)
    
    $\begin{array}{l}
          0\overset{?}{=}v^\intercal(x+Hx)=v^\intercal x+v^\intercal Hx=v^\intercal x+v^\intercal\left(x-\dfrac{2vv^\intercal}{v^\intercal v}\right)=v^\intercal x+v^\intercal x-\dfrac{2\cancel{v^\intercal v}v^\intercal}{\cancel{v^\intercal v}}=0\\
          x-Hx\text{ proporcional a $x$}\\
          x-Hx=x-x+\dfrac{2vv^\intercal}{v^\intercal v}\cdot x=v\dfrac{2v^\intercal x}{v^\intercal v}\equiv\text{ proporcional a $v$}\\
          x=\underbrace{\dfrac{1}{2}(x+Hx)}_{\begin{subarray}{c}
                      \rotatebox{-90}{$\in$}\\
                      u
          \end{subarray}}+\underbrace{\dfrac{1}{2}(x-Hx)}_{\begin{subarray}{c}
          \rotatebox{-90}{$\in$}\\
          u^\intercal
    \end{subarray}}\qquad Hx=\underbrace{\dfrac{1}{2}(x+Hx)}_{\begin{subarray}{c}
    \rotatebox{-90}{$\in$}\\
    u
\end{subarray}}+\dfrac{1}{2}(x-Hx)
    \end{array}$
    
    \item \lb{Calcula la proyección del vector (2,3,4) sobre el subespacio $\mathrm{Col}(A)$, donde $A=\begin{bmatrix}
            1 & 1\\
            0 & 1\\
            0 & 0
        \end{bmatrix}$. Hazlo de tres formas disitntas: como en el ejemplo 5.17, como en el ejemplo 5.18 y como en el ejemplo 5.21.}
  
  $\begin{array}{l}
        u=\mathrm{Col}(A)=<(1,0,0),(1,1,0)>\\
        M_{C\to C}(P_u)=\left[\begin{array}{c:c:c}
         ~ & ~ & ~\\
         P_u(e_1) & P_u(e_2) & P_u(e_3)\\
          & & 
  \end{array}\right]\\
  e_1=(1,0,0),\:e_2=(0,1,0),\:e_3=(0,0,1)
  \end{array}$
  
  $\{u_1,u_2,\dots,u_m\}$ base ortogonal de $u$, entonces \[ P_u(u)=\dfrac{u\cdot u_1}{\|u_1\|^2}u_1+\cdots+\dfrac{u\cdot u_m}{\|u_m\|^2}u_m \]
  
  \begin{enumerate}[label=\color{lightblue}\arabic*)]
        \item Calculamos por Gram-Schmidt una base ortogonal de $u=\mathrm{Col}(A)$
        \begin{enumerate}[label=\color{lightblue}\alph*)]
              \item $u_1=(1,0,0)$
              \item $u_2=(1,1,0) +\alpha(1,0,0)$
              
              Imponemos la condición
              
              $0=u_1\cdot u_2=(1,0,0)\cdot(1,1,0)+\alpha(1,0,0)\cdot(1,0,0)\longrightarrow\alpha=-\dfrac{(1,0,0)\cdot(1,1,0)}{(1,0,0)\cdot(1,0,0)}=-1$
        \end{enumerate}
  \end{enumerate}
  
  $\begin{array}{l}
        P_u(1,0,0)=\underbrace{(1,0,0)\cdot(1,0,0)}_1\cdot(1,0,0)=(1,0,0)\\
        P_u(0,1,1)=(0,1,0)\\
        P_u(0,0,1)=\underbrace{(0,0,1)\cdot(1,0,0)}_0\cdot u_1+(0,0,1)\cdot(0,1,0)\cdot u_2=(0,0,0)\\
        P_u(2,3,4)=\begin{bmatrix}
              1 & 0 & 0 \\
              0 & 1 & 0 \\
              0 & 0 & 0
        \end{bmatrix}\cdot\begin{bmatrix}
        2\\
        3\\
        4
        \end{bmatrix}=\begin{bmatrix}
        2\\
        3\\
        0
        \end{bmatrix}
  \end{array}$
  
    \item \lb{¿Qué combianción lienal de los vectores $(1,2,-1)$ y $(1,0,1)$ está más "cerca" del vector $(2,1,1)$?}
    
    $\|(1,2,1)x_1+(1,0,1)x_2-(2,1,1)\|$ mínima.
    
    $\underbrace{\begin{bmatrix}
                1 & 1 \\
                2 & 0 \\
                1 & 1
    \end{bmatrix}_{3\times2}}_{A}\cdot\underbrace{\begin{bmatrix}
    x_1\\
    x_2
\end{bmatrix}}_x\qquad\|\underbrace{Ax}_{P_{\mathrm{Col}(A)(b)}}-b\|$ minimizar.\quad$Ax=A\underbrace{(A^\intercal A)^{-1}A^\intercal b}_{x}$


    \item \lb{Encuentra una base ortonormal $\{u_1,u_2,u_3\}$ de $\R^3$ tal que $\mathrm{Col}(A)$ sea el subespacio generado por $u_1,u_2$ donde \[ A=\begin{bmatrix}
            1 &1 \\
            2 & -1\\
            -2 & 4
        \end{bmatrix} \]¿Cuál de los cuatro subespacios fundamentales de $A$ contiene a $u_3$? Calcula la proyección orotogonal de $b^\intercal=(1,2,7)$ sobre $\mathrm{Col}(A)$ y explica por qué esta proyección es la solución aproximada del sistema de ecuaciones $Ax=b$.} 
  
  Se tiene que cumplor que $\mathrm{Col}(A)=<u_1,\:u_2>$
  
  \bu{Gram-Schmidt}
  \begin{enumerate}[label=\color{lightblue}\arabic*)]
        \item $u_1=(1,2,-2)\longrightarrow\dfrac{u_1}{\|u_1\|}$
        \item $u_2=(1,-1,4)+\alpha(1,2,-2)$
        
        $\begin{array}{l}
              0=u_1\cdot u_2\longrightarrow\alpha=?\\
              \dfrac{u_2}{\|u_2\|}
        \end{array}$
        \item $u_3=\underbrace{(1,0,0)}_{\begin{subarray}{l}
                    (0,1,0)\\
                    (0,0,1)
        \end{subarray}}+\alpha\cdot u_1+\beta\cdot u_2\longrightarrow\begin{array}{l}
        0=u_1\cdot u_3\longrightarrow \alpha\\
        0=u_2\cdot u_3\longrightarrow \beta
  \end{array}$
  \end{enumerate}
  
  $\mathrm{Col}(A)\oplus\mathrm{nuc}(A^\intercal)=\R^3$
\end{enumerate}

