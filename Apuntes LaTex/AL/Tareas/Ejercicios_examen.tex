\begin{center}
	\large\textbf{\rc{Ejercicios Tipo Examen}}
\end{center}
\begin{enumerate}[label=\color{red}\textbf{\arabic*)}, leftmargin=*]
	\item \lb{En una red eléctrica se tienen dos fuentes de tensión $\vec{V}_1$ y $\vec{V}_2$, siendo $\vec{V}_2=5_{45^\circ}$, esto es, el número complejo de módulo 5 y argumento $45^\circ$. La ley de Ohm conduce al sistema de ecuaciones lineales $\vec{V}=R\vec{I}$, donde \[ \vec{V}=\begin{bmatrix}
			\vec{V}_1 \\
			0 \\
			\vec{V}_2
		\end{bmatrix},\qquad R=\begin{bmatrix}
			1+\jm & -5\jm & 0\\
			-5\jm & 0 & -3 \\
			0 & -3& 1
		\end{bmatrix}\quad\mathrm{e}\quad\vec{I}=\begin{bmatrix}
			\vec{I}_1 \\
			\vec{I}_2 \\
			\vec{I}_3
		\end{bmatrix}.\]Calcula el valor que debe tomar $\vec{V}_1$ de modo que $\vec{I}_2$ sea igual a cero.}

	Por la regla de Cramer se tiene que \[ \vec{I}_2=\dfrac{\begin{bmatrix}
				1+\jmath & \vec{V}_1 & 0\\
				-5\jmath & 0 & -3 \\
				0& \vec{V}_2 & 1
			\end{bmatrix}}{|R|} \] Por tanto, $\vec{I}_2=0$ si y sólo si \[
		\begin{array}{l}
			0=\begin{bmatrix}
				1+\jmath & \vec{V}_1& 0\\
				-5\jmath & 0& -3 \\
				0& 5_{45^\circ} & 1
			\end{bmatrix}=\begin{bmatrix}
				\sqrt{2}_{45^\circ} & \vec{V}_1& 0 \\
				5_{-90^\circ} & 0& 3_{180^\circ} \\
				0 & 5_{45^\circ} & 1_{0^\circ}
			\end{bmatrix} \\
			\begin{rcases}
				|1+j|=\sqrt{1^2+1^2}=\sqrt{2} \\
				\theta =\arctan\dfrac{1}{1}=45^\circ
			\end{rcases}\longrightarrow 1+j=\sqrt{2}_{45^\circ}\\
			-5\jmath=5(-\jmath)=5_{-90^\circ}=5_{270^\circ}\\
			-3=3_{180^\circ}
		\end{array} \] Hemos tomado el argumento principal en $[0,360^\circ[.$ Así:

	$0=\begin{bmatrix}
			\sqrt{2}_{45^\circ} & \vec{V}_1& 0 \\
			5_{270^\circ }& 0& 3_{180^\circ} \\
			0 & 5_{45^\circ} & 1_{0^\circ}
		\end{bmatrix}=-(5_{45^\circ}\cdot3_{180^\circ}\cdot\sqrt{2}_{45^\circ}+1\cdot5_{270^\circ}\cdot\vec{V}_1)=-15\sqrt{2}_{270^\circ}-5_{270^\circ}\cdot\vec{V}_1$

	Despejando: \[
		\vec{V}_1=\dfrac{-15\sqrt{2}_{270^\circ}}{5_{270^\circ}}=\dfrac{15\sqrt{2}_{270^\circ}\cdot1_{180^\circ}}{5_{270^\circ}}=3\sqrt{2}_{180^\circ}=\bboxed{-3\sqrt{2}}
	\]
	\item \lb{Sea \[ Q=\begin{bmatrix}
			a & b \\
			c & d
		\end{bmatrix} \] una matriz ortogonal con determinante $|Q|=1$. Consideremos
	la norma matricial \[
		\|Q_2\|=\underset{x\in\R^2,x\neq0}{\max}\dfrac{\|Qx\|_2}{\|x\|_2}, \] donde
	$\|v\|_2^2=v_1^2+v_1^2$ para todo $v=[v_1,v_2]\in\R^2$. Se pide:}
	\begin{enumerate}[label=\color{red}\alph*)]
		\item \db{Demuestra que el número de condicionamiento de $Q$, para la norma anterior, es igual a 1.}

		$\mathrm{c}(Q)=\|Q\|_2\cdot\|Q^{-1}\|_2$

		Recordemos que este tipo de matrices tienen la forma $Q=\begin{bmatrix}
				a& b \\
				-b & a
			\end{bmatrix}$ con $a^2+b^2=1$.

		También se tiene que \[ Q^{-1}=Q^\intercal=\begin{bmatrix}
				a & -b \\
				b & a
			\end{bmatrix} \]Por tanto:
		\[ \begin{aligned}
				\|Q\|_2 & =\underset{\begin{subarray}{c}
						 x\in\R^2\\
						 x\neq0
					 \end{subarray}}{\max}\dfrac{\|Qx\|_2}{\|x\|_2}=\underset{\begin{subarray}{c}
x\in\R^2\\
x\neq0
\end{subarray}}{\max}\dfrac{\left\|\begin{bmatrix}
a& b \\
-b & a
\end{bmatrix}\cdot\begin{bmatrix}
						 x_1 \\
						 x_2
					 \end{bmatrix}\right\|_2}{\sqrt{x_1^2+x_2^2}} \\
				& =\underset{\begin{subarray}{c}
						 x\in\R^2\\
						 x\neq0
					 \end{subarray}}{\max}\dfrac{\|(ax_1+bx_2-bx_1+ax_2)\|_2}{\sqrt{x_1^2+x_2^2}} \\
				& =\underset{\begin{subarray}{c}
						 x\in\R^2\\
						 x\neq0
					 \end{subarray}}{\max}\dfrac{\sqrt{(ax_1+bx_2)^2+(-bx_1+ax_2)^2}}{\sqrt{x_1^2+x_2^2}} \\
				&=\underset{\begin{subarray}{c}
										 x\in\R^2\\
										 x\neq0
									 \end{subarray}}{\max}\dfrac{\sqrt{a^2x_1^2+b^2x_2^2+2abx_1x_2+b^2x_1^2+a^2x_2^2-2abx_1x_2}}{\sqrt{x_1^2+x_2^2}}\\
				&=\underset{\begin{subarray}{c}
										 x\in\R^2\\
										 x\neq0
									 \end{subarray}}{\max}\dfrac{\sqrt{\cancelto{1}{(a^2+b^2)}(x_1^2+x_2^2)}}{\sqrt{x_1^2+x_2^2}}=1
			\end{aligned} \]Análogamente, 
			\[ \begin{aligned}
			\|Q^{-1}\|_2&=\underset{\begin{subarray}{c}
									 x\in\R^2\\
									 x\neq0
								 \end{subarray}}{\max}\dfrac{\|Q^{-1}x\|_2}{\|x\|_2}=\underset{\begin{subarray}{c}
								 						 x\in\R^2\\
								 						 x\neq0
								 					 \end{subarray}}{\max}\dfrac{\left\|\begin{bmatrix}
								 					 a & -b \\
								 					 b & a
								 					 \end{bmatrix}\cdot\begin{bmatrix}
								 					 x_1\\
								 					 x_2
								 					 \end{bmatrix}\right\|_2}{\|(x_1,x_2)\|_2}\\
									&=\underset{\begin{subarray}{c}
															 x\in\R^2\\
															 x\neq0
														 \end{subarray}}{\max}\dfrac{\|(ax_1-bx_2,bx_1+ax_2)\|_2}{\|(x_1,x_2)\|_2}\\
									&=\underset{\begin{subarray}{c}
															 x\in\R^2\\
															 x\neq0
														 \end{subarray}}{\max}\dfrac{\sqrt{(ax_1-bx_2)^2+(bx_1+ax_2)^2}}{\sqrt{x_1^2+x_2^2}}\\
									&=\underset{\begin{subarray}{c}
															 x\in\R^2\\
															 x\neq0
														 \end{subarray}}{\max}\dfrac{\sqrt{a^2x_1^2+b^2x_2^2-2abx_1x_2+b^2x_1^2+a^2x_2^2+2abx_1x_2}}{\sqrt{x_1^2+x_1^2}}
									&=\underset{\begin{subarray}{c}
															 x\in\R^2\\
															 x\neq0
														 \end{subarray}}{\max}\dfrac{\sqrt{(a^2+b^2)\cdot(x_1^2+x_2^2)}}{\sqrt{x_1^2+x_2^2}}=1
			\end{aligned} \]Por tanto, \[ \mathrm{c}(Q)=\|Q\|_2\cdot\|Q^{-1}\|_2=1\cdot1=1 \]
		\item \db{Se puede probar que la propiedad anterior es cierta si la matriz $Q$ del apartado anterior es de orden $ n$, siendo $n\ge2$ un número natural. Supongamos que queremos resolver un sistema lineal de la forma $Qx=b$ y que, debido a errores de redondeo o de medida, cometemos un error relativo en el término independiente de aproximadamente $10^{-3}$. Obtén una cota del error relativo que se cometería en la solución $x$. Se ha de justificar la respuesta.}
		
		Recordemos que \[ \dfrac{\|\triangle x\|}{\|x\|}\le\mathrm{c}(Q)\dfrac{\|\triangle b\|}{\|b\|} \]Como $\mathrm{c}(Q)=1$, si $\dfrac{\|\triangle b\|}{\|b\|}=10^{-3}$, entonces \[ \dfrac{\|\triangle x\|}{\|x\|}\le1\cdot10^{-3}=10^{-3}. \]
	\end{enumerate}

	\item \lb{Explica en qué consisten las factorizaciones $LU,\,PLU$ y Cholesky de una matriz cuadrada $A$. Explica también cómo se resolvería un sistema lineal de la forma $Ax=b$ usando ambas factorizaciones. Finalmente, supongamos que la matriz $A$ es simétrica y definida positiva. ¿Cuál de las factorizaciones usarías para resolver el sistema $Ax=b$? ¿Por qué?}


	\item \lb{Una clase de matrices que aparecen en varias aplicaciones en Ciencia de Datos (por ejemplo, en el famoso PageRank de Google) son las llamadas \textit{matrices de Markov}. Se caracterizan porque todas sus entradas son no negativas y además la suma de las entradas de cada columna siempre da como resultado 1. La siguiente matriz es un ejemplo de matriz de Markov: \[
			M=\begin{bmatrix}
				\frac{1}{2} & \frac{1}{3} & \frac{1}{4} \\
				\frac{1}{2} & \frac{1}{3} & \frac{1}{4} \\
				0 & \frac{1}{3} & \frac{1}{2}
			\end{bmatrix} \]Se pide:}
	\begin{enumerate}[label=\color{red}\alph*)]
		\item \db{Encuentra matrices $P$ y $Q$ de modo que $B=PMQ$, con $B$ una matriz por bloques que tiene a la identidad $I_r$ en el primer bloque, y en el resto de bloques vale cero.}


		\item \db{¿Cuál es el rango de $M$?}


		\item \db{Resuleve el sistema lineal $Mx=x$, donde
		$x=[x_1,x_2,x_3]^{\intercal}$ es la incógnita.}
	\end{enumerate}
\end{enumerate}
