\documentclass{article}
\usepackage{fullpage}
\usepackage[utf8]{inputenc}
\usepackage{pict2e}
\usepackage{amsmath}
\usepackage{enumitem}
\usepackage{eurosym}
\usepackage{mathtools}
\usepackage{amssymb, amsfonts, latexsym, cancel}
\setlength{\parskip}{0.3cm}
\usepackage{graphicx}
\usepackage{fontenc}
\usepackage{slashbox}
\usepackage{setspace}
\usepackage{gensymb}
\usepackage{accents}
\usepackage{adjustbox}
\setstretch{1.35}
\usepackage{bold-extra}
\usepackage[document]{ragged2e}
\usepackage{subcaption}
\usepackage{tcolorbox}
\usepackage{xcolor, colortbl}
\usepackage{wrapfig}
\usepackage{empheq}
\usepackage{array}
\usepackage{parskip}
\usepackage{arydshln}
\graphicspath{ {images/} }
\renewcommand*\contentsname{\color{black}Índice} 
\usepackage{array, multirow, multicol}
\definecolor{lightblue}{HTML}{007AFF}
\usepackage{color}
\usepackage{etoolbox}
\usepackage{listings}
\usepackage{mdframed}
\setlength{\parindent}{0pt}
\usepackage{underscore}
\usepackage{hyperref}
\usepackage{tikz}
\usepackage{tikz-cd}
\usetikzlibrary{shapes, positioning, patterns}
\usepackage{tikz-qtree}
\usepackage{biblatex}
\usepackage{pdfpages}
\usepackage{pgfplots}
\usepackage{pgfkeys}
\addbibresource{biblatex-examples.bib}
\usepackage[a4paper, left=1cm, right=1cm, top=1cm,
bottom=1.5cm]{geometry}
\usepackage{titlesec}
\usepackage{titletoc}
\usepackage{tikz-3dplot}
\usepackage{kbordermatrix}
\usetikzlibrary{decorations.pathreplacing}
\newcommand{\Ej}{\textcolor{lightblue}{\underline{Ejemplo}}}
\setlength{\fboxrule}{1.5pt}

% Configura el formato de las secciones utilizando titlesec
\titleformat{\section}
{\color{red}\normalfont\LARGE\bfseries}
{Tema \thesection:}
{10 pt}
{}

% Ajusta el formato de las entradas de la tabla de contenidos
\addtocontents{toc}{\protect\setcounter{tocdepth}{4}}
\addtocontents{toc}{\color{black}}

\titleformat{\subsection}
{\normalfont\Large\bfseries\color{red}}{\thesubsection)}{1em}{\color{lightblue}}

\titleformat{\subsubsection}
{\normalfont\large\bfseries\color{red}}{\thesubsubsection)}{1em}{\color{lightblue}}

\newcommand{\bboxed}[1]{\fcolorbox{lightblue}{lightblue!10}{$#1$}}
\newcommand{\rboxed}[1]{\fcolorbox{red}{red!10}{$#1$}}

\DeclareMathOperator{\N}{\mathbb{N}}
\DeclareMathOperator{\Z}{\mathbb{Z}}
\DeclareMathOperator{\R}{\mathbb{R}}
\DeclareMathOperator{\Q}{\mathbb{Q}}
\DeclareMathOperator{\K}{\mathbb{K}}
\DeclareMathOperator{\im}{\imath}
\DeclareMathOperator{\jm}{\jmath}
\DeclareMathOperator{\col}{\mathrm{Col}}
\DeclareMathOperator{\fil}{\mathrm{Fil}}
\DeclareMathOperator{\rg}{\mathrm{rg}}
\DeclareMathOperator{\nuc}{\mathrm{nuc}}
\DeclareMathOperator{\dimf}{\mathrm{dimFil}}
\DeclareMathOperator{\dimc}{\mathrm{dimCol}}
\DeclareMathOperator{\dimn}{\mathrm{dimnuc}}
\DeclareMathOperator{\dimr}{\mathrm{dimrg}}
\DeclareMathOperator{\dom}{\mathrm{Dom}}
\DeclareMathOperator{\infi}{\int_{-\infty}^{+\infty}}
\newcommand{\dint}[2]{\int_{#1}^{#2}}

\newcommand{\bu}[1]{\textcolor{lightblue}{\underline{#1}}}
\newcommand{\lb}[1]{\textcolor{lightblue}{#1}}
\newcommand{\db}[1]{\textcolor{blue}{#1}}
\newcommand{\rc}[1]{\textcolor{red}{#1}}
\newcommand{\tr}{^\intercal}

\renewcommand{\CancelColor}{\color{lightblue}}

\newcommand{\dx}{\:\mathrm{d}x}
\newcommand{\dt}{\:\mathrm{d}t}
\newcommand{\dy}{\:\mathrm{d}y}
\newcommand{\dz}{\:\mathrm{d}z}
\newcommand{\dth}{\:\mathrm{d}\theta}
\newcommand{\dr}{\:\mathrm{d}\rho}
\newcommand{\du}{\:\mathrm{d}u}
\newcommand{\dv}{\:\mathrm{d}v}
\newcommand{\tozero}[1]{\cancelto{0}{#1}}
\newcommand{\lbb}[2]{\textcolor{lightblue}{\underbracket[1pt]{\textcolor{black}{#1}}_{#2}}}
\newcommand{\dbb}[2]{\textcolor{blue}{\underbracket[1pt]{\textcolor{black}{#1}}_{#2}}}
\newcommand{\rub}[2]{\textcolor{red}{\underbracket[1pt]{\textcolor{black}{#1}}_{#2}}}

\author{Francisco Javier Mercader Martínez}
\date{}
\renewcommand{\arraystretch}{1}
\setlength{\arraycolsep}{6pt}
\setstretch{1.15}
\title{Álgebra Lineal\\Examen Convocatoria Mayo 2024}

\begin{document}
\maketitle
\begin{enumerate}[label=\color{red}\textbf{\arabic*)}]
    \item \lb{Sea $A$ una matriz de tamaño $m\times n$. Demuestra que existen matrices ortogonales $U$ y $V$, y una matriz $\Sigma$, de tamaño  $m\times n$ y diagonal, de modo que $A=U\Sigma V^\intercal$ (factorización SVD).\newline Tienes que explicar con detalles cuáles son las entradas de la diagonal de $\Sigma$ y cómo se obtienen las columnas de  $U$ y  $V$.}
    \item \lb{Sea $u$ un vector de $\R^n$ con $\|u\|=1$ y consideremos la matriz  $A=I-2uu^\intercal$}
        \begin{enumerate}[label=\color{red}\textbf{\alph*)}]
            \item \db{Prueba que $A$ es simétrica y  $A^2=I$.}
            \item \db{Prueba que $u$ es un vector propio de $A$ de valor propio $-1$ y su  $v$ es un vectori no nulo orotogonal a  $u$, prueba que  $v$ es un vector propio de  $A$ de valor propio  $1$.}
            \item \db{Si $A$ es  $2\times 2$, ¿es $A$ diagonalizable?} 
        \end{enumerate}
    \item \lb{Consideremos los números complejos $z_1=\dfrac{1}{2}-\dfrac{1}{2}j$ y $z_2=\dfrac{1}{-1-\sqrt{3} j}$} 
        \begin{enumerate}[label=\color{red}\textbf{\alph*)}]
            \item \db{Calcula la forma exponencial de $z_1$ y $z_2$.}
            \item \db{Prueba que $z_1^{20}+z_2^{9}=\dfrac{1}{2^{10}}$} 
        \end{enumerate}
    \item \lb{En $\R^3$ se considera el subespacio vectorial $U=\left<u_1,u_2,u_3,u_4 \right>$, donde \[
   u_1=(1,1,1),\quad u_2=(0,1,-1),\quad u_3=(1,2,0), \quad u_4=(1,0,2), 
    \] } 
    \begin{enumerate}[label=\color{red}\textbf{\alph*)}]
        \item \db{Calcula una base y la dimensión de $U$} 
        \item \db{Prueba que el vector $u=(-1,2,-4)$ pertence a  $U$ y halla sus coordenadas respecto de la base hallada en el apartado anterior.}
        \item \db{Calcula la proyección ortogonal del vector $(-3,1,3)$ sobre  $U$.}

        \item \db{¿Qué ecuaciones debe satisfacer un vector $(x,y,z)$ para pertenecer a  $U$?} 
    \end{enumerate}
\item \lb{Sea la matriz simétrica $A=\begin{bmatrix} 
            2 & 1 \\ 1 & 2 
\end{bmatrix} $.}
\begin{enumerate}[label=\color{red}\textbf{\alph*)}]
    \item \db{Halla una matriz $Q$ ortogonal tal que  $A=QDQ^{-1}$ con $D$ diagonal.}
    \item \db{Halla la descomposición espectral de $A$, es decir, escribe  $A$ como  $A=\lambda_1u_1u_1^\intercal+\lambda_2u_2u_2^\intercal$ con $\lambda_1,\lambda_2$ los valores propios de $A$ y  $u_1,u_2$ vectores propios adecuados.} 
\end{enumerate}
\item \lb{Resuelve el siguiente sistema de ecuaciones mediante una factorización $LU$  \[
\begin{rcases}
    2x-3y+z=2\\
    -4x+9y+2z=4\\
    6x-12y-2z=-2
\end{rcases}
\] } 
\end{enumerate}
\end{document}
