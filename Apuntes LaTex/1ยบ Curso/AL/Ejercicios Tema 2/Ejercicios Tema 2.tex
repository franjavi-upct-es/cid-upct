\documentclass{article}
\usepackage{fullpage}
\usepackage[utf8]{inputenc}
\usepackage{pict2e}
\usepackage{amsmath}
\usepackage{enumitem}
\usepackage{eurosym}
\usepackage{mathtools}
\usepackage{amssymb, amsfonts, latexsym, cancel}
\setlength{\parskip}{0.3cm}
\usepackage{graphicx}
\usepackage{fontenc}
\usepackage{slashbox}
\usepackage{setspace}
\usepackage{gensymb}
\usepackage{accents}
\usepackage{adjustbox}
\setstretch{1.35}
\usepackage{bold-extra}
\usepackage[document]{ragged2e}
\usepackage{subcaption}
\usepackage{tcolorbox}
\usepackage{xcolor, colortbl}
\usepackage{wrapfig}
\usepackage{empheq}
\usepackage{array}
\usepackage{parskip}
\usepackage{arydshln}
\graphicspath{ {images/} }
\renewcommand*\contentsname{\color{black}Índice} 
\usepackage{array, multirow, multicol}
\definecolor{lightblue}{HTML}{007AFF}
\usepackage{color}
\usepackage{etoolbox}
\usepackage{listings}
\usepackage{mdframed}
\setlength{\parindent}{0pt}
\usepackage{underscore}
\usepackage{hyperref}
\usepackage{tikz}
\usepackage{tikz-cd}
\usetikzlibrary{shapes, positioning, patterns}
\usepackage{tikz-qtree}
\usepackage{biblatex}
\usepackage{pdfpages}
\usepackage{pgfplots}
\usepackage{pgfkeys}
\addbibresource{biblatex-examples.bib}
\usepackage[a4paper, left=1cm, right=1cm, top=1cm,
bottom=1.5cm]{geometry}
\usepackage{titlesec}
\usepackage{titletoc}
\usepackage{tikz-3dplot}
\usepackage{kbordermatrix}
\usetikzlibrary{decorations.pathreplacing}
\newcommand{\Ej}{\textcolor{lightblue}{\underline{Ejemplo}}}
\setlength{\fboxrule}{1.5pt}

% Configura el formato de las secciones utilizando titlesec
\titleformat{\section}
{\color{red}\normalfont\LARGE\bfseries}
{Tema \thesection:}
{10 pt}
{}

% Ajusta el formato de las entradas de la tabla de contenidos
\addtocontents{toc}{\protect\setcounter{tocdepth}{4}}
\addtocontents{toc}{\color{black}}

\titleformat{\subsection}
{\normalfont\Large\bfseries\color{red}}{\thesubsection)}{1em}{\color{lightblue}}

\titleformat{\subsubsection}
{\normalfont\large\bfseries\color{red}}{\thesubsubsection)}{1em}{\color{lightblue}}

\newcommand{\bboxed}[1]{\fcolorbox{lightblue}{lightblue!10}{$#1$}}
\newcommand{\rboxed}[1]{\fcolorbox{red}{red!10}{$#1$}}

\DeclareMathOperator{\N}{\mathbb{N}}
\DeclareMathOperator{\Z}{\mathbb{Z}}
\DeclareMathOperator{\R}{\mathbb{R}}
\DeclareMathOperator{\Q}{\mathbb{Q}}
\DeclareMathOperator{\K}{\mathbb{K}}
\DeclareMathOperator{\im}{\imath}
\DeclareMathOperator{\jm}{\jmath}
\DeclareMathOperator{\col}{\mathrm{Col}}
\DeclareMathOperator{\fil}{\mathrm{Fil}}
\DeclareMathOperator{\rg}{\mathrm{rg}}
\DeclareMathOperator{\nuc}{\mathrm{nuc}}
\DeclareMathOperator{\dimf}{\mathrm{dimFil}}
\DeclareMathOperator{\dimc}{\mathrm{dimCol}}
\DeclareMathOperator{\dimn}{\mathrm{dimnuc}}
\DeclareMathOperator{\dimr}{\mathrm{dimrg}}
\DeclareMathOperator{\dom}{\mathrm{Dom}}
\DeclareMathOperator{\infi}{\int_{-\infty}^{+\infty}}
\newcommand{\dint}[2]{\int_{#1}^{#2}}

\newcommand{\bu}[1]{\textcolor{lightblue}{\underline{#1}}}
\newcommand{\lb}[1]{\textcolor{lightblue}{#1}}
\newcommand{\db}[1]{\textcolor{blue}{#1}}
\newcommand{\rc}[1]{\textcolor{red}{#1}}
\newcommand{\tr}{^\intercal}

\renewcommand{\CancelColor}{\color{lightblue}}

\newcommand{\dx}{\:\mathrm{d}x}
\newcommand{\dt}{\:\mathrm{d}t}
\newcommand{\dy}{\:\mathrm{d}y}
\newcommand{\dz}{\:\mathrm{d}z}
\newcommand{\dth}{\:\mathrm{d}\theta}
\newcommand{\dr}{\:\mathrm{d}\rho}
\newcommand{\du}{\:\mathrm{d}u}
\newcommand{\dv}{\:\mathrm{d}v}
\newcommand{\tozero}[1]{\cancelto{0}{#1}}
\newcommand{\lbb}[2]{\textcolor{lightblue}{\underbracket[1pt]{\textcolor{black}{#1}}_{#2}}}
\newcommand{\dbb}[2]{\textcolor{blue}{\underbracket[1pt]{\textcolor{black}{#1}}_{#2}}}
\newcommand{\rub}[2]{\textcolor{red}{\underbracket[1pt]{\textcolor{black}{#1}}_{#2}}}

\author{Francisco Javier Mercader Martínez}
\date{}
\title{Álgebra Lineal\\ Ejercicios Tema 2: Vectores, matrices y tensores}

\begin{document}
\maketitle
\begin{enumerate}[label=\color{red}\textbf{\arabic*)}]
    \item \lb{Determina la verdad o falsedad de la siguiente afirmación: si $u_1$ es combinación lineal de $u_2$ y $u_3$, entonces $u_3$ es combinación lineal de $u_1$ y $u_2$.}
\item \lb{Consideremos los vectores $u=(1,1,0)$ y  $v=(0,1,1)$. Encuentra un vector $w$ ortogonal a $u$ y $v$. Comprueba que $w$ es ortogonal a cualquier combinación lineal de $u$ y $v$. Encuentra ahora un vector que \underline{no} sea lineal de $u,v$ y comprueba que no es ortogonal a $w$.}
\item \lb{Haz un dibujo de los siguientes conjuntos de $\R^2$: \[
\begin{array}{l}
    \{(x,y)\in \R^2\text{ tal que }\|(x,y)\|_1=1\}\\ 
    \{(x,y)\in \R^2\text{ tal que }\|(x,y)\|_2=1\}\\ 
    \{(x,y)\in \R^2\text{ tal que }\|(x,y)\|_\infty=1\}\\ 
\end{array}
\] } 
\item \lb{Prueba que $\|u\|_2\le \sqrt{\|u\|_1\|u\|_\infty} $.}

\item \lb{Dos vectores son ortogonales cuando su producto escalar es cero, pero ¿qué pasa si su producto escalar es próximo a cero? Sean $x=[1,-0.75]$ e $y=[0.3,0.3]$. Calcula el producto escalar  $x\cdot y$ y el ángulo que forman. ¿Qué conclusión puedes sacar?\newline Si dos vectores $x,y$ son unitarios, entonces  $-1\le x\cdot y\le 1$ (¿por qué?). En este caso, ¿qué podemos decir si $x\cdot y$ es aproximadamente $-1,1$ o cero?}

\item \lb{Sean $u,v$ dos vectores unitarios de  $\R^{n}$ que forman un ángulo de 60º. Calcula $\|2u+v\|$.}

\item \lb{Sena $u,v$ dos vectores de  $\R^n$ de norma $2$ y  $3$ respectivamente que forman un ángulo de  $60º$. ¿Qué ángulo forman los vectores $u$ y  $2u-v$?} 

\item \lb{Calcula $A+B,(A+B)^\intercal, AB, BA, (AB)^\intercal, A^\intercal B^\intercal$ y $B^\intercal A^\intercal$ para las matrices  \[
A=\begin{bmatrix} 
    1 & 0 & 3\\
    2 & 2 & 3\\
    3 & 0 & 3
\end{bmatrix}\quad\text{y} \quad B=\begin{bmatrix} 
    2 & 1 & 0\\
    1 & 2 & 0\\
    0 & 1 & 2
\end{bmatrix} 
\] } 
\item \lb{Prueba que no existe ninguna matriz $A$ tal que  $A^2=\begin{bmatrix} 
            0 & 1\\
            0 & 0
\end{bmatrix} $.}

\item \lb{¿Existen matrices reales no nula $2\times 2$ tales que $A\cdot A^\intercal=0$? ¿y si son matrices complejas?}

\item \lb{Sean $A,B$ matrices tales que  $I+AB$ es invertible y sea  $S$ la inversa de  $I+AB$. Prueba que  $I+BA$ también es invertible y su inversa es  $I-BSA$.} 

\item \lb{Sea $A$ una matriz  $n\times m$ y sea $B$ una matriz $m\times n$. Suponiendo que las matrices $I+AB$ y  $I+BA$ sea invertibles, prueba que se cumple la igualdad  \[
            (I+AB)^{-1}A=A(1-BA)^{-1}
\]Si $m$ es mucho más pequeño que  $n$, ¿cuál de las dos expresiones es más fácil de calcular desde el punto de vista computacional?}

\item \lb{Sea $A$ una matriz  $n\times p$ y $B$ una matriz $p\times m$. Si llamamos \textbf{flop} a una operación, ya sea una suma, resta, multiplicación o división, prueba que para calcular $AB$ son necesarios  $mn(2p-1)$ flops (haciendo la multiplicación de forma estándar). Si  $A$ es una matriz  $10\times 2$, $B$ una matriz $2\times 10$ y $C$ una matriz  $10\times 10$ y queremos calcular $ABC$, ¿qué es mejor desde el punto de vista computacional, calcular $(AB)C$ o $A(BC)$?} 

\item \lb{Dadas dos matrices cuadradas $A,B$, se define el  \textbf{conmutador} de $A,B$ como  \[
            [A,B]=AB-BA
\]Por una parte, el spin de un electrón se suele representar a través de las siguiente tres matrices, llamadas matrices de Pauli: \[
S_x=\dfrac{1}{2}\overline{h}\begin{bmatrix} 
    0 & 1\\
    1 & 0
\end{bmatrix} \quad S_y=\dfrac{1}{2}\overline{h}\begin{bmatrix} 
    0 & -j\\
    j & 0
\end{bmatrix} \quad S_z=\dfrac{1}{2}\overline{h}\begin{bmatrix} 
    1 & 0\\
    0 & -1
\end{bmatrix} 
\]donde $\overline{h}=\dfrac{h}{2\pi}$, con $h$ constante de Plank. Comprueba que \[
[S_x,S_y]=j\overline{h}S_z,\quad [S_y,S_z]= j\overline{h} S_x, \quad [S_z,S_x]=j\overline{h}S_y
\]y que \[
S_x ^2+S_y^2+S_z^2=\dfrac{3}{4}\overline{h}^2I_3
\] con $I_3$ la matriz identidad $3\times 3$. }

\item \lb{Si $A$ es una matriz simétrica, ¿son las matrices $B^\intercal AB,A+A^\intercal$ y $A-A^\intercal$ simétricas?}

\item \lb{Prueba que si $A$ es una matriz invertible simétrica, entonces  $A^{-1}$ es también simétrica.}

\item \lb{Sean $u_1,\dots,u_m$ vectores de $\mathbb{K}^n$ y supongamos que para ciertos escalares $x_1,\dots,x_m$ se tiene $x_1u_1+\cdots+x_mu_m=0$. Expresa esta última igualdad en forma matricial.}

\item \lb{Sean $u,v$ dos vectores no nulos vistos como matrices columna  $n\times 1$. Observa que $1+v^\intercal u$ es un escalar, que suponemos no nulo. Prueba que la matriz $I+uv^\intercal$ es no singular y su inversa es \[
            (I+uv^\intercal)^{-1}=I-\dfrac{uv^\intercal}{1+v^\intercal u}
\] } 
\item \lb{Sea $u$ un vector de $\R^n$ con $\|u\|=1$ y consideremos la matriz  $A=I-2uu^\intercal$. Prueba que $A$ es simétrica y  $A^2=I$.}

\item \lb{Sea $A$ la matriz dada por bloques  \[
A=\begin{bmatrix} 
    A_{11} & A_{12}\\
    A_{21} & A_{22}
\end{bmatrix} 
\]con $A_{11}$ invertible. Prueba que existen matrices $X,Y$ tales que  \[
A=\begin{bmatrix} 
    1 & 0\\
    X & 1
\end{bmatrix} \begin{bmatrix} 
    A_{11} & 0\\
    0 & S
\end{bmatrix} \begin{bmatrix} 
    I & Y\\
    0 & I
\end{bmatrix} 
\] donde $S=A_{22}-A_{21}A_{11}^{-1}A_{12}$ e $I$ es la matriz identidad del tamaño adecuado (la matriz  $S$ se denomina el  \textbf{complemento dde Schur} de $A_{11}$ ).}

\item \lb{Expresa la matriz $AB$ como suma de matrices de rango  $1$, donde  \[
A=\begin{bmatrix} 
    1 & -1 & 3\\
    2 & 1 & 4
\end{bmatrix},\qquad B=\begin{bmatrix} 
    1 & 2 & 3\\
    0 & 1 & 4\\
    1 & 0 & 5
\end{bmatrix}  
\] } 

\item \lb{Sea $u^\intercal=\left[ \dfrac{1}{3},\dfrac{2}{3},-\dfrac{2}{3} \right], v^\intercal=\left[ \dfrac{2}{3},\dfrac{1}{3},\dfrac{2}{3} \right]  $ y $w^\intercal=[a,b,c]$. Halla $a,b,c$ para que la matriz  $Q=[u,v,w]$ sea ortogonal de determinante 1.}

\item \lb{\textbf{(Traza de una matriz)} Dada una matriz cuadrada $A$, se define la \textbf{traza} de $A$ como $\mathrm{tr}(A)=\sum_{i=1}^{n} [A]_{ii}$, es decir, como la suma de los elementos de la diagonal principal. Prueba las siguientes propiedades:}
    \begin{enumerate}[label=\color{red}\textbf{\alph*)}]
        \item \db{$\mathrm{tr}(A+B)=\mathrm{tr}(A)+\mathrm{tr}(B)$.} 
        \item \db{$\mathrm{tr}(A)=\mathrm{tr}(A^\intercal)$.} 
        \item \db{$\mathrm{tr}(AB)=\mathrm{tr}(BA)$.} 
        \item \db{$\mathrm{tr}(P^{-1}AP)=\mathrm{tr}(A)$ con $P$ invertible.} 
    \end{enumerate}
\end{enumerate}
\end{document}

