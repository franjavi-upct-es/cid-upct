\documentclass{article}
\usepackage{fullpage}
\usepackage[utf8]{inputenc}
\usepackage{pict2e}
\usepackage{amsmath}
\usepackage{enumitem}
\usepackage{eurosym}
\usepackage{mathtools}
\usepackage{amssymb, amsfonts, latexsym, cancel}
\setlength{\parskip}{0.3cm}
\usepackage{graphicx}
\usepackage{fontenc}
\usepackage{slashbox}
\usepackage{setspace}
\usepackage{gensymb}
\usepackage{accents}
\usepackage{adjustbox}
\setstretch{1.35}
\usepackage{bold-extra}
\usepackage[document]{ragged2e}
\usepackage{subcaption}
\usepackage{tcolorbox}
\usepackage{xcolor, colortbl}
\usepackage{wrapfig}
\usepackage{empheq}
\usepackage{array}
\usepackage{parskip}
\usepackage{arydshln}
\graphicspath{ {images/} }
\renewcommand*\contentsname{\color{black}Índice} 
\usepackage{array, multirow, multicol}
\definecolor{lightblue}{HTML}{007AFF}
\usepackage{color}
\usepackage{etoolbox}
\usepackage{listings}
\usepackage{mdframed}
\setlength{\parindent}{0pt}
\usepackage{underscore}
\usepackage{hyperref}
\usepackage{tikz}
\usepackage{tikz-cd}
\usetikzlibrary{shapes, positioning, patterns}
\usepackage{tikz-qtree}
\usepackage{biblatex}
\usepackage{pdfpages}
\usepackage{pgfplots}
\usepackage{pgfkeys}
\addbibresource{biblatex-examples.bib}
\usepackage[a4paper, left=1cm, right=1cm, top=1cm,
bottom=1.5cm]{geometry}
\usepackage{titlesec}
\usepackage{titletoc}
\usepackage{tikz-3dplot}
\usepackage{kbordermatrix}
\usetikzlibrary{decorations.pathreplacing}
\newcommand{\Ej}{\textcolor{lightblue}{\underline{Ejemplo}}}
\setlength{\fboxrule}{1.5pt}

% Configura el formato de las secciones utilizando titlesec
\titleformat{\section}
{\color{red}\normalfont\LARGE\bfseries}
{Tema \thesection:}
{10 pt}
{}

% Ajusta el formato de las entradas de la tabla de contenidos
\addtocontents{toc}{\protect\setcounter{tocdepth}{4}}
\addtocontents{toc}{\color{black}}

\titleformat{\subsection}
{\normalfont\Large\bfseries\color{red}}{\thesubsection)}{1em}{\color{lightblue}}

\titleformat{\subsubsection}
{\normalfont\large\bfseries\color{red}}{\thesubsubsection)}{1em}{\color{lightblue}}

\newcommand{\bboxed}[1]{\fcolorbox{lightblue}{lightblue!10}{$#1$}}
\newcommand{\rboxed}[1]{\fcolorbox{red}{red!10}{$#1$}}

\DeclareMathOperator{\N}{\mathbb{N}}
\DeclareMathOperator{\Z}{\mathbb{Z}}
\DeclareMathOperator{\R}{\mathbb{R}}
\DeclareMathOperator{\Q}{\mathbb{Q}}
\DeclareMathOperator{\K}{\mathbb{K}}
\DeclareMathOperator{\im}{\imath}
\DeclareMathOperator{\jm}{\jmath}
\DeclareMathOperator{\col}{\mathrm{Col}}
\DeclareMathOperator{\fil}{\mathrm{Fil}}
\DeclareMathOperator{\rg}{\mathrm{rg}}
\DeclareMathOperator{\nuc}{\mathrm{nuc}}
\DeclareMathOperator{\dimf}{\mathrm{dimFil}}
\DeclareMathOperator{\dimc}{\mathrm{dimCol}}
\DeclareMathOperator{\dimn}{\mathrm{dimnuc}}
\DeclareMathOperator{\dimr}{\mathrm{dimrg}}
\DeclareMathOperator{\dom}{\mathrm{Dom}}
\DeclareMathOperator{\infi}{\int_{-\infty}^{+\infty}}
\newcommand{\dint}[2]{\int_{#1}^{#2}}

\newcommand{\bu}[1]{\textcolor{lightblue}{\underline{#1}}}
\newcommand{\lb}[1]{\textcolor{lightblue}{#1}}
\newcommand{\db}[1]{\textcolor{blue}{#1}}
\newcommand{\rc}[1]{\textcolor{red}{#1}}
\newcommand{\tr}{^\intercal}

\renewcommand{\CancelColor}{\color{lightblue}}

\newcommand{\dx}{\:\mathrm{d}x}
\newcommand{\dt}{\:\mathrm{d}t}
\newcommand{\dy}{\:\mathrm{d}y}
\newcommand{\dz}{\:\mathrm{d}z}
\newcommand{\dth}{\:\mathrm{d}\theta}
\newcommand{\dr}{\:\mathrm{d}\rho}
\newcommand{\du}{\:\mathrm{d}u}
\newcommand{\dv}{\:\mathrm{d}v}
\newcommand{\tozero}[1]{\cancelto{0}{#1}}
\newcommand{\lbb}[2]{\textcolor{lightblue}{\underbracket[1pt]{\textcolor{black}{#1}}_{#2}}}
\newcommand{\dbb}[2]{\textcolor{blue}{\underbracket[1pt]{\textcolor{black}{#1}}_{#2}}}
\newcommand{\rub}[2]{\textcolor{red}{\underbracket[1pt]{\textcolor{black}{#1}}_{#2}}}

\author{Francisco Javier Mercader Martínez}
\date{}
\renewcommand{\arraystretch}{1}
\setlength{\arraycolsep}{6pt}

\title{Álgebra Lineal\\Ejercicios Tema 4: Subespacios vectoriales, bases y coordenadas}

\begin{document}
\maketitle
\begin{enumerate}[label=\color{red}\textbf{\arabic*)}]
    \item \lb{Determina cuáles de ls siguientes subonjunto de $\R^3$ son subespacios vectoriales:}
        \begin{enumerate}[label=\color{red}\textbf{\alph*)}]
            \item \db{$\{(x,y,z|x=1)\} $} 

                No es un subespacio vectorial porque no contiene al $(0,0,0)$.
            \item \db{$\{(x,y,z)|xyz=0\} $} 

                $W=\{(x,y,z)|xyz=0\} $ 

                Tomamos $(x_1,y_1,z_1),(x_2,y_2,z_2)\in W\longrightarrow \underbrace{(x_1,y_1,z_1)+(x_2,y_2,z_2)}_{(x_1+x_2,y_1+y_2,z_1+z_2)}\in W$ 

                $(x_1+x_2)\cdot (y_1+y_2)\cdot (z_1+z_2)=0$

                Claramente no es un subespacio vectorial porque \textit{"las tres componentes  están multiplicando"} 

                Para verlo más claramente: \[
                \begin{array}{l}
                    v_1=(1,1,0)\in W\text{ ya que }1\cdot 1\cdot 0=0\\
                    v_2=(0,0,1)\in W\text{ ya que }0\cdot 0\cdot 1=0
                \end{array}
                \]
                Sin embargo, $v_1+v_2=(1,1,1)\notin W$ ya que \[
                1\cdot 1\cdot 1=1\neq 0.
                \] 
            \item \db{$\{(x,y,z)|x+y+z=0\} $} 

                $(x_1,y_1,z_1),(x_2,y_2,z_2)\in W\longrightarrow (x_1+x_2,y_1+y_2,z_1+z_2)\in W$

                $(x_1+x_2)+(y_1+y_2)+(z_1+z_2)=\underbrace{(x_1+y_1+z_1)}_{=0}+\underbrace{(x_2+y_2+z_2)}_{=0}=0$

                Sean ahora $(x_1,y_1,z_1)\in W$ y $\alpha\in \R\longrightarrow \alpha(x_1,y_1,z_1)\in W$

                $\begin{array}{l}
                    \alpha(x_1,y_1,z_1)=(\alpha x_1,\alpha y_1,\alpha z_1)\\
                    \alpha x_1+\alpha y_1+\alpha z_1=\alpha\underbrace{(x_1+y_1+z_1)}_{=0}=0
                \end{array}$
        \end{enumerate}
    \item \lb{Comprueba que, en $\R^3$, el subespacio generado por los vectores $(1,2,1)$ y  $(6,1,-16)$ coincide con el subespacio generador por los vectores  $(-3,7,20)$ y  $(4,9,6)$.} 

        Para verificar si el subespacio generador po $\{(1,2,1),(6,1,-16)\} $ coincide con el generado por $\{(-3,7,20),(4,9,6)\} $ en $\R^3$, debemos comprobar si cada conjunto de vectores puede ser expresado como una combinación lineal de los vectores del otro conjunto. Esto implica que ambos subconjuntos generan el mismo campo vectorial.

        Paso a seguir:
        \begin{enumerate}[label=\arabic*)]
            \item \textbf{Matriz ampliada:} Consideramos todos los vectores como filas de una matriz y verificamos si los vectores de un conjunto son combinación lineal de los del otro. Esto se hace calculando las filas escalonadas (rango). \[
            A_1=\begin{bmatrix} 
                1 & 2 & 1\\
                6 & 1 & -16\\
                -3 & 7 & 20\\
                4 & 9 & 6
            \end{bmatrix} 
            \]  
            Reducimos la matriz a su forma escalonada para identificar si el rango es igual a 2 (dimensión de un plano en $\R^3$).
        \item \textbf{Dependencia lineal:} Si se confirma que el rango es 2, probamos si los vectores de un conjunto pertenecen al subespacio generado por los vectores del otro. Para esto, verificamos si existe una relación lineal. 

            $\begin{bmatrix} 
                1 & 2 & 1 \\
                6 & 1 & -16\\
                -3 & 7 & 20\\
                4 & 9 & 6
            \end{bmatrix}\xrightarrow[\begin{subarray}{l}
                F_3\to F_3+3F_1\\
                F_4\to F_4-4F_1
            \end{subarray}]{F_2\to F_2-6F_1}\begin{bmatrix} 
                1 & 2 & 1\\
                0 & -11 & -22\\
                0 & 13 & 23\\
                0 & 1 & 2
            \end{bmatrix}\xrightarrow{F_2\leftrightarrow F_4}\begin{bmatrix} 
                1 & 2 & 1\\
                0 & 1 & 2\\
                0 & 13 & 23\\
                0 & -11 & -22
            \end{bmatrix}\xrightarrow[\begin{subarray}{l}
                F_3\to F_3-13F_2\\
                F_4\to F_4-F_2
            \end{subarray}]{F_1\to F_1-2F_2}\begin{bmatrix} 
                1 & 0 & 3\\
                0 & 1 & 2\\
                0 & 0 & -3\\
                0 & 0 & 0
            \end{bmatrix}\xrightarrow{F_3\to -\frac{1}{3} F_3}\begin{bmatrix} 
                1 & 0 & -3\\
                0 & 1 & 2\\
                0 & 0 & 1\\
                0 & 0 & 0
            \end{bmatrix}\xrightarrow[F_2\to F_2-2F_3]{F_1\to F_1+3F_3}\begin{bmatrix} 
                1 & 0 & 0\\
                0 & 1 & 0\\
                0 & 0 & 1\\
                0 & 0 & 0
            \end{bmatrix}$

            La matriz tiene rango 3, lo cual indica que los cuatro vectores son linealmente independientes. Por lo tanto, los subespacios generados por los conjuntos $\{(1,2,1),(6,1,-16)\} $ y $\{(-3,7,20),(4,9,6)\} $ \textbf{no coinciden}, ya que el espacio generado por los cuatro vectores abarca todo $\R^3$. 
            
        \end{enumerate}

\item \lb{Usando determinantes, halla una base del subespacio $W$ del ejercicio anterior que esté contenida en el conjunto generador dado.}

    \item \lb{Sean $v_1,v_2,v_3\in \mathbb{K}^n$ con $n\ge 3$. Pon un ejemplo de una matriz $A$ tal que los sistemas de ecuaciones $Ax=v_1$ y $Ax=v_2$ tengan solución pero el sistema $Ax=v_3$ no lo tenga.}

        
    \item \lb{Completa la frase "el vector $b$ pertenee a $\mathrm{Col}(A)$ cuando \dots\dots tiene solución".}


    \item \lb{Cierto o falso: "si el vector cero pertenece a $\mathrm{Fil}(A)$, entonces las filas de $A$ son linealmente dependientes".}

    \item \lb{Halla una base del subespacio vectorial $W$ de $\R^5$ generado por los vectores \[
                (-1,0,0,1,-2)\quad(2,1,0,-1,2)\quad(1,3,1,0,-1)\quad(0,2,1,0,-1)\quad(3,1,0,-2,4).
    \] } 
    \item \lb{Amplía el conjunto $\{(1,-1,1)\} $ a una base ortonormal de $\R^3$.}
    \item \lb{En $\R^5$, si las ecuaciones del subespacio $U$ son $\left\{ \begin{array}{l}
        x+y+z+t+u=0\\
        x-y+z-t+u=0
    \end{array} \right\} $, encuentra las de $U^{\perp}$.}

    \item \lb{Halla una base de siguientes subespacios vectoriales de $\R^4$:}
        \begin{enumerate}[label=\color{red}\textbf{\alph*)}]
            \item \db{$\{(x,y,z,t)|x=y=z=t\} $} 
            \item \db{$\{(x,y,z,t)|x+y+z+t=0\} $} 
        \end{enumerate}
    \item \lb{Dados los subespacios \[
    U=\{(a,b,a-b,a-2b):a,b\in \R\} \qquad W=\{(x,x-y,-x+y,y):x,y\in \R\} 
    \]calcula una base de cada uno de ellos y determina si están en suma directa.}

    \item \lb{Dada la matriz $U=\begin{bmatrix} 
                1 & 0 & 1 & 0 & 1\\
                0 & 1 & 0 & 1 & 0
    \end{bmatrix} $ calcula:}
    \begin{enumerate}[label=\color{red}\textbf{\alph*)}]
        \item \db{tres bases distintas de $\mathrm{Col}(U)$.} 
        \item \db{dos bases distintas de $\mathrm{Fil}(U)$.} 
    \end{enumerate}
    \item \lb{Sea $S$ un conjunto de 6 vectores de $\R^4$. De las opciones entre paréntesis escoge la(s) correcta(s):}
        \begin{enumerate}[label=\color{red}\textbf{\alph*)}]
            \item \db{$S$ (es)(no es)(no necesariamente es) un conjunto generadr de $\R^4$.}
            \item \db{$S$ (es)(no es)(puede ser) un conjunto linealmente independiente.}
            \item \db{un subconjunto de $S$ con 4 vectores (es)(no es)(puede ser) una base de $\R^4$.} 
        \end{enumerate}
    \item \lb{Determina los valores que faltan en la siguiente matriz sabiendo que tiene rango 1: \[
    \begin{bmatrix} 
        7 & - & -\\
        - & 8 & -\\
        - & 12 & 6\\
        - & - & 2\\
        21 & 6 & -
    \end{bmatrix} 
    \] }
    \item \lb{Dadas dos matrices $A$ y $B$, prueba las siguientes afirmaciones:}
        \begin{enumerate}[label=\color{red}\textbf{\alph*)}]
            \item \db{$\mathrm{rango}(A+B)\le \mathrm{rango}(A)+\mathrm{rango}(B)$}
            \item \db{$\mathrm{rango}(AB)\le \mathrm{rango}(A),\mathrm{rango}(B)$} 
        \end{enumerate}
    \item \lb{Sea $A$ una matriz $m\times n$ de rango $k$. Prueba que existe una matriz  $B$ de tamaño $m\times k$ y rango $k$ y una matriz $C$ de tamaño $k\times n$ tales que $A=BC$ (esta factorización se llama \textbf{full rank factorization}). Si $k$ es mucho menor que $n$ y $m$, razona si es mejor, en términos de memoria, almacenar $A$ o almacenar $B$ y $C$.} 

    \item \lb{Calcula la "full rank factorization" para la matriz \[
    \begin{bmatrix} 
        1 & 0 & 2 & 3\\
        -1 & -3 & 1 & 0
    \end{bmatrix} 
    \] } 
    \item \lb{Utiliza la "full rank factorization" y el ejercicio 15 para ver que una matriz $A$ tiene rango $k$ si y solo si se puede expresar como suma de $k$ matrices de rango 1, pero no se puede expresar como suma de menos de $k$ matrices de rango 1. Expresa la matriz del ejercicio anterior como suma de 2 matrices de rango 1.}

    \item \lb{Se consideran las matrices \[
    A=\begin{bmatrix} 
        1 & 3 & 2\\
        0 & 1 & 1\\
        1 & 3 & 2
    \end{bmatrix} \qquad B=\begin{bmatrix} 
        1 & 3 & 2\\
        0 & 1 & 1\\
        0 & 0 & 0
    \end{bmatrix} 
    \] en donde $B$ se obtiene de $A$ restando la fila uno a la fila tres. ¿Qué relación hay entre los cuatro subespacios fundamentales de las dos matrices? Calcula una base para cada uno de ellos.}
    \item \lb{Pon un ejemplo de una matriz cuadradada $A$ en la que $\mathrm{Col}(A)=\mathrm{Fil}(A)$ y otro en el que $\mathrm{Col}(A)\neq \mathrm{Fil}(A)$.}

    \item \lb{Sea $A$ una matriz $10\times 10$ que cumple $A^2=0$. Prueba que $\mathrm{Col}(A)\subseteq \mathrm{Nuc}(A)$ y en consecuencia, el rango de $A$ es menor o igual que 5.}

    \item \lb{Sea $A$ una matriz cuadrada invertible. Calcula una base para cada uno de los subespacios fundamentales de las matrices $A$ y $B=\begin{bmatrix} 
                A & A
    \end{bmatrix} $.} 
    \item \lb{Sin calcular la matriz $A$, encuentra bases para sus cuatro subespacios fundamentales \[
    A=\begin{bmatrix} 
        1 & 0 & 0\\
        6 & 1 & 0\\
        9 & 8 & 1
    \end{bmatrix}\begin{bmatrix} 
        1 & 2 & 3 & 4\\
        0 & 1 & 2 & 3\\
        0 & 0 & 1 & 2
    \end{bmatrix}  
    \] } 
\item \lb{Halla una base del subespacio de $\R^4$ dado por el núcleo de $A$, donde \[
A=\begin{bmatrix} 
    3 & 0 & 1 & 2\\
    1 & 1 & 0 & 1 
\end{bmatrix} 
\]Comprueba que el vector $(1,1,1-2)$ pertenece a $\mathrm{Nuc}(A)$ y calcula sus coordenadas respecto de la base obtenida.}

\item \lb{Dadas las siguientes bases de $\R^4$ \[
\begin{array}{c}
    \mathcal{B}_1=\{v_1=(1,1,0,0),v_2=(0,1,0,0),v_3=(0,0,1,1),v_4=(0,0,0,1)\} \\
    \mathcal{B}_2=\{w_1=(1,2,0,0),w_2=(0,1,2,-1),w_3=(0,1,1,1),w_4=(0,1,2,0)\}
\end{array}
\]encuentra la matriz de cambio de base de $\mathcal{B}_1$ y $\mathcal{B}_2$ y la del cambio inverso. ¿Qué coordenadas tiene el vector $3v_1-v_3+2v_2$ con respecto a la base $\mathcal{B}_2$? ¿Qué coordenadas tiene el vector $3w_1-w_3+2w_2$ con respecto a la base $\mathcal{B}_1$?} 

\item \lb{Calcula una base y unas ecuaciones implícitas de $U+V$, donde  \[
\begin{array}{c}
    U=\left<(1,2,3,4,5),(5,4,3,2,1),(-1,0,1,2,3) \right>\\
    V=\left\{ (x,y,z,r,s): \begin{array}{l}
        3x+r=2y\\
        3x+s=2y
    \end{array} \right\} 
\end{array}
\] } 
\end{enumerate}
\end{document}
