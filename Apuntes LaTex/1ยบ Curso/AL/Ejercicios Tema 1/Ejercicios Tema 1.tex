\documentclass{article}
\usepackage{fullpage}
\usepackage[utf8]{inputenc}
\usepackage{pict2e}
\usepackage{amsmath}
\usepackage{enumitem}
\usepackage{eurosym}
\usepackage{mathtools}
\usepackage{amssymb, amsfonts, latexsym, cancel}
\setlength{\parskip}{0.3cm}
\usepackage{graphicx}
\usepackage{fontenc}
\usepackage{slashbox}
\usepackage{setspace}
\usepackage{gensymb}
\usepackage{accents}
\usepackage{adjustbox}
\setstretch{1.35}
\usepackage{bold-extra}
\usepackage[document]{ragged2e}
\usepackage{subcaption}
\usepackage{tcolorbox}
\usepackage{xcolor, colortbl}
\usepackage{wrapfig}
\usepackage{empheq}
\usepackage{array}
\usepackage{parskip}
\usepackage{arydshln}
\graphicspath{ {images/} }
\renewcommand*\contentsname{\color{black}Índice} 
\usepackage{array, multirow, multicol}
\definecolor{lightblue}{HTML}{007AFF}
\usepackage{color}
\usepackage{etoolbox}
\usepackage{listings}
\usepackage{mdframed}
\setlength{\parindent}{0pt}
\usepackage{underscore}
\usepackage{hyperref}
\usepackage{tikz}
\usepackage{tikz-cd}
\usetikzlibrary{shapes, positioning, patterns}
\usepackage{tikz-qtree}
\usepackage{biblatex}
\usepackage{pdfpages}
\usepackage{pgfplots}
\usepackage{pgfkeys}
\addbibresource{biblatex-examples.bib}
\usepackage[a4paper, left=1cm, right=1cm, top=1cm,
bottom=1.5cm]{geometry}
\usepackage{titlesec}
\usepackage{titletoc}
\usepackage{tikz-3dplot}
\usepackage{kbordermatrix}
\usetikzlibrary{decorations.pathreplacing}
\newcommand{\Ej}{\textcolor{lightblue}{\underline{Ejemplo}}}
\setlength{\fboxrule}{1.5pt}

% Configura el formato de las secciones utilizando titlesec
\titleformat{\section}
{\color{red}\normalfont\LARGE\bfseries}
{Tema \thesection:}
{10 pt}
{}

% Ajusta el formato de las entradas de la tabla de contenidos
\addtocontents{toc}{\protect\setcounter{tocdepth}{4}}
\addtocontents{toc}{\color{black}}

\titleformat{\subsection}
{\normalfont\Large\bfseries\color{red}}{\thesubsection)}{1em}{\color{lightblue}}

\titleformat{\subsubsection}
{\normalfont\large\bfseries\color{red}}{\thesubsubsection)}{1em}{\color{lightblue}}

\newcommand{\bboxed}[1]{\fcolorbox{lightblue}{lightblue!10}{$#1$}}
\newcommand{\rboxed}[1]{\fcolorbox{red}{red!10}{$#1$}}

\DeclareMathOperator{\N}{\mathbb{N}}
\DeclareMathOperator{\Z}{\mathbb{Z}}
\DeclareMathOperator{\R}{\mathbb{R}}
\DeclareMathOperator{\Q}{\mathbb{Q}}
\DeclareMathOperator{\K}{\mathbb{K}}
\DeclareMathOperator{\im}{\imath}
\DeclareMathOperator{\jm}{\jmath}
\DeclareMathOperator{\col}{\mathrm{Col}}
\DeclareMathOperator{\fil}{\mathrm{Fil}}
\DeclareMathOperator{\rg}{\mathrm{rg}}
\DeclareMathOperator{\nuc}{\mathrm{nuc}}
\DeclareMathOperator{\dimf}{\mathrm{dimFil}}
\DeclareMathOperator{\dimc}{\mathrm{dimCol}}
\DeclareMathOperator{\dimn}{\mathrm{dimnuc}}
\DeclareMathOperator{\dimr}{\mathrm{dimrg}}
\DeclareMathOperator{\dom}{\mathrm{Dom}}
\DeclareMathOperator{\infi}{\int_{-\infty}^{+\infty}}
\newcommand{\dint}[2]{\int_{#1}^{#2}}

\newcommand{\bu}[1]{\textcolor{lightblue}{\underline{#1}}}
\newcommand{\lb}[1]{\textcolor{lightblue}{#1}}
\newcommand{\db}[1]{\textcolor{blue}{#1}}
\newcommand{\rc}[1]{\textcolor{red}{#1}}
\newcommand{\tr}{^\intercal}

\renewcommand{\CancelColor}{\color{lightblue}}

\newcommand{\dx}{\:\mathrm{d}x}
\newcommand{\dt}{\:\mathrm{d}t}
\newcommand{\dy}{\:\mathrm{d}y}
\newcommand{\dz}{\:\mathrm{d}z}
\newcommand{\dth}{\:\mathrm{d}\theta}
\newcommand{\dr}{\:\mathrm{d}\rho}
\newcommand{\du}{\:\mathrm{d}u}
\newcommand{\dv}{\:\mathrm{d}v}
\newcommand{\tozero}[1]{\cancelto{0}{#1}}
\newcommand{\lbb}[2]{\textcolor{lightblue}{\underbracket[1pt]{\textcolor{black}{#1}}_{#2}}}
\newcommand{\dbb}[2]{\textcolor{blue}{\underbracket[1pt]{\textcolor{black}{#1}}_{#2}}}
\newcommand{\rub}[2]{\textcolor{red}{\underbracket[1pt]{\textcolor{black}{#1}}_{#2}}}

\author{Francisco Javier Mercader Martínez}
\date{}
\title{Álgebra Lineal\\Ejercicios Tema 1: Números complejos}

\begin{document}
\maketitle
\begin{enumerate}[label=\color{red}\textbf{\arabic*)}]
    \item \lb{Calcula las siguientes operaciones de números complejos.}
        \begin{enumerate}[label=\color{red}\textbf{\textbf{\alph*)}}]
            \item \db{$(-2+j)+\left( -\dfrac{1}{2}-3j \right) $} 

                $(-2+j)+\left( -\dfrac{1}{2}-3j \right)=-\dfrac{5}{2-2j} $
            \item \db{$(-2-3j)-\left( -3+\dfrac{1}{2}j \right) $}
                
                $(-2-3j)-\left( -3+\dfrac{1}{2}j \right)=1-\dfrac{7}{2}j $
            \item \db{$-2\cdot (-1-j)-3\cdot (-2+j)+2\cdot (1-2j)$} 

                $-2\cdot (-1-j)-3\cdot (-2+j)+2\cdot (1-2j)=2+2j+6-3j+2-4j=10-5j$
            \item \db{$j\cdot (-1+j)$} 

                $j\cdot (-1+j)=-j-1$
            \item \db{$(-2-j)\cdot \left( -3+\dfrac{1}{2}j \right) $} 

                $(-2-j)\cdot \left( -3+\dfrac{1}{2}j \right) =6-j+3j+\dfrac{1}{2}=\dfrac{13}{2}+2j$
            \item \db{$\dfrac{1}{-1-2j}$} 

                $\dfrac{1}{-1-2j}=\dfrac{1}{-1-2j}\cdot \dfrac{-1+2j}{-1+2j}=\dfrac{-1+2j}{5}=-\dfrac{1}{5}+\dfrac{2}{5}j $
            \item \db{$\dfrac{-j}{2-3j}$} 

                $\dfrac{-j}{2-3j}=\dfrac{-j}{2-3j}\cdot \dfrac{2+3j}{2+3j}=\dfrac{-j\cdot (2-3j)}{13}=\dfrac{-2j+3}{13}=\dfrac{3}{13}-\dfrac{2}{13}j$
            \item \db{$\dfrac{-1-j}{-2+j}$} 

                $\dfrac{-1-j}{-2+j}=\dfrac{-1-j}{-2+j}\cdot \dfrac{-2-j}{-2-j}=\dfrac{(-1-j)\cdot (-2-j)}{5}=\dfrac{1+3j}{5}=\dfrac{1}{5}+\dfrac{1}{3}j$

            \item \db{$\dfrac{1-j}{j} -\dfrac{j}{1-j} $} 

                $\dfrac{1-j}{j}-\dfrac{j}{1-j}=-1-j+\dfrac{1}{2}-\dfrac{1}{2}j=-\dfrac{1}{2}-\dfrac{3}{2}j$

                $\begin{array}{l}
                    \dfrac{1-j}{j}=\dfrac{1-j}{j}\cdot \dfrac{-j}{-j}=\dfrac{(1-j)\cdot (-j)}{1}=-1-j\\
                    \dfrac{j}{1-j}=\dfrac{j}{1-j}\cdot \dfrac{1+j}{1+j}=\dfrac{j\cdot (1+j)}{2}=\dfrac{-1+j}{2}=-\dfrac{1}{2}+\dfrac{1}{2}j
                \end{array}$
        \end{enumerate}
    \item \lb{Obtén las formas polares y trigonométricas de los siguientes números complejos:}
        \begin{enumerate}[label=\color{red}\textbf{\textbf{\textbf{\alph*)}}}]
            \item \db{$1+j$} 

                $\begin{cases}
                    |z| = \sqrt{1^2+1^2}=\sqrt{2}\\
                    \theta=\arctan\left( \dfrac{1}{1} \right) =\dfrac{\pi}{4}
                \end{cases}\longrightarrow \sqrt{2}\cdot e^{\frac{\pi}{4} j}  $
            \item \db{$-j$} 

                $\begin{cases}
                    |z|=\sqrt{0^2+(-1)^2}=1\\
                    \theta=\arctan(-j)=-\dfrac{\pi}{2}\equiv \dfrac{3\pi}{2}
                \end{cases}\longrightarrow 1\cdot e^{\frac{3\pi}{2} j} $
            \item \db{$-1+\sqrt{3} j$} 

                $\begin{cases}
                    |z|=\sqrt{(-1)^2+(\sqrt{3})^2 } =2\\
                    \theta=\arctan\left( \frac{\sqrt{3}}{-1}  \right) =\dfrac{2\pi}{3}
                \end{cases}\longrightarrow 2e^{\frac{2\pi}{3} j} $
            \item \db{$2\sqrt{3} -2j$} 

                $\begin{cases}
                   |z|=\sqrt{(2\sqrt{3})^2+(-2)^2 }  = 4\\
                   \theta=\arctan\left( -\dfrac{2}{2\sqrt{3} } \right) =-\dfrac{\pi}{6}\equiv \dfrac{11\pi}{6}
                \end{cases}\longrightarrow 4e^{\frac{11\pi}{6} j} $
            \item \db{$-1-j$} 

                $\begin{cases}
                    |z|=\sqrt{(-1)^2+(-1)^2}=\sqrt{2}\\
                    \theta=\arctan\left( \dfrac{-1}{-1}  \right) =-\dfrac{3\pi}{4}\equiv \dfrac{5\pi}{4}
                \end{cases}\longrightarrow \sqrt{2} e^{\frac{5\pi}{4} j} $
            \item \db{$-2+j$} 

                $\begin{cases}
                    |z|=\sqrt{(-2)^2+1^2}=\sqrt{5}\\
                    \theta=\arctan\left( -\dfrac{1}{2} \right) \simeq2.6779
                \end{cases}\longrightarrow \sqrt{5}e^{2.6779j}  $
        \end{enumerate}
    \item \lb{Obtén la forma binómica de los siguientes números complejos:}
        \begin{enumerate}[label=\color{red}\textbf{\textbf{\textbf{\alph*)}}}]
            \item \db{$2_{\frac{\pi}{3} }$} 

            $2\cdot \left( \cos\left( \dfrac{\pi}{3} \right) +j\sin\left( \dfrac{\pi}{3} \right)  \right)=1+\sqrt{3} j $
            \item \db{$1_\pi$} 

                $\cos(\pi)+j\sin(\pi)=-1$
            \item \db{$3_{\frac{5\pi}{4} }$} 

                $3\cdot \left( \cos\left( \dfrac{5\pi}{4} \right) +j\sin\left( \dfrac{5\pi}{4} \right)  \right)=-\dfrac{3\sqrt{2} }{2} -\dfrac{3\sqrt{2} }{2}j  $
        \end{enumerate}
    \item \lb{Calcula las siguientes operaciones de números complejos, expresando el resultado en forma exponencial:}
        \begin{enumerate}[label=\color{red}\textbf{\textbf{\textbf{\alph*)}}}]
            \item \db{$2_{\frac{5\pi}{3} }\cdot 3_{\frac{\pi}{2}}$} 
                
                $2_{\frac{5\pi}{3} }\cdot 3_{\frac{\pi}{2} }=(1-\sqrt{3}j)\cdot 3j=3\sqrt{3} +3j $

                $\begin{array}{l}
                    2_{\frac{5\pi}{3} }=2\cdot \left( \cos\left( \dfrac{5\pi}{3} \right) +j\sin\left( \dfrac{5\pi}{3} \right)  \right) =1-\sqrt{3} j\\
                    3_{\frac{\pi}{2} }=3\cdot \left( \cos\left( \dfrac{\pi}{2} \right) +j\sin\left( \dfrac{\pi}{2} \right)  \right) =3j
                \end{array}$
            \item \db{$1 _{\frac{7\pi}{4} }\cdot 2_{\frac{7\pi}{3} }$} 

                $1_{\frac{7\pi}{4} }\cdot 2_{\frac{7\pi}{3} }=\left( \dfrac{\sqrt{2} }{2}-\dfrac{\sqrt{2} }{2}j \right) \cdot (1+\sqrt{3} j)=\dfrac{\sqrt{6} +\sqrt{2} }{2}+\dfrac{\sqrt{6} -\sqrt{2} }{2}j$

                $\begin{array}{l}
                    1_{\frac{7\pi}{4} }=\cos\left( \dfrac{7\pi}{4} \right) + j\sin\left( \dfrac{7\pi}{4} \right) =\dfrac{\sqrt{2} }{2}-\dfrac{\sqrt{2} }{2}j\\
                    2_{\frac{7\pi}{3} }=2\cdot \left( \cos\left( \dfrac{7\pi}{3} \right) +j\sin\left( \dfrac{7\pi}{3} \right)  \right) =1+\sqrt{3} j
                \end{array}$
            \item \db{$2\left( \cos\dfrac{\pi}{2}+j\sin\dfrac{\pi}{2} \right) \cdot 3_{\frac{11\pi}{6} }$} 

                $2\left( \cos\dfrac{\pi}{2} +j\sin\dfrac{\pi}{2}  \right) \cdot 3_{\frac{11\pi}{6} } =2j\cdot \left( \dfrac{2\sqrt{3} }{2}-\dfrac{3}{2}j \right) =3+3\sqrt{3} j$

                $\begin{array}{l}
                2\left( \cos\dfrac{\pi}{2} +j\sin\dfrac{\pi}{2}  \right) =2j\\
                3_{\frac{11\pi}{6} }=3\cdot \left( \cos \dfrac{11\pi}{6}+j\sin \dfrac{11\pi}{6} \right) =\dfrac{2\sqrt{3} }{2}-\dfrac{3}{2}j
                \end{array}$

            \item \db{$\dfrac{4_{\frac{5\pi}{2} }}{2_{\frac{2\pi}{3} }} $} 

                $\dfrac{4_{\frac{5\pi}{2} }}{2_{\frac{2\pi}{3} }}=\dfrac{4j}{-1+\sqrt{3}j }=\dfrac{4j}{-1+\sqrt{3}j }\cdot \dfrac{-1-\sqrt{3}j }{-1-\sqrt{3} j}=\dfrac{4j\cdot (-1-\sqrt{3}j) }{4}=\dfrac{4\sqrt{3} -4j}{4}=\sqrt{3}-j $

                $\begin{array}{l}
                    4_{\frac{5\pi}{2} }=4\cdot \left( \cos \dfrac{5\pi}{2}+j\sin \dfrac{5\pi}{2} \right) = 4j\\
                    2_{\frac{2\pi}{3} }=2\cdot \left( \cos \dfrac{2\pi}{3}+j\sin \dfrac{2\pi}{3} \right) = -1+\sqrt{3}j 
                \end{array}$
            \item \db{$\dfrac{1_{\pi}}{2_{\frac{7\pi}{6} }}$} 

                $\dfrac{1_{\pi}}{2_{\frac{7\pi}{6} }}= \dfrac{-1}{-\sqrt{3} -j}=\dfrac{-1}{-\sqrt{3}-j }\cdot \dfrac{-\sqrt{3}+j }{-\sqrt{3} +j}=\dfrac{(-1)\cdot (-\sqrt{3}+j) }{4}=\dfrac{\sqrt{3}-j }{4}=\dfrac{\sqrt{3} }{4}-\dfrac{1}{4}j$ 

                $\begin{array}{l}
                    1_{\pi}=\cos(\pi)+j\sin(\pi)=-1\\
                    2_{\frac{7\pi}{6} }=2\cdot \left( \cos \dfrac{7\pi}{6}+j\sin \dfrac{7\pi}{6} \right) = -\sqrt{3} - j
                \end{array}$
            \item \db{$\dfrac{12\left( \cos\frac{5\pi}{4} +j\sin\frac{5\pi}{4}  \right) }{8_{\frac{5\pi}{4} }}$}

                $\dfrac{12\cdot \left( \cos\frac{5\pi}{4} +j\sin \frac{5\pi}{4}   \right) }{5_{\frac{5\pi}{4} }}=\dfrac{12\cdot \cancel{ \left( \cos\frac{5\pi}{4}+j\sin \frac{5\pi}{4}   \right) }}{8\cdot \cancel{\left( \cos\frac{5\pi}{4} +j\sin \frac{5\pi}{4}  \right) } }=\dfrac{12}{8}=\dfrac{3}{2}$
        \end{enumerate}
    \item \lb{Calcula las siguientes potencias de números complejos:}
        \begin{enumerate}[label=\color{red}\textbf{\textbf{\textbf{\alph*)}}}]
            \item \db{$(1-\sqrt{3} j)^6$}

                $(1-\sqrt{3} j)^6=64$
            \item \db{$(1-j)^8$} 

                $(1-j)^8=16$
            \item \db{$(-\sqrt{3} +j)^{10}$} 

                $(-\sqrt{3}+j)^{10}=512+886.81j $
            \item \db{$\left( \dfrac{1-j}{1+j} \right)^5 $} 

                $\left( \dfrac{1-j}{1+j} \right)^5=\lb{(\ast)}= (-j)^5=-j $ 

                $\lb{(\ast)=}\dfrac{1-j}{1+j}=\dfrac{1-j}{1+j}\cdot \dfrac{1-j}{1-j}=\dfrac{-2j}{2}=-j $
        \end{enumerate}
    \item \lb{Expresa los siguientes números complejos en forma binómica y en forma exponencial:}
        \begin{enumerate}[label=\color{red}\textbf{\textbf{\textbf{\alph*)}}}]
            \item \db{$(1+j)^3$}

                $(1+j)^3=-2+2j$

                $\begin{cases}
                    |z|=\sqrt{(-2)^2+2^2}=2\sqrt{2}  \\
                    \theta=\arctan\left( -2+2j \right) = \dfrac{3\pi}{4}
                \end{cases}\longrightarrow 2\sqrt{2}\cdot e^{\frac{3\pi}{4}j}  $
            \item \db{$j^5+j^{16}$} 

                $j^5+j^{16}=1+j$ 

                $\begin{cases}
                    |z|=\sqrt{1^2+1^2}=\sqrt{2}\\
                    \theta=\arctan(1+j)=\dfrac{\pi}{4}
                \end{cases}\longrightarrow \sqrt{2}\cdot e^{\frac{\pi}{4} j}  $
            \item \db{$1+3e^{j\pi} $} 

                $1+3e^{j\pi}=1+3\cdot \left( \cos\pi+j\sin\pi \right) =1-3=-2$ 

                $\begin{cases}
                    |z|=\sqrt{(-2)^2+0^2}=2\\
                    \theta=\arctan(-2)=\pi
                \end{cases}\longrightarrow 2e^{\pi j} $
            \item \db{$\dfrac{2+3j}{3-4j}$} 

                $\dfrac{2+3j}{3-4j}=\dfrac{2+3j}{3-4j}\cdot \dfrac{3+4j}{3+4j}=\dfrac{-6+17j}{25}=-\dfrac{6}{25}+\dfrac{17}{25}j$

                $\begin{cases}
                    |z|=\sqrt{\left( -\dfrac{6}{25} \right) +\left( \dfrac{17}{25} \right) }=\dfrac{\sqrt{13} }{5}\\
                    \theta=\arctan\left( -\dfrac{6}{25}+\dfrac{17}{25}j \right) = 1.91
                \end{cases}\longrightarrow \dfrac{\sqrt{13} }{5}\cdot e^{1.91j} $
            \item \db{$2_{\frac{3\pi}{2} }+j$} 

                $2_{\frac{3\pi}{2} }+j=2\cdot \left( \cos \dfrac{3\pi}{2}+j\sin\dfrac{3\pi}{2}  \right)+j=-2j+j=-j $

                $\begin{cases}
                    |z|=\sqrt{0^2+(-1)^2}=1\\
                    \theta=\arctan(-j)=-\dfrac{\pi}{2}\equiv \dfrac{3\pi}{2}
                \end{cases}\longrightarrow 1\cdot e^{\frac{3\pi}{2} j} $
            \item \db{$\dfrac{1}{j}$} 

                $\dfrac{1}{j}=\dfrac{1}{j}\cdot \dfrac{-j}{-j}=-j$ 

                $\begin{cases}
                    |z|=\sqrt{0^2+(-1)^2}=1\\
                    \theta=\arctan(-j)=-\dfrac{\pi}{2}\equiv \dfrac{3\pi}{2}
                \end{cases}\longrightarrow 1\cdot e^{\frac{3\pi}{2} j} $
        \end{enumerate}
    \item \lb{Calcula el módulo y el argumento principal de los siguientes números complejos:}
        \begin{enumerate}[label=\color{red}\textbf{\textbf{\textbf{\alph*)}}}]
            \item \db{$2j$} 

                $\begin{cases}
                    |z|=\sqrt{0^2+2^2}=2\\
                    \theta=\arctan(2j)=\dfrac{\pi}{2}
                \end{cases}$
            \item \db{$1-j$}

                $\begin{cases}
                    |z|=\sqrt{1^2+(-1)^2}=\sqrt{2}\\
                    \theta=\arctan(1-j)=-\dfrac{\pi}{4}\equiv \dfrac{7\pi}{4}
                \end{cases}$
            \item \db{$-1$} 

                $\begin{cases}
                    |z|=\sqrt{(-1)^2+0^2}=1\\
                    \theta=\arctan(-1)=\pi
                \end{cases}$
            \item \db{$\dfrac{1+j}{1-j} $} 

                $\dfrac{1+j}{1-j}=\dfrac{1+j}{1-j}\cdot \dfrac{1+j}{1+j}=\dfrac{2j}{2}=j$

                $\begin{cases}
                    |z|=\sqrt{0^2+1^2}=1\\
                    \theta=\arctan(j)=\dfrac{\pi}{2}
                \end{cases}$
            \item \db{$\dfrac{1}{j\pi}$} 

                $\dfrac{1}{j\pi}=\dfrac{1}{j\pi}\cdot \dfrac{-j}{-j}=-\dfrac{j}{\pi}$

                $\begin{cases}
                    |z|=\sqrt{0^2+\left( -\dfrac{1}{\pi} \right)^2 }=\dfrac{1}{\pi}\\
                    \theta=\arctan\left( -\dfrac{j}{\pi} \right) =-\dfrac{\pi}{2}\equiv \dfrac{3\pi}{2}
                \end{cases}$
            \item \db{$-3+j\sqrt{3} $} 

                $\begin{cases}
                    |z|=\sqrt{(-3)^2+(\sqrt{3})^2}=2\sqrt{3}\\
                    \theta=\arctan(-3+\sqrt{3} j)=\dfrac{5\pi}{6}
                \end{cases}$
        \end{enumerate}
    \item \lb{Calcula las raíces cúbicas de los números $-8$ y  $1+j$} 

        \begin{itemize}[label=\textbullet]
            \item $\sqrt[3]{-8}=-2 $ 
            \item $\sqrt[3]{1+j}$

                La raíz cúbic de un número complejo se calcula expresándolo en su \textbf{forma polar} y luego aplicando la fórmula de las raíces de un número complejo.
                \begin{enumerate}[label=Paso \arabic*:]
                    \item Convertir $1+j$ a su forma polar  \[
                    \begin{cases}
                        r=|z|=\sqrt{1^2+1^2}=\sqrt{2}\\
                        \theta=\arctan\left( \dfrac{1}{1} \right) =\dfrac{\pi}{4}
                    \end{cases}
                    \] 
                    Por lo tanto, la forma polar de $1+j$ es:  \[
                    1+j=\sqrt{2}e^{j\frac{\pi}{4} }  
                    \] 
                \item Calcular la raíz cúbica

                    La fórmula para las $n$-ésimas de un número complejo es:  \[
                    z_K=r^{\frac{1}{n} }e^{j\left( \frac{\theta+2k\pi}{n}  \right) }\quad k=0,1,\dots,n-1 
                    \] 
                    Aquí: 
                    \begin{itemize}[label=\textbullet]
                        \item $n=3$
                        \item  $r=\sqrt{2} $ 
                        \item $\theta=\dfrac{\pi}{4}$
                    \end{itemize}
                    Por lo tanto:
                    \[
                    z_k = \left(\sqrt{2}\right)^{1/3} \, e^{j\left(\frac{\frac{\pi}{4} + 2k\pi}{3}\right)}, \quad k = 0, 1, 2)
                    \] 
                \item Expresar las raíces

                    El módulo de las raíces es: \[
                    r^{\frac{1}{3} }=\left( \sqrt{2}  \right) ^{\frac{1}{3} }
                    \] 
                    El argumento de las raíces es: \[
                    \theta_k=\dfrac{\pi}{4} +\dfrac{2k\pi}{3},\quad k=0,1,2
                    \] 
                    Por lo tanto, las raíces son: \[
                    z_k=\left( \sqrt{2}  \right) ^{\frac{1}{3} }\left[ \cos\left( \dfrac{\pi}{4}+\dfrac{2k\pi}{3}  \right) +j\sin\left( \dfrac{\pi}{4}+\dfrac{2k\pi}{3} \right)  \right] ,\quad k=0,1,2
                    \] 
                    \begin{enumerate}[label=\arabic*)]
                        \item Para $k=0$: \[
                        z_0=\left( \sqrt{2}  \right) ^{\frac{1}{3}}\left[ \cos\left( \dfrac{\pi}{4} \right) +j\sin\left( \dfrac{\pi}{4} \right)  \right] =\left( \sqrt{2}  \right) ^{\frac{1}{3} }\cdot \left( \dfrac{\sqrt{2} }{2}+\dfrac{\sqrt{2} }{2}j \right) =\dfrac{j+1}{\sqrt[3]{2} }
                        \] 
                    \item Para $k=1$:  \[
                    z_1=\left( \sqrt{2}  \right) ^{\frac{1}{3} }\left[ \cos\left( \dfrac{\pi}{4}+\dfrac{2\pi}{3} \right) +j\sin\left( \dfrac{\pi}{4}+\dfrac{2\pi}{3} \right)  \right] =\left( \sqrt{2}  \right) ^{\frac{1}{3} }\cdot \left( -\dfrac{\sqrt{6} +\sqrt{2} }{4}+\dfrac{\sqrt{6} -\sqrt{2} }{4}j \right) 
                    \] 
                \item Para $k=2$:
                     \[
                    z_2=\left( \sqrt{2}  \right) ^{\frac{1}{3} }\left[ \cos\left( \dfrac{\pi}{4}+\dfrac{4\pi}{3} \right) +j\sin\left( \dfrac{\pi}{4}+\dfrac{4\pi}{3} \right)  \right] =\left( \sqrt{2}  \right) ^{\frac{1}{3} }\cdot \left( \dfrac{\sqrt{6} -\sqrt{2} }{4}-\dfrac{\sqrt{6} +\sqrt{2} }{4}j \right) 
                
                    \] 
                    \end{enumerate}
                \end{enumerate}
        \end{itemize}
        
        \begin{enumerate}[label=\color{red}\textbf{\alph*)}]
            \item \db{Números complejos cuyo móudlo es igual a 1.}
            \item \db{$\{z^k\in C:z=e^{\frac{2\pi j}{8} },1\le  k\le 8 \} $} 
            \item \db{$\{z^k\in C:z=e^{-\frac{2\pi j}{8} },1\le  k\le 8 \} $} 
            \item \db{$\{z\in C:z-\overline{z}=j\} $} 
        \end{enumerate}
    \item \lb{Resuelve las siguientes ecuaciones algebraicas y expresa el resultado en forma binómica:}
        \begin{enumerate}[label=\color{red}\textbf{\alph*)}]
            \item \db{$2+4j+10e^{\frac{\pi}{3} j}=ze^{\frac{\pi}{3} j}  $} 
            \item \db{$5_{-\frac{\pi}{6} }+z+4+\sqrt{2} j=6e^{-\frac{\pi}{4} j} z$} 
            \item \db{$2_{\frac{\pi}{3} }+j+ze^{-\frac{\pi}{3} j} =0$} 
            \item \db{$z+4jz=1$} 
        \end{enumerate}
    \item \lb{Dados los números coomplejos $z_1=-1-j,z_2=2_{\frac{\pi}{3} },z_3=3e^{100\pi j} $, representa gráficamente los números $z_1,z_2,z_3,z_1+z_2+z_3,z_1\cdot z_2,z_1\cdot z_2\cdot z_3$.} 
    \item \lb{Consideremos la función $f(t)=e^{-t+j_{10}t},0\le t\le 2\pi $. Se pide:}
        \begin{enumerate}[label=\color{red}\textbf{\alph*)}]
            \item \db{Dibuja de manera aproximada el conjunto $\{(\mathrm{Re}f(t), \mathrm{Im}f(t)), 0\le t\le  2\pi\} $} 
            \item \db{Dibuja de manera aproximada las funciones $\mathrm{Re}f(t),\mathrm{Im}f(t)$, y $|f(t)|$} 
        \end{enumerate}
\end{enumerate}
\end{document}
