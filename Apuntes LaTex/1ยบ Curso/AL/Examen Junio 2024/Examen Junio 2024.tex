\documentclass{article}
\usepackage{fullpage}
\usepackage[utf8]{inputenc}
\usepackage{pict2e}
\usepackage{amsmath}
\usepackage{enumitem}
\usepackage{eurosym}
\usepackage{mathtools}
\usepackage{amssymb, amsfonts, latexsym, cancel}
\setlength{\parskip}{0.3cm}
\usepackage{graphicx}
\usepackage{fontenc}
\usepackage{slashbox}
\usepackage{setspace}
\usepackage{gensymb}
\usepackage{accents}
\usepackage{adjustbox}
\setstretch{1.35}
\usepackage{bold-extra}
\usepackage[document]{ragged2e}
\usepackage{subcaption}
\usepackage{tcolorbox}
\usepackage{xcolor, colortbl}
\usepackage{wrapfig}
\usepackage{empheq}
\usepackage{array}
\usepackage{parskip}
\usepackage{arydshln}
\graphicspath{ {images/} }
\renewcommand*\contentsname{\color{black}Índice} 
\usepackage{array, multirow, multicol}
\definecolor{lightblue}{HTML}{007AFF}
\usepackage{color}
\usepackage{etoolbox}
\usepackage{listings}
\usepackage{mdframed}
\setlength{\parindent}{0pt}
\usepackage{underscore}
\usepackage{hyperref}
\usepackage{tikz}
\usepackage{tikz-cd}
\usetikzlibrary{shapes, positioning, patterns}
\usepackage{tikz-qtree}
\usepackage{biblatex}
\usepackage{pdfpages}
\usepackage{pgfplots}
\usepackage{pgfkeys}
\addbibresource{biblatex-examples.bib}
\usepackage[a4paper, left=1cm, right=1cm, top=1cm,
bottom=1.5cm]{geometry}
\usepackage{titlesec}
\usepackage{titletoc}
\usepackage{tikz-3dplot}
\usepackage{kbordermatrix}
\usetikzlibrary{decorations.pathreplacing}
\newcommand{\Ej}{\textcolor{lightblue}{\underline{Ejemplo}}}
\setlength{\fboxrule}{1.5pt}

% Configura el formato de las secciones utilizando titlesec
\titleformat{\section}
{\color{red}\normalfont\LARGE\bfseries}
{Tema \thesection:}
{10 pt}
{}

% Ajusta el formato de las entradas de la tabla de contenidos
\addtocontents{toc}{\protect\setcounter{tocdepth}{4}}
\addtocontents{toc}{\color{black}}

\titleformat{\subsection}
{\normalfont\Large\bfseries\color{red}}{\thesubsection)}{1em}{\color{lightblue}}

\titleformat{\subsubsection}
{\normalfont\large\bfseries\color{red}}{\thesubsubsection)}{1em}{\color{lightblue}}

\newcommand{\bboxed}[1]{\fcolorbox{lightblue}{lightblue!10}{$#1$}}
\newcommand{\rboxed}[1]{\fcolorbox{red}{red!10}{$#1$}}

\DeclareMathOperator{\N}{\mathbb{N}}
\DeclareMathOperator{\Z}{\mathbb{Z}}
\DeclareMathOperator{\R}{\mathbb{R}}
\DeclareMathOperator{\Q}{\mathbb{Q}}
\DeclareMathOperator{\K}{\mathbb{K}}
\DeclareMathOperator{\im}{\imath}
\DeclareMathOperator{\jm}{\jmath}
\DeclareMathOperator{\col}{\mathrm{Col}}
\DeclareMathOperator{\fil}{\mathrm{Fil}}
\DeclareMathOperator{\rg}{\mathrm{rg}}
\DeclareMathOperator{\nuc}{\mathrm{nuc}}
\DeclareMathOperator{\dimf}{\mathrm{dimFil}}
\DeclareMathOperator{\dimc}{\mathrm{dimCol}}
\DeclareMathOperator{\dimn}{\mathrm{dimnuc}}
\DeclareMathOperator{\dimr}{\mathrm{dimrg}}
\DeclareMathOperator{\dom}{\mathrm{Dom}}
\DeclareMathOperator{\infi}{\int_{-\infty}^{+\infty}}
\newcommand{\dint}[2]{\int_{#1}^{#2}}

\newcommand{\bu}[1]{\textcolor{lightblue}{\underline{#1}}}
\newcommand{\lb}[1]{\textcolor{lightblue}{#1}}
\newcommand{\db}[1]{\textcolor{blue}{#1}}
\newcommand{\rc}[1]{\textcolor{red}{#1}}
\newcommand{\tr}{^\intercal}

\renewcommand{\CancelColor}{\color{lightblue}}

\newcommand{\dx}{\:\mathrm{d}x}
\newcommand{\dt}{\:\mathrm{d}t}
\newcommand{\dy}{\:\mathrm{d}y}
\newcommand{\dz}{\:\mathrm{d}z}
\newcommand{\dth}{\:\mathrm{d}\theta}
\newcommand{\dr}{\:\mathrm{d}\rho}
\newcommand{\du}{\:\mathrm{d}u}
\newcommand{\dv}{\:\mathrm{d}v}
\newcommand{\tozero}[1]{\cancelto{0}{#1}}
\newcommand{\lbb}[2]{\textcolor{lightblue}{\underbracket[1pt]{\textcolor{black}{#1}}_{#2}}}
\newcommand{\dbb}[2]{\textcolor{blue}{\underbracket[1pt]{\textcolor{black}{#1}}_{#2}}}
\newcommand{\rub}[2]{\textcolor{red}{\underbracket[1pt]{\textcolor{black}{#1}}_{#2}}}

\author{Francisco Javier Mercader Martínez}
\date{}
\renewcommand{\arraystretch}{1}
\setlength{\arraycolsep}{6pt}
\setstretch{1.15}
\title{Álgebra Lineal\\ Examen Junio 2024}

\begin{document}
\maketitle
\begin{enumerate}[label=\color{red}\textbf{\arabic*)}]
    \item \lb{Sea $A$ una matriz. Explica en qué consisten, cuándo se pueden calcular, y dónde se pueden utilizar las siguientes factorizaciones de  $A$.}
        \begin{enumerate}[label=\color{red}\textbf{\alph*)}]
            \item \db{Factorización $LU$ y Cholesky.}
            \item \db{Factorización QR.}
            \item \db{Factorización en valores propios.}
            \item \db{Factorización en valores singulares.} 
        \end{enumerate}
    \item \lb{Sean $u$ y $v$ dos vectores unitarios de  $\R^n$. Consideremos la matriz $A=uu^\intercal+vv^\intercal$. Supongamos que $u$ es un vector propio de $A$ asociado al valor propio $\lambda=1$. Demuestra que los vectores  $u$ y $v$ son ortogonales.}
    \item \lb{Consideremos los números complejos $z_1=-\sqrt{2}+\sqrt{2}j  $ y $z_2=\dfrac{1}{1+\sqrt{3}j }$} 
        \begin{enumerate}[label=\color{red}\textbf{\alph*)}]
            \item \db{Calcula la forma exponencial de $z_1$ y $z_2$.}
            \item \db{Calcula $z_1\cdot z_2$ y expresa el resultado en forma binómica.}
        \end{enumerate}
    \item \lb{Consideremos el sistema de ecuaciones $Ax=b$, donde  \[
    A=\begin{bmatrix} 
        1 & 2\\ 2 & 4 
    \end{bmatrix} ,\quad b=\begin{bmatrix} 
    1\\1 
    \end{bmatrix} .
    \]Se pide: }
    \begin{enumerate}[label=\color{red}\textbf{\alph*)}]
        \item \db{Comprueba que el sistema de ecuaciones $Ax=b$ no tiene solución, y calcula la proyección ortogonal del vector  $v$ sobre el subespacio  $\mathrm{Col}(A)$.}
        \item \db{Calcula la factorización SVD de la matriz $A$, es decir, calcula matrices ortogonales  $U$ y  $V$, y una matriz diagonal  $\Sigma$, de modo que  $A=U\Sigma V^\intercal$.}
        \item \db{Se define la pseudo-inversa de Moore-Penrose como la matriz $$A_{-1}=V\Sigma_{-1}U^\intercal$$ donde la matriz diagonal $\Sigma_{-1}$ contiene en su diagonal, los inversos de los valores singulares de $A$ no nulos. Calcula  $A_{1}$.}
        \item \db{Halla todos los vectores $X^\intercal=(x,y)$ tales que $Ax=u$, donde  $u$ es la proyección ortogonal que has calculado en el primer apartado, y comprueba que el que tiene la forma mínimia es precisamente $A_{-1}b$.} 
    \end{enumerate}
\item \lb{Consideremos la base $\mathcal{B}=\{v_1=(1,2),v_2=(-2,1)\} $ en $\R^2$, y una aplicación lineal que cumple que $f(v_1)=v_2$ y $f(v_2)=v_1$. Se pide:}
    \begin{enumerate}[label=\color{red}\textbf{\alph*)}]
        \item \db{Encuentra la matriz asociada a la aplicación lineal $f$ en la base  $\mathcal{B}$.}
        \item \db{Encuentra la matriz asociada a la aplicación lineal $f$ en la base canónica de  $\R^2$.}
        \item \db{Si $A$ es la matriz del apartado anterior, halla la descomposición espectral de $A$, es decir, escribe  $A$ como  $\lambda_1u_1u_1^\intercal+\lambda_2u_2u_2^\intercal$ con $\lambda_1,\lambda_2$ los valores propios de $A$ y  $u_1,u_2$ vectores propios adecuados.} 
    \end{enumerate}
\end{enumerate}
\end{document}
