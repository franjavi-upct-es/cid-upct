\documentclass{article}
\usepackage{fullpage}
\usepackage[utf8]{inputenc}
\usepackage{pict2e}
\usepackage{amsmath}
\usepackage{enumitem}
\usepackage{eurosym}
\usepackage{mathtools}
\usepackage{amssymb, amsfonts, latexsym, cancel}
\setlength{\parskip}{0.3cm}
\usepackage{graphicx}
\usepackage{fontenc}
\usepackage{slashbox}
\usepackage{setspace}
\usepackage{gensymb}
\usepackage{accents}
\usepackage{adjustbox}
\setstretch{1.35}
\usepackage{bold-extra}
\usepackage[document]{ragged2e}
\usepackage{subcaption}
\usepackage{tcolorbox}
\usepackage{xcolor, colortbl}
\usepackage{wrapfig}
\usepackage{empheq}
\usepackage{array}
\usepackage{parskip}
\usepackage{arydshln}
\graphicspath{ {images/} }
\renewcommand*\contentsname{\color{black}Índice} 
\usepackage{array, multirow, multicol}
\definecolor{lightblue}{HTML}{007AFF}
\usepackage{color}
\usepackage{etoolbox}
\usepackage{listings}
\usepackage{mdframed}
\setlength{\parindent}{0pt}
\usepackage{underscore}
\usepackage{hyperref}
\usepackage{tikz}
\usepackage{tikz-cd}
\usetikzlibrary{shapes, positioning, patterns}
\usepackage{tikz-qtree}
\usepackage{biblatex}
\usepackage{pdfpages}
\usepackage{pgfplots}
\usepackage{pgfkeys}
\addbibresource{biblatex-examples.bib}
\usepackage[a4paper, left=1cm, right=1cm, top=1cm,
bottom=1.5cm]{geometry}
\usepackage{titlesec}
\usepackage{titletoc}
\usepackage{tikz-3dplot}
\usepackage{kbordermatrix}
\usetikzlibrary{decorations.pathreplacing}
\newcommand{\Ej}{\textcolor{lightblue}{\underline{Ejemplo}}}
\setlength{\fboxrule}{1.5pt}

% Configura el formato de las secciones utilizando titlesec
\titleformat{\section}
{\color{red}\normalfont\LARGE\bfseries}
{Tema \thesection:}
{10 pt}
{}

% Ajusta el formato de las entradas de la tabla de contenidos
\addtocontents{toc}{\protect\setcounter{tocdepth}{4}}
\addtocontents{toc}{\color{black}}

\titleformat{\subsection}
{\normalfont\Large\bfseries\color{red}}{\thesubsection)}{1em}{\color{lightblue}}

\titleformat{\subsubsection}
{\normalfont\large\bfseries\color{red}}{\thesubsubsection)}{1em}{\color{lightblue}}

\newcommand{\bboxed}[1]{\fcolorbox{lightblue}{lightblue!10}{$#1$}}
\newcommand{\rboxed}[1]{\fcolorbox{red}{red!10}{$#1$}}

\DeclareMathOperator{\N}{\mathbb{N}}
\DeclareMathOperator{\Z}{\mathbb{Z}}
\DeclareMathOperator{\R}{\mathbb{R}}
\DeclareMathOperator{\Q}{\mathbb{Q}}
\DeclareMathOperator{\K}{\mathbb{K}}
\DeclareMathOperator{\im}{\imath}
\DeclareMathOperator{\jm}{\jmath}
\DeclareMathOperator{\col}{\mathrm{Col}}
\DeclareMathOperator{\fil}{\mathrm{Fil}}
\DeclareMathOperator{\rg}{\mathrm{rg}}
\DeclareMathOperator{\nuc}{\mathrm{nuc}}
\DeclareMathOperator{\dimf}{\mathrm{dimFil}}
\DeclareMathOperator{\dimc}{\mathrm{dimCol}}
\DeclareMathOperator{\dimn}{\mathrm{dimnuc}}
\DeclareMathOperator{\dimr}{\mathrm{dimrg}}
\DeclareMathOperator{\dom}{\mathrm{Dom}}
\DeclareMathOperator{\infi}{\int_{-\infty}^{+\infty}}
\newcommand{\dint}[2]{\int_{#1}^{#2}}

\newcommand{\bu}[1]{\textcolor{lightblue}{\underline{#1}}}
\newcommand{\lb}[1]{\textcolor{lightblue}{#1}}
\newcommand{\db}[1]{\textcolor{blue}{#1}}
\newcommand{\rc}[1]{\textcolor{red}{#1}}
\newcommand{\tr}{^\intercal}

\renewcommand{\CancelColor}{\color{lightblue}}

\newcommand{\dx}{\:\mathrm{d}x}
\newcommand{\dt}{\:\mathrm{d}t}
\newcommand{\dy}{\:\mathrm{d}y}
\newcommand{\dz}{\:\mathrm{d}z}
\newcommand{\dth}{\:\mathrm{d}\theta}
\newcommand{\dr}{\:\mathrm{d}\rho}
\newcommand{\du}{\:\mathrm{d}u}
\newcommand{\dv}{\:\mathrm{d}v}
\newcommand{\tozero}[1]{\cancelto{0}{#1}}
\newcommand{\lbb}[2]{\textcolor{lightblue}{\underbracket[1pt]{\textcolor{black}{#1}}_{#2}}}
\newcommand{\dbb}[2]{\textcolor{blue}{\underbracket[1pt]{\textcolor{black}{#1}}_{#2}}}
\newcommand{\rub}[2]{\textcolor{red}{\underbracket[1pt]{\textcolor{black}{#1}}_{#2}}}

\author{Francisco Javier Mercader Martínez}
\date{}
\title{Álgebra Lineal\\ Examen Convocatoria Enero 2023}
\renewcommand{\arraystretch}{1}
\setlength{\arraycolsep}{6pt}

\begin{document}
\maketitle
\begin{enumerate}[label=\color{red}\textbf{\arabic*)}]
    \item \lb{Consideremos los número complejos \[
    z_1=-1-j,\quad z_2=\sqrt{2}e^{\frac{\pi}{4} j}  .
    \]Calcula $z_1+z_2,z_1\cdot z_2$ y $\dfrac{z_2}{z_1}$ y expresa el resultado en forma exponencial.}

    $z_2=\sqrt{2}e^{\frac{\pi}{4} j} =\sqrt{2} \cdot \left( \cos\dfrac{\pi}{4}+j\sin\dfrac{\pi}{4} \right)= 1+j $
\begin{itemize}[label=\textbullet]
    \item $z_1+z_2=(-1-j)+1+j=0$
    \item $z_1\cdot z_2=(-1-j)\cdot (1+j)=-1-j-j+1=-2j$
    \item $\dfrac{z_2}{z_1}=\dfrac{1+j}{-1-j}=\dfrac{1+j}{-1-j}\cdot \dfrac{-1+j}{-1+j}=\frac{(1+j)\cdot (-1+j)}{2}=\dfrac{-2}{2}=-1$
\end{itemize}
\item \lb{Dada una matriz cuadrada $D$, se llama de  \textit{similitud producto-escalar} a la matriz $S=DD^\intercal$, con $D^\intercal$ la traspuesta de $D$. Se pide:}
    \begin{enumerate}[label=\color{red}\textbf{\alph*)}]
        \item \db{Demuestra que $S$ simétrica.}

            Sabemos que: \[
            S=DD^\intercal
            \] donde $D^\intercal$ es la traspuesta de $D$. Para verificar que  $S$ es simétrica, necesitamos comprobar que $S=S^\intercal$.

            El cálculo de la traspuesta de $S$ es:  \[
            S^\intercal=( D D^\intercal )^\intercal
            \] 
            Utilizando la propiedad de la trasposición de un producto de matrices: $(AB)^\intercal=B^\intercal A^\intercal$. Entonces: \[
            S^\intercal=(D D^\intercal)^\intercal=(D^\intercal)^\intercal D^\intercal
            \] 
            Sabmos que $(D^\intercal)^\intercal=D$. Por lo tanto: \[
            S^\intercal=D D^\intercal
            \] 
            Dado que $S=DD^\intercal$, se concluye que $S=S^\intercal$, lo que prueba que $S$ es simétrica.
        \item \db{Sea $P$ una matriz ortogonal del mismo tamaño que $D$ y consideremos la matriz $DP$. Denotemos por  $S$ y  $S'$ a las matrices de similitud producto-escalar de  $D$ y  $DP$, respectivamente. Comprueba que  $S=S'$.} 
            
            Definimos:
            \begin{itemize}[label=\textbullet]
                \item $S=D D^\intercal$ como la matriz de similitud producto escalar de $D$.
                \item  $S'$ como la matriz de similitud producto-escalar de  $DP$, donde $P$ es una matriz ortogonal.
            \end{itemize}
            Dado que $P$ es ortogonal, satisface $P^\intercal P=I$, donde $I$ es la matriz identidad. La matriz  $S'$ está dada por: \[
            S'=(DP)(DP)^\intercal
            \] 
            Calculamos $(DP)^\intercal$ utilizando la propiedad de la transposición de producto de matrices: \[
                (DP)^\intercal=P^\intercal D^\intercal
            \] 
            Sustituimos esto en la definición de $S'$:  \[
            S'=(DP)(DP)^\intercal=(DP)(P^\intercal D^\intercal)
            \] 
            Distribuimos el producto: \[
            S'=D(PP^\intercal)D^\intercal=DID^\intercal=D D^\intercal=S
            \] 
            Esto prueba que $S'=S$.
    \end{enumerate}
\item \lb{Sea $A$ una matriz. Explica con detalle en qué consiste la factorización en valores singulares  $(SVD)$ de  $A$. Por supuesto, se ha de explicar qué son los valores singulares y cómo se calculan las matrices que aparecen en dicha factorización. Pon también un ejemplo de aplicación de la factorización  $SVD$ en Ciencia de datos.}

    La \textbf{Factorización en Valores Singulares (SVD)} descompone cualquier matriz $A$ (de tamaño $m\times n$) en el producto de tres matrices específicas: \[
    A=U\Sigma V^\intercal,
    \]donde:
    \begin{enumerate}[label=\arabic*)]
        \item $U$ es una matriz ortogonal de tamaño $m\times m$.
        \item $\Sigma$ es una matriz diagonal de tamaño  $m\times n$, cuyos elementos no nulos en la diagonal se llaman \textbf{valores singulares}.
        \item $V^\intercal$ es la traspuesta de una matriz ortogonal $V$ de tamaño  $n\times n$.
    \end{enumerate}
    \textbf{Definición de los componentes}

    \begin{enumerate}[label=\arabic*)]
        \item \textbf{Matriz $U$:}
            \begin{itemize}[label=\textbullet]
                \item Sus columans son los \textbf{vectores singulares izquierdos} de $A$, que son vectores propios de $A A^\intercal$.
                \item Representan las direcciones principales del espacio de salida de $A$.
                \item  $U$ es ortogonal:  $U^\intercal U=I_m$.
            \end{itemize}
        \item \textbf{Matriz $\Sigma$:}
            \begin{itemize}[label=\textbullet]
                \item Es una matriz diagonal cuyos elementos $\sigma_1\ge \sigma_2\ge \cdots\ge \sigma_r\ge 0$ (donde $r$ es el rango de  $A$) se llaman \textbf{valores singulares}.
                \item Los valores singulares son las raíces cuadradas de los valores propios de $A^\intercal A$ o $A A^\intercal$: \[
                \sigma_i=\sqrt{\lambda_i},\quad \text{donde $\lambda_i$ son los valores propios de  $A^\intercal A$.} 
                \] 
                \end{itemize}
            \item \textbf{Matriz $V$:}
                \begin{itemize}[label=\textbullet]
                    \item Sus columnas son los \textbf{vectores singulares derechos} de $A$, que son vectores propios de  $A^\intercal A$.
                    \item Representan las direcciones principales del espacio de entrada de $A$.
                    \item  $V$ es ortogonal:  $V^\intercal V=I_n$.
            \end{itemize}
    \end{enumerate}
\textbf{Cálculo del SVD:}
\begin{enumerate}[label=Paso \arabic*:]
    \item Calcular los valores propios de $A^\intercal A$ y $A A^\intercal$:
        \begin{itemize}[label=\textbullet]
            \item Los valores propios de $A^\intercal A$ (o $A A^\intercal$) determinan lso valores singulares $\sigma_i^2$.
                
        \end{itemize}
    \item Encontrar los vectores propios:
        \begin{itemize}[label=\textbullet]
            \item Los vectores propios de $A^\intercal A$ son las columnas de $V$.
            \item Los vectores propios de  $A A^\intercal$ son las columnas de $U$.
        \end{itemize}
    \item Construir $\Sigma$: 
         \begin{itemize}[label=\textbullet]
            \item Los valores singulares se colocan en la diagonal de $\Sigma$.
        \end{itemize}
\end{enumerate}
\item \lb{Responde a las siguientes preguntas:}
    \begin{enumerate}[label=\color{red}\textbf{\alph*)}]
        \item \db{Explica qué es una matriz ortogonal.}

            Una matriz $Q$ de tamaño  $n\times n$ es \textbf{ortogonal} si sus columnas forma un conjunto ortonormal. Esto significa que:
           \begin{enumerate}[label=Paso \arabic*:]
               \item Las columnas de $Q$ son  \textbf{ortogonales entre sí} (el producto escalar entre dos columnas distintas es $0$).
               \item Cada columna de $Q$ tiene  \textbf{norma 1} (su longitud es igual a 1). 
           \end{enumerate} 
           En términos matemáticos, una matriz $Q$ es ortogonal si satisface:  \[
           Q^\intercal Q=I_n\quad \text{o}\quad Q Q^\intercal=I_n,
           \] 
           donde: 
           \begin{itemize}[label=\textbullet]
               \item $Q^\intercal$ es la traspuesta de $Q$.
               \item  $I_n$ es la matriz identidad de tamaño  $n\times n$.
           \end{itemize}
        \item \db{Explica en qué consiste la factorización $QR$ de una matriz cuadrada  $A$.}

            La factorización $QR$ de una matriz cuadrada  $A$ descompone  $A$ en el producto de dos matrices:  \[
            A=QR,
            \] donde:
            \begin{enumerate}[label=\arabic*)]
                \item $Q$: Es una matriz ortogonal ($Q^\intercal Q=I$). Las columnas de $Q$ son ortonormales, es decir, forman una base de ortonormal para el espacio columna de  $A$.
                \item  $R$: Es una matriz triangular superior (todos los elementos debajo de la diagonal son 0).
            \end{enumerate}
        \item \db{¿Qué propiedad tiene que cumplir $A$ para que se pueda calcular su factorización  $QR$?}

            Anque cualquier matriz $A$ tiene una factorización  $QR$, en ciertos contextos puede haber condiciones adicionales:
             \begin{itemize}[label=\textbullet]
                \item Rango completo de columnas: Para que el proceso de Gram-Schmidt sea válido, se requiere que las columnas de $A$ sean linealmente independientes.
            \end{itemize}
        \item \db{Razona si la siguiente afirmación es cierta o falsa: sea $A$ una matriz no singular. El sistema lineal  $Ax=b$ siempre se puede resolver mediante factorización $LU$ y Choleski. Además, siempre hemos de elegir el método de Cholesky porque su coste computacional es menor que el de la  $LU$.} 

            La afirmación es \textbf{falsa}, porque: 
            \begin{enumerate}[label=\arabic*)]
                \item Aunque $Ax=b$ siempre se puede resolver mediante  $LU$ si $A$ es no singular (invertible), \textbf{Cholesky no siempre es aplicable}. La factorización de Cholesky requiere que $A$ sea simétrica ($A=A^\intercal$) y definida positiva ($x^\intercal Ax>0$ para todo $x\neq 0$), lo cual no está garantizado para cualquier matriz no singular.
                \item Aunque el coste computacional de Cholesky es menor que el de $LU$,  \textbf{no siempre se puede elegir Cholesky} debido a las restricciones mencionadas. 
            \end{enumerate}
    \end{enumerate}
\item \lb{Consideremos la matriz \[
A=\begin{bmatrix} 
    1 & 1 & 1 & 1\\
    1 & 2 & 3 & 4\\
    1 & 3 & 6 & 10\\
    1 & 4 & 10 & 20
\end{bmatrix} .
\]Calcula la factorización $LU$ de  $A$ y utiliza dicha factorización para calcular el determinante de $A$. Por curiosidad, la matriz  $L$ que aparece en la factorización anterior contiene el llamado  \textit{triángulo de Pascal}, que se usa, por ejemplo, en Combinatoria.}

$$\begin{aligned}
    \left[ \begin{array}{cccc|cccc}
            1 & 1 & 1 & 1 & 1 & 0 & 0 & 0\\
            1 & 2 & 3 & 4 &  & 1 & 0 & 0\\
            1 & 3 & 6 & 10 &  &  & 1 & 0\\
            1 & 4 & 10 & 20 &  &  &  & 1\\
        \end{array} \right]& \xrightarrow[\begin{subarray}{l}
            F_3\to F_3-F_1\\
            F_4\to F_4-F_1
    \end{subarray}]{F_2\to F_2-F_1}\left[ \begin{array}{cccc|cccc}
            1 & 1 & 1 & 1 & 1 & 0 & 0 & 0\\
            0 & 1 & 2 & 3 & 1 & 1 & 0 & 0\\
            0 & 2 & 5 & 9 & 1 &  & 1 & 0\\
            0 & 3 & 9 & 19 & 1 & & & 1
    \end{array} \right] \\
    &\xrightarrow[F_4\to F_4-3F_2]{F_3\to F_3-2F_2} \left[ \begin{array}{cccc|cccc}
            1 & 1 & 1 & 1 & 1 & 0 & 0 & 0\\
            0 & 1 & 2 & 3 & 1 & 1 & 0 & 0\\
            0 & 0 & 1 & 3 & 1 & 2 & 1 & 0\\
            0 & 0 & 3 & 10 & 1 & 3 & & 1
    \end{array} \right] \\
    &\xrightarrow{F_4\to F_4-3F_3}\left[ \begin{array}{cccc|cccc}
            1 & 1 & 1 & 1 & 1 & 0 & 0 & 0\\
            0 & 1 & 2 & 3 & 1 & 1 & 0 & 0\\
            0 & 0 & 1 & 3 & 1 & 2 & 1 & 0\\
            0 & 0 & 0 & 1 & 1 & 3 & 3 & 1 
    \end{array} \right] 
\end{aligned}$$
Tenemos: \[
L=\begin{bmatrix} 
    1 & 0 & 0 & 0\\
    1 & 1 & 0 & 0\\
    1 & 2 & 1 & 0\\
    1 & 3 & 3 & 1
\end{bmatrix}, \qquad U=\begin{bmatrix} 
    1 & 1 & 1 & 1\\
    0 & 1 & 2 & 3\\
    0 & 0 & 1 & 3\\
    0 & 0 & 0 & 1
\end{bmatrix} 
\] 
El determinante de $A$ se calcula como el producto de elentos de la diagonal de $U$:  \[
\mathrm{det}(A)=1\cdot 1\cdot 1\cdot 1=1.
\] 
\item \lb{Consideremos la matriz $A=\begin{bmatrix} 
            3 & 0\\ 4 & 5 
\end{bmatrix}. $ Se pide:}
\begin{enumerate}[label=\color{red}\textbf{\alph*)}]
    \item \db{Calcula los valores propios y una base ortonormal de vectores propios de la matriz $A^\intercal A$. Denotemos por $\lambda_1,\lambda_2$ dichos vectores propios y por $v_1,v_2$ dicha base ortonormal de vectores propios asociados a $\lambda_1$ y $\lambda_2$, respectivamente. LLamaremos $V$ a la matriz cuyas columnas son, en este orden,  $v_1$ y $v_2$.}

        \[
        A^\intercal A=\begin{bmatrix} 
            3 & 4\\
            0 & 5
        \end{bmatrix} \begin{bmatrix} 
            3 & 0\\
            4 & 5
        \end{bmatrix} =\begin{bmatrix} 
            25 & 20\\
            20 & 25
        \end{bmatrix}
        \] 
        $\mathrm{det}(A^\intercal-\lambda I)=\mathrm{det}\begin{bmatrix} 
            25-\lambda & 20\\
            20 & 25-\lambda
        \end{bmatrix}=(25-\lambda)^2-400=\lambda^2-50\lambda+625-400=\lambda^2-50\lambda+225=0 $ \[
        \lambda=\dfrac{50\pm\sqrt{(-50)^2-4\cdot 1\cdot 225} }{2\cdot 1}=\dfrac{50\pm\sqrt{1600} }{2}=\begin{cases}
            \dfrac{50+40}{2}=45=\lambda_1\\
            \dfrac{50-40}{2}=5=\lambda_2
        \end{cases}
        \]
        \begin{itemize}[label=\textbullet]
            \item Para $\lambda_1=45$ \[
           \mathrm{Nuc}(A^\intercal A-45I )=\begin{bmatrix} 
               -20 & 20\\
               20 & -20
           \end{bmatrix}\begin{bmatrix} 
           x\\y 
           \end{bmatrix}=\begin{bmatrix} 
           0\\0 
           \end{bmatrix} \longrightarrow \begin{cases}
               -20x+20y=0\\
               20x-20y=0
           \end{cases}\longrightarrow 20x=20y\longrightarrow x=y\longrightarrow (1,1)
            \] 
            $v_1=\dfrac{1}{\|v_1\|}\begin{bmatrix} 
            1\\1 
            \end{bmatrix}=\dfrac{1}{\sqrt{2} }\begin{bmatrix} 
            1\\1 
            \end{bmatrix}=\begin{bmatrix} 
            \tfrac{\sqrt{2} }{2} \\ \tfrac{\sqrt{2} }{2}  
            \end{bmatrix}  $ 
        \item Para $\lambda_2=5$ \[
        \mathrm{Nuc}(A^\intercal A-5I)=\begin{bmatrix} 
            20 & 20\\
            20 & 20
        \end{bmatrix} \begin{bmatrix} 
        x\\ y 
        \end{bmatrix} =\begin{bmatrix} 
        0\\0 
        \end{bmatrix} \longrightarrow \begin{cases}
            20x+20y=0\\
            20x+20y=0
        \end{cases}\longrightarrow 20x=-20y\longrightarrow x=-y\longrightarrow (1,-1)
        \] 
        $v_2=\dfrac{1}{\|v_2\|}\begin{bmatrix} 
        1\\-1 
        \end{bmatrix}=\dfrac{1}{\sqrt{2} }\begin{bmatrix} 
        1\\-1 
        \end{bmatrix}=\begin{bmatrix} 
        \tfrac{\sqrt{2} }{2}\\-\tfrac{\sqrt{2} }{2}   
        \end{bmatrix}   $
        \end{itemize}
\[
V=\begin{bmatrix} 
    \tfrac{\sqrt{2} }{2}  & \tfrac{\sqrt{2} }{2} \\
    \tfrac{\sqrt{2} }{2}  & -\tfrac{\sqrt{2} }{2} 
\end{bmatrix} 
\] 

    \item \db{Sean $\sigma_1=\sqrt{\lambda_1} $ y $\sigma_2=\sqrt{\lambda_2} $. Calcula los vectores $u_1=\dfrac{1}{\sigma_1}Av_1$ y $u_2=\dfrac{1}{\sigma_2}Av_2$ y comprueba que forman una base ortonormal de vectores propios asociados a $\lambda_1$ y $\lambda_2$ para la matriz $A A^\intercal$.} 

        \[
        \begin{array}{l}
            \sigma_1=\sqrt{45}\\
            \sigma_2=\sqrt{5} 
        \end{array}
        \] 
        $\begin{array}{l}
            u_1=\dfrac{1}{\sqrt{45} }\begin{bmatrix} 
                3 & 0\\
                4 & 5
            \end{bmatrix} \begin{bmatrix} 
            \tfrac{\sqrt{2} }{2} \\ \tfrac{\sqrt{2} }{2}  
            \end{bmatrix} =\dfrac{\sqrt{10} }{30}\begin{bmatrix} 
            3\\9 
            \end{bmatrix} =\begin{bmatrix} 
            \tfrac{\sqrt{10} }{10} \\ \tfrac{3\sqrt{10} }{10}
            \end{bmatrix}\\
            u_2=\dfrac{1}{\sqrt{5} }\begin{bmatrix} 
                3 & 0\\ 4 & 5 
            \end{bmatrix} \begin{bmatrix} 
            \tfrac{\sqrt{2} }{2}\\ -\tfrac{\sqrt{2}}{2}   
            \end{bmatrix} =\dfrac{\sqrt{10} }{10}\begin{bmatrix} 
            3\\-1 
            \end{bmatrix} =\begin{bmatrix} 
            \tfrac{3\sqrt{10} }{10}\\ -\tfrac{\sqrt{10} }{10} 
            \end{bmatrix} 
        \end{array}$


    Para comprobar que $u_1$ y $u_2$ son vectores propios de $A A^\intercal$, debemos verificar que satisfacen: \[
    A A^\intercal u_i=\lambda_iu_i,\quad i=1,2.
    \] 
    \begin{itemize}[label=\textbullet]
        \item Para $u_1$: 

            Sustituimos  $u_1=\dfrac{1}{\sigma_1}Av_1$ en $A A^\intercal u_1$: \[
            A A^\intercal u_1=A A^\intercal\left( \dfrac{1}{\sigma_1}Av_1 \right)=\dfrac{1}{\sigma_1}A(A^\intercal Av_1). 
            \] 
            Sabemos que $A^\intercal Av_1=\lambda_1v_1$, así que: \[
                A A^\intercal u_1=\dfrac{1}{\sigma_1}A(\lambda_1v_1)=\lbb{\dfrac{\lambda_1}{\sigma_1}}{\sqrt{\lambda_1} }Av_1=\sigma_1Av_1=\lambda_1u_1.
            \] 
        \item Para $u_2$: 

            Sustituimos $u_2=\dfrac{1}{\sigma_2}Av_2$ en $A A^\intercal u_2$: \[
            A A^\intercal u_2=A A^\intercal\left( \dfrac{1}{\sigma_2}Av_2 \right) =\dfrac{1}{\sigma_2}A(A^\intercal Av_2).
            \] 

            Sabemos que $A^\intercal Av_2=\lambda_2v_2$, así que: \[
            A A^\intercal u_2=\dfrac{1}{\sigma_2}A(\lambda_2v_2)=\lbb{\dfrac{\lambda_2}{\sigma_2}}{\sqrt{\lambda_2} } Av_2=\sigma_2Av_2=\lambda_2u_2.
            \] 
    \end{itemize}
    Por lo tanto, $u_1$ y $u_2$ son vectores propios de $A A^\intercal$ asociados a $\lambda_1$ y $\lambda_2$, respectivamente.

Para demostrar que $u_1$ y $u_2$ son ortogonales, calulamos el producto escalar: \[u_1\cdot u_2=\left(\dfrac{1}{\sigma_1}Av_1\right)\cdot\left(\dfrac{1}{\sigma_2}Av_2\right)=\dfrac{1}{\sigma_1\sigma_2}(Av_1)^\intercal(Av_2).\]

Sabemos que $(Av_1)^\intercal(Av_2)=v_1^\intercal(A^\intercal A)v_2$. Usamos que $v_1$ y $v_2$ son ortogonales (porque son vectores propios de $A^\intercal A$ asociados a diferentes valores propios): \[v_1^\intercal(A^\intercal A)v_2=\lambda_2v_1^\intercal v_2=0\]
Por lo tanto: \[u_1\cdot u_2=0,\]lo que demuestra que $u_1$ y $u_2$ son ortogonales.

    \item \db{Sean $U$ la matriz que tiene por columnas los vectores $u_1$ y $u_2$, en ese orden, y $\Sigma$ la matriz diagonal que tiene en su diagonal los números  $\sigma_1$ y $\sigma_2$. Comprueba que \[
    A=\sigma_1u_1v_1^\intercal+\sigma_2u_2v_2^\intercal=U\Sigma V^\intercal.
    \] } 

    Por definición de $\sigma_i,u_i$, y  $v_i$, tenemos:  \[
    A=\sigma_1u_1v_1^\intercal+\sigma_2u_2v_2^\intercal.
    \] 
    Expandimos cada términos: \[
    \begin{array}{c}
        \sigma_1u_1v_1^\intercal=\sigma_1\begin{bmatrix} 
        u_{11}\\u_{12} 
        \end{bmatrix} \begin{bmatrix} 
        v_{11} & v_{12} 
        \end{bmatrix} \\
        \sigma_2u_2v_2^\intercal=\sigma_2\begin{bmatrix} 
        u_{21}\\u_{22} 
        \end{bmatrix} \begin{bmatrix} 
        v_{21} & v_{22} 
        \end{bmatrix} 
    \end{array}
    \] 
    La suma es:
    \[
    A=\sigma_1\begin{bmatrix} 
        u_{11}\\u_{12} 
        \end{bmatrix} \begin{bmatrix} 
        v_{11} & v_{12} 
        \end{bmatrix}+\sigma_2\begin{bmatrix} 
        u_{21}\\u_{22} 
        \end{bmatrix} \begin{bmatrix} 
        v_{21} & v_{22} 
        \end{bmatrix} 
    \] 

    Agrupamos los términos para formar $A$, ya que  $u_1,u_2$ y $v_1,v_2$ forman las bases ortonormales en $U$ y $V$.

    La forma general de $U\Sigma V\intercal$ es: \[
    U\Sigma V^\intercal=\begin{bmatrix} 
        u_1 & u_2 
    \end{bmatrix}\begin{bmatrix} 
        \sigma_1 & 0\\0 & \sigma_2 
    \end{bmatrix} \begin{bmatrix} 
    v_1^\intercal\\ v_2^\intercal 
    \end{bmatrix}  .
    \] 
    Realizamos el producto: \[
    U\Sigma = \begin{bmatrix} 
        u_1 & u_2 
    \end{bmatrix}\begin{bmatrix} 
        \sigma_1 & 0\\0 & \sigma_2 
    \end{bmatrix}  =\begin{bmatrix} 
        \sigma_1u_1 & \sigma_2u_2 
    \end{bmatrix} .
    \] 

    Multiplicamos por $V^\intercal$: 
    \[
    U\Sigma V^\intercal=\begin{bmatrix} 
        \sigma_1u_1 & \sigma_2u_2 
    \end{bmatrix} \begin{bmatrix} 
    v_1^\intercal\\v_2^\intercal 
    \end{bmatrix} =\sigmau_1v_1^\intercal+\sigma_2u_2v_2^\intercal.
    \] 
    
    Por lo que queda demostrado que: \[
    A=\sigma_1u_1v_1^\intercal+\sigma_2u_2v_2^\intercal=U\Sigma V^\intercal.
    \] 
\item \db{¿Cómo se llaman los números $\sigma_1$ y $\sigma_2$? ¿Cómo se llama la factorización de $A$ que hemos hecho en el aparato anterior?} 

    Los números $\sigma_1$ y  $\sigma_2$ son los \textbf{valores singulares} de la matriz $A$.
    \begin{itemize}[label=\textbullet]
        \item Los valores singulares son raíces cuadradas de los valores propios de la matriz $A^\intercal A$, es decir: \[
        \sigma_i=\sqrt{\lambda_i},\quad \text{donde $\lambda_i$ son los valores propios de  $A^\intercal A$.} 
        \] 
    \item Los valores singulares miden \textit{"la magnitud"} de la transformación lineal inducida por $A$ en diferentes direcciones del espacio.
    \end{itemize}
    La factorización de $A$ que hemos realizado se llama \textbf{Factorización en Valores Singulares (SVD)}. En esta factorización descomponemos $A$ como: \[
    A=U\Sigma V^\intercal,
    \] 
    donde:
    \begin{itemize}[label=\textbullet]
        \item $U$ es la matriz ortogonal formada por los vectores singulares izquierdos $(u_i)$ de  $A$, que son los vectores propios de  $A A^\intercal$.
        \item $V$ es una matriz ortogonal formada por los vectores singulares derechos  $(v_i)$ de  $A$, que son vectores propios de  $A^\intercal A$.
        \item $\Sigma$ es una matriz diagonal cuyos elementos son los valores singulares ($\sigma_1,\sigma_2$) de $A$.
    \end{itemize}
\end{enumerate}
\item \lb{Consideremos la matriz $A=\begin{bmatrix} 
            1 & 1 & 1\\
            1 & 2 & 3
\end{bmatrix} $.}
\begin{enumerate}[label=\color{red}\textbf{\alph*)}]
    \item \db{Calcula la dimensión y una base de los cuatro subespacios fundamentales de $A$, es decir,  $\mathrm{Fil}(A),\mathrm{Col}(A),\mathrm{Nuc}(A)$ y $\mathrm{Nuc}(A^\intercal)$.}
        \begin{enumerate}[label=\arabic*)]
            \item $\mathrm{Fil}(A)$: 

                \[
                \begin{bmatrix} 
                    1 & 1 & 1\\
                    1 & 2 & 3
                \end{bmatrix}\xrightarrow{F_2\to F_2-F_1}\begin{bmatrix} 
                    1 & 1 & 1\\
                    0 & 1 & 2
                \end{bmatrix}  
                \]  
                Los vectores independientes son $(1,1,1)$ y  $(0,1,2)$. Por lo tanto:
                $$\mathrm{Fil}(A)= <(1,1,1),(0,1,2)>.$$
                La dimensión de $\mathrm{Fil}(A)$ es: \[
                \mathrm{dimFil}(A)=2.
                \] 
            \item $\mathrm{Col}(A):$ 

                \[
                \begin{bmatrix} 
                    1 & 1 & 1\\
                    1 & 2 & 3
                \end{bmatrix}\xrightarrow[C_3\to C_3-C_1]{C_2\to C_2-C_1} \begin{bmatrix} 
                    1 & 0 & 0\\
                    1 & 1 & 2
                \end{bmatrix}\xrightarrow{C_3\to C_3-2C_2}\begin{bmatrix} 
                    1 & 0 & 0\\
                    1 & 1 & 0
                \end{bmatrix}  
                \] 
                Las columnas $(1,1)^\intercal$ y $(1,2)^\intercal$ son independientes. Por lo tanto: \[
                \mathrm{Col}(A)= <(1,1),(1,2)>.
                \] 
                La dimensión de $\mathrm{Col}(A)$ es: \[
                \mathrm{dimCol}(A)=2.
                \] 
            \item $\mathrm{Nuc}(A):$ \[
            \begin{bmatrix} 
                1 & 1 & 1\\
                1 & 2 & 3
            \end{bmatrix} \begin{bmatrix} 
            x\\ y\\z 
            \end{bmatrix} =\begin{bmatrix} 
            0\\0 
            \end{bmatrix} \longrightarrow \begin{cases}
                x+y+z=0\\
                x+2y+3z=0
            \end{cases}\longrightarrow \begin{cases}
                x+y+z=0\\
                y+2z=0
            \end{cases}\longrightarrow y=-2z\longrightarrow x=2z-z=z
            \] 
            Por lo tanto, el núcleo es: \[
            \mathrm{Nuc}(A)= <(1,-2,1) >.
            \] 
            La dimensión de $\mathrm{Nuc}(A)$ es: \[
            \mathrm{dimNuc}(A)=1.
            \] 
        \item $\mathrm{Nuc}(A^\intercal):$ \[
        \begin{bmatrix} 
            1 & 1\\
            1 & 2\\
            1 & 3
        \end{bmatrix} \begin{bmatrix} 
        x\\y 
        \end{bmatrix} =\begin{bmatrix} 
        0\\0\\0 
        \end{bmatrix} \longrightarrow \begin{cases}
            x+y=0\\
            x+2y=0\\
            x+3y
        \end{cases}\longrightarrow x=-y
        \] 
        Por lo tanto, el núcleo es: \[
        \mathrm{Nuc}(A^\intercal)=<(-1,1)>
        \] 
        La dimensión de $\mathrm{Nuc}(A^\intercal)$ es: \[
        \mathrm{dimNuc}(A^\intercal)=1.
        \] 
        \end{enumerate}
    \item \db{En general, ¿qué relaciones pueden dar entre los subespacios anteriores?} 
        \begin{enumerate}[label=\arabic*)]
            \item \textbf{Relaciones entre dimensiones:} Por el \textbf{teorema fundamental del álgebra lineal:} \[
            \mathrm{dimFil}(A)=\mathrm{dimCol}(A)=\mathrm{rango}(A),
            \] y \[
            \mathrm{dimNuc}(A)+\mathrm{dimFil}(A)=n, \quad \mathrm{dimNuc}(A^\intercal)+\mathrm{dimCol}(A)=m,
            \]   donde $n$ es el número de columnas y  $m$ es el número de filas de $A$.
        \item \textbf{Ortogonalidad:} Los vectores en $\mathrm{Nuc}(A)$ son ortogonales a los vectores de $\mathrm{Fil}(A)$, y los vectores en $\mathrm{Nuc}(A^\intercal)$ son ortogonales a los vectores de $\mathrm{Col}(A)$.
        \end{enumerate}
\end{enumerate}
\end{enumerate}
\end{document}
