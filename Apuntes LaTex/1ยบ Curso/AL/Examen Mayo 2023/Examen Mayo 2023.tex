\documentclass{article}
\usepackage{fullpage}
\usepackage[utf8]{inputenc}
\usepackage{pict2e}
\usepackage{amsmath}
\usepackage{enumitem}
\usepackage{eurosym}
\usepackage{mathtools}
\usepackage{amssymb, amsfonts, latexsym, cancel}
\setlength{\parskip}{0.3cm}
\usepackage{graphicx}
\usepackage{fontenc}
\usepackage{slashbox}
\usepackage{setspace}
\usepackage{gensymb}
\usepackage{accents}
\usepackage{adjustbox}
\setstretch{1.35}
\usepackage{bold-extra}
\usepackage[document]{ragged2e}
\usepackage{subcaption}
\usepackage{tcolorbox}
\usepackage{xcolor, colortbl}
\usepackage{wrapfig}
\usepackage{empheq}
\usepackage{array}
\usepackage{parskip}
\usepackage{arydshln}
\graphicspath{ {images/} }
\renewcommand*\contentsname{\color{black}Índice} 
\usepackage{array, multirow, multicol}
\definecolor{lightblue}{HTML}{007AFF}
\usepackage{color}
\usepackage{etoolbox}
\usepackage{listings}
\usepackage{mdframed}
\setlength{\parindent}{0pt}
\usepackage{underscore}
\usepackage{hyperref}
\usepackage{tikz}
\usepackage{tikz-cd}
\usetikzlibrary{shapes, positioning, patterns}
\usepackage{tikz-qtree}
\usepackage{biblatex}
\usepackage{pdfpages}
\usepackage{pgfplots}
\usepackage{pgfkeys}
\addbibresource{biblatex-examples.bib}
\usepackage[a4paper, left=1cm, right=1cm, top=1cm,
bottom=1.5cm]{geometry}
\usepackage{titlesec}
\usepackage{titletoc}
\usepackage{tikz-3dplot}
\usepackage{kbordermatrix}
\usetikzlibrary{decorations.pathreplacing}
\newcommand{\Ej}{\textcolor{lightblue}{\underline{Ejemplo}}}
\setlength{\fboxrule}{1.5pt}

% Configura el formato de las secciones utilizando titlesec
\titleformat{\section}
{\color{red}\normalfont\LARGE\bfseries}
{Tema \thesection:}
{10 pt}
{}

% Ajusta el formato de las entradas de la tabla de contenidos
\addtocontents{toc}{\protect\setcounter{tocdepth}{4}}
\addtocontents{toc}{\color{black}}

\titleformat{\subsection}
{\normalfont\Large\bfseries\color{red}}{\thesubsection)}{1em}{\color{lightblue}}

\titleformat{\subsubsection}
{\normalfont\large\bfseries\color{red}}{\thesubsubsection)}{1em}{\color{lightblue}}

\newcommand{\bboxed}[1]{\fcolorbox{lightblue}{lightblue!10}{$#1$}}
\newcommand{\rboxed}[1]{\fcolorbox{red}{red!10}{$#1$}}

\DeclareMathOperator{\N}{\mathbb{N}}
\DeclareMathOperator{\Z}{\mathbb{Z}}
\DeclareMathOperator{\R}{\mathbb{R}}
\DeclareMathOperator{\Q}{\mathbb{Q}}
\DeclareMathOperator{\K}{\mathbb{K}}
\DeclareMathOperator{\im}{\imath}
\DeclareMathOperator{\jm}{\jmath}
\DeclareMathOperator{\col}{\mathrm{Col}}
\DeclareMathOperator{\fil}{\mathrm{Fil}}
\DeclareMathOperator{\rg}{\mathrm{rg}}
\DeclareMathOperator{\nuc}{\mathrm{nuc}}
\DeclareMathOperator{\dimf}{\mathrm{dimFil}}
\DeclareMathOperator{\dimc}{\mathrm{dimCol}}
\DeclareMathOperator{\dimn}{\mathrm{dimnuc}}
\DeclareMathOperator{\dimr}{\mathrm{dimrg}}
\DeclareMathOperator{\dom}{\mathrm{Dom}}
\DeclareMathOperator{\infi}{\int_{-\infty}^{+\infty}}
\newcommand{\dint}[2]{\int_{#1}^{#2}}

\newcommand{\bu}[1]{\textcolor{lightblue}{\underline{#1}}}
\newcommand{\lb}[1]{\textcolor{lightblue}{#1}}
\newcommand{\db}[1]{\textcolor{blue}{#1}}
\newcommand{\rc}[1]{\textcolor{red}{#1}}
\newcommand{\tr}{^\intercal}

\renewcommand{\CancelColor}{\color{lightblue}}

\newcommand{\dx}{\:\mathrm{d}x}
\newcommand{\dt}{\:\mathrm{d}t}
\newcommand{\dy}{\:\mathrm{d}y}
\newcommand{\dz}{\:\mathrm{d}z}
\newcommand{\dth}{\:\mathrm{d}\theta}
\newcommand{\dr}{\:\mathrm{d}\rho}
\newcommand{\du}{\:\mathrm{d}u}
\newcommand{\dv}{\:\mathrm{d}v}
\newcommand{\tozero}[1]{\cancelto{0}{#1}}
\newcommand{\lbb}[2]{\textcolor{lightblue}{\underbracket[1pt]{\textcolor{black}{#1}}_{#2}}}
\newcommand{\dbb}[2]{\textcolor{blue}{\underbracket[1pt]{\textcolor{black}{#1}}_{#2}}}
\newcommand{\rub}[2]{\textcolor{red}{\underbracket[1pt]{\textcolor{black}{#1}}_{#2}}}

\author{Francisco Javier Mercader Martínez}
\date{}
\title{Álgebra Lineal\\Examen Convocatoria Mayo 2023}

\begin{document}
\maketitle
\begin{enumerate}[label=\color{red}\textbf{\arabic*)}]
    \item \lb{Se pide:}
        \begin{enumerate}[label=\color{red}\textbf{\alph*)}]
            \item \db{Expresa en forma binomial y exponencial el número complejo \[
            z=\dfrac{3+i}{1+2i}
            \] } 
        \item \db{Calcula y expresa en forma binomial el número complejo $z^{7}$.} 
        \end{enumerate}
    \item \lb{Consideremos la matriz $A=\begin{bmatrix} 
                0.5 & 0.3\\-0.5 & 0.1 
    \end{bmatrix} $.}
    \begin{enumerate}[label=\color{red}\textbf{\alph*)}]
        \item \db{Recordemos que la norma 1 de una matriz $B=(b_{ij})$ se define como \[
       \|B\|_1=\max_{1\le j\le n}\left\{ \sum_{i=1}^{n} |b_{ij}| \right\}  
        \]Calcula el número de condición de $A$ utilizando la norma 1.}
    \item \db{Supongamos que queremos resolver el sistema lineal $Ax=b$ y que, debido a errores de redondeo o medida, cometemos un error relativo en el término independiente de aproximadamente  $10^{-2}$. Obtén una cota del error relativo que se cometería en la solución $x$.}
        
    \end{enumerate}
\item \lb{Consideremos la matriz $A=\begin{bmatrix} 
            1 & 1 & 1 & -1\\
            2 & 1 & 5 & 2\\
            3 & 2 & 6 & 1
\end{bmatrix} $.}
\begin{enumerate}[label=\color{red}\textbf{\alph*)}]
    \item \db{Calcula una base ortogonal de $\mathrm{Col}(A)$ (subespacio de columnas de $A$) ¿Cuál es el rango de $A$?}
    \item \db{¿Qué ecuaciones debe cumplor un vector $(x,y,z)$ para pertenecer a  $\mathrm{Col}(A)$?} 
    \item \db{Calcula la proyección ortogonal del vector $b=(1,1,0)$ sobre el subespacio  $\mathrm{Col}(A)$.}
\end{enumerate}
\item \lb{Consideremos la matriz $A=\begin{bmatrix} 
            0 & 1 & -1\\
            1 & 0 & 0\\
            1 & 1 & 0
\end{bmatrix} $}
\begin{enumerate}[label=\color{red}\textbf{\alph*)}]
    \item \db{Calcula una matriz de permutación $P$, una matriz triangular inferior  $L$ y una matriz triangular superior  $U$ tales que  $PA=LU$ (factorización $PLU$).}
    \item \db{Utilizando la factorización anterior, explica cómo resolverías el sistema de ecuaciones $Ax=b$.} 
\end{enumerate}
\item \lb{Consideremos la matriz $A=\begin{bmatrix} 
            3 & 8\\
            0 & 3
\end{bmatrix} $} 
\begin{enumerate}[label=\color{red}\textbf{\alph*)}]
    \item \db{Calcula los valores propios y los valores singulares de $A$.}
    \item \db{¿Es $A$ diagonalizable? Justifica la respuesta.}
    \item \db{Calcula matrices $U$ y  $V$ ortogonales tales que  $A=U\Sigma V^\intercal$, donde $\Sigma=\begin{bmatrix} 
                \sigma_1 & 0\\ 0 & \sigma_2 
    \end{bmatrix} $ con $\sigma_1,\sigma_2$ los valores singulares de $A$.} 
\end{enumerate}
\end{enumerate}
\end{document}
