\documentclass[12pt]{article}
\usepackage{fullpage}
\usepackage[utf8]{inputenc}
\usepackage{pict2e}
\usepackage{amsmath}
\usepackage{enumitem}
\usepackage{eurosym}
\usepackage{pict2e}
\usepackage{mathtools}
\usepackage{amssymb, amsfonts, latexsym, cancel}
\setlength{\parskip}{0.3cm}
\usepackage{graphicx}
\usepackage{fontenc}
\usepackage{slashbox}
\usepackage{setspace}
\usepackage{gensymb}
\usepackage{accents}
\usepackage{adjustbox}
\setstretch{1.5}
\usepackage{bold-extra}
\usepackage[document]{ragged2e}
\usepackage{subcaption}
\usepackage{tcolorbox}
\usepackage{xcolor, colortbl}
\usepackage{wrapfig}
\usepackage{empheq}
\usepackage{array}
\usepackage{parskip}
\usepackage{arydshln}
\graphicspath{ {images/} }
\renewcommand*\contentsname{\color{black}Índice} 
\usepackage{array, multirow, multicol}
\definecolor{lightblue}{HTML}{007AFF}
\usepackage{color}
\usepackage{etoolbox}
\usepackage{listings}
\usepackage{mdframed}
\setlength{\parindent}{0pt}
\usepackage{underscore}
\usepackage{hyperref}
\usepackage{tikz}
\usepackage{tikz-cd}
\usetikzlibrary{shapes, positioning, patterns}
\usepackage{tikz-qtree}
\usepackage{biblatex}
\usepackage{pdfpages}
\usepackage{pgfplots}
\usepackage{pgfkeys}
\addbibresource{biblatex-examples.bib}
\usepackage[a4paper, left=1.5cm, right=1.5cm, top=1cm,
bottom=1.5cm]{geometry}
\everymath{\displaystyle}
\usetikzlibrary{decorations.pathreplacing}
\usepackage{titlesec}
\usepackage{titletoc}
\usepackage{tikz-3dplot}
\usetikzlibrary{decorations.pathreplacing}
\newcommand{\Ej}{\textcolor{lightblue}{\underline{Ejemplo}}}
\setlength{\fboxrule}{1.5pt}
\renewcommand{\arraystretch}{1.35}
\setlength{\arraycolsep}{0.3cm}

% Configura el formato de las secciones utilizando titlesec
\titleformat{\section}
{\color{red}\normalfont\LARGE\bfseries}
{Tema \thesection:}
{10 pt}
{}

% Ajusta el formato de las entradas de la tabla de contenidos
\addtocontents{toc}{\protect\setcounter{tocdepth}{4}}
\addtocontents{toc}{\color{black}}

\titleformat{\subsection}
{\normalfont\Large\bfseries\color{red}}{\thesubsection)}{1em}{\color{lightblue}}

\titleformat{\subsubsection}
{\normalfont\large\bfseries\color{red}}{\thesubsubsection)}{1em}{\color{lightblue}}

\newcommand{\bboxed}[1]{\fcolorbox{lightblue}{lightblue!10}{$#1$}}

\DeclareMathOperator{\N}{\mathbb{N}}
\DeclareMathOperator{\Z}{\mathbb{Z}}
\DeclareMathOperator{\R}{\mathbb{R}}
\DeclareMathOperator{\Q}{\mathbb{Q}}
\DeclareMathOperator{\K}{\mathbb{K}}
\DeclareMathOperator{\im}{\imath}
\DeclareMathOperator{\jm}{\jmath}
\DeclareMathOperator{\col}{\mathrm{Col}}
\DeclareMathOperator{\fil}{\mathrm{Fil}}
\DeclareMathOperator{\rg}{\mathrm{rg}}
\DeclareMathOperator{\nuc}{\mathrm{nuc}}
\DeclareMathOperator{\dimf}{\mathrm{dimFil}}
\DeclareMathOperator{\dimc}{\mathrm{dimCol}}
\DeclareMathOperator{\dimn}{\mathrm{dimnuc}}
\DeclareMathOperator{\dimr}{\mathrm{dimrg}}

\newcommand{\bu}[1]{\textcolor{lightblue}{\underline{#1}}}
\newcommand{\lb}[1]{\textcolor{lightblue}{#1}}
\newcommand{\db}[1]{\textcolor{blue}{#1}}
\newcommand{\rc}[1]{\textcolor{red}{#1}}
\newcommand{\tr}{^\intercal}

\renewcommand{\CancelColor}{\color{lightblue}}

\newcommand{\dx}{\:\mathrm{d}x}
\newcommand{\dt}{\:\mathrm{d}t}
\newcommand{\dy}{\:\mathrm{d}y}
\newcommand{\dz}{\:\mathrm{d}z}
\newcommand{\dth}{\:\mathrm{d}\theta}
\newcommand{\dr}{\:\mathrm{d}\rho}
\newcommand{\du}{\:\mathrm{d}u}
\newcommand{\dv}{\:\mathrm{d}v}
\newcommand{\tozero}[1]{\cancelto{0}{#1}}
\newcommand{\lbb}[2]{\textcolor{lightblue}{\underbracket[1pt]{\textcolor{black}{#1}}_{#2}}}
\newcommand{\dbb}[2]{\textcolor{blue}{\underbracket[1pt]{\textcolor{black}{#1}}_{#2}}}
\title{Machine Learning II\\Ejercicio de ampliación}

\begin{document}
\maketitle
\begin{itemize}[label=\color{red}\textbullet, leftmargin=*]
    \item \lb{He instalado una alarma antirrobo en mi casa, pudiendo ésta activarse $(A)$ tanto si hay un robo $(B)$ como si se produce un terremoto $(E)$.} 
    \item \lb{Las especificaciones de la alarma son las siguientes: \[ 
            \begin{array}{ccc} 
                \hline 
                B & E & P(A|B,E)\\ \hline 
                0 & 0 & 0.001\\ 
                0 & 1 & 0.29\\ 
                1 & 0 & 0.94\\ 
                1 & 1 & 0.95\\ \hline 
            \end{array} 
        \]} 
    \item \lb{La tasa diaria de robos en mi vecindario es del $0.1\%$, mientras que la tasa diaria de terremotos en esta zona es del $0.2\%$.} 
    \item \lb{Si suena la alarma, alguno de mis dos vecinos podría avisarme por teléfono} 
    \item \lb{Mi vecino Juan me llamará $(J)$ con una probabilidad del $90\%$ si realmente se produce una alarma en mi domicilio, aunque también podría confundirla con el timbre de la puerta con una probabilidad de $5\%$.} 
    \item \lb{Mi vecina María me llamará $(M)$ si oye la alarma, lo cual tiene una probabilidad del $70\%$ puesto que siempre escucha la múscia muy alta, pudiendo además equivocarse con una probabilidad del $1\%$.} 
\end{itemize}
\lb{¿Qué probabilidad hay de que se esté produciendo un robo en mi casa si recibo una llamada de María?} 
\begin{enumerate}[label=\arabic*)]
    \item Datos de problema
        \begin{itemize}[label=\textbullet]
            \item $P(B)=0.001,\, P(\neg B)=(1-P(B))=0.999$.
            \item $P(E)=0.002,\,P(\neg E)=(1-P(E))=0.998$. 
            \item Para María:
                \begin{itemize}[label=\textbullet]
                    \item $P(M|A)=0.70$ (llama si oye la alarma, 70\%).
                    \item  $P(M|\neg A)=0.01$ (se confunde, 1\%).
                \end{itemize}
        \end{itemize}
    \item Probabilidad de que suene la alarma $P(A)$
         \[
            P(A)=\lbb{P(A|\neg B,\neg E)P(\neg B)P(\neg E)}{t_1}+\lbb{P(A|\neg B,E)P(\neg B)P(E)}{t_2}+\lbb{P(A|B,\neg E)P(B)P(\neg E)}{t_3}+\lbb{P(A|B,E)P(B)P(E)}{t_4}.
        \] 
        Cálculo numérico:
        \begin{itemize}[label=\textbullet]
            \item $t_1=0.001\cdot 0.999\cdot 0.998=0.000997002$
            \item $t_2=0.29\cdot 0.999\cdot 0.002=0.00057942$
            \item $t_3=0.94\cdot 0.001\cdot 0.998=0.00093812$
            \item $t_4=0.95\cdot 0.001\cdot 0.002=0.0000019$
        \end{itemize}
        \[
        P(A)=t_1+t_2+t_3+t_4=0.002516442
        \] 
    \item Probabilidad de recibir llamada de María $P(M)$
         \[
        P(M)=P(M|A)P(A)+P(M|\neg A)P(\neg A)=0.7\cdot P(A)+0.01\cdot (1-P(A))=0.01173634498
        \] 
    \item Probabilidad conjunto $P(B,M)$

        Descomponemos por  $A$:  \[
        P(B,M)=P(M|A)P(A,B)+P(M|\neg A)P(\neg A,B).
        \] 
        Primero $P(A,B)$:
         \[
        P(A,B)=P(A|B,\neg E)P(B)P(\neg E)+P(A|B,E)P(B)P(E)=0.00093812+0.0000019=0.00094002
        \] 
        Luego $P(\neg A,B)=P(B)-P(A,B)=0.001-0.00094002=0.00005998$.

        Así, \[
            \begin{aligned}
                P(B,M)&= 0.7\cdot 0.00094002+0.01\cdot 0.00005998\\
                &= 0.000658014+0.0000005998 \\
                &= 0.0006586138 \\
            \end{aligned}
        \] 
    \item Posterior buscado $P(B|M)$
         \[
        P(B|M)=\dfrac{P(B,M)}{P(M)}=\dfrac{0.0006586138}{0.01173634498}\approx 0.5612.
        \] 
\end{enumerate}
\end{document}
