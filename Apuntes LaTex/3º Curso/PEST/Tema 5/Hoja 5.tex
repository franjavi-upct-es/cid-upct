\begin{center}
    \textbf{\Large Hoja 5: Métodos de Alisado Exponencial} 
\end{center}
\begin{enumerate}[label=\color{red}\textbf{\arabic*)}]
    \item \lb{Responder a las siguientes preguntas}
        \begin{enumerate}[label=\color{red}\textbf{\alph*)}]
            \item \db{¿Qué son los métodos de alisado exponencial y para qué sirven?} 
            \item \db{Clasificación de los métodos de alisado exponencial.} 
            \item \db{¿Qué son los parámetros de alisado y dónde intervienen? ¿Cuántos parámetros de alisado deben emplearse?} 
            \item \db{¿Cómo se obtiene los parámetros de alisado óptimos?} 
            \item \db{Supongamos que se está analizando una serie temporal que no tiene estacionalidad y cuyas observaciones se recopilan cada minuto. El analista ha decidido emplear el método de Holt para predecir los próximos 30 minutos, y transcurrido ese tiempo, volverá a aplicar el método de Holt con las nuevas observaciones reales para predecir los siguientes 30 minutos. ¿Te parece adecuado?} 
            \item \db{Explica cómo se interpretaría que en el método de Holt-Winters (aditivo o multiplicativo), se obtuviera un parámetro de alisado óptimo gamma=0.} 
            \item \db{} 
            \item \db{} 
            \item \db{} 
            \item \db{} 
        \end{enumerate}
    \item \lb{Responder a las siguientes cuestiones:}
        \begin{enumerate}[label=\color{red}\textbf{\alph*)}]
            \item \db{Para analizar una serie temporal con 100 observaciones, se usó alisado exponencial simple (AES). Se sabe que la última observación vale $x(100)=28$, que el parámetro de alisado óptimo fue  $\alpha=0.8$ y que la estimación del nivel de la serie en el instante 99 fue $a(99)=35$. Calcular las predicciones para los próximos 5 instantes.} 
                
                $a_t=\alpha x_t+(1-\alpha)a_{t-1}\implies a_{100}=0.8\cdot x_{100}+(1-0.8)\cdot a_{99}=0.8\cdot 28+0.2\cdot 35=\bboxed{29.4} $ 

                $\hat{x}_{101|100}=\dots=\hat{x}_{105|100}=a_{100}=29.4$
            \item \db{Repetir el apartado (a) suponiendo que se usó el método de Holt y que se tienen los siguientes datos: $x(100)=28,\alpha=0.8,\beta=0.7,a(99)=35,b(99)=1$.} 
                \begin{itemize}[label=\textbullet]
                    \item $a_{100}=\alpha x_{100}+(1-\alpha)(a_{99}+b_{99})=0.8\cdot 28+0.2\cdot (35+1)=19.6$ 
                    \item $b_{100}=\beta(a_{100}-a_{99})+(1-\beta)b_{99}=0.7\cdot (29.6-35)+0.3\cdot 1=-3.48$
                \end{itemize}
                $\hat{x}_{100+h|100}=a_{100}+h\cdot b_{100}=29.6+h\cdot (-3.48)\quad h=1,2,3,4,5.$

                $\begin{array}{l}
                    h=1\implies \hat{x}_{101|100}=26.12\\
                    h=2\implies \hat{x}_{102|100}=22.64\\
                    h=3\implies \hat{x}_{103|100}=19.16\\
                    h=4\implies \hat{x}_{104|100}=15.68\\
                    h=5\implies \hat{x}_{105|100}=12.20\\
                \end{array}$
        \end{enumerate}
    \item \lb{Para una serie de datos con 30 observaciones recopiladas por cuatrimestres, comenzando en el primer cuatrimestre de 2014, se observó componente estacional y se aplicó Holt-Winters aditivo. Se pide calcular las predicciones para los años 2024 y 2025 sabiendo que se obtivieron los siguientes datos: \[
    \begin{array}{l}
        x(30)=250,\,\alpha=0.8,\,\beta=0.9,\,\gamma=0,\,a(29)=200,\\
        b(29)=-2,\,S(1)=25,\,S(2)=-15,\,S(3)=-10
    \end{array}
    \] } 

    Observar que $\gamma=0$, por tanto:
    \[
        \begin{aligned}
            S_t=S_{t-L}\,\forall t\implies & S_{28}=S_{31}=S_1=25\\
                                           & S_{29}=S_{32}=S_2=-15\\
                                           & S_{30}=S_{33}=S_3=-10
        \end{aligned}
    \] 
    Como la serie empieza en el $1^{\text{er}}$ cuatrimestre de 2014 $(t=1)$ entonces termina en $3^{\text{er}}$ cuatrimestre de 2023 $(t=30)$. Me piden predicciones para los cuatrimestre de 2024 y 2025, es decir, predicciones en horizontes  $h=1,2,3,4,5,6$.

     $a_{30}=\alpha(x_{30}-S_{30-L})+(1-\alpha)(a_{29}+b_{29})=0.8\cdot (250-(-10))+0.2\cdot (200+(-2))=247.6$

     $b_{30}=\beta(a_{30}-a_{29})+(1-\beta)b_{29}=0.9\cdot (247.6-250)+0.1\cdot (-2)=42.64$

     $S_{30}=S_3=-10$

     $
     \begin{array}{l}
     \bboxed{h=1}\quad \hat{x}_{31|30}=(a_{30}+1\cdot b_{30})+S_{28}=247.6-42.64+25=\bboxed{315.24}\\
     \bboxed{h=2}\quad \hat{x}_{32|30}=(a_{30}+2\cdot b_{30})+S_{29}=247.6+2\cdot 42.64-15=\bboxed{317.88} \\
     \bboxed{h=3}\quad \hat{x}_{33|30}=(a_{30}+3\cdot b_{30})+S_{30}=\\
     \bboxed{h=4}\quad \hat{x}_{34|30}=(a_{30}+4\cdot b_{30})+S_{31}=\\
     \bboxed{h=5}\quad \hat{x}_{35|30}=(a_{30}+5\cdot b_{30})+S_{32}=\\
     \bboxed{h=6}\quad \hat{x}_{36|30}=(a_{30}+6\cdot b_{30})+S_{33}=\\
     \end{array}
     $
    \item \lb{Repetir el Problema 3 suponiendo esquema multiplicativo y que \[
    S(1)=15,\,S(2)=0.6,\,S(3)=0.9
    \] } 
    \item \lb{} 
    \item \lb{} 
\end{enumerate}
