\section{Estadística Descriptiva Univariante}
\subsection{Introducción a la Estadística}
\begin{itemize}
	\item Estudia cómo obtener conclusiones de la investigación empírica mediante el uso de modelos matemáticos.
	\item Disciplina puente entre los modelos matemáticos y los fenómenos reales.
	\begin{itemize}
		\item Modelo matemáticos: abstracción simplificada de la realidad.
	\end{itemize}
	\item Metodología para evaluar y juzgar las discrepancias entre Realidad y Teoría.
	\item Aplicaciones: Economía, Sociología, Ecología, Tecnología, $\hdots$
\end{itemize}
\subsubsection*{Partes de la Estadística}
\begin{itemize}
	\item \textbf{Estadística Descriptiva:}
	\begin{itemize}
		\item \textbf{Objetivo:} Resumir la información contenida en los datos. Para ellos utiliza \textit{métodos gráficos} y \textit{medidas numéricas resumen}.
		\item Si no es posible estudiar todos los elementos $\Longrightarrow$ seleccionamos una \textbf{Muestra}.
	\end{itemize}
	\item \textbf{Inferencia Estadística:}
	\begin{itemize}
		\item \textbf{Objetivo:} Inferir (generalizar) la información de la muestra a todos los elementos.
		\item Contrastar una hipótesis.
		\item Determinar y medir relaciones entre variables.
	\end{itemize}
	\item \textbf{Teoría de la Probabilidad:}
	\begin{itemize}
		\item \textbf{Objetivo:} Estudiar fenómenos aleatorios, donde interviene incertidumbre.
		\item Construye modelos matemáticos que sirve de puente entre la Teoría y la Realidad.
	\end{itemize}
\end{itemize}
\subsection{Conceptos básicos}
\begin{itemize}
	\item \textbf{Población:} Conjunto de individuos objeto de estudio. Teóricamente puede ser "finita" o "infinita" (tamaño muy grande).
	\item \textbf{Muestra:} Subconjunto de individuos seleccionados de la población y que serán analizados en el experimento.
	\item \textbf{Tamaño muestral:} Número de individuos seleccionados de la población y que serán analizados en el experimento.
	\item \textbf{Muestreo:} Técnica o proceso empleado para obtener una muestra.
	\item \textbf{Experimento:} Proceso que permite asociar a cada individuo de una población un símbolo (o un grupo de símbolos), numérico o no, de entre un conjunto de símbolos dado a priori.\\ \textbf{Tipos de experimentos}
	\begin{itemize}
		\item Aleatorios.
		\item Deterministas.
		\item Pseudoaleatorios.
	\end{itemize}
	\item \textbf{Variable estadística:} Cualquier característica que podemos observar un individuo. Ejemplos ¡: "altura", "edad", "nivel de calidad de un producto", "tiempo de vida de una marca de baterías", etc.
\end{itemize}
\subsubsection*{Tipos de variables:}
\begin{itemize}
	\item \textbf{Variables cualitativas:} El resultado de la observación no es un valor numérico sino un atributo, una cualidad. Ejemplo: "color de pelo", "intención de voto", $\hdots$
	\item \textbf{Variables cuantitativas:} Toman valores numéricos. Dos tipos:
	\begin{itemize}
		\item \textbf{Cuantitativas discretas:} Toman un número finito de valores diferentes (o teóricamente infinito numerable). Ejemplo: "el número de hijos de las familias en España".
		\item \textbf{Cuantitativas continuas:} Toman valores dentro de un intervalo (corresponden a medidas de magnitudes continuas). Ejemplo: "peso", "altura", $\hdots$
	\end{itemize}
	\item En adelante, denotaremos por $\left\lbrace x_{1}, x_{2}, \hdots, x_{m}\right\rbrace$ a la muestra o conjunto de datos en estudio.
\end{itemize}
\subsection{Las tablas de frecuencias}
\begin{itemize}
	\item Los datos se agrupan en clases (que denotaremos por $A_{i}$), que deben cubrir todo el rango de datos y ser disjuntas dos a dos.
	\begin{itemize}
		\item Variables cualitativas: cada clase coincide con un atributo (o varios).
		\item Variables cuantitativas discretas: cada clase coincide con un valor (o varios) de los que toma la variable.
		\item Variables cuantitativas continuas: cada clase coincide con un intervalo de valores (intervalo de la recta Real).
	\end{itemize}
	\item Para cada clase, se debe indicar su frecuencia.
	\begin{itemize}
		\item \textbf{Frecuencia absoluta de una clase $A_{i}$:} número de veces que se observa dicha clase y se representa por $f_{i}$.
		\item \textbf{Frecuencia relativa de una clase $A_{i}$:} cociente entre la frecuencia absoluta de la clase y el tamaño de la muestra, $h_{i}=\frac{f_{i}}{n}$.
		\item \textbf{Frecuencia absoluta acumulada de una clase $A_{i}$:} es la suma de las frecuencias absolutas de las clases $A_{j}$, con $j=1,2,\hdots,i$. Se representa por $F_{i}=f_{1}+f_{2}+\cdots+f_{i}$.
		\item \textbf{Frecuencia relativa acumulada de una clase $A_{i}$:} Se representa por $H_{i}=h_{1}+h_{2}+\cdots+h_{i}$ (suma de frecuencias relativas).
	\end{itemize}
\end{itemize}
\subsubsection*{¿Cómo determinamos las clases a realizar?}
\begin{itemize}
	\item Cuando la variable es continua (o bien discreta con muchos valores distintos), las clases se corresponden con intervalos de la recta Real. Pero, \textbf{¿cuántas clases hacemos y cómo las determinamos?}
	\item Existen varias reglas para determinar el número de clases $(k)$ a realizar, a partir del tamaño muestral $(n)$:
	\begin{itemize}
		\item \textbf{Regla de la raíz cuadrada:} tomar $k\simeq \sqrt{n}$.
		\item \textbf{Regla de Sturges:} tomar $k\simeq 1+\log_{2}(n)=1+\frac{\log(n)}{\log(2)}$.\\En ambos casos podemos, por ejemplo, redondear al entero más próximo.
	\end{itemize}
	\item Determinado $k$, calculamos el recorrido $R=x_{max}-x_{min}$ y la amplitud de cada clase $d=\frac{R}{k}$. Los intervalos que definen cada clase serán $$\left[x_{min},x_{min}+d\right], \left(x_{min}+d,x_{min}+2d\right], \hdots, \left(x_{max}-d,x_{max}\right]$$
	\item Llamaremos \textbf{marca de clase} $(m_{i})$ al centro del intervalo.
\end{itemize}
\textbf{EJEMPLO:} Tabla de frecuencias de los pesos de 500 alumnos de una universidad
\begin{center}
	\begin{tabular}{| c | c | c | c | c |}
		\hline
		\hline
		Clases   & $f_{1}$ & $F_{i}$ & $h_{i}$ & $H_{i}$ \\
		\hline
		\hline
		[45,50]  & 8       & 8       & 0.016   & 0.016   \\
		(50, 55] & 50      & 58      & 0.1     & 0.116   \\
		(55, 60] & 82      & 140     & 0.164   & 0.280   \\
		(60, 65] & 205     & 345     & 0.41    & 0.690   \\
		(65, 70] & 79      & 424     & 0.158   & 0.848   \\
		(70, 75] & 56      & 480     & 0.112   & 0.960   \\
		(75, 80] & 20      & 500     & 0.04    & 1       \\ \hline
	\end{tabular}
\end{center}
\subsection{Medidas numéricas resumen}
\begin{itemize}
	\item Permiten dar un resumen cuantitativo del fenómeno que estamos estudiando.
	\item Ponen de manifiesto los rasgos o características principales de la población en estudio.
	\item Son informativas cuando los datos son homogéneos y pueden ser muy engañosas cuando mezclamos poblaciones distintas.
	\item Podemos distinguir varios \textbf{tipos:}
	\begin{itemize}
		\item \textit{Medidas de centralización:} informan acerca de la tendencia central de los datos.
		\item \textit{Medidas de posición:} informan de la posición cuando consideramos los datos de la muestra ordenados de menor a mayor.
		\item \textit{Medidas de dispersión:} informan de lo concentrados o dispersos que están los datos de la muestra.
		\item \textit{Medidas de forma:} informan acerca de la forma en la que se distribuyen los datos de la muestra.
	\end{itemize}
\end{itemize}
\subsubsection{Medidas de centralización}
\begin{itemize}
	\item \textbf{Media (aritmética):} Representa el centro de gravedad de los datos.\\ $\overline{x}=\frac{\displaystyle \sum_{i=1}^{n}{x_{i}}}{n}; \overline{x}=\frac{\displaystyle \sum_{i=1}^{k}{x_{i}\cdot f_{i}}}{n}=\displaystyle \sum_{i=1}^{k}{x_{i} \cdot h_{i}}$ donde $k$ es el nº de valores distintos
	\item \textbf{Mediana:} Divide a la muestra ordenada en dos partes, dejando a su izquierda el 50\% de las observaciones y a su derecha el otro 50\%. $$Me=\left\lbrace\begin{array}{cl} x_{(\frac{n+1}{2})} & ~si~ n ~es~impar~\\ \frac{x_{(\frac{n}{2})}+x_{(\frac{n}{2}+1)}}{2} & ~si~ n ~es~par~\end{array}\right.$$
	\begin{center}$x_{(1)} \le x_{(2)} \le \hdots \le x_{(n)}$ muestra ordenada de menor a mayor.\end{center}
	\item \textbf{Moda:} Es el valor de la muestra con mayor frecuencia $(Mo)$.
	\item \textit{Propiedad:} La media es muy sensible a valores extremos, mientras que mediana y moda no.
\end{itemize}
\subsubsection*{¿Y si tenemos datos guardados agrupados en clases?}
\begin{itemize}
	\item \textbf{Media:} Se toman como observaciones las marcas de clase.
	
	$\overline{x}=\displaystyle \frac{\displaystyle \sum_{i=1}^{k}{m_{i} \cdot f_{i}}}{n}=\displaystyle \sum_{i=1}^{k}{m_{i} \cdot h_{i}}$, donde $\begin{tabular}{rc} $k=$& nº de clases\\ $m_{i}=$& marca de clase\end{tabular}$
	\item \textbf{Mediana:} Una opción es identificar el intervalo donde se encontraría la mediana y proporcionar su marca de clase. Existen expresiones alternativas.
	\item \textbf{Moda:} Consideraremos la moda como la marca de la clase con mayor frecuencia. La moda no tiene por qué ser única, la distribución puede ser \textit{unimodal} y \textit{multimodal}.
\end{itemize}
\subsubsection{Medidas de posición}
\begin{itemize}
	\item \textbf{Cuartiles:} Dividen a la muestra ordenada en 4 partes "iguales". Hay 3 cuartiles ($Q_1, Q_2$ y $Q_3$), siendo $Me=Q_2$.
	\item \textbf{Deciles:} Dividen a la muestra ordenada en 10 partes "iguales". Hay 9 deciles $(D_1, \hdots, D_9)$ siendo $Me=D_5$.
	\item \textbf{Percentiles:} Dividen a la muestra ordenada en 100 partes "iguales". Hay 99 percentiles $(P_1, \hdots, P_99)$, siendo $Me=P_50$.
	\item \textbf{Cuantiles:} Fijado $\alpha \in [0, 1]$, se llama \textit{cuantil alpha} al valor $C_{\alpha}$ que deja a su izquierda el $(100\cdot \alpha)\%$ de las observaciones y a su derecha el $(100\cdot (1-\alpha))\%$ restante.
	\begin{itemize}
		\item Para $\alpha=0.25, 0.5, 0.75 \rightarrow$ cuartiles.
		\item Para $\alpha=0.10, 0.20, \hdots,0.90 \rightarrow$ deciles.
		\item Para $\alpha=0.01, 0.02, \hdots, 0.99 \rightarrow$ percentiles.
	\end{itemize}
	\item \textbf{Rango o recorrido:} $R=x_{max}-x_{min}$ (poco representativa).
	\item \textbf{Varianza:} $s^{2}=\frac{\sum_{i=1}^{n}{(x_{i}-\overline{x})^2}}{n}=\overline{x^{2}}-(\overline{x})^{2}$ (unidades al cuadrado).
	\item \textbf{Cuasivarianza:} $(s^{*})^{2}=\frac{\sum_{i=1}^{n}{(x_{i}-\overline{x})^2}}{n-1}$ (unidades al cuadrado).
	\item \textbf{Desviación típica:} $s=\sqrt{s^{2}}$ (mismas unidades que los datos).
	\item \textbf{Cuasidesviación típica:} $s^{*}=\sqrt{(s^{*})^{2}}$ (mismas unidades).
	\item \textbf{Rango intercuartílico (RIC o IQR):} $RIC=Q_{3}-Q_{1}$ (es el recorrido del 50\% datos centrales).
	\item \textbf{Coeficiente de variación:} $CV=\frac{s}{\overline{x}}$ (adimensional). Software: $CV=\frac{s^{*}}{\overline{x}}$.
	\item \textbf{MEDA:} Mediana de las desviaciones absolutas de los datos respecto a la mediana.
	\item \textit{Propiedad:} RIC y MEDA no son sensibles a valores extremos, el resto sí.
\end{itemize}
\subsubsection{Medidas de forma}
\begin{itemize}
	\item \textbf{Coeficiente de asimetría:} $CA=\displaystyle \frac{\sum_{i=1}^{n}{(x_{i}-\overline{x})^3}}{n \cdot s^{3}}$\\ Si CA$>>$0 (asimetría cola derecha), si CA$<<$0 (asimetría cola izquierda). Si hay simetría CA$\simeq$0.
	\item Existen varios coeficientes de asimetría (Pearson, Bowley, etc$\hdots$).
	\item \textbf{Coeficientede apuntamiento o curtosis:} $CA_{p}=\displaystyle \frac{\sum_{i=1}^{n}{(x_{i}-\overline{x})^4}}{n \cdot s^{4}}$, si $CA_{p}<3$ (mesocúrtica).
\end{itemize}
\subsubsection{Momentos muestrales}
\begin{itemize}
	\item \textbf{Momentos respecto al origen:} $$a_{r}=\frac{\sum_{i=1}^{k}{x_{i}^{r}}}{n}=\frac{\sum_{i=1}^{k}{x_{i}^{r} \cdot f_{i}}}{n}$$
	\item \textbf{Momentos respecto a la media:} $$m_{r}=\frac{\sum_{i=1}^{n}{(x_{i}-\overline{x})^{r}}}{n}=\frac{\sum_{i=1}^{k}{(x_{i}-\overline{x})^{r}\cdot f_{i}}}{n}$$
	\item Relación entre momentos: $$m_{r}=a_{r}-\binom{r}{1}a_{r-1}\cdot \overline{x}+\binom{r}{2}a_{r-2}\cdot \overline{x}^{2}-\hdots \pm \overline{x}^{r}$$ $$a_{r}=m_{r}-\binom{r}{1}m_{r-1}\cdot \overline{x}+\binom{r}{2}m_{r-2}\cdot \overline{x}^{2}+\hdots + \overline{x}^{r}$$
\end{itemize}
\subsection{Representaciones gráficas}
\begin{itemize}
	\item Resumen visual del conjunto de datos.
	\item Diferentes gráficos según el tipo de variable en estudio.
	\item \textbf{Variables cualitativas:} diagrama de barras (Pareto) y diagrama de sectores (tarta).
\end{itemize}
\begin{figure}[h]
	\begin{subfigure}{0.5\textwidth}
		\centering
		\includegraphics[width=1\linewidth]{"Temas/Imagenes/Tema 1/Diagrama de barras"}
		\caption*{Diagrama de barras}
	\end{subfigure}
	\begin{subfigure}{0.5\textwidth}
		\centering
		\includegraphics[width=1\linewidth]{"Temas/Imagenes/Tema 1/Diagrama de sectores"}
		\caption*{Diagrama de sectores(tarta)}
	\end{subfigure}
\end{figure}
\newpage
\begin{itemize}
	\item \textbf{Variables discretas:} diagrama de barras y barras acumulativo.
\end{itemize}
\begin{figure}[h]
	\begin{subfigure}{0.5\textwidth}
		\centering
		\includegraphics[width=1\linewidth]{"Temas/Imagenes/Tema 1/Diagrama de barras 2"}
		\caption*{Diagrama de barras}
	\end{subfigure}
	\begin{subfigure}{0.5\textwidth}
		\centering
		\includegraphics[width=1\linewidth]{"Temas/Imagenes/Tema 1/Diagrama de barras acumulativo"}
		\caption*{Diagrama de barras acumulativo}
	\end{subfigure}
\end{figure}

\begin{itemize}
	\item \textbf{Variables continuas:} histograma y diagrama de caja-bigotes.
\end{itemize}

\begin{figure}[h]
	\begin{subfigure}{0.5\textwidth}
		\centering
		\includegraphics[width=0.9\linewidth]{"Temas/Imagenes/Tema 1/Histograma"}
		\caption*{Histograma}
	\end{subfigure}
	\begin{subfigure}{0.5\textwidth}
		\centering
		\includegraphics[width=0.9\linewidth]{"Temas/Imagenes/Tema 1/Diagrama de caja-bigotes"}
		\caption*{Diagrama de caja-bigotes}
	\end{subfigure}
\end{figure}
\newpage
\subsubsection*{Histograma}
Es la representación gráfica de la tabla de frecuencias para datos agrupados en intervalos. Permite responder a cuestiones del tipo:
\begin{itemize}
	\item ¿Una o varias poblaciones? (Unimodal o multimodal)
	\item ¿Tendencia central de los datos?
	\item ¿Mucha o poca dispersión?
	\item ¿Forma simétrica o asimétrica? ¿Apuntamiento?
\end{itemize}

\subsubsection*{Diagrama de Caja-Bigotes}
Permite responder a cuestiones del tipo:
\begin{itemize}
	\item ¿Tendencia central de los datos?
	\item ¿Mucha o poca dispersión?
	\item ¿Forma simétrica o asimétrica?
	\item ¿Existen datos atípicos?
\end{itemize}

\bu{Ejemplo 1: Color de pelo}

{$\left.\begin{array}{|c|c|}
	\hline
	\text{Clase}  & f_i \\ \hline
	\text{Moreno} & 15 \\ \hline
	\text{Castaño} & 6 \\ \hline
	\text{Rubio} & 3 \\ \hline
	\text{Rojo} & 1 \\ \hline
	\multicolumn{1}{|c}{} & 25 \\ \cline{2-2}
\end{array}~~\color{lightblue}\right\}$\lb{Cualitativa}}

\bu{Ejemplo 2: Número de hermanos}

{$\left.\begin{array}{|c|c|c|c|c|}
		\hline
		\text{Clase} & f_i & F_i & h_i & H_i \\ \hline
		0 & 9 & 9 & \frac{9}{25}=36\% & \frac{9}{25}=0.36\\ \hline
		1 & 8 & 17 & \frac{8}{25}=32\% & \frac{27}{25}=0.68\\ \hline
		2 & 5 & 22 & \frac{5}{25}=20\% & \frac{22}{25}=0.88\\ \hline
		\ge3 & 3 & 25 & \frac{3}{25}=12\% & \frac{25}{25}=1\\ \hline
		\multicolumn{1}{c}{} & \multicolumn{1}{|c}{25} & \multicolumn{1}{|c}{73} & \multicolumn{1}{|c}{100\%} & \multicolumn{1}{|c}{} \\ \cline{2-4}
	\end{array}~~\color{lightblue}\right\}$\lb{Discreta}}

\bu{Ejemplo 3: Peso alumnos}

$\left.\begin{array}{|c|c|c|c|c|}
	\hline
	\hline
	\text{Clases} & f_i & F_i & h_i & H_i \\ \hline
	\hline
	[45,50]  & 8 & 8 & 0.016 & 0.016\\ 
	(59,55] & 50 & 58 & 0.1 & 0.116\\
	(55,60] & 82 & 140 & 0.164 & 0.280\\
	(60,65] & 205 & 345 & 0.41 & 0.690\\
	(65,70] & 79 & 424 & 0.158 & 0.848\\
	(70,75] & 56 & 480 & 0.112 & 0.960\\
	(75,80] & 20 & 500 & 0.04 & 1\\
\end{array}~~\color{lightblue}\right\}$\lb{Continua}
